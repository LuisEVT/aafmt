%%% CPT need more explanation of U notation. Make it clear that there is always a universal set
%%% Descriptions of sets need to be concise.


\chap{Set Theory}{sets}



\section{Set Basics \quad
\sectionvideohref{ciswhn56NJc&index=12&list=PL2uooHqQ6T7PW5na4EX8rQX2WvBBdM8Qo}}\label{basics}

You've probably seen sets, set relations, and set operations in previous classes.  In fact, in the previous two chapters of this book you've already been working with sets.  So we'll review them quickly before moving on to further properties and proofs concerning sets and their accessories.
\medskip

This chapter is an adapted and expanded version of a chapter by D. and J. Morris.
 
\subsection{Definition and examples}

First of all, let's give a precise mathematical definition for ``set'':

\begin{defn} A \term{set}\index{Set!definition of} is a well-defined collection of objects: that is, it is defined in such a manner that we can determine for any given object $x$ whether or not $x$ belongs to the set.  The objects that belong to a set are called its \term{elements}\index{Element!of a set}\index{Set!elements} or \term{members}. We will denote sets by capital letters, such as $A$ or $X$; if $a$ is an element of the set $A$, we write $a \in A$\label{sets_membership}.
\end{defn}
% \subsection{How to specify sets}
Two common ways of specifying sets are:
\begin{itemize}
\item
by listing all of its elements inside a pair of braces; or 
\item
by stating the property that determines whether or not an object $x$ belongs to the set. 
\end{itemize}
For example, we could define a particular set $E$ by listing its elements: 
\[
E = \{2, 4, 6, \ldots \}, \]
or by specifying properties which characterize its elements:
\[ E = \{ x :  x > 0  \textrm{ and } x \textrm{ is divisible by 2}\}.
\]
(here the ``:'' signifies ``such that'').  
We can also describe $E$  in a less mathy way by simply calling it ``the set of positive even numbers''.

We write $2 \in E$ when we want to say that 2 is in the set $E$, and $-3 \notin E$ to say that $-3$ is not in the set $E$.

Sets don't have to involve numbers. For example, we could define a certain set $X$ by listing:
\[
X = \{\textrm{Sunday, Monday, Tuesday, Wednesday, Thursday, Friday, Saturday} \},
\]
or by property:
\[
X = \{ x :x \text{ is the name of a weekday (in English)}\}.
\]
For the purposes of this book, it would be good enough to say, ``$X$ is the set of weekday names (in English)'' (we're not so snobby about set brackets).



%%% CPT More exercises here might be good
\begin{exercise}{set1}
\begin{enumerate}[(a)]
%%% I took out the following item because the question isn't well-specified. 
% \item
% What elements are in the following set:  $P = \{x : x$ is a regular polygon\}?  Write the set as a list of objects (you can use '$\ldots$' as we've done above to indicate 'etcetera', in case it's an infinite set).
\item
What elements are in the following set:  $S = \{x : x  \textrm{ is the name of a U.S. state and } x  \textrm{ begins with `W'} \}$?  Write the set as a list of objects.
\item
Rewrite the following as a list =$\{x : x \textrm{ is a type of regular polygon with less than 6 sides} \}$.
\item
Rewrite the following set of dates by using a property:  $T = \{ \textrm{Jan. 4th  2011, Jan. 11th  2011, Jan. 18  2011, Jan. 25  2011}, \ldots, \textrm{Dec. 27 2011} \}$ (Note: January 1 2011 was on a Saturday).
\item
Write the set of odd integers $O$: (i) as a list, and (ii) by using a property.
%% CPT the following is also not well-specified
% \item
% Consider the following collection: \{numbers that are divisible by 4\}. 
% Is this collection well-defined (i.e. is it a set)? Give an example to support your answer. If the collection is not well-defined, change the description so that it specifies a set.
\end{enumerate}
\end{exercise}

It is possible for the elements of a set to be sets in their own right. For instance,  we could define
\[T = \{ x : x \mathrm{~is~a~National~League~baseball~team} \}. \]
A more mathematical (but less interesting) example would be
\[S = \{ x : x \mathrm{~is~a~set~of~integers} \}. \]
Then elements of $S$ would include the sets $\{1,2,3,4\}$, $\{\textrm{the set of odd integers}\}$, $\{ 0 \}$, and so on. 

We can even go farther, and define sets of sets of sets. For instance, the set $L$ of major baseball leagues in the U.S. has two elements: 
\[L = \{ \text{American League, National League} \}. \]
However, the American League $A$ consists of a set of teams:
\[A = \{ \text{Yankees, Red Sox,} \ldots \}, \]
whle the National League $N$ also consists of a set of teams:
\[N = \{ \text{Cubs, Phillies,} \ldots \}. \]
Each of these teams consists of a set of players: so altogether the set $L$ is a set of sets of sets!

\begin{exercise}{SetSet}
\begin{enumerate}[(a)]
\item
Describe the $21^{st}$ century as a set of sets of sets of sets of sets of sets of sets.
\hyperref[sec:set_chapter:hints]{(*Hint*)}
\item
(For you biologists out there)  Describe the animal kingdom as a set of sets of sets of sets of sets of sets of sets of sets
\item
Voters (in the U.S.) can be described as Democrat, Republican, or Independent.  Each voter in these groups can be described as liberal or conservative. Why does this division \emph{not} provide a description of  the set of American voters as a set of sets of sets? 
\hyperref[sec:set_chapter:hints]{(*Hint*)}
\end{enumerate}
\end{exercise}

This notion of ``sets of sets'' can bring us into dangerous territory. For example, consider the set 
\[S = \{ x : x \mathrm{~is~a~set~which~is~not~an~element~of~itself} \}. \]
We may then pose the question: is $S$ an element of itself?\footnote{This question is called \term{Russell's paradox}, and plays an important role in the history of set theory.} 

Let us consider the possibilities:
\begin{itemize}
\item
 Suppose first that $S$ is an element of itself. 
Then $S$ must satisfy the defining property of elements of  $S$ -- that is, $S$ must be an example of a set $x$ for which ``$x$ is  not an element of itself.'' It follows that $S$ is not an element of itself.  This contradicts our supposition -- so apparently our supposition is wrong, and $S$ must not be an element of itself.
\item
On the other hand, suppose that $S$ is not an element of itself. Then $S$ satisfies the defining property of elements of  $S$ -- that is, $S$ is an example of a set $x$ for which ``$x$ is  not an element of itself.'' It follows that $S$ is an element of $S$.  Once again this contradicts our supposition -- so apparently $S$ must be an element of itself!
\end{itemize}
How do we get out of this mess? No matter what we assume, we end up with a contradiction! The problem, as is often the case, lies in  \emph{hidden assumptions} that we have made. Our definition of $S$ makes reference to the unknown $x$, where $x$ is an ``arbitrary'' set. Herein lies the rub:  the notion of ``arbitrary'' set is \emph{not well-defined}. Put another way: the set of ``all possible sets'' is NOT a set!

In the following discussion we will avoid this problem by always starting out with a well-defined set that contains all the sets and elements of interest in a particular example or problem. Such an all-encompassing set is referred to as a \term{universal set}\index{Universal set}. Note each particular problem will have its own universal set. For instance, if we are talking about public opinion polls  in the United States, an appropriate universal set might be the set of American citizens. If we're talking about sets of prime and composite numbers, our universal set could be either the set of integers, or the set of natural numbers. If we are talking about roots of algebraic equations, depending on our particular interest we might choose the universal set to be the set of real numbers, or the set of complex numbers. When we talk about sets in a general way, we often denote sets by captial  letters $A, B, C,...$, and it's assumed that all these sets are subsets of some universal set $U$.

\subsection{Important sets of numbers}
We will refer often to the following sets of numbers. Although we are presuming that these sets are ``given'', the reader should be aware that it's not at all easy to formally define them in a mathematically precise way.  (Although we won't give any definitions here,  you may encounter them in other mathematics courses, such as logic or analysis.)
\begin{itemize}
\item
${\mathbb N}\label{naturalnum}  = \{n: n \text{ is a natural number}\}  = \{1, 2, 3, \ldots \};$ (Note that according to our definition the natural numbers do \emph{\underline{not}} include $0$. Some books include $0$ as a natural number.)\index{Number!natural}\index{Natural number} 
\item
${\mathbb Z}\label{sets_integers}  = \{n : n \text{ is an integer} \} = \{\ldots, -1, 0, 1,  2, \ldots \};$
\item
${\mathbb Q}\label{rationals} = \{r : r \text{ is a rational number}\};$
\item
${\mathbb R}\label{reals} = \{ x : x \text{ is a real number} \};$
\end{itemize}

You may recall that in  Chapter~\ref{complex}, we defined  the set of complex numbers ${\mathbb C}$ as
\[ {\mathbb C} := \{ x + iy, \text{ such that } x,y \in {\mathbb R} \}. \] 
This is just one example of a favorite gambit of mathematicians, namely creating new sets from existing sets in various imaginative ways. You'll be seeing many more examples of this as we go along.


\subsection *{Subsets and proper subsets}
% Back in grade school, you were introduced to \term{relations} between integers, such as 'less than' ($<$).\index{Relation}

% \begin{exercise}{}
% Name five different relations between integers that you saw in grade school.
% \end{exercise}
% Of course, these relations can be applied to rational numbers and real numbers as well:

% \begin{exercise}{}
% \begin{enumerate}[(a)]
% \item
% Given two rational numbers $a/b$ and $c/d$, where $a,b,c,d$ are positive integers, define the relation '$<$' using only \emph{integer} operations by completing the following sentence:
% \[ a/b < c/d \mbox{ if and only if } \ldots \]
% Note that you can't use any division in your answer, because integers are not closed under division.
% \item
% Now suppose that $a,c$ can be \emph{any} integers, and $b,d$ can be any \emph{nonzero} integers. Show by example that the statement in (a) is no longer true.
% \item
% ** Modify the statement so that it is true for any integers $a,c$ and any nonzero integers $b,d.$  \emph{(Hint: consider four separate cases, depending on the denominators. What are the four different cases?)}
% \end{enumerate}
% \end{exercise}

% \begin{exercise}{}
% Can you define a relation '$<$'  on the complex numbers ${\mathbb C}$? If so, give the definition; if not, explain why not.
% \end{exercise}

% It turns out that 'relation' is an important concept in higher mathematics that extends far beyond numbers. In fact, we can define a relation between sets. You've seen it before: here we give the mathematical definition:

\begin{defn}\label{definition:sets:subset}
A set $A$ is a \term{subset}\index{Subset!definition of} of $B$, written $A \subset B$\label{setcontain} or $B \supset A$, if every element of $A$ is also an element of $B$.  
\end{defn}

For example, using this notation we may write:
\[
\{\text{sons of John and Jane Doe}\} \subset \{\text{children of John and Jane Doe\}}
\]
and 
\[
\{4,5,8\} \subset \{2, 3, 4, 5, 6, 7, 8, 9 \}
\]
and
\[
{\mathbb N} \subset {\mathbb Z} \subset {\mathbb Q} \subset {\mathbb R} \subset {\mathbb C}.
\]
According to Definition~\ref{definition:sets:subset}, every set is a subset of itself.  That is, for any set $A$, $A \subset A$, since every element in $A$ is (of course) in $A$.  Sometimes though we may want to take about subsets of $A$ that really are strictly contained in $A$, without being all of $A$. Such subsets are called \term*{proper subsets}\index{Subset!proper}. Formally, a set $B$ is a \term{proper subset} of a set $A$ if $B \subset A$ and $B \neq A$. For instance, if John and Jane Doe had only sons, then
$\{\textrm{sons of John and Jane Doe}\}$ is not a proper subset of $\{\textrm{children of John and Jane Doe}\}$.

\begin{rem} 
In this book, we use `$\subset$' for subset, and we have no special symbol to distinguish ``proper subset'' from ``subset''.
Some authors use  `$\subseteq$' to denote  subset, and  `$\subset$' to denote proper subset. This has the advantage that then `$\subseteq$' and `$\supseteq$' are similar to `$\le$' and `$\ge$', while `$\subset$' and `$\supset$' are like `$<$' and `$>$'.
But we rarely have to distinguish the case of proper subsets, so it's not worth defining a special symbol for them.
\end{rem}

If $A$ is not a subset of $B$, we write $A \not \subset B$; for example, $\{4, 7, 9\} \not \subset \{2, 4, 5,  8, 9 \}$.  Two sets are \term{equal}, written $A = B$, if we can show that $A \subset B$ and $B \subset A$.  

It is convenient to have a set with no elements in it.  This set is called the \term{empty set}\index{Empty set}\index{Set!empty set} and is denoted by $\emptyset$\label{theemptyset}.  For instance, if John and Jane Doe had only daughters, then
\[
\mbox{\{sons of John and Jane Doe\}} = \emptyset
\]

Note that the empty set is a subset of every set.  

\begin{exercise}{s16}
Let $S$ be a set with a single element.
\begin{enumerate}[(a)]
\item
How many subsets does it have?
\item
How many proper subsets does it have?
\item
How many nonempty subsets does it have?
\item
How many nonempty proper subsets does it have?
\end{enumerate}
\end{exercise}

\begin{exercise}{7}
\begin{enumerate}[(a)]
\item
Can you give an example of a set with exactly three subsets? How about exactly three proper subsets?
\item
What is the smallest number of elements a set must have in order to have at least eight proper subsets?
\end{enumerate}
\end{exercise}

\subsection{Operations on sets}
In our halcyon days of youth, we were introduced to \term{operations} on integers, rational numbers, etc.. An operation on the integers takes two integers and always comes up with another integer. For instance, the '+' operation gives $2+3=5$ (of course, we know now that this means that + has the property of \emph{closure}).

\begin{exercise}{}
What's wrong with the following statement: ``Subtraction is an operation on the natural numbers.''
\end{exercise}

In a similar way, we can construct new sets out of old sets using \emph{set operations}.\index{Set operations} The mathematical definitions of the basic set operations are as follows:

\begin{defn}
The \term{union} $A \cup B$ of two sets $A$ and $B$ is defined as\index{Set operations!union}\index{Union}  
\[
A \cup B\label{union} = \{x : x \in A \text{ or } x \in B \};
\]
\end{defn}
\begin{defn}
the \term{intersection} of $A$ and $B$  is defined by \index{Set operations!intersection}\index{Intersection}
\[
A \cap B\label{intersection} = \{x :  x \in A \text{ and } x \in B \}.
\]
\end{defn}
For example: if $A = \{1, 3, 5\}$ and $B = \{ 1, 2, 3, 9 \}$, then
\[
A \cup B = \{1, 2, 3, 5, 9 \}
\quad \text{and} \quad
A \cap B = \{ 1, 3 \}.
\]
We may also consider the union and the intersection of more than two sets.  For instance,  the union of three sets $A_1, A_2,$ and $A_3$ can be written   $A_{1} \cup A_2 \cup A_3$ or $\bigcup_{i = 1}^{3} A_{i}. $ 

\medskip{}
Similarly, the intersection of the same three sets can be written as $A_{1} \cap A_2 \cap A_3$ or $\bigcap_{i = 1}^{3} A_i.$

\begin{rem}
There's actually a technical difficulty with our notations for  $A_{1} \cup A_2 \cup A_3$ and $A_{1} \cap A_2 \cap A_3$. The problem is that the notation is ambiguous: does $A_{1} \cup A_2 \cup A_3$ mean $(A_{1} \cup A_2) \cup A_3$ or $A_{1} \cup (A_2 \cup A_3)$? As it turns out, it doesn't make any difference (we'll show this in the next section). Since it doesn't matter which order we do the $\cup$, we just leave off the parentheses (and the same for $\cap$). This is really nothing new: you're used to writing $3 + 4 + 7 + 9$ instead of $((3+4)+7)+9$, because it doesn't matter what order you add the numbers.
\end{rem}

\begin{exercise}{12}
\begin{enumerate}[(a)]
\item
Find three sets $A_1, A_2, A_3$ such that $A_{1} \cup A_2 \cup A_3 = {\mathbb Z}$ and $A_{1} \cap A_2 \cap A_3 = \emptyset$
\item
Find three sets $A_1, A_2, A_3$ such that (i) $A_1, A_2, A_3 \subset {\mathbb C}$; (ii) $A_1 \cap A_2 \neq \emptyset, A_2 \cap A_3 \neq  \emptyset, A_1 \cap A_3 \neq \emptyset$; and (iii) $A_{1} \cap A_2 \cap A_3 = \emptyset$
\item
Find three sets that satisfy all conditions of part (b) and in addition satisfy $A_{1} \cup A_2 \cup A_3 = {\mathbb C}$.
\end{enumerate}
\end{exercise}

\noindent
We may generalize to intersections and unions of collections of $n$ sets by writing:
\[
\bigcup_{i = 1}^{n} A_{i} = A_{1} \cup \ldots \cup A_n
\]
and
\[
\bigcap_{i = 1}^{n} A_{i} = A_{1} \cap \ldots \cap A_n
\]
for the union and intersection, respectively, of the collection of sets $A_1, \ldots A_n$.



\begin{example}{}
Specify the following sets, either by:
\begin{itemize}
\item
listing the elements;
\item
describing with a property; or
\item
giving another set that we've already defined that has the same elements.
\end{itemize}
\begin{enumerate}[(a)]
\item
$\bigcup_{i = 1}^{n}  \{i\}$
\item
$\bigcup_{i = 1}^{n}  \{1, \ldots, i\}$
\item
$\bigcup_{i = 1}^{\infty}  \{1, \ldots, i\}$
\end{enumerate}
\noindent
Solutions:
\begin{enumerate}[(a)]
\item
$\bigcup_{i = 1}^{n}  \{i\} = \{1\} \bigcup \{2\} \bigcup \{ 3 \} \bigcup \ldots \bigcup \{n\} $

$\qquad \quad = \{1, \ldots , n\} \qquad \qquad [list~of~elements] $

$ \qquad \quad = \text{all integers from 1 to }n. \quad [property]$
\item
$\bigcup_{i = 1}^{n}  \{1, \ldots, i\} = \{1\} \bigcup \{1, 2\} \bigcup \{1, 2, 3 \} \bigcup \ldots \bigcup \{1, \ldots , n\} $

$\qquad \quad = \{1, \ldots , n\} \qquad \qquad [list~of~elements]$

$\qquad \quad = \text{all integers from 1 to }n. \quad [property]$
\item
$\bigcup_{i = 1}^{\infty}  \{1, \ldots, i\} = \textrm{[by part (b)]   } \{1, \ldots , \infty\} = {\mathbb N}$
\end{enumerate}
\end{example} 

\begin{exercise}{14}
Specify the following sets, either by:
\begin{itemize}
\item
listing the elements;
\item
describing with a property; or
\item
giving another set that we've already defined that has the same elements.
\end{itemize}
\begin{enumerate}[(a)]
\item
$\bigcap_{i = 1}^{n}  \{i\}$
\item
$\bigcap_{i = 1}^{n}  \{1, \ldots, i\}$
\item
$\bigcap_{i = 1}^{\infty}  \{1, \ldots, i\}$
\item
$\bigcup_{r = 0}^{n-1}  \{\mbox{Integers that have remainder }r \mbox{ when divided by }n\}$
\item
$\bigcap_{r = 0}^{n-1}  \{\mbox{Integers that have remainder }r \mbox{ when divided by }n\}$
\end{enumerate}
\end{exercise}


\begin{exercise}{15}
\begin{enumerate}[(a)]
\item
Find an infinite collection of sets $\{A_i\}, i = 1,2,3,\ldots$ such that (i) $A_i \subset {\mathbb R}, i = 1,2,3, \dots$; (ii)
 each $A_i$ is a closed  interval of length 1 (that is, $A_i = [a_i, a_i+1]$ for some $a_i$; and (iii) $\bigcup_{i = 1}^{\infty} A_{i} = [0, \infty)$. (That is, the union of all the $A_i$'s is the set of all nonnegative real numbers.)
\item
Find an infinite collection of sets $\{A_i\}, i = 1,2,3,\ldots$ such that (i) $A_i \subset {\mathbb R}, i = 1,2,3, \dots$; (ii)
 each $A_i$ is an open  interval of length 1 (that is, $A_i = (a_i, a_i+1)$ for some $a_i$; and (iii) $\bigcup_{i = 1}^{\infty} A_{i} = (0, \infty)$. (That is, the union of all the $A_i$'s is the set of all positive real numbers.)
\item
Find an infinite collection of sets $\{A_n\}, n = 1,2,3,\ldots$ such that (i) $A_n \subset  [-1/2,1/2], n = 1,2,3, \dots$; (ii)
 each $A_n$ is an open interval of length $1/n$; and (iii) $\bigcap_{n = 1}^{\infty} A_{n} = \{0\}$.
 \item
**Find an infinite collection of sets $\{A_n\}, n = 1,2,3,\ldots$ such that (i) $A_n \subset [0,1], n = 1,2,3, \dots$; (ii)
 each $A_n$ is an open interval of length $1/n$; (iii) $A_{n+1} \subset A_{n}, n = 1,2,3, \dots$; and (iv) $\bigcap_{n = 1}^{\infty} A_{n} = \emptyset$.
\end{enumerate}
\end{exercise}

When two sets have no elements in common, they are said to be \term{disjoint}\index{Sets!disjoint}\index{Set operations!disjoint}\index{Disjoint!definition of}; for example, if $E$ is the set of even integers and $O$ is the set of odd integers, then $E$ and $O$ are disjoint.  Two sets $A$ and $B$ are disjoint exactly when $A \cap B = \emptyset$. 

\begin{exercise}{16}
\begin{enumerate}[(a)]
\item
Find four disjoint sets $A_1, A_2, A_3, A_4$ such that $\bigcup_{i = 1}^{4} A_i = {\mathbb Z}$.
\item
Find four disjoint sets $A_1, A_2, A_3, A_4$ such that $\bigcup_{i = 1}^{4} A_i = {\mathbb R}$.
\item
Find four disjoint sets $A_1, A_2, A_3, A_4$ such that $\bigcup_{i = 1}^{4} A_i = {\mathbb C}$.
\end{enumerate}
\end{exercise} 


If we are working within the universal set $U$ and $A \subset U$, we define the \term{complement}\footnote{Please note the spelling: 'complement', not 'compliment', thank you!} of $A$ (denoted by $A'$)\index{Complement!definition}\index{Set operations!complement}, to be the set
\[
A' = \{ x : x \in U \text{ and } x \notin A \}.
\]


\begin{defn}\label{setdifference}\index{Set operations!set difference}
The \term{difference} of two sets $A$ and $B$ is defined as\index{Set difference}
\[
A \setminus B = A \cap B'  = \{ x : x \in A \text{ and } x \notin B \}.
\]
\end{defn}
\emph{Note} that it's not necessary for $B$ to be inside $A$ to define $A \setminus B$. In fact, $A \setminus (A \cap B)$ is exactly the same thing as $A \setminus B$ (you may draw a picture to see why this is true).

\begin{exercise}{18}
Suppose that $A \subset B$. What is the largest subset of $B$ that is disjoint from $A$?
\end{exercise}

\noindent
The set difference concludes our set operations for now.  The following example and exercises will give you an opportunity to sharpen your set operation skillls. 
 
\begin{example}{operations}
Let ${\mathbb N}$ be the universal set, and suppose that
\begin{align*}
A = \{ x \in {\mathbb N} : x \text{ is divisible by 2}\} \\ 
B = \{ x \in {\mathbb N} : x \text{ is divisible by 3}\} \\ 
C = \{ x \in {\mathbb N} : x \text{ is divisible by 6}\} \\
D = \{\text{the odd natural numbers}\}
\end{align*} 
Then specify the following sets:
\begin{enumerate}[(a)]
\item 
$A \cap B$
\item
$C \cup A$
\item
 $D \setminus B$
\item
$B'$
\end{enumerate}

\noindent
Solutions:
\begin{enumerate}[(a)]
\item
\begin{align*}
A \cap B &= \{ x \in {\mathbb N} : x \text{ is divisible by 2} \text{ and } x \text{ is divisible by 3}\} \\
& = \{ x \in {\mathbb N} : x \text{ is divisible by 6}\} \\
& = C
\end{align*}
\item
\begin{align*}
C \cup A &= \{ x \in {\mathbb N} : x \text{ is divisible by 6} \text{ or } x \text{ is divisible by 2}\} \\
&= \{2, 4, 6, 8, 10, 12, \ldots \} \\
&= A
\end{align*}
\item
\begin{align*}
D \setminus B &= \{ x \in {\mathbb N} : x \in D \text{ and } x \notin B \} \\
&= \{ x \in {\mathbb N} : x \text{ is an odd natural number and } x \text{ is } not \text{ divisible by} 3 \} \\
&= \{  x \in {\mathbb N} : x \text{ is an odd natural number that is not divisible by 3} \} 
\end{align*} 
\item
\begin{align*}
B' &= \{ x \in {\mathbb N} : x \text{ is divisible by 3}\}' \\
&= \{ x \in {\mathbb N} : x \text{ is not divisible by 3}\}
\end{align*}
\end{enumerate} 
\end{example}

\begin{exercise}{20}
Let ${\mathbb N}$ be the universal set and suppose that
\begin{align*}
A = \{ x \in {\mathbb N} : x \text{ is divisible by 2}\} \\ 
B = \{ x \in {\mathbb N} : x \text{ is divisible by 3}\} \\ 
C = \{ x \in {\mathbb N} : x \text{ is divisible by 6}\} \\
D = \{\text{the odd natural numbers}\}
\end{align*} 

\noindent
Specify each of the following sets. You may specify a set either by describing a property, by enumerating the elements, or as one of the four sets $A, B, C, D$:
\begin{multicols}{2}
\begin{enumerate}[(a)]
\item
$(A \cap B) \setminus C$
\item
$A \cap B \cap C \cap D$
\item
$A \cup B \cup C \cup D$

\end{enumerate}
\end{multicols}
\end{exercise}

\begin{exercise}{21}
Let ${\mathbb N}$ be the universal set and suppose that
\begin{align*}
A & = \{ x : \mbox{$x \in {\mathbb N}$ and $x$ is even} \}, \\
B & = \{x : \mbox{$x \in {\mathbb N}$ and $x$ is prime}\}, \\
C & = \{ x : \mbox{$x \in {\mathbb N}$ and $x$ is a multiple of $5$}\}.
\end{align*}
Describe each of the following sets. Make your description as concise as possible.
\begin{multicols}{2}
\begin{enumerate}[(a)]

\item
$A \cap B$

\item
$(A \cap B)'$

\item
$A' \cap B'$

\item
$A \cup B$

\item
$(A \cup B)'$

\item
$A' \cup B'$


\item
$B \cap C$

\item
$A \cap (B \cup C)'$

\end{enumerate}
\end{multicols}
\end{exercise} 


\section{Properties of set operations \quad
\sectionvideohref{ciswhn56NJc&index=12&list=PL2uooHqQ6T7PW5na4EX8rQX2WvBBdM8Qo}} \label{propset}

Now that we have the basics out of the way, let's look at the some of  the properties of set operations.
The individual steps of the following proofs depend on \emph{logic}; and a rigorous treatment  of these proofs would require that we introduce formal logic and its rules. However, many of these logical rules are intuitive, and it should be possible for you to follow the proofs even if you haven't studied mathematical logic.

 First, we give two rather obvious (but very useful) properties of $\cup$ and $\cap$:

\begin{prop}{cupcapincl}
Given any sets $A,B$, It is always true that
\[
A \cap B \subset A \mathrm{~~~and~~~}A \subset A \cup B.
\]
\end{prop}
\begin{proof}
The style of proof we'll use here  is often described as \emph{element by element},\index{Element by element proof} because the proofs make use of the definitions of $A\cap B$ and $A\cup B$ in terms of their elements. 

First, suppose that $x$ is an element of $A \cap B$. we then have:
\begin{align*}
&x \in A \cap B &[\text{supposition}]\\
\implies &x \in A \text{ and } x\in B &[\text{def. of }\cap]\\
\implies &x \in A.&[\text{logic}]
\end{align*}
Since every element of $A \cap B$ is an element of $A$, it follows by the definition of $\subset$ that $A \cap B \subset A$.

\begin{exercise}{23}
Give a similar proof of the second part of Proposition~\ref{proposition:sets:cupcapincl}.
\end{exercise}
\end{proof}

Many useful properties of set operations are summarized in the following multi-part proposition: 

\begin{prop}{sets_theorem_set_ops}
Let $A$, $B$, and $C$ be subsets of a universal set $U$. Then
\begin{enumerate}
 
\item
$A \cup A' = U$ and $A \cap A' = \emptyset$

\item
$A \cup A = A$, $A \cap A = A$, and $A \setminus A = \emptyset$;
 
\item
$A \cup \emptyset = A$ and $A \cap \emptyset = \emptyset$;

\item
$A \cup U = U$ and $A \cap U = A$;
 
\item
$A \cup (B \cup C) = (A \cup B) \cup C$ and  $A \cap (B \cap C) = (A \cap B) \cap C$;
 
\item
$A \cup B = B \cup A$ and $A \cap B = B \cap A$;
 
\item
$A \cup (B \cap C) = (A \cup B) \cap (A \cup C)$;
 
\item
$A \cap (B \cup C) = (A \cap B) \cup (A \cap C)$.
 
\end{enumerate}
\end{prop}


%%% CPT I think the following is good, but not necessary here
%Element-by-element proofs sometimes use the following basic strategy: two sets $A$ and $B$ are equal if every element of $A$ is also an element of $B$, and every element of $B$ is also an element of $B$. Recalling our definition of ``subset'' above, this amounts to showing that $A \subset B$ and $B \subset A$. In other words:
%
%\[ \textrm{Two~sets~}A \textrm{~and~} B \textrm{~are~equal~iff~}A \subset B \textrm{~and~} B\subset A \].
%
%
\begin{proof}
We'll prove parts (1), (2), (5), and (7), and leave the rest to you!

\noindent
%%% CPT I rewrote (1) to be consistent with the other proofs.
(1)
From our definitions we have:
\begin{align*}
A \cup A' & =  \{ x : x \in A \mathrm{~or~} x \in A' \}  & [\text{def. of } \cup ]\\
& =  \{ x : x \in A \mathrm{~or~} x \notin A \}  & [\text{def. of complement} ]  \\
\end{align*}
But every $x \in U$ must satisfy either $x \in A$ or $x \notin A$. It follows that $A \cup A'$ includes all elements of $U$; so $A \cup A' = U$.

\noindent
We also have 
\begin{align*}
A \cap A' & =  \{ x : x \in A \text{ and } x \in A' \}  & [\text{def. of } \cap ]\\
& =  \{ x : x \in A \text{ and } x \notin A \}  & [\text{def. of complement} ]  \\
\end{align*}
But there is no element $x$ that is both in $A$ and not in $A$, it follows that there are no elements in $A \cap A'$; so $A \cap A' = \emptyset$.

\noindent
(2)
Observe that
\begin{align*}
A \cup A & =  \{ x : \mbox{ $x \in A$ or $x \in A$} \}    & [\text{def. of } \cup]  \\
& =  \{ x : \mbox{ $x \in A$} \} \\
& =  A
\end{align*}
and
\begin{align*}
A \cap A & =  \{ x : \mbox{ $x \in A$ and $x \in A$} \}    &\text{ [def. of } \cap] \\
& =  \{ x : \mbox{ $x \in A$}  \} \\
& =  A.
\end{align*}
Also, 
\begin{align*}
A \setminus A &= A \cap A'    &\mbox{ [def. of } \setminus] \\
&= \emptyset.    &\mbox{ [by part 1] }
\end{align*}

\noindent 
(5)
For sets $A$, $B$, and $C$,
\begin{align*}
A \cup (B \cup C)
& =
A \cup \{ x : \mbox{ $x \in B$ or $x \in C$} \}   &\text{ [def. of } \cup]  \\
& =
\{ x : \mbox{ $x \in A$ or $x \in B$ or $x \in C$} \}    &\text{ [def. of } \cup]  \\
& =
\{ x : \mbox{ $x \in A$ or $x \in B$} \} \cup C    &\text{ [def. of } \cup]  \\
& =
(A \cup B) \cup C.    &\text{ [def. of } \cup] 
\end{align*}
A  similar argument proves that  $A \cap (B \cap C) = (A \cap B) \cap
C$. 

\noindent 
(7)
We show that these two sets are equal by showing that:
\begin{enumerate}[(I)]
\item
Every element $x$ in $A \cup (B \cap C)$ is also an element of $(A \cup B) \cap (A \cup C)$;
\item
Every element $x$ in $(A \cup B) \cap (A \cup C)$ is also an element of $A \cup (B \cap C)$.
\end{enumerate}
(It's actually a rather common strategy to prove that two sets are equal by showing that every element of one set is an element of the other set, and vice versa.)

Let's begin by proving (I). Take any element $x \in A \cup (B \cap C)$.  Then
$x \in A$  or $(x \in B \cap C)$, by the definition of $\cup$. We may therefore consider two cases: (i) $x \in A$, or (ii) $x \in B \cap C$.  (Actually some $x$'s are included in both cases, but that's not a problem.)

\noindent
\emph{Case i}:  If $x \in A$, the by Proposition~\ref{proposition:sets:cupcapincl} we know $x \in A \cup B$ and $x \in A \cup C$. By the definition of $\cap$, we then have   $x \in (A \cup B) \cap (A \cup C)$.

\noindent
\emph{Case ii}:  If $x \in  B  \cap C$, then by Proposition~\ref{proposition:sets:cupcapincl} we know $x \in  B$ and $x \in  C$. 
By Proposition~\ref{proposition:sets:cupcapincl},  then $x \in A \cup  B$ and $x \in A  \cup  C$.  By the definition of $\cap$, this means that $x \in (A \cup  B) \cap (A  \cup  C)$.  

This completes the proof of (I). Now we'll  prove (II). Take any element $x \in (A \cup B) \cap (A \cup C)$.  Then we may consider two cases: 
(i) $x \in A$, or (ii) $x \not\in A$.

\noindent
\emph{Case i}:  If $x \in A$, then by by Proposition~\ref{proposition:sets:cupcapincl} it's also true that 
$x \in A \cup (B \cap C)$.  

\noindent
\emph{Case ii}:  Suppose $x \not\in  A$. Now, since $x \in (A \cup B) \cap (A \cup C)$, by the definitions of $\cap$ and $\cup$ we know that $(x \in A \text{ or } x \in B)$ and $(x \in A \text{ or } x \in C)$. But since  $x \not\in  A$, it must be true that $x\in B$, and also $x\in C$. By the definition of $\cap$, this means that $x\in B \cap C$. by Proposition~\ref{proposition:sets:cupcapincl}, we have that  $x \in A \cup (B \cap C)$. This completes the proof of (II), which completes the proof of (7).
\end{proof}

\begin{exercise}{}
Fill in the blanks in the following proof of Proposition~\ref{proposition:sets:sets_theorem_set_ops} part (3):

\medskip{}
\noindent
Observe that
\begin{align*}
A \cup \emptyset & =  \{ x : x \in A \mathrm{~or~}x \in \emptyset \}    & [\text{Def. of }\cup] \\
& = \{ x : x \in \_\_\_\_\_\_\_ \}     & [\emptyset \mathrm{~has~no~elements}] \\
& =  A & \text{Def. of set }A 
\end{align*}
and
\begin{align*}
A \cap \emptyset & =  \{ x : \mbox{ $x \in \_\_\_\_\_\_\_$ and $x \in \_\_\_\_\_\_\_$} \}     & \_\_\_\_\_\_\_\_\_\_\_\_ \\
& =  \emptyset     & \_\_\_\_\_\_\_\_\_\_\_\_\_\_\_\_\_\_\_ .
\end{align*}
\end{exercise}

\begin{exercise}{26}
Prove parts 4,6,8 of Proposition~\ref{proposition:sets:sets_theorem_set_ops} using element-by-element proofs.
\end{exercise}

\medskip{}
\noindent
The following rules that govern the operations $\cap, \cup$ and $'$  follow from the definitions of these operations:

\begin{thm}[De Morgan's Laws]\index{De Morgan's laws for sets}\label{sets_de_morgan}\index{Set operations!De Morgan's laws for sets}
Let $A$ and $B$ be sets. Then 
\begin{enumerate}[(1)]
 \item
$(A \cup B)' = A' \cap B'$; 
 \item
$(A \cap B)' = A' \cup B'$.
 \end{enumerate}
\end{thm}
 
 We will use the same strategy we used to prove Proposition~\ref{proposition:sets:sets_theorem_set_ops} part (7)-that is, we show that sets are equal by showing they are subsets of each other.
\medskip{}

\begin{proof} 

\noindent
We'll prove (1), and leave (2) as an exercise. The proof will show that the sets on the left and right sides of the equality in (1)  are both subsets of each other.

First we show that $(A \cup B)' \subset A' \cap B'$.  Let $x \in (A \cup B)'$.  Then $x \notin A \cup B$. So $x$ is neither in $A$ nor in $B$, by the definition of $\cup$.  By the definition of $'$, $x \in A'$ and $x \in B'$.  Therefore, $x \in A' \cap B'$ and we have $(A \cup B)' \subset A' \cap B'$.

 To show the reverse inclusion, suppose that $x \in A' \cap B'$.  Then $x \in A'$ and $x \in B'$, and so $x \notin A$ and $x \notin B$.  Thus $x \notin A \cup B$ and so $x \in (A \cup B)'$.  
\end{proof}

\begin{exercise}{s27}
Prove Proposition~\ref{sets_de_morgan} part (2).
\end{exercise}
 
\medskip{}
\noindent
Proposition~\ref{proposition:sets:sets_theorem_set_ops} and Proposition~\ref{sets_de_morgan} provide us with an arsenal of rules for set operations. You should consider these as your ``rules of arithmetic'' for sets: just as you used arithmetic rules in high school to solve algebraic equations, so now you can use these rules for set operations to solve set equations.  Here is an example of how to do this:

\begin{example}{other_relations}
Prove that
\[
( A \setminus B) \cap (B \setminus A) = \emptyset.
\]

\begin{proof}
To see that this is true, observe that
\begin{align*}
( A \setminus B) \cap (B \setminus A)
& =
( A \cap B') \cap (B \cap A')   & \mbox{[definition of } \setminus] \\
& =
A \cap A' \cap B \cap B'    &\mbox{[by Proposition~\ref{proposition:sets:sets_theorem_set_ops} parts 5 and 6]} \\
&= \emptyset \cap \emptyset    &\mbox{[by Proposition~\ref{proposition:sets:sets_theorem_set_ops} part 1]} \\
& = \emptyset. \\
\end{align*}
\end{proof}
\end{example}


%\begin{exercise}{}
%If $A = \{ a, b, c \}$, $B = \{ 1, 2, 3 \}$, $C = \{ x \}$, and $D = \emptyset$, list all of the elements in each of the %following sets. 
%\begin{multicols}{2}
%\begin{enumerate}

%\item
%$A \times B$

%\item
%$B \times A$

%\item
%$A \times B \times C$

%\item
%$A \times D$

%\end{enumerate}
%\end{multicols}
%\end{exercise} 
 
%\begin{exercise}{}
%Find an example of two nonempty sets $A$ and $B$ for which $A \times B = B \times A$ is true. 
%\end{exercise} 
%\item
%Prove $A \cup \emptyset = A$ and $A \cap \emptyset = \emptyset$.
 
%\item
%Prove $A \cup B = B \cup A$ and $A \cap B = B \cap A$.
 
%\item
%Prove $A \cup (B \cap C) = (A \cup B) \cap (A \cup C)$.
 
%\item
%Prove $A \cap (B \cup C) = (A \cap B) \cup (A \cap C)$.
 
%\item
%Prove  $A \subset B$ if and only if $A \cap B = A$.
 
%\item
%Prove $(A \cap B)' = A' \cup B'$.

%\item
%Prove  $A \cup B = (A \cap B) \cup (A \setminus B) \cup (B \setminus A)$. 
 
%\item
%Prove  $(A \cup B) \times C = (A \times C ) \cup (B \times C)$.
 
\begin{exercise}{30}
Prove the following statements by mimicking the style of proof in Example~\ref{example:sets:other_relations}; that is use the definitions of $\cap, \cup, \setminus$, and $'$ as well as their properties listed in Proposition~\ref{proposition:sets:sets_theorem_set_ops} and  Proposition~\ref{sets_de_morgan}. This type of proof is called an ``algebraic'' proof.  Every time you use a property, remember to give a reference!

We should mention that the best strategy is usually to begin with the more complicated side of the equality, and try to simplify to the point where it agrees with the other side. if you can't get it to agree, then start working on the other side and simplify until the simplified versions of both sides finally agree.

\begin{enumerate}[(a)]
\item
$(A \cap B) \setminus B = \emptyset$.
\item
$(A \cup B) \setminus B = A \setminus B$.
\item
$A \setminus (B \cup C) = (A \setminus B) \cap (A \setminus C)$. 
\item
 $A \cap (B \setminus C) = (A \cap B) \setminus (A \cap C)$. 
\item
$(A \setminus B) \cup (B \setminus A) = (A \cup B) \setminus (A \cap B)$. 
\item
$(A \cup B \cup C) \cap D) = (A \cap D) \cup (B \cap D)\cup (C \cap D)$. 
\item
$(A \cap B \cap C) \cup D = (A \cup D) \cap (B \cup D)\cap (C \cup D)$. 
\end{enumerate}
\end{exercise}

\section{Do the subsets of a set form a group? \quad
\sectionvideohref{ciswhn56NJc&index=12&list=PL2uooHqQ6T7PW5na4EX8rQX2WvBBdM8Qo}}\label{SetGroup} 
Some of the properties in Proposition~\ref{proposition:sets:sets_theorem_set_ops} may ring a bell. Recall that in the Section~\ref{DefOfGroup} of the Modular Arithmetic chapter  we defined a \emph{group}\index{Group!definition} to be a set combined with an operation that has the following properties:
\begin{enumerate}
\item
The set is \emph{closed} under the operation (in other words, the operation has the property of \emph{closure});
\item
The set has  a unique \emph{identity};
\item
Every element of the set has its own \emph{inverse};
\item
The set elements satisfy the \emph{associative property} under the group operation;
\item
\emph{Some} groups satisfy the \emph{commutative property} under the group operation.
\end{enumerate}

\noindent
If you forgot what these properties mean, look back at Section~\ref{sec:ClosureZn} and the following subsections, where we discuss these properties as applied to the integers mod $n$.

What we're going to do now is a first taste of a magic recipe that you're going to see again and again in Abstract Algebra. We're going to turn \emph{sets} into \emph{elements}. Abracadabra!

What do we mean by this? Let's take an example. Take the 3-element set $S = \{a,b,c\}$. 

\begin{exercise}{31}
\begin{enumerate}[(a)]
\item
List the \emph{subsets} of $S =  \{a,b,c\}$. Include the empty set and non-proper subsets of $S$. How many subsets are in your list?
\item
If you listed the subsets of $\{a,b\}$, how many subsets would be in your list?
\item
If you listed the subsets of $\{a,b,c,d\}$, how many subsets would be in your list?
\item
**If you listed the subsets of $\{a,b,c,\ldots,x,y,z\}$, how many subsets would be in your list?
\hyperref[sec:set_chapter:hints]{(*Hint*)}
\end{enumerate}
\end{exercise}

Let's take the list of subsets of $\{a,b,c\}$ that you came up with in part (a) of the previous exercise. We can consider this list as a set of 8 elements, where each element is a subset of the original set $S = \{a,b,c\}$. Let's call this 8-element set $G$. Remember, the elements of $G$ are \emph{subsets} of the original set $S$.

So now let's face the question:  Is $G$ a group? 

Recall that a group has a single \emph{operation}\index{Operation!definition of}: that is, a way of combining two elements to obtain a third element. We actually have two candidates for an operation for $G$: either intersection or union. So we actually have two questions:
\begin{itemize}
\item
 Is $G$ with the operation $\cup$ a group?
\item
Is $G$ with the operation $\cap$ a group?
\end{itemize}

We'll take these questions one at a time. First we investigate group properties for the set $G$ with the operation $\cup$:

\begin{exercise}{cup_group}
Let $G$ be the set of subsets of the set $\{a,b,c\}$.
\begin{enumerate}[(a)]
\item
Does the set $G$  with the operation $\cup$ have the closure property? \emph{Justify} your answer.
\item
Does the set $G$  with the operation $\cup$ have an identity? If so, what is it? Which part of  Proposition~\ref{proposition:sets:sets_theorem_set_ops} enabled you to draw this conclusion?
\item
Is the operation $\cup$ defined on the set $G$ associative? Which part of  Proposition~\ref{proposition:sets:sets_theorem_set_ops} enabled you to draw this conclusion?
\item
Is the operation $\cup$ defined on the set $G$ commutative? Which part of  Proposition~\ref{proposition:sets:sets_theorem_set_ops} enabled you to draw this conclusion?
\item
Does each element of $G$ have a unique inverse under the operation $\cup$? If so, which part of  Proposition~\ref{proposition:sets:sets_theorem_set_ops} enabled you to draw this conclusion? If not, provide a counterexample.
\item
Is the set $G$ a group under the $\cup$ operation?  \emph{Justify} your answer.
\end{enumerate}
\end{exercise} 

Although Exercise~\ref{exercise:sets:cup_group} deals with a particular set of subsets,  the results of the  exercise are completely general and apply to the set of any subsets of \emph{any} set (and not just $\{a,b,c\}$.  
 
Now we'll consider $\cap$:

\begin{exercise}{cap_group}
Given a set $A$, let $G$ be the set of all subsets of $A$. 
\begin{enumerate}[(a)]
\item
Does the set $G$  with the operation $\cap$ have the closure property? \emph{Justify} your answer.
\item
Does the set $G$  with the operation $\cap$ have an identity? If so, what is it? Which part of  Proposition~\ref{proposition:sets:sets_theorem_set_ops} enabled you to draw this conclusion?
\item
Is the operation $\cap$ defined on the set $G$ associative? Which part of  Proposition~\ref{proposition:sets:sets_theorem_set_ops} enabled you to draw this conclusion?
\item
Is the operation $\cap$ defined on the set $G$ commutative? Which part of  Proposition~\ref{proposition:sets:sets_theorem_set_ops} enabled you to draw this conclusion?
\item
Does each element of $G$ have a unique inverse under the operation $\cap$? If so, which part of  Proposition~\ref{proposition:sets:sets_theorem_set_ops} enabled you to draw this conclusion? If not, provide a counterexample.
\item
Is the set $G$ a group under the $\cap$ operation?  \emph{Justify} your answer.
\end{enumerate}
\end{exercise} 
 

No doubt you're bitterly disappointed that neither $\cap$ nor $\cup$ can be used to define a group. However, take heart! Mathematicians use these operations to define a different sort of algebraic structure called (appropriately enough) a \emph{Boolean algebra}. We won't deal further with Boolean algebras in this course: suffice it to say that mathematicians have defined a large variety of abstract algebraic structures for different purposes.

Although $\cap$ and $\cup$ didn't work, there is a consolation prize:
 
\begin{exercise}{}
 Besides $\cup$ and $\cap$, there is another set operation called \emph{symmetric difference},\index{Set operations!symmetric difference} which is sometimes denoted by the symbol $\Delta$ and is defined as:
\begin{equation*}
A \Delta B = (A \setminus B) \cup (B\setminus A).
\end{equation*}
Given a set $A$, let $G$ be the set of all subsets of $A$.  Repeat parts (a)--(f) of Exercise~\ref{exercise:sets:cap_group}, but this time for the set operation $\Delta$ instead of $\cap$.
\end{exercise} 

