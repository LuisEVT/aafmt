\section{Hints for ``Modular Arithmetic'' exercises}\label{sec:modular_arithmetic:hints}

\noindent Exercise \ref{exercise:modular:eqproof}: Use the alternative definition of modular equivalence in Proposition~\ref{proposition:modular:equivalence_alt}.

\noindent Exercise \ref{exercise:modular:UPCSymbols}(f): Prove by contradiction: 
suppose the codes  $d_1, d_2, \ldots  d_{10}$  and $e_1, e_2, \ldots  e_{10}$ are both valid, and suppose that all digits are equal except for the $n$'th digit (so $d_n \neq e_n$).
There are two cases: (a) $n$ is even; (b) $n$ is odd. In case (a), show that this implies $e_n - d_n \equiv 0  \pmod{10}$, and derive a contradiction. Prove case (b) similarly.

\noindent Exercise \ref{exercise:modular:ISBNCodes}(d): Use the fact that $10 \equiv -1 \pmod{11}$.

\noindent Exercise \ref{exercise:modular:ISBNCodes}(i): Prove by contradiction: Suppose the codes  $d_1, d_2, \ldots  d_{10}$  and $e_1, e_2, \ldots  e_{10}$ are both valid, and suppose that all digits are equal except for the $n$'th digit (so $d_n \neq e_n$). Show that $d_n - e_n$ satisfies $(d_n - e_n)n \equiv 0 \pmod{11}$, and show that the only solution is $d_n - e_n = 0$.

\noindent Exercise \ref{exercise:modular:ISBNCodes}(j): Suppose the code $d_1, d_2, \ldots  d_{10}$  is valid, and suppose the code is still valid when the digits $d_n$ and $d_{n+1}$ are exchanged. Write down two modular equations, and take the difference between the two modular equations.  Use this to find an equation involving $d_n$ and $d_{n+1}$.

\noindent Exercise \ref{exercise:modular:modeq3}(c): Find a \emph{negative} number that is equivalent to $856 \pmod{123}$.

\noindent Exercise \ref{exercise:modular:ModPower}(a): Let $m=\ell$ and $b=a$. Check the conditions of the proposition still hold, and apply the proposition.

\noindent Exercise \ref{exercise:modular:ops}(a): You will need to use Proposition~\ref{proposition:modular:number_remainder} twice.

\noindent Exercise \ref{exercise:modular:52}: Use the definitions of $\oplus$ and $\odot$.

\noindent Exercise \ref{exercise:modular:53}: Be careful about 0!

\noindent Exercise \ref{exercise:modular:58}(b)(i): Use the fact that $0 < a < n$.

\noindent Exercise \ref{exercise:modular:diophant}(a): The left-hand side is always even, no matter what $m$ and $n$ are.

\noindent Exercise \ref{exercise:modular:prop74}: Use Proposition~\ref{proposition:modular:Lin_comb}.

\noindent Exercise \ref{exercise:modular:propiff}: Use Proposition~\ref{proposition:modular:74}.

\noindent Exercise \ref{exercise:modular:EuclidLemmaProof}(a): Use  Proposition~\ref{proposition:modular:74}.~~(b): $p$ must divide the left-hand side of the multiplied equation (explain why).~~(c): Consider two cases (I) $a$ is relatively prime to $p$; (II) $a$ is not relatively prime to $p$.

\noindent Exercise \ref{exercise:modular:93}: Use Proposition~\ref{proposition:modular:mod_eq_solution}. 

\noindent Exercise \ref{exercise:modular:94}: Use the previous exercise.

\noindent Exercise \ref{exercise:modular:congruence}(b):  If  $y$ is a particular solution to $ax \equiv b \pmod 3$, then $x=y+3k$ is also a solution. Similarly, if $z$ is 
 a particular solution to $cx \equiv d \pmod 7$, then $x=z+7\ell$ is also a solution. Set the two expressions equal, and show there is always a solution for $k, \ell$ regardless of the values of $y,z$.

\noindent Exercise \ref{exercise:modular:congruence}  (c): Follow the method used in the Chinese Remainder Theorem, and for each modular equivalence obtained show that a solution exists.

\noindent Exercise \ref{exercise:modular:congruence}  (d): Suppose that $x$ and $y$ are both solutions to the given pair of congruences. Show that 
$ x-y \equiv 0 \pmod{m}$ and $x-y \equiv 0 \pmod{n}$. This implies that both $m$ and $n$ divide $x-y$ (explain why).