\section{Study guide  for ``Cosets and Quotient Groups (a.k.a. Factor Groups)''  chapter}
\label{sec:Cosets:study} 


\subsection*{Section \ref{Cosets:def}, Definition of cosets}
\subsubsection*{Concepts}
\begin{enumerate}
\item 
Review:
	\begin{enumerate}
	\item
	Two numbers are equivalent mod $m$ if they have the same remainder under division by $m$. (Definition~\ref{definition:ModularArithmetic:equivalence})
	\item
	Given $a, b, m \in {\mathbb Z}$, then $a \equiv b \pmod{m} \iff m | (a - b)$. (Proposition~\ref{proposition:ModularArithmetic:equivalence_alt})
	\item
	Equivalence class for integers mod 3 (or ${\mathbb Z}_3$) is $\{ [0]_3, [1]_3, [2]_3 \}$ this process can be used for all ${\mathbb Z}_n$ (Definition~\ref{mod_eqiv_def_3})
	\item
	Using Example~\ref{example:Groups:Z2subgroup} and Exercise~\ref{exercise:Groups:63} you can see that $m{\mathbb Z}$ is equivalent to  $[0]_m$.
	\end{enumerate}
	
\item
Using (d) from above you can create the equivalence classes using the new notion of:
	\begin{align*}
	0 \pmod{m} &= 0 + m{\mathbb Z} = [0]_m
	\\
	1 \pmod{m} &= 1 + m{\mathbb Z} = [1]_m
	\\
	2 \pmod{m} &= 2 + m{\mathbb Z} =  [2]_m
	\\
	&\vdots
	\\
	n \pmod{m} &= n + m{\mathbb Z} =  [n]_m
	\end{align*}

\item
If you have group $G$ and $H$ as a subgroup of $G$, then the left coset of $H$ with representative $g \in G$ is defined by:
\\
$$gH = {gh:h \in H}$$
\\
and the right cosets is defined by:
\\
$$Hg = {hg:h \in H}$$
which is (subgroup)(group operation)(group element).
\\
Note that both operations above are representative of $\circ$. . (Definition~\ref{definition:Cosets:def_coset})

\item
Left and right cosets are not always equal, because they are not commutative. (Example~\ref{example:Cosets:S3_Cosets})
\end{enumerate}

\subsubsection*{Competencies}
\begin{enumerate}
\item
Write the equivalence classes that make up ${\mathbb Z}_n$ in the new notation (Exercise~\ref{exercise:Cosets:equiv_class_mod5})
\item
Determine the left and/or right cosets and determine if they are equal. (Exercises~\ref{exercise:Cosets:Z6_cosets}, \ref{exercise:Cosets:left_right_cosets})
\item
Show that group $G$ is an abelian group and $H$ is a subgroup of $G$. (Exercise~\ref{exercise:Cosets:abelian_cosets})
\end{enumerate}


\subsection*{Section \ref{sec:Cosets:CosetsPartitions}, Cosets and partitions of groups}
\subsubsection*{Concepts}
\begin{enumerate}
\item 
%(Proposition~\ref{proposition:Cosets:%proposition name%})
\
%(Example~\ref{example:Cosets:%example name%})
\
%(Table~\ref)
\
\item
\end{enumerate}

\subsubsection*{Competencies}
\begin{enumerate}
\item
%(\ref{exercise:Cosets:%exercise name%})
\item
%(\ref{exercise:Cosets:%exercise name%})
\end{enumerate}


\subsection*{Section \ref{sec:Cosets:LagrangeTheorem}, Lagrange's theorem, and some consequences}
\subsubsection*{Concepts}
\begin{enumerate}
\item 
%(Proposition~\ref{proposition:Cosets:%proposition name%})
\
%(Example~\ref{example:Cosets:%example name%})
\
%(Table~\ref)
\
\item
\end{enumerate}

\subsubsection*{Competencies}
\begin{enumerate}
\item
%(\ref{exercise:Cosets:%exercise name%})
\item
%(\ref{exercise:Cosets:%exercise name%})
\end{enumerate}


\subsection*{Section \ref{subsec:Cosets:NormalSubAndFactorGroup:NormalSubgroups}, Quotient groups and normal subgroups}
\subsubsection*{Concepts}
\begin{enumerate}
\item 
%(Proposition~\ref{proposition:Cosets:%proposition name%})
\
%(Example~\ref{example:Cosets:%example name%})
\
%(Table~\ref)
\
\item
\end{enumerate}

\subsubsection*{Competencies}
\begin{enumerate}
\item
%(\ref{exercise:Cosets:%exercise name%})
\item
%(\ref{exercise:Cosets:%exercise name%})
\end{enumerate}





