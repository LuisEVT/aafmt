
\section{Solutions for ``Cosets and Factor Groups''}
\noindent\textbf{\textit{ (Chapter \ref{cosets}})}\bigskip
\\
\textbf{Exercise \ref{exercise:cosets:equiv_class_mod5}:}
\begin{multicols}{2}
\begin{enumerate}[(a)]
\item
%Write the 5 equivalence classes (subsets of ${\mathbb Z}$)  which make up ${\mathbb Z}_5$ using our new notation.
	\begin{itemize}
	\item
	$[1]_5 = 5{\mathbb Z} + 1$
	\item
	$[2]_5 = 5{\mathbb Z} + 2$
	\item
	$[3]_5 = 5{\mathbb Z} + 3$
	\item
	$[4]_5 = 5{\mathbb Z} + 4$
	\item
	$[0]_5 = 5{\mathbb Z} + 0$
	\end{itemize}
\columnbreak	
\item
%Write all elements of ${\mathbb Z}_7$ using our new notation.
	\begin{itemize}
	\item
	$[1]_7 = 7{\mathbb Z} + 1$
	\item
	$[2]_7 = 7{\mathbb Z} + 2$
	\item
	$[3]_7 = 7{\mathbb Z} + 3$
	\item
	$[4]_7 = 7{\mathbb Z} + 4$
	\item
	$[5]_7 = 7{\mathbb Z} + 5$
	\item
	$[6]_7 = 7{\mathbb Z} + 6$
	\item
	$[0]_7 = 7{\mathbb Z} + 0$
	\end{itemize}
\end{enumerate}
\end{multicols}

\noindent\textbf{Exercise \ref{exercise:cosets:Z6_cosets}:}
%Let $H$ be the subgroup of ${\mathbb Z}_6 = \{0,1,2,3,4,5\}$ consisting of the elements 0 and 3.
%(We are using our simplified notation here: `0' represents $\overline{0}$, etc.)  The left cosets are 
%\begin{gather*}
%0 + H = 3 + H = \{ 0, 3 \} \\
%1 + H = 4 + H = \{ 1, 4 \} \\
%2 + H = 5 + H = \{ 2, 5 \}.
%\end{gather*}
%What are the right cosets? Are the left and right cosets equal?
%\\
%$H = {0, 3}$ and ${\mathbb Z}_6 = {0, 1, 2, 3, 4, 5}$
\\
Right coset:
\begin{gather*}
H + 0 = H + 3 = \{0, 3\}
\\
H + 1 = H + 4 = \{ 1, 4 \} 
\\
H + 2 = H + 5 = \{ 2, 5 \}.
\end{gather*}
Yes, the left and right cosets are equal.
\\
\\
\textbf{Exercise \ref{exercise:cosets:left_right_cosets}:}
%List the left and right cosets of the subgroups in each of the following.  Tell whether the left and right cosets are equal.
%
%\noindent
%(Recall that $A_n$ (the alternating group) is the set of even permutations, on $n$ objects; $D_4$ is the group of symmetries of a square; and ${\mathbb T}$ is the group of complex numbers with modulus 1, under the operation of multiplication.)
\begin{enumerate}[(a)]

\skipitems{1}

\item
%$\langle 3 \rangle$ in $U(8)$
%\\
%$G = U(8) = \{1, 3, 5, 7\}$ and $H = \langle 3 \rangle = \{1, 3\}.  (3 \equiv 3, 3 \odot 3 \equiv 1)$ and $U(8)$ is *
%\\
Left coset:
\begin{gather*}
1H = 3H = \{1, 3\}
\\
5H = 7H = \{5, 7\} 
\end{gather*}
\\
Right coset:
\begin{gather*}
H1 = H3 = \{1, 3\}
\\
H5 = H7 = \{5, 7\} 
\end{gather*}
Yes, the left and right cosets are equal.

\skipitems{1}

\item
%$H = \{ (1), (123), (132) \}$ in $S_4$
%\\
%$H = \{ (1), (123), (132) \}$ and $S_4 = \{(1), (12), (13), (14), (23), (24), (34), (123), (124), (132),(134), (142),
%\\
%(143), (234), (243), (1234), (1243), (1324), (1342), (1423), (1432), (12)(34), (13)(24), (14)(23)\}$
%\\
Left coset:
\begin{gather*}
(1)H = (123)H = (132)H = \{(1), (123), (132)\}
\\
(12)H =(13)H = (23)H = \{(12), (13), (23)\} 
\\
(14)H =(1234)H = (1324)H = \{(14), (1234), (1324)\} 
\\
(24)H =(1423)H = (1342)H = \{(24), (1423), (1342)\}
\\
(34)H =(1243)H = (1432)H = \{(34), (1243), (1432)\}
\\
(124)H =(14)(23)H = (134)H = \{(124), (14)(23), (134)\}
\\
(142)H =(234)H = (13)(24)H = \{(142), (234), (13)(24)\}
\\
(143)H =(12)(34)H = (243)H = \{(143), (12)(34), (243)\}
\end{gather*}
\\
Right coset:
\begin{gather*}
H(1) = H(123) = H(132) = \{(1), (123), (132)\}
\\
H(12) = H(23) = H(13) = \{(12), (23), (13)\} 
\\
H(14) = H(1423) = H(1432)  = \{(14), (1423), (1432)\} 
\\
H(24)H = H(1243) = H(1324)  = \{(24), (1243), (1324)\}
\\
H(34) = H(1234) = H(1342) = \{(34), (1234), (1342)\}
\\
H(124) = H(243) = H(13)(24) = \{(124), (243), (13)(24)\}
\\
H(134) = H(234) = H(12)(34) =  \{(134), (234), (12)(34)\}
\\
H(142) = H(143) = H(14)(23) =  \{(142), (143), (14)(23)\}
\end{gather*}
No, the left and right cosets are not equal.

\item
%$A_4$ in $S_4$, $A_4$ is the even permutations of $S_4$
%%\\
%$A_4 = \{(1), (123), (124), (132), (134), (142), (143), (234), (243), (12)(34), (13)(24), (14)(23)\}$
%\\
%and $S_4 = \{(1), (12), (13), (14), (23), (24), (34), (123), (124), (132), (134), (142), (143), 
%\\
%(234), (243), (1234), (1243), (1324), (1342), (1423), (1432), (12)(34), (13)(24), (14)(23)\}$
%\\
Left cosets:
\\
$A_4$ in $S_4$
\\
$S_4\setminus A_4$
\\
%\begin{gather*}
%(1)H = (123)H = (124)H = (132)H = (134)H = (142)H =  (143)H = (234)H = (243)H = (12)(34)H = 
%\\
%(13)(24)H = (14)(23)H = \{(1), (123), (124), (132), (134), (142), (143), (234), (243), (12)(34), (13)(24), (14)(23)\}
%\\
%\\
%(12)H = (13)H = (14)H = (23)H = (24)H = (34)H = (1234)H = (1243)H = (1324)H = 
%\\
%(1342)H = (1423)H = (1432)H = \{(12), (13), (14), (23), (24), (34), (1234), (1243), (1324), (1342), (1423), (1432)\}
%\end{gather*}
\\
Right cosets:
\\
$A_4$ in $S_4$
\\
$S_4\setminus A_4$
\\
%\begin{gather*}
%H(1) =  H(123) = H(124) = H(132) = H(134) = H(142) = H(143) = H(234) = H(243) = H(12)(34) = 
%\\
%H(13)(24) = H(14)(23) =\{(1), (123), (124), (132), (134), (142), (143), (234), (243), (12)(34), (13)(24), (14)(23)\}
%\\
%\\
%H(12) = H(13) = H(14) = H(23) = H(24) = H(34) = H(1234) = H(1243) = H(1324) = 
%\\
%H(1342) = H(1423) = H(1432) = \{(12), (13), (14), (23), (24), (34), (1234), (1243), (1324), (1342), (1423), (1432)\}
%\end{gather*}

Yes, the left and right cosets are equal.

\item
%$A_n$ in $S_n$
%\\
%$A_n$ = \{even permutations $\in S_n\} = H$
%\\
%$S_n$ = \{all the even and odd permutations of a symmetric group, where $n \in {\mathbb N}\} = G$
%\\
%\\
Left coset, as shown in part e:
\\
$g_{even}H$ = \{all even permutations in $S_n$\}
\\
$g_{odd}H$ = \{all odd permutation in $S_n$\} 
\\
\\
Right coset, as shown in part e:
\\
$Hg_{even}$ = \{all even permutations in $S_n$\}
\\
$Hg_{odd}$ = \{all odd permutation in $S_n$\} 
\\
\\
Yes, the left and right cosets are equal.

\item
%$D_4$ in $S_4$
%\\
%$D_4 = \{(1), (13), (24), (1234), (1432), (12)(34), (13)(24), (14)(23)\}$ and 
%\\
%$S_4 = \{(1), (12), (13), (14), (23), (24), (34), (123), (124), (132), (134), (142), (143), (234), (243), (1234), (1243), (1324), (1342), (1423), (1432), (12)(34), (13)(24), (14)(23)\}$
%\\
Left coset:
\begin{gather*}
(1)H = (13)H = (24)H = (1234)H = (1432)H =  (12)(34)H = (13)(24)H = (14)(32)H = 
\\
\{(1), (13), (24), (1234), (1432), (12)(34), (13)(24), (14)(23)\}
\\
\\
(12)H = (34)H = (124)H = (132)H = (143)H = (234)H = (1324)H = (1423)H = 
\\
\{(12), (34), (124), (132), (143), (234), (1324), (1423)\}
\\
\\
(14)H = (23)H = (123)H = (134)H = (142)H = (243)H = (1243)H = (1342)H = 
\\
\{(14), (23), (123), (134), (142), (243), (1243), (1342)\}
\end{gather*}
\\
Right coset:
\begin{gather*}
H(1) = H(13) = H(24) = H(1234) = H(1432) =  H(12)(34) = H(13)(24) = H(14)(32) = 
\\
\{(1), (13), (24), (1234), (1432), (12)(34), (13)(24), (14)(23)\}
\\
\\
H(12) = H(34) = H(123) = H(134) = H(142)  = H(243) = H(1324) = H(1423) = 
\\
\{(12), (34), (123), (134), (142), (243), (1324), (1423)\}
\\
\\
H(14) = H(23) = H(124) = H(132) = H(143)  = H(234) = H(1243) = H(1342) = 
\\
\{(14), (23), (124), (132), (143), (234), (1243), (1342)\}
\end{gather*}

No, the left and right cosets are not equal.
\\

\item
%${\mathbb T}$ in ${\mathbb C}^\ast$ 
%\\
%$H$ = ${\mathbb T}$ is the unit circle = \{$z$ such that $|z| = 1$\}
%\\
%$G$ = ${\mathbb C}^\ast$ is the nonzero complex numbers under multiplication.
%\\
The cosets of ${\mathbb T}$ are the sets of the form $w{\mathbb T}$, where $w$ is a complex number.
\\
Suppose that $z \in {\mathbb T}$.  In polar form, $z = \cis (\theta)$, where $z \in {\mathbb T}$.
\\
So, $wz =( r \cis (\phi))(\cis (\theta)) = r \cis (\phi + \theta)$
\\
$\implies |wz| = r \implies wz$ lies on the circle with center 0 and radius $r$.
\\
Also consider, complex number $v$, that lies on the circle with center 0 and radius r. 
\\
Then $v = r \cis(\alpha)$ for any angle $\alpha$.
\\
It follows that $v = r \cis(\phi) \cis(-\phi) \cis(\alpha) = r \cis(\phi) \cis(\alpha - \phi) = wz'$, where $|z'| = 1$.
\\
So, $v \in w{\mathbb T}$.
\\
$\therefore$ the left cosets are all the circles of the complex plane with center 0.
\\
Since, ${\mathbb C}^\ast$ is abelian from Exercise \ref{exercise:groups:C_star}d, left and right cosets are equal.
\end{enumerate}

\noindent\textbf{Exercise \ref{exercise:cosets:abelian_cosets}:}
%\\
%Show that if $G$ is an abelian group and $H$ is a subgroup of $G$, then any left coset $gH$ is equal to the right coset $Hg$.  
%\\
\\
Suppose $G$ is an abelian group and $H$ is a subgroup of $G$.
\\
The left coset: $gH = \{gh: h \in H\}$.
\\
The right coset: $Hg = \{hg: h \in H\}$.
\\
Since $G$ is abelian we have: $gh = hg, \forall \  h \in H$.
\\
Therefore, the left and right cosets are equal.
\\
\\
\textbf{Exercise \ref{exercise:cosets:CosetEquiv}:}
%Theorem \ref{cosets_theorem_1}  Let $H$ be a subgroup of a group $G$ and suppose that $g_1, g_2 \in G$.  The following conditions are equivalent.  
%\begin{multicols}{2}
%\begin{enumerate}
%\item
%$g_1 H = g_2 H$; 
%
%\item
%$g_1^{-1} g_2 \in H$.
%
%\item
%$g_2 \in g_1 H$; 
%
%\item
%$g_2 H \subset g_1 H$; \qquad (\emph{Note:} ``$\subset$'' means that equality is also possible)
%
%\item
%$H g_1^{-1}  = H g_2^{-1}$; 
%\end{enumerate}
%\end{multicols}
%\noindent\makebox[\linewidth]{\rule{\paperwidth}{0.2pt}}
\begin{enumerate}[(a)]
\item
%Show that condition (1) implies condition (2).  
%\\
$g_2 = g_2 e$
\\
It follows that $g_2 \in g_2 H$
\\
By (1), it follows that $g_2 \in g_1 H$ since $g_1H = g_2 H$
\\
By def. $g_1 H$, it follows that $g_2 = g_1 h$ for some $h \in H$
\\
By composition on left, $(g_1)^{-1} g_2 = g_1^{-1} g_1 h$
\\
By def. of inverse $(g_1)^{-1} g_2 = eh$
\\
By def. of identity $(g_1)^{-1} g_2 = h$ 
\\
Since, $h \in H, g_1^{-1}g_2 \in H$.

\item
%Show that condition (2) implies condition (3).
%\\
Let $g_1^{-1}g_2 \in H$
\begin{align*}
g_1^{-1}g_2 &= h
\\
g_1g_1^{-1}g_2 &= g_1h &\text{by composition on left by\ } g_1
\\
eg_2 &= g_1h &\text{by def. of inverse}
\\
g_2 &= g_1h &\text{by def. of identity}
\end{align*}
Since, $h \in H, g_2 \in g_1H$.

\skipitems{1}

\item
%Show that condition (4) implies condition (1).
%\\
Given $g_2H \subset g_1H$. Let $g_1h \in g_1H$.
\\
$\implies g_2(id) \in g_1H$
\\
$\implies g_2 = g_1h'$, where $h' \in H$.
\\
\begin{align*}
g_2h'^{-1} &=g_1h'h'^{-1} &\text{by composition on right by\ } h'^{-1}
\\
g_2g'^{-1} &= g_1e &\text{def. of inverse}
\\
g_2g'^{-1} &= g_1 &\text{def. of identity}
\\
g_2h'^{-1}h &= g_1h &\text{by composition on right by\ } h
\end{align*}
Since $h'^{-1} h \in H$, it follows that $g_1 h \in g_2 H$.
\\
Since $g_1 h$ is an arbitrary element of $g_1H$, it follows that $g_1H \subset g_2 H$.
\\
By condition (4), $g_2H \subset g_1 H$.
\\
It follows that $g_1 H = g_2 H$


\item
%Show that condition (2) implies condition (5).
%\\
Given $g_1^{-1}g_2 \in H \implies Hg_1^{-1} = Hg_2^{-1}$.
\\
Let $h_1g_1^{-1} \in Hg_1^{-1}$
\\
By given, $g_1^{-1}g_2 = h$
\begin{align*}
g_1^{-1}g_2g_2^{-1} &= hg_2^{-1} &\text{ by right compose of} g_2^{-1}
\\
g_1^{-1}e &= hg_2^{-1} &\text{by def of inverse}
\\
g_1^{-1} &= hg_2^{-1} &\text{def of identity}
\\
h_1 g_1^{-1} \in H g_2^{-1} &\text{def\ } H g_2^{-1}
\end{align*}
Since $h_1 g_1^{-1}$ was an arbitrary element of $H g_1^{-1}$, it follows that $H g_1^{-1} \subset H g_2^{-1}$.
\\
A similar argument shows that $H g_2^{-1} \subset H g_1^{-1}$.
\\
It follows that $H g_1^{-1} = H g_2^{-1}$.

\item
%Show that condition (5) implies condition (2).
%\\
Given $h_1 g_1^{-1}$, there exists some $h_2$ such that $h_1 g_1^{-1} = h_2 g_2^{-1}$.
\\
Let $h_1, h_2 \in H$
\begin{align*}
hg_1^{-1} &= h g_2^{-1} &\text{by given}
\\
h_1g_1^{-1}g_2 &= h_2g_2^{-1}g_2 &\text{ by right compose of\ } g_2
\\
h_1g_1^{-1}g_2 &= h_2e &\text{by def of inverse}
\\
h_1g_1^{-1}g_2 &= h_2 &\text{def of identity}
\\
h_1^{-1}h_1g_1^{-1}g_2 &= h_1^{-1}h_2  &\text{by left compose of\ } h_1^{-1}
\\
eg_1^{-1}g_2 &= h_1^{-1}h_2 &\text{by def of inverse}
\\
g_1^{-1}g_2 &= h_1^{-1}h_2 &\text{def of identity}
\end{align*}
Since $h_1^{-1}, h_2 \in H$, it follows that $g_1^{-1}g_2 \in H$
\end{enumerate}

\noindent\textbf{Exercise \ref{exercise:cosets:equiv_conditions}:} %10.2.3
%Proposition~\ref{cosets_theorem_1} deals with \emph{left} cosets. A parallel proposition holds for right cosets. List the five equivalent conditions for \emph{right} cosets that correspond to the five conditions given in Proposition~\ref{cosets_theorem_1}.
\\
Right cosets:
\begin{multicols}{3}
\begin{enumerate}
\item
$Hg_1 = Hg_2$

\item
$g_2 g_1^{-1} \in H$

\item
$g_2 \in Hg_1$

\item
$Hg_2 \in Hg_1$

\item
$g_1^{-1}H = g_2^{-1}H$
\end{enumerate}
\end{multicols}

\noindent\textbf{Exercise \ref{exercise:cosets:disjoint_union_proof}:} %10.2.5
%Complete part (2) of the proof: that is, prove that $\bigcup_{g \in G} g H = G$.
\\
Suppose $g \in G$.
\\
Then $g \in gH$, where $gH$ is an arbitrary coset.
\\
\bigskip
$g \in \bigcup\limits_{g \in G} gH \implies G \subset \bigcup\limits_{g \in G} g_iH$
\\
\bigskip
Suppose $x \in \bigcup\limits_{g \in G} gH$.
\\
Then $x \in gH$ for some $ g \in G$.
\\
$gH \subset G$ by definition of closure.
\\
So, $x \in gH \subset G$.
\\
\bigskip
$x \subset G \implies \bigcup\limits_{g \in G} g_iH \subset G$.
\\
Since $G \subset \bigcup\limits_{g \in G} g_iH$ and $\bigcup\limits_{g \in G} g_iH \subset G$ then, $\bigcup\limits_{g \in G} g_iH = G$.
\\
\\
\\
\textbf{Exercise \ref{exercise:cosets:S3_right_index}:}
%How many right cosets of  $H = \{ (1),(123), (132) \}$ in $S_3$ were there?  How about right cosets of  $K= \{ (1), (12) \}$ in $S_3$?
\\
$G = S_3 \implies$ order of 6.
\\
$H$ has an order of 3.
\\
Left cosets = 2
\\
Right cosets = 2
\\
\\
$G = S_3 \implies$ order of 6.
\\
$K$ has an order of 2.
\\
Left cosets = 3
\\
Right cosets  = 3
\\
\\
\textbf{Exercise \ref{exercise:cosets:index_right_cosets}:}
%Using your work from Exercise~\ref{exercise:cosets:left_right_cosets}, find:
\begin{enumerate}[(a)]
\skipitems{1}

\item
%$[ U(8) : \langle 3 \rangle ]$ and the number of right cosets of $\langle 3 \rangle$ in $U(8)$.
%\\
%\\
%$G = U(8) = \{1, 3, 5, 7\}$ and $H = \langle 3 \rangle = \{1, 3\}$
%\\
$[ U(8) : \langle 3 \rangle ] = \frac{4}{2} = 2$
\\
The number of right cosets = 2.

\skipitems{1}

\item
%$[  S_4 : \{ (1), (123), (132) \}  ]$ and the number of the right cosets of $\{ (1), (123), (132) \}$ in $S_4$.
%\\
%\\
%$G = S_4 = \{(1), (12), (13), (14), (23), (24), (34), (123), (124), (132), (134), (142), (143), (234), (243), (1234), (1243), (1324), (1342), (1423), (1432), (12)(34), (13)(24), (14)(23)\}$ and $H = \{ (1), (123), (132) \}$
%\\
$[S_4 : \{(1), (123), (132)\}] = \frac{24}{3} = 8$
\\
The number of right cosets = 8.

\item
%$[ S_4 : A_4 ]$ and the number of right cosets of $A_4$ in $S_4$.
%\\
%\\
%$G = S_4 = \{(1), (12), (13), (14), (23), (24), (34), (123), (124), (132), (134), (142), (143), (234), (243), (1234), (1243), (1324), (1342), (1423), (1432), (12)(34), (13)(24), (14)(23)\}$ and $H = A_4 = \{(1), (123), (124), (132), (134), (142), (143), (234), (243), (12)(34), (13)(24), (14)(23)\}$
%\\
$[S_4 : A_4] = \frac{24}{12} = 2$
\\
The number of right cosets = 2.

\item
%$[ S_n : A_n ]$ and the number of right cosets of $A_n$ in $S_n$.
%\\
%\\
$[ S_n : A_n ] = 2$
\\
The number of right cosets = 2.

\item
%$[S_4 : D_4  ]$ and the number of right cosets of $D_4$ in $S_4$.
%\\
%\\
%$G = S_4 = \{(1), (12), (13), (14), (23), (24), (34), (123), (124), (132), (134), (142), (143), (234), (243), (1234), (1243), (1324), (1342), (1423), (1432), (12)(34), (13)(24), (14)(23)\}$ and $H = D_4 = \{(1), (13), (24), (1234), (1432), (12)(34), (13)(24), (14)(23)\}$
$[S_4 : D_4] = \frac{24}{8} = 3$

\item
%$[ {\mathbb C}^\ast : {\mathbb T} ]$ and the number or right cosets of ${\mathbb T}$ in ${\mathbb C}^\ast$.
%\\
%\\
$[ {\mathbb C}^\ast : {\mathbb T} ] = \infty$
\\
The number of right cosets = $\infty$.

\end{enumerate}

\noindent\textbf{Exercise \ref{exercise:cosets:SL2_cosets}:}
%Consider the left cosets of $SL_2( {\mathbb R} )$ in $GL_2( {\mathbb R})$.  Show that two matrices in $GL_2( {\mathbb R})$ are in the same left coset of $SL_2( {\mathbb R} )$ if and only if they have the same determinant. Is the same true for right cosets? (Prove your answer.) 
%\\
\\
Let $G = GL_2({\mathbb R})$  and $H = SL_2({\mathbb R})$. Suppose $a$ and $b$ are matrices that are in the same coset of $H$.  By Condition (3) of Exercise~\ref{exercise:cosets:CosetEquiv}, $b \in aH$. 
\begin{align*}
\implies b &= ah
\\
\implies \det(b) &= \det(ah)  & &\text{[substitution]}
\\
&= \det(a)\det(h) &= \det(a)  &\text{[ by det product rule and\ } h \in SL_2{\mathbb R} ]
\\
\text{So\ } \det b &= \det a
\\
\\
\text{Suppose\ } \det b &= \det a
\\
\text{Show\ } \det(a^{-1} b) &= 1
\\
\implies a^{-1}b &\in H
\end{align*}
By condition … this implies that $a$ and $b$ are in the same coset.
\\
\\
\noindent\textbf{Exercise \ref{exercise:cosets:order5and7}:}
%Suppose that $G$ is a finite group with an element $g$ of order 5 and an element $h$ of order 7. 
%
\begin{enumerate}[(a)]
\item
%Show that $G$ has subgroups of order 5 and 7.
%\\
Proof: Given a finite group with $g \in G, h \in G$ such that $|g| = 5, |h| = 7$, then by Proposition~\ref{proposition:groups:OrbitIsSubgroup}, $\langle g \rangle$ is a subgroup of $G$.
\\
By Proposition~\ref{proposition:groups:OrderEltCyclic}, $|\langle g \rangle| = |g|$, so $|\langle g \rangle| = 5$, where $\langle g \rangle$ is a subgroup of $G$ of order 5.
\\
Also by Proposition~\ref{proposition:groups:OrbitIsSubgroup}, $\langle h \rangle$ is a subgroup of $G$.
\\
So by Proposition~\ref{proposition:groups:OrderEltCyclic}, $|\langle h \rangle| = |h|$, so $|\langle h \rangle| = 7$, where $\langle h \rangle$ is a subgroup of $G$ of order 7.
\\
Therefore, if $g, h \in G$ such that $|g| = 5, |h| = 7, G$ has subgroups of order 5 and 7.

\item
%Why must $|G| \geq 35$?
%\\
$|G|$ must be $\geq$ based on Lagrange's Theorem.
\\
Since $\langle g \rangle$ is a subgroup of $G, |\langle g \rangle|$ must divide $|G|$, so $|G|$ is a multiple of 5.
\\
Also, $\langle h \rangle$ is also a subgroup of $G$, so $|\langle h \rangle| =7$ must also divide $|G|$.  Then $|G|$ is also a multiple of 7.
\\
Since $\lcm(5, 7) = 35, |G|$ must be $\geq 35$. 
\end{enumerate}

\noindent\textbf{Exercise \ref{exercise:cosets:finite_60}:}
%Suppose that $G$ is a finite group with 60 elements.  What are the possible orders for subgroups of $G$?
%\\
\\
Proposition~\ref{proposition:cosets:LagrangeTheorem} tells us that $|G|$ is evenly divided by $|H|$ and we know that $|G| = 60$.
\\
$\implies |H| = \{1, 2, 3, 4, 5, 6, 10, 12, 15, 20, 30, 60\}$.
\\
\\
\noindent\textbf{Exercise \ref{exercise:cosets:phivals}:}
%Evaluate the following:
\begin{enumerate}[(a)]
\item
%$\phi(12)$
%\\
%$U(12) = \{1, 5, 7, 11\}$
%\\
$\phi(12) = 4$ 

\item
%$\phi(16)$
%\\
%$U(16) = \{1, 3, 5, 7, 9, 12, 13, 15\}$
%\\
$\phi(16) = 8$
 
\item
%$\phi(20)$
%\\
%$U(20) = \{1, 3, 7, 9, 11, 13, 17, 19\}$
%\\
$\phi(20) = 8$

\item
%$\phi(23)$
%\\
%$U(23) = \{1, 2, 3, 4, 5, 6, 7, 8, 9, 10, 11, 12, 13, 14, 15, 16, 17, 18, 19, 20, 21, 22\}$
%\\
$\phi(23) = 22$

\item
%$\phi(51)$
%\\
%$U(51) = \{1, 2, 4, 5, 7, 8, 10, 11, 13, 14, 16, 19, 20, 22, 23, 25, 26, 28, 29, 31, 32, 35, 37, 38, 40, 41, 43, 44, 46, 47, 49, 50\}$
%\\
$\phi(51) = 32$

\item
%$\phi(p)$, where $p$ is prime.
%\\
$\phi(p) = p - 1$

\item
%$\phi(p^2)$, where $p$ is prime (\emph{justify} your answer).
%\\
Every number from 1 to $p^2$, except multiples of $p$. 
\\
For example, $U(5^2) = \{1, 2, 3, 4, 6, 7, 8, 9, 11, 12, 13, 14, 16, 17, 18, 19, 21, 22, 23, 24\}$
\\
$\phi(5^2) = 20$ or $5^2 - 5 = 20$
\\ 
$\phi(p^2) = p^2 - p$

\skipitems{1}

\item
%$\phi(pq)$, where $p$ and $q$ are primes and $p \neq q$ (\emph{justify} your answer).
%\\
Every number except $pq, p, q$ or multiples of $p, q$
\\
For example, $U(3 \cdot 5) = U(15) = \{1, 2, 4, 7, 8, 11, 13, 14\} = 8$
\\
$\phi(pq) = pq - p - q + 1$.
\end{enumerate}

\noindent\textbf{Exercise \ref{exercise:cosets:modvals}:}
%Evaluate the following, using the results of Exercise 10.3.10
\begin{enumerate}[(a)]
\item
%$\mod(5^{200},12)$
%\\
%Theorem~\ref{cosets:Eulers_theorem} $gcd(a, n) = 1$ and $a^{\phi(n)} \equiv 1 \pmod{n}$
%\\
gcd(5, 12) = 1
\\
$5^{\phi(12)} \equiv 1 \pmod{12} \quad \quad \phi(12) = 4$
\\
$(5^{50})^4 \equiv 1 \pmod{12}$

\skipitems{1}

\item
%$\mod(15^{221}, 23)$
%\\
%Theorem~\ref{cosets:Eulers_theorem} $gcd(a, n) = 1$ and $a^{\phi(n)} \equiv 1 \pmod{n}$
%\\
gcd(15, 23) = 1
\\
$15^{\phi(23)} \equiv 1 \pmod{23} \quad \quad \phi(23) = 22$
\\
$(15^{10})^{22} \equiv 1 \pmod{23}$
\\
$\mod(15^{221}, 23) \equiv 1 \pmod{23} \cdot 15 \pmod{23} = 15 \pmod{23}$


\skipitems{1}

\item
%$\mod(10^{195},221)$
%\\
%Theorem~\ref{cosets:Eulers_theorem} $gcd(a, n) = 1$ and $a^{\phi(n)} \equiv 1 \pmod{n}$
%\\
gcd(10, 221) = 1
\\
$10^{\phi(221)} \equiv 1 \pmod{221} \quad \quad \phi(221) = 192$
\\
$10^{192} \equiv 1 \pmod{221}$
\\
$\mod(10^{195},221) \equiv 1 \pmod{221} \cdot 10^3 \pmod{221} = 116 \pmod{221}$

\item
%$\mod \left( \left( \frac{p+1}{2} \right)^p,p\right)$, where $p$ is prime.
%\\
%Theorem~\ref{cosets:Eulers_theorem} $gcd(a, n) = 1$ and $a^{\phi(n)} \equiv 1 \pmod{n}$
%\\
gcd$(\frac{p+1}{2}, p) = 1$
\\
$\left(\frac{p+1}{2}\right)^{\phi(p)} \equiv 1 \pmod{p} \quad \quad \phi(p) = p - 1$
\\
\\
$\left(\frac{p+1}{2}\right)^{p-1} \equiv 1 \pmod{p}$
\\
$\mod \left( \left( \frac{p+1}{2} \right)^p,p\right) \equiv 1 \pmod{p} \cdot \left( \frac{p+1}{2} \right)^p \pmod{p} =  \left( \frac{p+1}{2} \right)^p \pmod{p}$
\end{enumerate}

\noindent\textbf{Exercise \ref{exercise:cosets:FermatLittle}:}
%Suppose that $p$ is a prime number, and $a$ is a natural number which is relatively prime to $p$. Show that  $a^{p-1} \equiv 1 \bmod{p}$.
%\\
\\
Let gcd$(a, p) = 1$.  Then we know that $a^{\phi(p)} \equiv 1 \pmod{p}$, by Proposition~\ref{proposition:cosets:cosets:Eulers_theorem}.
\\
By Exercise~\ref{exercise:cosets:phivals}f we know that $\phi(p) = p - 1$. So, $a^{p - 1} \equiv 1 \pmod{p}$.
\\
\\

\noindent\textbf{Exercise \ref{exercise:cosets:primeGroups}:}
%Let $G$ be a group such that $|G| = p$, where $p$ is a prime number.
\begin{enumerate}[(a)]
\item
%Let $a$ be an element of $G \setminus \{e\}$.  What does Proposition~\ref{cosets_theorem_6} tell us about $|a|$? (Recall that `$\setminus$' is the set difference operation, defined in Definition \ref{setdifference}).  \hyperref[sec:cosets:hints]{(*Hint*)}
%\\
%\\
Suppose $G$ is a group such that $|G| = p$ where $p$ is prime.
\\
Suppose $a \in G \setminus \{e\}$.
\\
Since $p$ is prime its subgroups must be of the order 1 or $p$.
\\
Since $a \in G \setminus \{e\}$ we know $a$ is not $\{e\}$
\\
$\implies |a| = p$

\item
%Prove that $G$ is cyclic.
%\\
%\\
Then by part a showing $|a| = p \implies a^p = e \implies | \langle a \rangle | = p$.
\\
Since $\langle a \rangle \subset G$ and $| \langle a \rangle | = p = |G| \implies \langle a \rangle = G \implies G$ is cyclic.

\item
%Describe the set of generators of $G$  (recall that $g \in G$ is a generator of $G$ if $\langle g \rangle = G$.)
%\\
%\\
Any besides \{e\} is a generator.
\end{enumerate}

\noindent\textbf{Exercise \ref{exercise:cosets:prime_simple}:}
%Let $G$ be a group of prime order. Use Proposition~\ref{cosets_theorem_7} to show that the only proper subgroup of $G$ is the trivial subgroup $\{e\}$.
%\\
\\
Suppose $G$ is a group of prime order $\implies |G| = p$.
\\
Since $|G|$ is prime then the order of the subgroups of $G$ are 1 and $p$.  The only proper subgroup of $G$ would have order 1, which means it is the trivial subgroup containing the identity, $\{e\}$.
\\
$\{e\}$ is a subgroup by Exercise~\ref{exercise:groups:trivial}.
\\
\\
\noindent\textbf{Exercise \ref{exercise:cosets:which_normal}:}
%Looking back at Exercise~\ref{exercise:cosets:left_right_cosets}, which of the subgroups were normal?
%\\
\begin{enumerate}[(a)]

\skipitems{1}

\item
$\langle 3 \rangle$ in $U(8)$ is normal.

\skipitems{1}

\item
$H = \{ (1), (123), (132) \}$ in $S_4$ is not normal.

\item
$A_4$ in $S_4$, $A_4$ is the even permutations of $S_4$ is normal.

\item
$A_n$ in $S_n$ is normal.

\item
$D_4$ in $S_4$ is normal.

\item
${\mathbb T}$ in ${\mathbb C}^\ast$ is normal.

\end{enumerate}


\noindent\textbf{Exercise \ref{exercise:cosets:SL2_normal}:}
%Is $SL_2( {\mathbb R} )$ a normal subgroup of $GL_2( {\mathbb R})?$  Prove or disprove.
%\\
\\
By Exercise~\ref{exercise:complex:56}a, we proved that matrix multiplication is not always abelian. Therefore, $SL_2{\mathbb R}$ will not be a normal subgroup of $GL_2{\mathbb R}$.

\noindent\textbf{Exercise \ref{exercise:cosets:e_normal}:}
%Prove that for \emph{any} group $G$, the set $\{e\}$ is a normal subgroup of $G$ (in other words the identity of group is always a normal subgoup).
%\\
\\
Suppose $G$ is a group and let $H = \{e\}$.  Also, suppose that $g_i \in G, i \in {\mathbb N}$.
\\
$\implies g_ie = eg_i = g_i$
\\
Therefore the left and right cosets are equal, because the identity element can be compose on either side and output the $g_i$ it is composed with.
\\
\\
\noindent\textbf{Exercise \ref{exercise:cosets:abelian_normal}:}
%Prove that any subgroup of an abelian group is normal.
%\\
\\
Suppose you have an abelian group, $G$, and it's subgroup, $H$. Let $g \in G$.
\\
Left coset = $gH$ and the Right coset = $Hg$.
\\
Since $G$ is abelian $\implies gH = Hg \implies H$ is a normal subgroup.
\\
\\

\noindent\textbf{Exercise \ref{exercise:cosets:normalk}:}
%In the following exercises, $G$ is a group and $H$ is a subgroup of $G$.
\begin{enumerate}[(a)]
\item
%Show that for any $g \in G$ then $gHg^{-1}$ is also a subgroup of $G$.
%\\
%\\
Suppose that $H$ is a subgroup of $G$.  Also suppose that $g \in G$ and $h \in H$.
\\
Since $e$ is an element of all groups, $e \in H$.
\begin{align*}
&gHg^{-1}
\\
\implies &geg^{-1} &\text{by substitution}
\\
\implies &gg^{-1} &\text{def of identity}
\\
\implies &e &\text{def of inverse}
\\
\implies &e \in gHg^{-1}
\end{align*}
So we've shown that $H$ has an identity element.
\\
\\
Suppose that $h_1 \in H \implies h^{-1} \in H$ (by def of subgroup).
\begin{align*}
&(gh_1g^{-1})^{-1}
\\
&g^{-1}h_1^{-1}g &\text{by algebra}
\\
&g^{-1}h_1^{-1}g \in gHg^{-1} 
\end{align*}
So we've shown that every element has an inverse.
\\
\\
Suppose $gh_2g^{-1} \in gHg^{-1}$ and$h_2 \in H$.
\begin{align*}
(gh_1g^{-1})(gh_2g^{-1}) &= gh_1(g^{-1}g)h_2g^{-1} &\text{by associativity}
\\
&= gh_1eh_2g^{-1} &\text{by def of inverse}
\\
&= gh_1h_2g^{-1} &\text{by def of identity}
\\
&= g(h_1h_2)g^{-1} &\text{by associativity}
\\
&= gHg^{-1} &\text{since\ } h_1, h_2 \in H
\end{align*}
Since H has an identity, inverse, and closure we know that $H$ is a subgroup.
 
\item
%Define a function $f: H \rightarrow gHg^{-1}$ as follows:  $f(h) = ghg^{-1}$.  Show that $f$ is a bijection, and thus $|H| = |gHg^{-1}|$.
%\\
%\\
Let $f: H \mapsto gHg^{-1}$ be defined as $f(h) = ghg^{-1}$. Also, $g \in G$ and $h_1, h_2 \in H$.
\begin{align*}
f(h_1) &= f(h_2)
\\
gh_1g^{-1} &= gh_2g^{-1} &\text{(def of mapping)}
\\
g^{-1}gh_1g^{-1} &= g^{-1}gh_2g^{-1} &\text{(left compose by\ } g^{-1}\text{)}
\\
eh_1g^{-1} &= eh_2g^{-1} &\text{(def of inverse)}
\\
h_1g^{-1} &= h_2g^{-1} &\text{(def of identity)}
\\
h_1g^{-1}g &= h_2g^{-1}g &\text{(right compose by\ } g\text{)}
\\
h_1e &= h_2e &\text{(def of inverse)}
\\
h_1 &= h_2 &\text{(def of identity)}
\end{align*}
This shows the $f$ is one to one.
\\
\\
Suppose $y \in gHg^{-1}$, then $y = ghg^{-1}$, for all $g, g^{-1} \in G$ and $h \in H$.
\\
Since $f(h) = y$ (by def of $f(h)$), $f$ is onto.
\\
Therefore, since $f(h)$ is both one-to-one and onto, $f(h)$ is a bijection.
\end{enumerate}


\noindent\textbf{Exercise \ref{exercise:cosets:normal_mult}:}
\begin{enumerate}[(a)]
\item
%Let $H \subset G$ be a normal subgroup, and let $g \in G, h \in H$.  Show that $g^{-1}hg \in H$.
%\\
%\\
Suppose that $H \subset G$ and that $H$ is a normal subgroup.  
\\
Let $g \in G, h \in H$.
\\
Since $H$ is a normal subset, we can say that 
\begin{align*}
gHg^{-1} \subset H \text{\ and\ } gHg^{-1} &= H &\text{(by Proposition~\ref{proposition:cosets:normal:normalequivalents})}
\\
ghg^{-1} &= h' &h' \in H
\\
\text{Let\ } g = g^{-1}\text{, then\ } 
\\
g^{-1}h(g^{-1})^{-1} &= h' &\text{by substitution}
\\
g^{-1}hg &= h' &\text{by exponent rules}
\\
g^{-1}hg &\in H &h' \in H 
\end{align*}

\item
%Let $H \subset G$ be a normal subgroup, and let $g \in G, h \in H$. Use part (a) to show that there exists an $h' \in H$ such that $hg = g h'$.
%\\
%\\
Suppose that $H \subset G$ and that $H$ is a normal subgroup.  
\\
Let $g \in G$ and $h, h' \in H$.
\\
\begin{align*}
h &= ghg^{-1} &\text{part a}
\\
hg &= ghg^{-1}g &\text{right compose by\ } g
\\
hg &= ghe &\text{def of inverse}
\\
hg &= gh &\text{def of identity}
\end{align*}

\item
%Let $H \subset G$ be a normal subgroup, and suppose $x_1 \in g_1H$ and $x_2 \in g_2H$. Prove that $x_1x_2 \in g_1g_2H$.
%\\
%\\
Suppose that $H \subset G$ and that $H$ is a normal subgroup.  
\\
Suppose $x_1 \in g_1H$ and $x_2 \in g_2H$. $\implies x_1 = g_1h_1$ and $x_2 = g_2h_2$.
\\
Let $g_1, g_2 \in G$ and $h_1, h_2 \in H$.
\\
\begin{align*}
x_1x_2 &= (g_1h_1)(g_2h_2) &\text{by algebra}
\\
&= g_1h_1g_2h_2 &\text{by algebra}
\\
\text{Let\ } g = g_2 \text{\ and\ } h = h_1
\\
 g_1h_1g_2h_2 &= g_1hgh_2 &\text{substitution}
\\
&= g_1ghh_2 &\text{part b}
\\
\text{Let\ } g = g_2 \text{\ and\ } h = h_1
\\
&= g_1g_2h_1h_2 &\text{part b}
\\
&= g_1g_2H &h_1, h_2 \in H 
\end{align*}
Then, $x_1x_2 \in g_1g_2H$.

\end{enumerate}

\noindent\textbf{Exercise \ref{exercise:cosets:factor_cayley_prac}:}
%Give the Cayley tables for the following factor groups:
\begin{enumerate}[(a)]
\item
% ${\mathbb Z}/ 4 {\mathbb Z}$
\begin{center}
\begin{tabular}{c|cccc}
$\cdot$             & $0 + 4{\mathbb Z}$ & $1 + 4{\mathbb Z}$ & $2 + 4{\mathbb Z}$ & $3 + 4{\mathbb Z}$ \\\hline
$0 + 4{\mathbb Z}$ & $0 + 4{\mathbb Z}$ & $1 + 4{\mathbb Z}$ & $2 + 4{\mathbb Z}$ & $3 + 4{\mathbb Z}$\\
$1 + 4{\mathbb Z}$ & $1 + 4{\mathbb Z}$ & $2 + 4{\mathbb Z}$ & $3 + 4{\mathbb Z}$ & $0 + 4{\mathbb Z}$\\
$2 + 4{\mathbb Z}$ & $2 + 4{\mathbb Z}$ & $3 + 4{\mathbb Z}$ & $0 + 4{\mathbb Z}$ & $1 + 4{\mathbb Z}$\\
$3 + 4{\mathbb Z}$ & $3 + 4{\mathbb Z}$ & $0 + 4{\mathbb Z}$ & $1 + 4{\mathbb Z}$ & $2 + 4{\mathbb Z}$\\
\end{tabular}
\end{center}

\skipitems{1}
 
\item
% ${\mathbb Z}_{24} / \langle 8 \rangle$ 
% $ \langle 8 \rangle =  0, 8, 16$
\begin{center}
\begin{tabular}{c|cccccccc}
$\cdot$             & $\langle 8 \rangle$ & $1 + \langle 8 \rangle$ & $2 + \langle 8 \rangle$ & $3 + \langle 8 \rangle$ & $4 + \langle 8 \rangle$ & $5 + \langle 8 \rangle$ & $6 + \langle 8 \rangle$ & $7 + \langle 8 \rangle$\\\hline
$\langle 8 \rangle$  & $\langle 8 \rangle$ & $1 + \langle 8 \rangle$ & $2 + \langle 8 \rangle$ & $3 + \langle 8 \rangle$ & $4 + \langle 8 \rangle$ & $5 + \langle 8 \rangle$ & $6 + \langle 8 \rangle$ & $7 + \langle 8 \rangle$\\
$1 + \langle 8 \rangle$ & $1 + \langle 8 \rangle$ & $2 + \langle 8 \rangle$ & $3 + \langle 8 \rangle$ & $4 + \langle 8 \rangle$ & $5 + \langle 8 \rangle$ & $6 + \langle 8 \rangle$ & $7 + \langle 8 \rangle$ & $\langle 8 \rangle$ \\
$2 + \langle 8 \rangle$ & $2 + \langle 8 \rangle$ & $3 + \langle 8 \rangle$ & $4 + \langle 8 \rangle$ & $5 + \langle 8 \rangle$ & $6 + \langle 8 \rangle$ & $7 + \langle 8 \rangle$  & $\langle 8 \rangle$ & $1 + \langle 8 \rangle$ \\
$3 + \langle 8 \rangle$ & $3 + \langle 8 \rangle$ & $4 + \langle 8 \rangle$ & $5 + \langle 8 \rangle$ & $6 + \langle 8 \rangle$ & $7 + \langle 8 \rangle$ & $\langle 8 \rangle$ & $1 + \langle 8 \rangle$ & $2 + \langle 8 \rangle$ \\
$4 + \langle 8 \rangle$ & $4 + \langle 8 \rangle$ & $5 + \langle 8 \rangle$ & $6 + \langle 8 \rangle$ & $7 + \langle 8 \rangle$ &  $\langle 8 \rangle$ & $1 + \langle 8 \rangle$ & $2 + \langle 8 \rangle$ & $3 + \langle 8 \rangle$ \\
$5 + \langle 8 \rangle$ & $5 + \langle 8 \rangle$ & $6 + \langle 8 \rangle$ & $7 + \langle 8 \rangle$ & $\langle 8 \rangle$ & $1 + \langle 8 \rangle$ & $2 + \langle 8 \rangle$ & $3 + \langle 8 \rangle$ & $4 + \langle 8 \rangle$ \\
$6 + \langle 8 \rangle$ & $6 + \langle 8 \rangle$ & $7 + \langle 8 \rangle$ & $\langle 8 \rangle$ & $1 + \langle 8 \rangle$ & $2 + \langle 8 \rangle$ & $3 + \langle 8 \rangle$ & $4 + \langle 8 \rangle$ & $5 + \langle 8 \rangle$ \\
$7 + \langle 8 \rangle$ & $7 + \langle 8 \rangle$ & $\langle 8 \rangle$ & $1 + \langle 8 \rangle$ & $2 + \langle 8 \rangle$ & $3 + \langle 8 \rangle$ & $4 + \langle 8 \rangle$ & $5 + \langle 8 \rangle$ & $6 + \langle 8 \rangle$ 
\end{tabular}
\end{center}
\skipitems{3}
 
\item
%$U(8) / \langle 3 \rangle$
%\begin{gather*}
%1H = 3H = \{1, 3\}
%\\
%5H = 7H = \{5, 7\} 
%\end{gather*}
\begin{center}
\begin{tabular}{c|cc}
                         & & $5 \langle 3 \rangle$ \\
$\cdot$             & $\langle 3\rangle$ & $\langle 3 \rangle 5$ \\\hline
$\langle 3\rangle$ & $\langle 3 \rangle$ & $5 \langle 3 \rangle$ or $\langle 3 \rangle 5$ \\
$5 \langle 3 \rangle$ or $\langle 3 \rangle 5$ & $5 \langle 3 \rangle$ or $\langle 3 \rangle 5$ & $\langle 3\rangle$\\

\end{tabular}
\end{center}
\end{enumerate}

\noindent\textbf{Exercise \ref{exercise:cosets:cayley_dn_rn}:}
%Construct the Cayley table for  $D_n / R_n$. 
%\\
%\\
\begin{center}
\begin{tabular}{c|cc}
$\cdot$             & $R_n$ & $F_n$ \\\hline
$R_n$ & $R_n$ & $F_n$ \\
$F_n$ & $F_n$ & $R_n$\\

\end{tabular}
\end{center}

