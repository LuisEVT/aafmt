\chap{Appendix: Induction proofs--patterns and examples}{Induction}


\section{ Basic examples of induction proofs}
\label{sec:Induction:BasicInduction}

Below is a complete proof of the
formula for the sum of the first $n$ integers, that can serve as a model
for proofs of similar sum/product formulas. \footnote{This section was taken (with permission!) from A. J. Hildebrand's excellent notes on induction (reformatted and minor edits  by C.T.).}

\begin{prop}{Induction1} For all $n\in\NN$, the following equation (which we denote as $P(n)$)  is true:
\[
\sum_{i=1}^n i=\frac{n(n+1)}{2}.
\tag{$P(n)$}
\]
\end{prop}
\begin{proof} \emph{(By induction)}:

\noindent
\textbf{Base case:} When $n=1$, the left side of $P(n)$ is $1$, and the
right side is $1(1+1)/2=1$, so both sides are equal and $P(n)$ holds  
for $n=1$.

\noindent
\textbf{Induction step:} Let $k\in\NN$ be given and suppose 
formula $P(n)$ holds for $n=k$. Then
\begin{align*}
\sum_{i=1}^{k+1}i&=\sum_{i=1}^k i + (k+1)\quad\text{(by 
definition of $\sum$ notation)}
\\
&=\frac{k(k+1)}{2}+(k+1)\quad \text{(by induction hypothesis)}
\\
&=\frac{k(k+1)+2(k+1)}{2}\quad \text{(by algebra)}
\\
&=\frac{(k+1)((k+1)+1)}{2}\quad \text{(by algebra)}.
\end{align*}
Thus, $P(n)$ holds for $n=k+1$, and the proof of the induction step is complete. 

\noindent
\textbf{Conclusion:} By the principle of induction, 
we have proved that $P(n)$ holds for all $n\in\NN$.  
\end{proof}


\section{Advice on writing up induction proofs}
\label{sec:Induction:Advice}
Here are four things to keep in mind as you write up induction proofs.

\noindent
 \textbf{\#1: Begin any induction proof by stating precisely, 
and prominently, the
statement  you plan to prove.} 
This statement typically involves an equation (or assertion) in the variable $n$, and we're trying to prove this equation (or assertion) for all natural numbers $n$ bigger than a certain value.  A good idea is to write out the statement and label it as ``$P(n)$'', so that it's easy to spot, and easy to reference; see the sample proofs for
examples.

\noindent
\textbf{\#2: Be sure to properly begin and end the induction step.} 
From a logical point of view, an induction step is a proof of a statement of the
form, ``for all  $ k\in\NN, P(k)\Rightarrow P(k+1)$''. To prove such a statement,
you need to start out by asserting,``let $k\in \NN$ be
given''), then assume $P(k)$ is true (``Suppose $P(n)$ is true for $n=k$''), 
and, after a sequence of logical deductions, derive $P(k+1)$ (``Therefore
$P(n)$ is true for $n=k+1$''). 

\noindent
 \textbf{\#3: Use different letters for the general variable appearing in the
statement you seek to prove ($n$ in the above example)
and the variable used for the induction step 
($k$ in the above example).} 
The reason for this distinction is that in the induction step 
you want to be able to say something like the following:
``Let $k\in\NN$ be given, and suppose $P(k)$ .... 
[Proof of induction step goes here] ...  
Therefore $P(k+1)$ is true.''  Without introducing a second variable
$k$, such a statement wouldn't make sense.

\noindent
\textbf{\#4: Always clearly state, at the appropriate place in the induction step,
when the induction hypothesis is being used.} E.g., say
``By the induction hypthesis we have ...'', or use a parenthetical
note ``(by induction hypothesis)'' in a chain of equations as in the above
example.  The induction hypothesis is the case $n=k$ of the statement we seek
to prove (i.e., the statement ``$P(k)$'' and it is what you assume at the
start of the induction step.  The place where this hypothesis is used is the
most crucial step in an induction argument, and you must get this hypothesis into
play at some point during the proof of the induction step---if not, you are
doing something wrong. 


\section{Induction proof patterns \& practice problems}
\label{sec:Induction:ProofPatternsAndPractice}
 


%%%%%%%%%%%%%%%%%%%%%%%%%%%%%%%%%%%%%%%%%%%%%%%%%%%%%%%%%%%%%%%%%%%%%%%%%
% induction, basic examples
%%%%%%%%%%%%%%%%%%%%%%%%%%%%%%%%%%%%%%%%%%%%%%%%%%%%%%%%%%%%%%%%%%%%%%%%%

\subsection*{Induction proofs, type I: Sum/product formulas}
 
The most common, and the easiest, application of induction is to prove
formulas for sums or products of $n$ terms. Many of these  proofs follow
the same pattern. Here are  some examples of formulas that can be proved by induction:

\begin{enumerate}[(i)]
\item 
$\sum_{i=1}^n i(i+1)=\frac{n(n+1)(n+2)}{3}$ 
\item
$\sum_{i=0}^n i! i = (n+1)!-1$. 
\item
$\sum_{i=0}^n r^i=\frac{1-r^{n+1}}{1-r}$ ($r\not=1$)
(sum of finite geometric series)
\item 
$\sum_{i=0}^n 2^i=2^{n+1}-1$ (sum of powers of $2$)
\end{enumerate}

In the following exercises, we will guide you through the proofs of (i) and (ii). For parts (iii) and (iv), you're on your own!

\begin{exercise}{}
Fill in the blanks for the following induction proof of formula (i) above.

\begin{proof}
We seek to show that, for all $n\in\NN$,  
\[
\sum_{i=1}^n i(i+1)=\frac{n(n+1)(n+2)}{3}. 
\tag{ 	$P(n)$}
\]

\noindent
\textbf{Base case:} When $n=1$, the left side of $P(1)$ is equal to $\underline{~<1>~}$ ,
and the
right side is equal to $\underline{~<2>~}$ , so both sides are equal and $P(1)$ is
true.

\noindent
\textbf{Induction step:} Let $k\in\NN$ be given and suppose 
$P(k)$ is true. Then
\begin{align*}
\sum_{i=1}^{k+1}i(i+1)
&=
\sum_{i=1}^{k}i(i+1)
+\underline{~<3>~}
\\
&=\frac{k(k+1)(k+2)}{3} 
+\underline{~<4>~} 
\quad \text{(by induction hypothesis)}
\\
&=\frac{(k+1)(k+2)(k+3)}{3}.
\end{align*}
Thus, $P(\underline{~<5>~})$ holds, and the proof of the induction step is complete, 

\noindent
\textbf{Conclusion:}
By the principle of induction,  it follows that
$\underline{~<6>~}$  is true for all $n\in\NN$.  
\end{proof}

\end{exercise}

\begin{exercise}{}
Provide an outline for the proof of formula (ii) by responding to each of the parts below.

\begin{enumerate}[(a)]
\item
What is the equation that must be shown  for all $n\in\NN$?  (Call this equation ``$P(n)$'').
\item
Identify the base case, and show that equation $P(n)$  holds for the base case.
\item
Write the left-hand side of  $P( k+1)$.
\item
Separate off the last term in the sum, so that you have a sum from 1 to $k$ plus an additional term.
\item
Use the induction hypothesis to replace the sum from 1 to $k$ with a simpler expression.
\item
Use algebra to obtain $P(k+1)$, which completes the proof of the induction step. 
\item 
What is the final conclusion which can be drawn from the above argument?
\end{enumerate}

\end{exercise}

\begin{exercise}{}
\begin{enumerate}[(a)]
\item
Prove formula (iii) above using induction.
\item
Prove formula (iv) above using induction.
\end{enumerate}
\end{exercise}


%We seek to show that, for all $n\in\NN$,
%\[
%\sum_{i=0}^n i! i
%	= \underline{~~~~~~~~~~~}. 
%	{$*$}
%\]
%
%\textbf{Base case:} When $n=1$, the left side of ($*$) is equal to \underline{~~~~~},
%and the
%right side is equal to \underline{~~~~~}, so both sides are equal and ($*$) is
%true for $n=$\underline{~~~}.
%
%\textbf{Induction step:} Let $k\in\NN$ be given and suppose 
%($*$) is true for $n=$\underline{~~~~}. Then
%\begin{align*}
%\sum_{i=1}^{k+1}i\cdot i!
%&=
%\underline{~~~~~~~} + (k+1)(k+1)!
%\\
%&= \underline{~~~~~} + (k+1)(k+1)!
%\quad \text{(by induction hypothesis)}
%\\
%&=(k+1)!(k+2)- \underline{~~~}
%\\
%&=(k+2)!-1.
%\end{align*}
%Thus, (2) holds for $n=$\underline{~~~~~}, and the proof of the induction step is complete. 
%
%\textbf{Conclusion:} By the principle of induction, 
%($*$) is true for all $n\in\NN$.  
% {\hspace\fill$\square$\par\medskip}


\subsection*{Induction proofs, type II: inequalities} 
A second general type of application of induction is to prove inequalities
involving a  natural number $n$.   These proofs also tend to be on the
routine side; in fact, the algebra required is usually very minimal, in
contrast to some of the summation formulas.

In some cases the inequalities don't ``kick in'' until $n$ is 
large enough. By checking the first few values of $n$ one can
usually quickly determine the first  $n$-value, say $n_0$,
for which the inequality holds.  
Then one may use $n=n_0$ as the base case, instead of $n=0$.

Here are some examples of integer inequalities that can be proved using induction:

\noindent
\begin{enumerate}[(i)]

\item $2^n>n$ 
\item $2^n\ge n^2$ ($n\ge4$)
\item $n!> 2^n$ ($n\ge4$)
\item $(1-x)^n\ge 1-nx$ ($0<x<1$)
\item $(1+x)^n\ge 1+nx$ ($x>0$)

\end{enumerate}

In the following exercises, we will guide you through the proofs of (iii) and (iv). For the others, you'll have to wing it.

\begin{exercise}{}
Fill in the blanks in the following proof of (iii).

\begin{proof}

We are trying to show that 
\smallskip

\[
n!>2^n
\tag{$P(n)$}
\]
\smallskip

holds for all $n\ge4$.
(Note that  the  inequality fails for $n=1,2,3$. But this doesn't matter, because we only have to show that it works for for all $n$ from $4$ onwards.) 


\noindent
\textbf{Base case:} For $n=4$, the left and right sides of $P(4)$ are equal to
$\underline{~<1>~}$   and $\underline{~<2>~}$ , respectively, so $P(4)$ is true. 

\noindent
\textbf{Induction step:} Let $k\ge4$ be given and suppose 
$\underline{~<3>~}$  is true. Then
\begin{align*}
(k+1)!&=k! \cdot (k+1)
\\
&>2^k \cdot \underline{~<4>~}
\qquad (\text{by } \underline{~<5>~} )
\\
&\ge 2^k\cdot 2
\qquad \qquad \text{(since $k\ge4$ and so $k+1\ge2$))}
\\
&=\underline{~<6>~}.
\end{align*}
Thus, $\underline{~<7>~}$  holds, and the proof of the induction step is complete. 

\noindent
\textbf{Conclusion:} By the principle of induction, 
it follows that  $P(n)$ is true for all $n\ge4$.  
\end{proof}

\end{exercise}

\begin{exercise}{}
Provide an outline for the proof of the inequality (iv) by giving answers for each of the parts below.

\begin{proof}
\begin{enumerate}[(a)]
\item
What statement do you need to prove for every  real number $0<x<1$  
and any $n\in\NN$?  Call this statement ``$P(n)$''.
\item  \textbf{Base case:} Show that  the left and right sides of $P(n)$ are 
 equal in the base case.
\item
\textbf{Induction step:} Let $k\in\NN$ be given and suppose $P(k)$ is true
for any real number $0<x<1$.   What do we seek to show?
\item
Rewrite $(1-x)^{k+1}$ as $(1-x)^k \cdot (1-x)$. Then use $P(k)$ to obtain an inequality. Using basic algebra, simplify the right-hand side until you obtain a quantity that is greater than $1-(k+1)x$.
\item
What may you conclude about $P(k+1)$?
\item
\textbf{Conclusion:} What is the ultimate conclusion of the argument?
\end{enumerate}
\end{proof}
\end{exercise}

\begin{exercise}{}
\begin{enumerate}[(a)]
\item
Prove inequality (i) above using induction.
\item
Prove inequality (ii) above using induction.
\item
Prove inequality (v) above using induction.
\end{enumerate}
\end{exercise}

%We will prove that for any real number $0<x<1$  
%\[
%(1-x)^n\ge 1-nx.
%\tag{$*$}
%\]
%holds for any $n\in\NN$.
%
%\textbf{Base case:} For $n=$\underline{~~}, the left and right sides of ($*$) are 
%both equal to  \underline{~~~~}, so ($*$) holds.
%
%\textbf{Induction step:} Let $k\in\NN$ be given and suppose ($*$) is true
%for $n=$\underline{~~} and any real number $0<x<1$.   We seek to show that ($*$) holds
%for $n=$\underline{~~~~} and any real number $0<x<1$.
%
%Let $0<x<1$ be given. Then
%\begin{align*}
%(1-x)^{k+1}
%&=(1-x)^k \cdot \underline{~~~~}
%\\
%&\ge (1-kx) \cdot \underline{~~~~} 
%\quad\text{(by \underline{~~~~} since $0<x<1$, so $(1-x)>0$)}
%\\
%&=1-(k+1)x +\underline{~~~~}
%\quad\text{(by algebra)}
%\\
%&\ge 1-(k+1)x 
%\quad\text{(since \underline{~~~~~~~})}.
%\end{align*}
%Hence ($*$) holds for $n=$\underline{~~~~~}, and the proof of the induction step is
%complete.
%
%\textbf{Conclusion:} By the principle of induction, it follows that ($*$)
%holds for all $n\in\NN$.
%  {\hspace\fill$\square$\par\medskip}


\subsection*{Induction proofs, type III:
Extension of theorems from $2$ variables to $n$ variables}
Another very common and usually routine application of induction is to
extend general results that have been proved for the case of $2$
variables to the case of $n$ variables.  Below are some examples.
In proving these results, use the case $n=2$ as base case. To see how to
carry out the general induction step (from the case $n=k$ to
$n=k+1$), it may be  helpful to first try to see how get from the base
case $n=2$ to the next case $n=3$.
\medskip

\noindent
Here are some examples of mulitple-variable theorems that can be proved using induction:
\begin{enumerate}[(i)]

\item{} 
% [Problem 3.15]
Show that if $x_1,\dots,x_n$ are odd, then $x_1x_2\dots x_n$ is odd.

\item{}
Show that if $a_i$ and $b_i$ ($i=1,2,\dots,n$) are real numbers such that 
$a_i\le b_i$ for all $i$, then 
\[
\sum_{i=1}^na_i\le \sum_{i=1}^nb_i. 
\]



\item
Show that if $x_1,\dots,x_n$ are real numbers, then
\[
\left|\sin\left(\sum_{i=1}^n x_i\right)\right|
\le \sum_{i=1}^n\left|\sin x_i\right|.
\]
(Use the trig identity for $\sin(\alpha+\beta)$.)  

\item{}
Show that if $A_1,\dots, A_n$ are sets, then 
\[
\left(A_1\cup \dots \cup A_n\right)^c
=A_1^c\cap\dots \cap A_n^c.
\]
(This is a generalization of De Morgan's Law to unions of $n$ sets.
Use De Morgan's Law for two sets ($(A\cup B)^c=A^c\cap B^c$) and induction
to prove this result.)
\end{enumerate}

We'll give outlines of the proofs of (i) and (ii).

\begin{exercise}{}
This exercise will provide a proof of (i).
\begin{enumerate}[(a)]
\item
We will need the following assertion in the proof:

\[ \text{If }x \text{ and } y \text{ are odd, then } xy \text{ is also odd.}\]

We know that $x$ is odd if and only if mod($x$,2)=1.  Use this and facts from modular arithmetic to prove the needed assertion.
\item
Fill in the blanks in the following proof of (i).

\begin{proof}
We will prove by induction on $n$ the following statement: 
\[
\tag{$P(n)$}
\text{ If $x_1,\dots,x_n$ are odd numbers, then 
$x_1x_2\dots x_n$ is odd.}
\]

\noindent
\textbf{Base case:} For $n=1$, the product $x_1\dots x_n$ reduces to 
$\underline{~<1>~}$, which  is odd whenever $x_1$ is odd. 
Hence $P(n)$ is true for $n=1$.

\noindent
\textbf{Induction step.}
\begin{itemize}
\item Let $k\ge 1$, and suppose ($*$) is true for $n=k$, i.e.,
suppose that any product of $\underline{~<2>~}$  odd numbers is again odd.
\item
We seek to show that $\underline{~<3>~}$ is true, i.e., that any product of $\underline{~<4>~}$
odd numbers is odd. 
\item
Let $x_1,\dots,x_{k+1}$ be odd numbers. 
\item
Applying the induction hypothesis to $x_1,\dots,x_k$, we obtain that the
product $\underline{~<5>~}$ is odd. 
\item
Since $x_{k+1}$ is $\underline{~<6>~}$  and, by part (a) the product of two odd numbers is
again odd, it follows that $x_1x_2\dots x_{k+1} = (x_1\dots x_k)x_{k+1}$
is odd.
\item
As $x_1,\dots,x_{k+1}$ were arbitrary odd numbers, we have proved
$\underline{~<7>~}$, so the induction step is complete.
\end{itemize}

\noindent
\textbf{Conclusion:} By the principle of induction, it follows that 
$P(n)$ is true for all $n\in\NN$.
\end{proof}
\end{enumerate}
\end{exercise}

\begin{exercise}{}
Complete an outline of a proof of (ii) by responding to the following items.
\begin{enumerate}[(a)]
\item
What statement do we want to prove  for all natural numbers $n$ and for all real numbers $a_i$ and $b_i$ ($i=1,\dots,n$)
such that $a_i\le b_i$? Call this statement ``P(n)''.
(Note that the condition ``for all real numbers $a_i$ and $b_i$''
must be part of the induction statement we seek to prove.)
\item
\textbf{Base case:} Show that $P(1)$ is true.
\item
\textbf{Induction step:}  Let $k\ge 1$. Write $P(k)$.
\item
We seek to prove that $P(k)$ implies $P(k+1)$.  We may rewrite $P(k+1)$ as follows (fill in the blanks):
Let $a_1,\dots,a_{k+1}$ and $b_1,\dots,b_{k+1}$ be given real numbers
such that \underline{~~~~~} for each $i$.
Then
\begin{align*}
\sum_{i=1}^{k+1}a_i
&=\underline{~~~~~} + a_{k+1}.
\end{align*}
\item
Assuming that $P(k)$ is true, use  Proposition~\ref{proposition:BeforeWeBegin:sumIneq} to show that $P(k+1)$ is also true.  This is equivalent to showing that $P(k)$ implies $P(k+1)$.
\item 
\textbf{Conclusion:} What is the final conclusion?
\end{enumerate}
\end{exercise}


%We will prove by induction on $n$ the following statement: 
%
%\[
%\tag*{$P(n)$:}
%\parbox{5in}{\slshape For all real numbers $a_i$ and $b_i$ ($i=1,\dots,n$)
%such that $a_i\le b_i$ for all $i$ we
%have}
%\]
%\[
%\sum_{i=1}^na_i\le \underline{~~~~~~~}. 
%\tag{$*$}
%\]
%(Note that the condition ``for all real numbers $a_i$ and $b_i$''
%must be part of the induction statement we seek to prove.)
%
%\textbf{Base case:} For $n=1$, the 
%left and right sides are \underline{~~~} and \underline{~~~~}, respectively, and the
%inequality ($*$) therefore follows from our hypothesis that
%$a_i\le b_i$ for all $i=1,\dots,n$.  Hence \underline{~~~~} is true.
%
%\textbf{Induction step:}
%\begin{itemize}
%\item Let $k\ge 1$, and suppose $P(k)$ is true, i.e.,
%suppose that 
%for $n=$\underline{~~~} and any choice of real numbers 
%$a_1,\dots,a_k$ and $b_1,\dots,b_k$ 
%satisfying \underline{~~~~} for each $i$, the inequality ($*$) holds.
%\item
%We seek to show that \underline{~~~} is true, i.e., that 
%for $n=$\underline{~~~~}
%any choice of real numbers $a_1,\dots,a_{k+1}$ and $b_1,\dots,b_{k+1}$
%satisfying \underline{~~~~~} for each $i$, the inequality ($*$) holds.
%\item
%Let $a_1,\dots,a_{k+1}$ and $b_1,\dots,b_{k+1}$ be given real numbers
%such that \underline{~~~~~} for each $i$.
%\item Then
%\begin{align*}
%\sum_{i=1}^{k+1}a_i
%&=\underline{~~~~~} + a_{k+1}
%\\
%&\le \underline{~~~~~~} + a_{k+1}
%\quad \text{(by induction hypothesis applied to $a_1,\dots a_k$)}
%\\
%&\le \sum_{i=1}^{k}b_i + \underline{~~~~~}
%\quad \text{(by assumption $a_{k+1}\le b_{k+1}$)}
%\\
%&=\sum_{i=1}^{k+1}b_i.
%\end{align*}
%\item 
%Thus, ($*$) holds for $n=$\underline{~~~~} and the given numbers $a_1,\dots,a_{k+1}$ 
%and $b_1,\dots,b_{k+1}$.
%\item Since the $a_1,\dots,a_{k+1}$ 
%and $b_1,\dots,b_{k+1}$ were arbitrary
%real numbers satisfying $a_i\le b_i$ for each $i$,
%we have obtained statement $P(k+1)$, 
%and the proof of the induction step is complete.
%\end{itemize}
%
%
%\textbf{Conclusion:} By the principle of induction, 
%it follows that $P(n)$ is true for all $n\in\NN$.  
%  {\hspace\fill$\square$\par\medskip}


%} 
%\longsol{
%We seek to prove by induction on $n$ the following statement: 
%\[
%\tag*{$P(n)$:}
%\parbox{5in}{\slshape For all real numbers $x_1,\dots,x_n$ 
%we have}
%\]
%\[
%\left|\sin\left(\sum_{i=1}^n x_i\right)\right|
%\le \sum_{i=1}^n\left|\sin x_i\right|.
%\tag{$*$}
%\]
%
%The key to the argument is the trig identity
%\[
%\sin(\alpha+\beta)=\sin\alpha \cos \beta+\sin\beta\cos\alpha,
%\]
%which is valid for any real $\alpha$ and $\beta$. Since $|\cos x|\le 1$,
%this identity implies, via the triangle inequality,
%\begin{align*}
%\tag{$**$}
%|\sin(\alpha+\beta)|
%&\le|\sin\alpha \cos \beta|+|\sin\beta\cos\alpha|
%\\
%&\le|\sin\alpha|+|\sin\beta|.
%\end{align*}
%The inequality $(**)$ is the case $n=2$ of the statement $(*)$ 
%we seek to prove, and will be needed in the induction proof. (One could
%also use it as the base case of an induction proof that starts with $n=2$, 
%but it is easier to start the induction with $n=1$, where the base case
%is trivial.)
%
%\bigskip
%
%
%\textbf{Base case:} For $n=1$, the 
%left and right sides of ($*$) are both equal to $|\sin x_1|$, so  
%($*$) holds trivially in this case. Hence $P(1)$ is true.
%
%\textbf{Induction step:}
%\begin{itemize}
%\item Let $k\ge 1$, and suppose $P(k)$ is true, i.e.,
%suppose that ($*$) holds for $n=k$ and any choice of real numbers 
%$x_1,\dots,x_k$.
%\item
%We seek to show that $P(k+1)$ is true, i.e., that for
%any choice of real numbers $x_1,\dots,x_{k+1}$
%the inequality ($*$) holds. 
%\item
%Let $x_1,\dots,x_{k+1}$  be given real numbers. 
%\item Then
%\begin{align*}
%\left|\sin\left(\sum_{i=1}^{k+1} x_i\right)\right|
%&=\left|\sin\left(\left(\sum_{i=1}^{k} x_i\right) + x_{k+1}\right)\right|
%\\
%&\le \left|\sin\left(\sum_{i=1}^{k} x_i\right)\right|
%+ \left|\sin x_{k+1}\right|
%\quad \text{(by ($**$)
%with $\alpha=\sum_{i=1}^k x_i$ and $\beta=x_{k+1}$)}
%\\
%&\le \sum_{i=1}^k\left|\sin x_i\right|
%+ \left|\sin x_{k+1}\right|
%\quad \text{(by induction hypothesis applied to $x_1,\dots, x_k$)}
%\\
%&=\sum_{i=1}^{k+1}\left|\sin x_i\right|.
%\end{align*}
%\item 
%Thus, ($*$) holds for $n=k+1$ and the given numbers $x_1,\dots,x_{k+1}$.
%\item
%Since the $x_1,\dots,x_{k+1}$ were
%arbitrary real numbers, we have obtained statement $P(k+1)$, and 
%proof of the induction step is complete.
%\end{itemize}
%
%\textbf{Conclusion:} By the principle of induction, 
%it follows that $P(n)$ is true for all $n\in\NN$.  
%} 

%%%%%%%%%%%%%%%%%%%%%%%%%%%%%%%%%%%%%%%%%%%%%%%%%%%%%%%%%%%%%%%%%%%%%%%%%
%\longsol{
%We seek to prove by induction on $n$ the following statement: 
%\[
%\tag*{$P(n)$:}
%\parbox{5in}{\slshape For all sets $A_1,\dots,A_n$  
%we have}
%\]
%\[
%\left(A_1\cup \dots \cup A_n\right)^c
%=A_1^c\cap\dots \cap A_n^c.
%\tag{$*$}
%\]
%
%The key to the argument is two set version of De Morgan's Law:
%\[
%(A\cup B)^c = A^c\cap B^c,
%\tag{$**$}
%\]
%which holds for any sets $A$ and $B$.
%
%
%\textbf{Base case:} For $n=1$, the 
%left and right sides of ($*$) are both equal to $A_1^c$,
%so ($*$) holds trivially in this case. Hence $P(1)$ is true.
%\iffalse
%Though not necessary, we can also easily verify the next case,
%$n=2$: In this case, the left and right sides of ($*$) are $(A_1\cup A_2)^c$
%and $A_1^c\cap A_2^c$, respectively, so the identity is just the two set
%version of De Morgan's Law, i.e., ($**$) with $A=A_1$ and $B=A_2$.
%\fi
%
%\textbf{Induction step:} 
%\begin{itemize}
%\item Let $k\ge 1$, and suppose $P(k)$ is true, i.e.,
%suppose that ($*$) holds for $n=k$ and any sets $A_1,\dots,A_k$.
%\item
%We seek to show that $P(k+1)$ is true, i.e., that for
%any sets $A_1,\dots,A_{k+1}$, ($*$) holds. 
%\item
%Let $A_1,\dots,A_{k+1}$  be given sets.
%\item
%Then
%\begin{align*}
%\left(A_1\cup \dots \cup A_{k+1}\right)^c
%&=\left(\left(A_1\cup \dots \cup A_{k}\right)\cup A_{k+1}\right)^c
%\\
%&=\left(A_1\cup \dots \cup A_{k}\right)^c\cap A_{k+1}^c
%\quad \text{(by ($**$) with $A=(A_1\cup \dots \cup A_k)$ and 
%$B=A_{k+1}$)}
%\\
%&=\left(A_1^c\cap\dots \cap A_k^c\right)\cap A_{k+1}^c
%\quad \text{(by induction hypothesis applied to $A_1,\dots, A_k$)}
%\\
%&=A_1^c\cap\dots \cap A_k^c\cap A_{k+1}^c.
%\end{align*}
%\item
%Thus, ($*$) holds for $n=k+1$
%and the given sets $A_1,\dots,A_{k+1}$.
%\item 
%Since the $A_1,\dots,A_{k+1}$ were
%arbitrary sets, we have obtained statement $P(k+1)$, and the 
%proof of the induction step is complete.
%\end{itemize}
%
%\textbf{Conclusion:} By the principle of induction, 
%it follows that $P(n)$ is true for all $n\in\NN$.  
%} 



\section{Strong Induction, with applications}
\label{sec:Induction:StrongInduction}


One of the most common applications of induction is to problems involving
recurrence sequences such as the Fibonacci numbers, and to representation
problems such as the representation of integers as a product of primes
(Fundamental Theorem of Arithmetic), sums of powers of $2$ (binary
representation), and sums of stamp denominations (postage stamp problem).

In applications of this type, the case $n=k$ in the induction step is
not enough to deduce the case $n=k+1$; one usually needs additional
predecessors predecessors to get the induction step to work, e.g., 
the two preceding cases $n=k$ and $n=k-1$, or 
\emph{all} preceding cases
$n=k,k-1,\dots,1$.  This variation of the induction
method is called \textbf{strong induction}.  The induction principle
remains valid in this modified form. 


\subsection*{Strong induction and recurrences}
In the induction proofs we've looked at so far, we first had to prove a base case, and then used a preceding case ($n=k$) to prove the case $n=k+1$ in the induction step. But when we aply  induction to two-term recurrence sequences like the Fibonacci
numbers, we'll need \emph{two} preceding cases, $n=k$ and $n=k-1$, in
the induction step, and \emph{two} base cases (e.g., $n=1$ and $n=2$) to get
the induction going.  The logical structure of such a proof is of the
following form:

\noindent
\textbf{Base step:}  $P(n)$ is true for $n=1,2$.
\smallskip

\noindent
\textbf{Induction step:} Let $k\in\NN$ with $k\ge 2$ be given
and assume $P(n)$ holds for $n=k$ and $n=k-1$.
\medskip

[... Work goes here ...] 
\medskip

\noindent
Therefore $P(k+1)$ holds.
\smallskip

\noindent
\textbf{Conclusion:} By the principle of strong induction, 
$P(n)$ holds for all $n\in\NN$.
\smallskip

\noindent
Note that in the induction step, one could also say ``Assume
$P(n)$ holds for `` $n=1,2,\dots, k$''; this is a bit redundant as only
the last two of the cases $n=1,2,\dots,k$ are needed, though logically
correct. 

Here is a worked-out example of a proof by strong induction.


\begin{prop}{}
Let $a_n$ be the sequence defined by 
$a_1=1$, $a_2=8$, and $a_n=a_{n-1}+2a_{n-2}$ for $n\ge3$.
Then
$a_n=3\cdot 2^{n-1}+2(-1)^n$ for all $n\in\NN$.
\end{prop}

\begin{proof}
We'll prove by strong induction that, for all $n\in\NN$, 
\[
\tag{$P(n)$}
a_n=3\cdot 2^{n-1}+2(-1)^n.
\]

\textbf{Base case:} 
When $n=1$, the left side of $P(1)$ is $a_1 =1$, and
the right side is $3\cdot 2^{0}+2\cdot(-1)^1=1$,
so both sides are equal and $P(1)$ is
true. 

When $n=2$, the left and right sides of $P(2)$ are
$a_2=8$ and $3\cdot 2^1+2\cdot(-1)^2=8$, so $P(2)$ also holds.


\textbf{Induction step:} Let $k\in\NN$ with $k\ge2$ be given and suppose 
$P(n)$ is true for $n=1,2,\dots,k$. Then
\begin{align*}
a_{k+1}&=a_{k}+2a_{k-1}
\quad \text{(by recurrence for $a_n$)}
\\
&=
3\cdot 2^{k-1}+2\cdot(-1)^k
+2\left(3\cdot 2^{k-2}+2\cdot(-1)^{k-1}\right)
\quad \text{(by $P(k)$ and $P(k-1)$)}
\\
&=
3\cdot \left(2^{k-1}+2^{k-1}\right) 
+2\left((-1)^k+2(-1)^{k-1}\right)
\quad \text{(by algebra)}
\\
&=
3\cdot 2^k+2(-1)^{k+1}
\quad \text{(more algebra).}
\end{align*}
Thus, $P(k+1)$) holds, and the proof of the induction step is complete. 

\textbf{Conclusion:} By the strong induction principle,  it follows that
$P(n)$ is true for all $n\in\NN$.  
\end{proof}

\subsection*{Strong Induction and representation problems}
For applications to representation problems one 
typically requires the induction hypothesis in its strongest possible
form, where one assumes \emph{all} preceding
cases (i.e., for $n=1,2,\dots,k$) instead of just the immediate
predecessor (as in simple induction) or two predecessors (as in strong induction applied to two-term recurrences). 

Below is a classic example of this type, a proof that every integer
$\ge2$ can be written as a product of prime numbers. This is the
existence part of what is called the Fundamental Theorem of Arithmetic;
the other part guarantees uniquess of the representation, which we will
not be concerned with here (it can also be proved by induction, but the proof is a little more complicated).  

Recall the definition of \term*{prime} from Chapter~\ref{ComplexNumbers}: an integer $n>1$ is called 
\term*{prime} if it has no factor greater than 1 other than itself. An integer $n>1$ that is not prime is called
\term*{composite}\index{Composite number!definition}: in other words, $n$ can be written as $n=ab$
with integers $a,b$ satisfying $2\le a,b<n$. Using these definitions, we may now state and prove:

\begin{prop}{}\emph{(Fundamental Theorem of Arithmetic: existence)}

Any integer $n\ge 2$
is either a prime or can be represented as a product of (not
necessarily distinct) primes, i.e., in the form $n=p_1p_2\dots p_r$,
where the $p_i$ are primes.
\end{prop}

\begin{proof}
We will prove by strong induction that the following statement  
holds for all integers $n\ge2$.
\[
\text{$n$ can be represented as a product of one or more primes.}
\tag{$P(n)$}
\]

\noindent
\textbf{Base case:} The integer $n=2$ is a prime since it cannot be
written as a product $ab$, with integers $a,b\ge 2$,  so $P(n)$ holds
for $n=2$.

\noindent
\textbf{Induction step:} 
\begin{itemize}

\item Let $k\ge2$ be given and suppose $P(n)$ is true for all 
integers $2\le n\le k$, i.e., suppose that all such $n$ can be represented 
as a product of one or more primes.
\item We seek to show that $k+1$ also has a
representation of this form.
\item
If $k+1$ itself is prime, then $P(n)$ holds for $n=k+1$, and we are done.
\item
Now consider the case when $k+1$ is composite.
\item 
By definition, this means that $k+1$
can be written in the form $k+1=ab$, where $a$ and $b$ are integers satisfying
$2\le a,b< k+1$, i.e., $2\le a,b\le k$. 
\item
Since $2\le a,b\le k$, the induction hypothesis can be applied to $a$ and $b$
and shows that $a$ and $b$ can be represented as products
of one or more primes.
\item
Multiplying these two representations gives a representation of $k+1$ as
a product of primes. 
\item
Hence $k+1$ has a representation of the desired form, so $P(n)$ holds
for $n=k+1$, and the induction step is
complete.
\end{itemize}

\noindent
\textbf{Conclusion:} By the strong induction principle, 
it follows that $P(n)$  is true for all $n\ge2$, i.e., every integer
$n\ge2$ is either a prime or can be represented as a product of primes.
\end{proof}

\section{More advice on induction and strong induction proofs}
\label{sec:Induction:MoreAdviceAndStrongInduction}

\noindent
\term{Should I use ordinary induction or strong induction?}
With some standard types of problems (e.g., sum formulas) 
it is clear ahead of time what type of induction is \emph{likely} to be
required, but usually this question answers itself during 
the exploratory/scratch phase of the argument. In the induction step you
will need to reach the $k+1$ case, and you should ask yourself which of
the previous cases you need to get there. If all you need to prove the
$k+1$ case is the case $k$ of the statement, then ordinary induction is
appropriate. If two preceding cases, $k-1$ and $k$, are necessary to get
to $k+1$, then (a weak form of) strong induction is appropriate. If one
needs the full range of preceding cases (i.e., all cases
$n=1,2,\dots,k$), then the full force of strong induction is needed.


\noindent
\term{How many base cases are needed?} 
The number of base cases to be checked depends on how far back one needs to
``look'' in the induction step. In standard induction proofs (e.g., for summation
formulas) the induction step requires only the immediately preceding case (i.e., the
case $n=k$), so a single base case is enough to start the induction.   
\begin{itemize}
\item 
For Fibonacci-type problems, 
the induction step usually requires the result for the two preceding cases,
$n=k$ and $n=k-1$. To get the induction started, one therefore needs to know 
the result for two consecutive cases, e.g., $n=1$ and $n=2$. 
\item 
In postage stamp type problems, getting the result for
$n=k+1$  might require knowing the result for $n=k-2$ and  
$n=k-6$, say. This amounts to ``looking back'' $7$ steps  (namely
$n=k,k-1,\dots,k-6$), so $7$ consecutive cases are needed to get the
induction started.
\item 
On the other hand, in problems involving the full strength of the strong
induction hypothesis (i.e., if in the induction step one needs to assume the
result for \emph{all} preceding cases $n=k,k-1,\dots,1$),  a single base
case may be sufficient.  An example is the Fundamental Theorem of
Arithmetic.
\end{itemize}

\noindent
 \term{How do I write the induction step?}
As in the case of ordinary induction, at the beginning of the 
induction step \emph{state precisely 
what you are assuming, including any constraints on the induction variable
$k$}.  Without an explicitly stated assumption, the argument is incomplete.  The
appropriate induction hypothesis depends on the nature of the problem and the
type of induction used.  Here are some common ways to start out an induction
step:
\begin{itemize}
\item ``Let $k\in \NN$ be given and assume $P(k)$ is true.''    
(typical form for standard induction proofs)
\item ``Let $k\ge 2$ be given and assume $P(n)$ holds for $n=k-1$ and $n=k$.''
(typical form for induction involving recurrences)
\item ``Let $k\in \NN$ be given and assume $P(n)$ holds for $n=1,2,\dots,k$.''
(typical form for representation problems)
\end{itemize}

\section{Common mistakes}
\label{sec:Induction:CommonMistakes}
 The following examples illustrate
some common mistakes in setting up base case(s) and the induction step.

\begin{itemize}

\item[]  \textbf{Example 1.}
\begin{itemize}
\item \textbf{Base step:} $n=3$.
\item \textbf{Induction step:} Let $k\in \NN$ with $k\ge 3$ be given and
assume $P(n)$ is true for $n=k$ and $n=k-1$. 
\item \textbf{Comment:}
\textbf{BAD:} When $k=3$ (the first case of the induction step), the induction 
step requires the cases $3$ and $2$, but only $2$ is covered in the base
step.\\
\textbf{FIX:} Add the case $n=2$ to the base step. 
\end{itemize}
\item[]   \textbf{Example 2.}
\begin{itemize}
\item \textbf{Base step:} $n=1$ and $n=2$.
\item \textbf{Induction step:} Let $k\in \NN$ with $k>2$ be given and
assume $P(n)$ is true for $n=k$ and $n=k-1$. 
\item \textbf{Comment:}
\textbf{BAD.} Gap between base case and the first case  
of the induction step: The first case $k=3$ of the induction step
requires the cases $3$ and $2$, but the base step only gives the cases
$1$ and $2$.\\
\textbf{FIX:} Start induction step at $k=2$ rather than $k=3$:
``Let $k\in\NN$ with $k\ge 2$ be given \dots''
\end{itemize}

\item[]   \textbf{Example 3.}
\begin{itemize}
\item \textbf{Base step:} $n=1$ and $n=2$.
\item \textbf{Induction step:} Assume 
$P(n)$ is true for $n=k$ and $n=k-1$.   Then ...
\item \textbf{Comment:} \textbf{BAD.} The variable $k$ in the  induction
step is not quantified.\\
\textbf{FIX:} Add ``Let $k\in\NN$ with $k\ge2$ be given.''
\end{itemize}

\item[]   \textbf{Example 4.}
\begin{itemize}
\item \textbf{Base step:} $n=1$ and $n=2$.
\item \textbf{Induction step:} Let $k\in \NN$ be given and
assume $P(n)$ is true for $n=k$ and $n=k-1$.
\item \textbf{Comment:} 
\textbf{BAD.}  Here the first case induction step is $k=1$, with the induction
hypothesis being the cases $n=k$ and $n=k-1$. But when $k=1$, the second of
these cases, $n=k-1=0$, is out of range.\\
\textbf{FIX:} Add the restriction $k\ge2$ to the induction step:
``Let $k\in\NN$ with $k\ge2$ be given.''

\end{itemize}

\end{itemize}




\section{Strong induction practice problems}
\label{sec:Induction:StrongInductionPractice} 


\begin{enumerate}


%%%%%%%%%%%%%%%%%%%%%%%%%%%%%%%%%%%%%%%%%%%%%%%%%%%%%%%%%%%%%%%%%%%%%%%%%
% recurrence sequences 
%%%%%%%%%%%%%%%%%%%%%%%%%%%%%%%%%%%%%%%%%%%%%%%%%%%%%%%%%%%%%%%%%%%%%%%%%


\item \textbf{Recurrences:} 
The first  few problems deal with properties of the Fibonacci sequence
and related recurrence sequences.
The Fibonacci sequence is defined by $F_1=1$, $F_2=1$, and
$F_n=F_{n-1}+F_{n-2}$ for $n\ge 3$. Its first few terms are
$1,1,2,3,5,8,13,21,34,55,89,144,\dots$.

In the following problems, use an appropriate form of
induction (standard induction or strong induction) 
to establish the desired properties and formulas. (Note that
some of these problems require only ordinary induction.)

\begin{enumerate}

%%%%%%%%%%%%%%%%%%%%%%%%%%%%%%%%%%%%%%%%%%%%%%%%%%%%%%%%%%%%%%%%%%%%%%%%%
%%%%%%%%%%%%%%%%%%%%%%%%%%%%%%%%%%%%%%%%%%%%%%%%%%%%%%%%%%%%%%%%%%%%%%%%%
\item 
\textbf{Fibonacci sums:}
Prove that $\sum_{i=1}^n F_i=F_{n+2}-1$ for all $n\in\NN$.

\item 
\textbf{Fibonacci matrix:}
\newcommand{\mat}[4]{\begin{pmatrix}#1 & #2 \\ #3 & #4\end{pmatrix}}
Show that, for all $n\in\NN$, 
\[
\tag{$P(n)$}
\mat 1110^n
=\mat
{F_{n+1}}{F_n}
{F_n}{F_{n-1}}.
\]


\item \textbf{Odd/even Fibonacci numbers:}
Prove that the Fibonacci numbers follow the pattern odd,odd,even:
that is, show that for any positive integer $m$, 
$F_{3m-2}$ and $F_{3m-1}$ are odd and $F_{3m}$ is even.


\item
\textbf{Inequalities for recurrence sequences:}
Let the sequence $T_n$ (``Tribonacci sequence'')
be defined by $T_1=T_2=T_3=1$ and $T_n=T_{n-1}+T_{n-2}+T_{n-3}$ for
$n\ge4$.  Prove that  
\[
T_n<2^n
\tag{$P(n)$}
\]
holds for all $n\in\NN$.


We'll give an outline for the proof of (d). 

%\longsol{
%We seek to show that, for all $n\in\NN$, 
%\[
%\tag{$*$}
%\sum_{i=1}^n F_i=F_{n+2}-1.
%\]
%
%\textbf{Base case:} When $n=1$, the left side of ($*$) is $F_1 =1$,
%and the
%right side is $F_3-1=2-1=1$, so both sides are equal and ($*$) is
%true for $n=1$.
%
%\textbf{Induction step:} Let $k\in\NN$ be given and suppose 
%($*$) is true for $n=k$. Then
%\begin{align*}
%\sum_{i=1}^{k+1}F_i
%&=
%\sum_{i=1}^{k}F_i
%+F_{k+1}
%\\
%&=F_{k+2}-1 + F_{k+1}
%\quad \text{(by ind. hyp. $(*)$ with $n=k$)}
%\\
%&=F_{k+3}-1 
%\quad \text{(by recurrence for $F_n$)}
%\end{align*}
%Thus, ($*$) holds for $n=k+1$, and the proof of the induction step is complete. 
%
%\textbf{Conclusion:} By the principle of induction,  it follows that
%($*$) is true for all $n\in\NN$.  
%
%\bigskip
%
%\textbf{Remark:} Here standard induction was sufficient, since we were
%able to relate the $n=k+1$ case directly to the $n=k$ case, in the
%same way as in the induction proofs for summation formulas like 
%$\sum_{i=1}^n i=n(n+1)/2$.  Hence, a single base case was sufficient.
%}

%(For convenience, we define $F_0=0$; with this definition, the recurrence
%relation $F_n=F_{n-1}+F_{n-2}$ holds for all $n\ge2$ and the above matrix
%is well defined for all $n\ge1$.)
%
%\textbf{Base case:} When $n=1$, 
%the four entries of the matrix on the right are $F_2=1$, $F_1=1$, $F_1=1$, 
%and $F_0=0$, so ($*$) holds in this case. 
%
%\textbf{Induction step:} Let $k\in\NN$ be given and suppose 
%($*$) holds for $n=k$. Then
%\begin{align*}
%\mat 1110^{k+1}
%&=\mat 1110^k\mat1110
%\\
%&=\mat{F_{k+1}}{F_k}
%{F_k}{F_{k-1}}\mat1110
%\quad\text{(by $(*)$ with $n=k$)}
%\\
%&=\mat{F_{k+1}+F_k}{F_{k+1}}{F_{k}+F_{k-1}}{F_k}
%\quad \text{(by matrix multiplication)}
%\\
%&=\mat{F_{k+2}}{F_{k+1}}{F_{k+1}}{F_k}
%\quad \text{(by recurrence for $F_n$)}.
%\end{align*}
%Thus, ($*$) holds for $n=k+1$, and the proof of the induction step is complete. 
%
%\textbf{Conclusion:} By the principle of induction,  it follows that
%($*$) is true for all $n\in\NN$.  
%}
%\longsol{
%We will use induction to show that the following statement 
%holds for all $m\in\NN$:
%\[
%\tag*{$P(m):$}
%\text{
%$F_{3m-2}$ and $F_{3m-1}$ are odd,
%and $F_{3m}$ is even.}
%\]
%
%\textbf{Base case:} When $m=1$, the three Fibonacci numbers appearing in 
%$P(m)$ are $F_1=1$, $F_2=1$, and $F_3=2$, and thus are of the required
%parity. Hence $P(1)$ is true.
%
%\textbf{Induction step:} Let $k\in\NN$ be given and suppose 
%$P(m)$ is true for $m=k$. Thus, $F_{3k-2}$ and $F_{3k-1}$ are odd and 
%$F_{3k}$ is even.
%
%By the recurrence for $F_n$, we have $F_{3k+1}=F_{3k}+F_{3k-1}$. Hence
%$F_{3k+1}$ is the sum of an even number $F_{3k}$, and an odd number,
%$F_{3k-1}$, and therefore odd. 
%
%Similarly, $F_{3k+2}=F_{3k+1}+F_{3k}$, so $F_{3k+2}$ is the sum of an odd
%number, $F_{3k+1}$, and an even number, $F_{3k}$, and hence odd.
%
%Finally, $F_{3k+3}=F_{3k+2}+F_{3k+1}$, so $F_{3k+3}$ is the sum of two odd
%numbers and hence even. 
%
%Altogether, we have shown that, of the three numbers
%$F_{3k+1},F_{3k+2},F_{3k+3}$, the first two are odd and the last one is
%even. Thus $P(m)$ holds for $m=k+1$, and the proof of the 
%induction step is complete. 
%
%\textbf{Conclusion:} By the principle of induction,  it follows that
%$P(m)$ is true for all $m\in\NN$.  
%
%\bigskip
%
%\textbf{Alternative argument:} The above proof lumps together groups of
%three consecutive Fibonacci numbers and establishes the desired parity
%properties simultaneously for all three numbers.  
%Alternatively, one can treat the sequences $F_{3m-2}$, $F_{3m-1}$, and
%$F_{3m}$ separately using  the following identity, valid for 
%all $n\ge4$:
%\[
%F_{n}=F_{n-1}+F_{n-2}=(F_{n-2}+F_{n-3})+F_{n-2}=2F_{n-2}+F_{n-3}.
%\]
%It follows from this identity that if $F_{n-3}$ is odd, then so is $F_{n}$,
%and if $F_{n-3}$ is even, then so is $F_n$.  Using 
%induction with $n=1$ as the base case then shows that the numbers
%$F_1=1,F_4,F_7,\dots$ are all odd.  With $n=2$ as base case one gets that 
%$F_2=1,F_5,F_8,\dots$ are all odd, and taking $n=3$ as base case 
%shows that $F_3=2,F_6,F_9,\dots$ are all even.
%}


We will prove $P(n)$ by strong induction.

\textbf{Base step:} 
For $n=1,2,3$, $T_n$ is equal to \underline{~~~}, whereas the
right-hand side of $P(n)$ is equal to $2^1=2$, $2^2=4$, and $2^3=8$,
respectively. Thus, $P(n)$ holds for $n=1,2,3$.

\textbf{Induction step:} Let $k\ge3$ be given and suppose 
$P(n)$ is true for all $n=1,2,\dots,k$. Then
\begin{align*}
T_{k+1}&=T_{k}+T_{k-1}+\underline{~~~~}
\quad \text{(by recurrence for $T_n$)}
\\
&<2^{k}+2^{k-1}+\underline{~~~~} \quad \text{(strong ind. hyp. \& ($P(k), P(k-1),P(k-2)$)}
\\
&=2^{k+1}\left(\frac12+\frac14+\underline{~~~~}\right)
\\
&=2^{k+1}\cdot \underline{~~~~}
<2^{k+1}.
\end{align*}
Thus, \underline{~~~~~~} holds, and the proof of the induction step is complete. 

\textbf{Conclusion:} By the strong induction principle,  it follows that
$P(n)$  is true for all $n\in\NN$.  
 
\end{enumerate}





%%%%%%%%%%%%%%%%%%%%%%%%%%%%%%%%%%%%%%%%%%%%%%%%%%%%%%%%%%%%%%%%%%%%%%%%%
% strong induction: representation problems
%%%%%%%%%%%%%%%%%%%%%%%%%%%%%%%%%%%%%%%%%%%%%%%%%%%%%%%%%%%%%%%%%%%%%%%%%

\item\textbf{Representation problems.}
One of the main applications of strong induction is to prove the existence of
representations of integers of various types. In these applications, strong
induction is usually needed in its full force, i.e., in the induction step, one
needs to assume that all predecessor cases $n=1,2,\dots,k$. 

\begin{enumerate}

%%%%%%%%%%%%%%%%%%%%%%%%%%%%%%%%%%%%%%%%%%%%%%%%%%%%%%%%%%%%%%%%%%%%%%%%%
% postage stamp problems
%%%%%%%%%%%%%%%%%%%%%%%%%%%%%%%%%%%%%%%%%%%%%%%%%%%%%%%%%%%%%%%%%%%%%%%%%

\item \textbf{The postage stamp problem:}
Determine which postage amounts can be created using the stamps  of $3$
and $7$ cents.  In other words, determine the exact set of positive
integers $n$ that can be written in the form $n=3x+7y$ with $x$ and $y$
nonnegative integers.  (\emph{Hint}: Check the first few values of $n$ directly,
then use strong induction to show that, from a certain point $n_0$
onwards, all numbers $n$ have such a representation.) 




%%%%%%%%%%%%%%%%%%%%%%%%%%%%%%%%%%%%%%%%%%%%%%%%%%%%%%%%%%%%%%%%%%%%%%%%%
% binary representation
%%%%%%%%%%%%%%%%%%%%%%%%%%%%%%%%%%%%%%%%%%%%%%%%%%%%%%%%%%%%%%%%%%%%%%%%%

\item \textbf{Binary representation:}
Using strong induction prove that every positive integer $n$ can be
represented as a sum of \emph{distinct} powers of $2$, i.e., in the form
$n=2^{i_1}+\dots + 2^{i_h}$ with integers $0\le i_1<\dots < i_h$.
(\emph{Hint}: To ensure distinctness, use the \emph{largest} 
power of $2$ as the first ``building block'' in the induction step.
)


\item
\textbf{Factorial representation.}
Show that any integer $n\ge1$ has a  represention in the form
$n=d_11!+d_22!+\cdots+d_rr!$
with ``digits'' $d_i$ in the range $d_i\in\{0,1,\dots,i\}$.
(\emph{Hint}: Use again the ``greedy'' trick (pick the largest factorial that ``fits'' 
as your first building block), and use the 
fact (established in an earlier problem) that
$\sum_{i=1}^ki!i=(k+1)!-1$.)


\end{enumerate}


%\longsol{A quick direct check shows that the 
%positive numbers $n<15$ that have a
%representation $n=3x+7y$ with $x,y\in\NN\cup \{0\}$) are exactly
%$3,6,7,9,10,12,13,14$.  We now use strong induction to show that from
%$12$ onwards every integer has a representation in the above form.
%In other words, we will prove that the following 
%statement holds for all $n\ge12$:
%\[
%\text{$n$ has a representation $(*)$ $n=3x+7y$ with $x,y\in\NN\cup \{0\}$}
%\tag{$P(n)$}
%\]
%\textbf{Base case:} For $n=12,13,14$, the representations $12=3\cdot 4$,
%$13=3\cdot 2+7$ and $14=7\cdot 2$ show that $P(n)$ is true.
%
%\textbf{Induction step:} 
%Let $k\ge14$ be given and suppose $P(k')$ is true for all $k'$ with 
%$k'=12,13,\dots,k$, i.e., suppose that all such $k'$ have a representation in
%the form $(*)$. We seek to show that $k+1$ also has a representation of
%this form. 
%
%Write  $k+1=3+k'$, so that $k'=k-2$. Note that $k'\le k$ and  also
%$k'\ge 12$ since we assumed $k\ge14$. Thus, we can apply the strong
%induction hypothesis to $k'$ and obtain a representation 
%\[
%k'=3x+7y,
%\]
%where $x,y\in\NN\cup\{0\}$. Adding $3$ to both sides of this
%representation, we get
%\[
%k+1=k'+3= 3x+7y+3=3(x+1)+7y,
%\]
%which is a representation of the desired form for $k+1$.
%Hence $P(k+1)$ is true, and the proof of the induction step is complete.
%
%\textbf{Conclusion:} By the strong induction principle, 
%it follows that $P(n)$  is true for all $n\ge12$.
%
%\bigskip
%
%\textbf{Remark:} Note that, in the induction step, in order to be able
%to apply the induction hypothesis with $k'=k-2$ we need to ensure that 
%$k'$ is at least $12$. This in turn requires $k$ 
%to be at least $14$ in the induction step,
%and the cases $k=12,13,14$ to be treated as base
%cases.
%}
%\longsol{
%We will prove by strong induction that the following statement  
%holds for all $n\in\NN$.
%\[
%\text{$n$ has a representation $(*)$ $n=2^{i_1}+\dots + 2^{i_h}$
%with distinct integers $i_1,\dots,i_h\in\NN\cup\{0\}$.}
%\tag{$P(n)$}
%\]
%\textbf{Base case:} The integer $n=1$ has the representation $1=2^0$,
%which is of the desired form. Hence $P(n)$ holds for $n=1$.
%
%
%\textbf{Induction step:} 
%Let $k\ge1$ be given and suppose $P(n)$ is true for all positive
%integers $ n\le k$, i.e., suppose that all such $n$ have a
%representation in the form $(*)$. We seek to show that $n=k+1$ also has a
%representation of this form. 
%
%Let $2^m$ be the largest (integer) power of $2$ that satisfies $2^m\le k+1$.
%
%If $2^m=k+1$, then $k+1$ has a representation of the desired form
%(namely as a sum of a single power of $2$, $2^m$), and we are done.
%
%If $2^m<k+1$, we let $k'=k+1-2^m$.  Since $2^m\ge1$
%and $2^m<k+1$, $k'$ is an integer with $1\le k'\le k$. 
%Hence we can apply the strong induction hypothesis to $k'$ and obtain a
%representation of $k'$ as a sum of distinct powers of $2$.
%
%Adding $2^m$ to this representation gives a representation of $k+1=k'+2^m$
%as a sum of powers of $2$. To complete the proof of the induction step,
%we still need to show that the powers of $2$ involved here are distinct.
%
%Since the powers of $2$ representing $k'$ were already distinct, it
%suffices to show that the added power $2^m$ cannot occur among the
%powers in the representation for $k'$. 
%To do this, we exploit the fact that $2^m$ was chosen as the largest
%power of $2$ below $k+1$. Thus, we have
%\[
%2^m< k+1 < 2^{m+1}.
%\]
%Subtracting $2^m$ from both sides, we get 
%\[
%0< k+1-2^m<2^{m+1}-2^m=2^m,
%\]
%and since $k'=k+1-2^m$, it follows that $k'$ is strictly less than
%$2^m$, and so $2^m$ cannot occur in the representation for $k'$.
%This is what we wanted to show.
%
%Thus, the representation for $k+1$ that we obtained is indeed a
%representation of the desired form and the proof of the 
%induction step is complete.
%
%\textbf{Conclusion:} By the strong induction principle, 
%it follows that $P(n)$  is true for all $n\ge1$.
%
%\bigskip
%
%\textbf{Remarks:} The argument used above in the induction step is called a
%``greedy'' algorithm: In constructing a binary representation for $k+1$, one
%starts  out by using the largest possible ``building block'' (namely, the
%largest power of $2$ that is $\le k+1$), and then uses the strong induction
%hypothesis to make up for the left over part. 
%
%
%An alternative, but less flexible, approach to the induction step is as
%follows: If $k+1$ is even, then $k+1=2k'$, where $k'$ is an integer in
%the range $1\le k'\le k$. By the strong induction hypothesis $k'$
%has a representation as sum of distinct powers of $2$, and multiplying this
%representation by $2$ gives a representation of the desired form for $k+1$. 
%If $k+1$ is odd, a similar argument based on the representation $k+1=2k'+1$
%yields the same conclusion. 
%
%The latter approach, however, relies heavily on specific arithmetic properties
%of the powers of $2$ and does not generalize to other sequences like Fibonacci
%numbers or factorials. By contrast, the ``greedy'' approach is one that can be
%used for many representation problems and, in fact, is the standard way to
%handle such problems.
%
%\bigskip
%
%\textbf{Uniquess of representation:} The above argument proves only the
%\emph{existence} of a representation, not its uniqueness.  One way to prove the
%uniqueness is by contradiction: Assume there are positive integers with
%multiple representations, let $n$  be the smallest of these exceptional
%integers, and derive a contradiction from this assumption.
%
%Another way to prove uniquess is to incorporate the uniqueness claim into the
%statement $P(n)$ to be proved. The strengthened statement requires an additional
%argument in the induction step showing that uniqueness holds for $k+1$,
%provided it holds for all $k'\le k$. This is not difficult; the key observation
%is that any representation of $k+1$ must necessarily involve the power $2^m$ 
%defined above.
%
%}
%\longsol{
%We will prove by strong induction that the following statement  
%holds for all $n\in\NN$.
%\[
%\text{$n$ has a representation 
%$(*)$ $\sum_{i=1}^rd_ii!$
%with $d_i\in\{0,1,\dots,i\}$.}
%\tag{$P(n)$}
%\]
%\textbf{Base case:} The integer $n=1$ has the representation $1=1\cdot 1!$,
%which is of the desired form. Hence $P(n)$ holds for $n=1$.
%
%
%\textbf{Induction step:} 
%Let $k\ge1$ be given and suppose $P(n)$ is true for all positive
%integers $ n\le k$, i.e., suppose that all such $n$ have a
%representation in the form $(*)$. We seek to show that $n=k+1$ also has a
%representation of this form. 
%
%Let $r$ be the \emph{largest} integer such that $r!\le k+1$; i.e., $r$ is the
%unique integer for which  
%\[
%r!\le k+1 < (r+1)!.
%\tag{1}
%\]
%
%If $r!=k+1$, then $k+1$ has a representation of the desired form,
%and we are done.
%
%If $r!<k+1$, we let $k'=k+1-r!$.  Since $1\le r!<k+1$, 
%$k'$ is an integer in the range $1\le k'\le k$. 
%Hence we can apply the strong induction hypothesis to $k'$ and obtain a
%representation of $k'$ as a finite sum of terms $d_ii!$, with ``digits'' 
%$d_i$ in the range $0\le d_i\le i$.  
%
%Adding $r!$ to this representation gives a representation of $k+1=k'+r!$
%as a sum of factorials.  To complete the induction step, we need to make sure
%that this new representation still satisfies the constraints $0\le d_i\le i$ 
%on the digits. 
%
%If $r!$ does not occur in the representation of $k'$, this is clearly the case.
%
%If $r!$ does occur in the representation of $k'$ with an associated
%``digit'' $d_r$ satisfying $d_r\le r-1$, 
%then adding $r!$ to this representation gives a
%representation with $d_r$ replaced by $d_r+1$ and all other digits unchanged, 
%and since $d_r\le r-1$, the new digit $d_r+1$ satisfies the required
%constraint, $d_r+1\le r$.
%
%It remains to consider the case when $k'$ has a representation involving $r!$
%in which the associated digit is maximal, i.e., $d_r=r$. But then 
%\[
%k+1=k'+r!\ge d_rr!+r! =r\cdot r! + r!=(r+1)!,
%\]
%so $(r+1)!\le k+1$, contradicting (1). Therefore this case is impossible. 
%
%Hence, in each case we have obtained a representation of $k+1$ of the 
%desired form and the proof of the 
%induction step is complete.
%
%\textbf{Conclusion:} By the strong induction principle, 
%it follows that $P(n)$  is true for all $n\ge1$.
%}

\end{enumerate}


\section{Non-formula induction proofs}
\label{sec:Induction:NonFormulaProofs}



Below is a sample proof of the statement that any $n$-element set (i.e.,
any set with $n$ elements) has $2^n$ subsets. 
This illustrates a case where the result we seek to prove is not a
formula, but a statement that must be expressed verbally, and where the
induction step requires some verbal explanation, and not just a chain of
equalities.  Additional practice  problems follow below.


\begin{prop}{}
For all $n\in\NN$, the following holds:
\[
{\text{Any $n$-element set has $2^n$ subsets.}}
\tag{$P(n)$}
\]
\end{prop}
\begin{proof}(\emph{By induction}):

\textbf{Base case:} Since any $1$-element set has $2$
subsets, namely the empty set and the set itself,
and $2^1=2$, the statement $P(n)$ is true for $n=1$.

\textbf{Induction step:} 
\begin{itemize}
\item 
Let $k\in\NN$ be given and suppose 
$P(k)$ is true, i.e., that any $k$-element set has $2^k$ subsets.
We seek to show that $P(k+1)$  is true as well,
i.e., that any $(k+1)$-element set has $2^{k+1}$ subsets.

\item 
Let $A$ be a set with $k+1$ elements.  

\item 
Let $a$ be an element of $A$, and let $A'=A-\{a\}$ (so that   
$A'$ is a set with $k$ elements).

\item 
We classify the subsets of $A$ into two types: (I) subsets that do
\emph{not} contain $a$, and (II) subsets that do contain $a$.

\item 
The subsets of type (I) are exactly the subsets of the set
$A'$. Since $A'$ has $k$ elements, the induction
hypothesis can be applied to this set and we conclude that there are $2^k$
subsets of type (I).

\item 
The subsets of type (II) are exactly the sets of the form $B=B'\cup
\{a\}$, where $B'$ is a subset of $A'$.
By the induction hypothesis there are $2^k$ such sets $B'$, and hence
$2^k$ subsets of type (II).

\item 
Since there are $2^k$ subsets of each of the two types, the total number
of subsets of  $A$ is $2^k+2^k=2^{k+1}$. 

\item 
Since $A$ was an arbitrary $(k+1)$-element set, we have proved that any
$(k+1)$-element set has $2^{k+1}$ subsets.
Thus $P(k+1)$ is true, completing the induction step. 
\end{itemize}

\textbf{Conclusion:} By the principle of induction, 
$P(n)$  is true for all $n\in\NN$.
\end{proof}

\section{Practice problems for non-formula induction}
\label{sec:Induction:PracticeNonFormulaInduction}

\begin{enumerate}


\item \textbf{Number of subsets with an even (or odd) number of elements:}
Using induction, prove that an $n$-element set  has $2^{n-1}$ subsets
with an even number of elements and $2^{n-1}$ subsets with an odd number
of elements.

%\sol{
%We use a variation of the above argument (showing that an
%$n$-element set has $2^n$ subsets).  For brevity, we call a subset with an odd
%number of elements an ``odd subset'', and a subset with an even number of
%elements an ``even subset.'' 
%
%Let $P(n)$ denote the statement that \textbf{any set with $n$ elements has
%$2^{n-1}$ odd subsets and $2^{n-1}$ even subsets.}
%We use induction to show that $P(n)$ holds for all $n\in\NN$.
%
%
%
%\textbf{Base case:} A $1$-element set $A=\{a_1\}$ has exactly 
%one even subset, the empty set $\emptyset$ (since the empty set has $0$
%elements, and $0$ is an even number), and one odd subset, $\{a_1\}$, so $P(1)$
%is true.
%
%\textbf{Induction step:} 
%Let $k\in\NN$ be given and suppose 
%$P(k)$ is true, i.e., that any $k$-element set has $2^{k-1}$ even subsets and
%$2^{k-1}$ odd subsets. 
%We seek to show that $P(k+1)$  is true as well, i.e., that
%any $(k+1)$-element set has $2^{k}$ even subsets and $2^k$ odd subsets.
%
%Let $A$ be a set with $(k+1)$ elements.  
%Choose an element $a$ in $A$, and set $A'=A-\{a\}$. 
%
%We again classify the subsets of $A$ into two types: (I) subsets that do
%\emph{not} contain $a$, and (II) subsets that do contain $a$.
%The subsets of type (I) are exactly the subsets of the set
%$A'$. Since $A'$ has $k$ elements, the induction
%hypothesis can be applied to this set and we get that there are 
%$2^{k-1}$ even subsets and $2^{k-1}$ odd subsets of type (I).
%
%The subsets of type (II) are exactly the sets of the form $B=B'\cup
%\{a\}$, where $B'$ is a subset of $A'$, and hence are in one-to-one
%correspondence with subsets $B'$ of $A'$.  Moreover, $B$ is
%an odd subset of
%$A$ if and only if the associated set $B'$ is an even subset of $A'$,
%and an even subset of
%$A$ if and only if the associated set $B'$ is an odd subset of $A'$.
%By the
%induction hypothesis there are $2^{k-1}$ even subsets of $A'$, and $2^{k-1}$
%odd subsets of $A'$.   Hence there are $2^{k-1}$ odd subsets of type (II), and
%$2^{k-1}$ even subsets of type (II).
%
%Since there are $2^{k-1}$ even subsets of each of the types (I) and
%(II), the total number of even subsets of  $A$ is $2^{k-1}+2^{k-1}=2^{k}$. 
%Similarly, the total number of odd subsets of $A$ is $2^{k-1}+2^{k-1}=2^k$.
%
%Since $A$ was an arbitrary $(k+1)$-element set, we have proved that any
%$(k+1)$-element set has $2^{k}$ even subsets and $2^k$ odd 
%subsets. Thus $P(k+1)$ is true,
%completing the induction step. 
%
%\textbf{Conclusion:}
%By the principle of induction, it follows that
%$P(n)$  is true for all $n\in\NN$.
%}


%%%%%%%%%%%%%%%%%%%%%%%%%%%%%%%%%%%%%%%%%%%%%%%%%%%%%%%%%%%%%%%%%%%%%%%%%
% regions created by n lines 
%%%%%%%%%%%%%%%%%%%%%%%%%%%%%%%%%%%%%%%%%%%%%%%%%%%%%%%%%%%%%%%%%%%%%%%%%

\item \textbf{Number of regions created by $n$ lines:}
How many regions are created by $n$ lines in the plane such that no two lines
are parallel and no three lines intersect at the same point? Guess the
answer from the first few cases, then use induction to prove your guess. 

%\sol{
%For brevity, we call a set of lines \emph{generic} if it satisfies the
%conditions in the statement, namely that no two lines are parallel, and no
%three lines intersect at the same point.
%
%Let $P(n)$ denote the statement that 
%\textbf{the number of regions created by $n$ generic lines in the plane is
%$1+\frac{n(n+1)}{2}$}. 
%We will use induction to show that $P(n)$ holds for all $n\in\NN$.
%
%
%\textbf{Base case:} A single line divides the plane into $2$ regions. Since 
%$1+1(1+1)/2=2$, this proves $P(1)$.
%for $n=1$. 
%
%\textbf{Induction step:} Let $k\in\NN$ be given, and suppose $P(n)$
%holds for $n=k$, i.e., suppose that any $k$ generic lines in the plane create
%$1+k(k+1)/2$ regions.  
%
%Let $k+1$ lines $L_1,L_2,\dots,L_{k+1}$ be given that are generic in the above
%sense.  Then the first $k$ lines, $L_1,\dots,L_k$ are also generic and, by the
%induction hypothesis, these $k$ lines divide the plane into $1+k(k+1)/2$
%regions.
%
%Now consider the line $L_{k+1}$.  By the ``generic'' property, this line
% intersects each of the lines $L_1,\dots,L_k$ at exactly one point,
%and the $k$ intersection points are all distinct and hence divide $L_{k+1}$
%into $k+1$ segments.  Each of these segments divides one of the regions created
%by the first $k$ lines into two parts, and hence increases the region count by
%$1$.  Since there are $k+1$ such segments, the added line $L_{k+1}$ increases
%the region count by $k+1$. Thus the total number of regions created by
%the lines $L_1,\dots,L_{k+1}$ is  
%\[
%1+\frac{k(k+1)}{2}+k+1=\frac{k(k+1)+2k+4}{2}=1+\frac{(k+1)(k+2)}{2},
%\]
%which is the desired formula for the number of regions created by $k+1$
%lines.  Hence $P(k+1)$ holds, and the proof of the induction step is
%complete.
%
%\textbf{Conclusion:} By the principle of induction, $P(n)$ holds for all
%$n\in\NN$.
%}

%%%%%%%%%%%%%%%%%%%%%%%%%%%%%%%%%%%%%%%%%%%%%%%%%%%%%%%%%%%%%%%%%%%%%%%%%
% sum angles in polygon 
%%%%%%%%%%%%%%%%%%%%%%%%%%%%%%%%%%%%%%%%%%%%%%%%%%%%%%%%%%%%%%%%%%%%%%%%%

\item \textbf{Sum of angles in a polygon:}
The sum of the interior angles in a triangle is $180$ degrees, or $\pi$.
Using this result and induction, prove that for any $n\ge3$, the sum of
the interior angles in an $n$-sided polygon is $(n-2)\pi$.

%\sol{
%Let $P(n)$ denote the statement that \textbf{the sum of the interior
%angles in an $n$-sided polygon is $(n-2)\pi$.}
%We will use induction to show that $P(n)$ holds for 
%all integers $n\ge3$.
%
%
%\textbf{Base case:} The sum of the angles in a triangle is $\pi$, which
%agrees with the formula 
%$(n-2)\pi$ when $n=3$. Thus the statement $P(n)$
%holds for $n=3$. 
%
%\textbf{Induction step:} Let $k\in\NN$ with $k\ge 3$ be given, and
%assume $P(n)$ holds for $n=k$, i.e., suppose that, for any $k$-sided polygon,
%the sum of the interior angles is $(k-2)\pi$. 
%
%
%Let $P$ be a $(k+1)$-sided polygon. Pick a vertext $P_1$ of $P$ at which the
%interior angle is $<\pi$. (It is clear that such a vertex must exist.)
%Let $P_0$ and $P_1$ denote the vertices of $P$ adjacent to $P_1$,
%let $T$ be the triangle $P_0P_1P_2$, and 
%$P'$ the polygon obtained from $P$ by removing the triangle $T$,
%i.e., with the two sides $P_0P_1$ and $P_1P_2$ replaced by $P_0P_2$
%
%Then $P'$ has $k$ sides, so by the induction hypothesis the sum of the
%interior angles in $P'$ is $(k-2)\pi$. Also, since $T$ is a triangle,
%the sum of the interior angles in $T$ is $\pi$.  
%The sum of the interior angles in the original polygon $P$ is equal
%to the sum of the interior angles of $P'$ plus the sum of the interior
%angles of $T$, i.e.,  $(k-2)\pi + \pi = ((k+1)-2)\pi$. 
%This is the desired formula for the sum of the interior angles of a
%$(k+1)$-sided polygon, so we have proved $P(k+1)$. 
%
%\textbf{Conclusion:} By the principle of induction, $P(n)$ holds for
%every integer $n\ge3$.
%}

%%%%%%%%%%%%%%%%%%%%%%%%%%%%%%%%%%%%%%%%%%%%%%%%%%%%%%%%%%%%%%%%%%%%%%%%%


\item \textbf{Pie-throwing problem:}
Here is a harder, but fun problem.  Consider a group of $n$ fraternity members
standing in a yard, such that their mutual distances are all distinct.  Suppose
each of throws a pie at his nearest neighbor.  Show that if $n$ is odd, then
there is one person in the group who does not get hit by a pie. (\emph{Hint}: Let
$n=2m+1$ with $m\in\NN$, and use $m$ as the induction variable. Consider first
some small cases, e.g., $n=3$ and $n=5$.)

%\sol{Let $P(m)$ denote the statement that,  \textbf{given any group of $2m+1$ people
%with pairwise distinct mutual distances, there is at least one survivor in the
%pie fight (in the sense of not getting hit by a pie)}.
%We will show by induction on $m$ that $P(m)$ holds for all $m\in\NN$.
%
%\textbf{Base case:} When $m=1$, there are $2m+1=3$ fraternity members in the
%group, By the ``distinct distance'' assumption, the triangle created by these
%three members has a unique minimal side. Therefore the two members standing at
%the endpoints of this side throw pies at each other, while the third person
%hits one of these two, but does not get hit. Thus, this third person
%``survives'' the fight, and hence the statement $P(m)$ holds for $m=1$.
%
%\textbf{Induction step:} Let $k\in\NN$ and suppose $P(m)$ holds for $m=k$.
%Consider a group of $2(k+1)+1=2k+3$ fraternity members, say
%$F_1,\dots,F_{2k+2},F_{2k+3}$,
%with distinct mutual distances.  Since the distances are
%distinct, there exists a unique minimal distance among the distances between
%pairs of fraternity members. Without loss of generality, we may assume that
%$F_{2k+2}$ and $F_{2k+3}$ are the two members whose mutual distance is minimal. 
%Then these two members throw pies at each other, while the remaining members, 
%i.e., $F_1,\dots, F_{2k+1}$, throw pies at each other or at $F_{2k+2}$
%or $F_{2k+3}$. 
%
%If all remaining members $F_1,\dots, F_{2k+1}$ only throw  pies at themselves
%(and  not at $F_{2k+2}$ or $F_{2k+3}$), then the induction hypothesis
%immediately guarantees that there is a survivor among these members. 
%
%Now consider the case when some of the members
%$F_1,\dots,F_{2k+1}$
%have  $F_{2k+2}$ or
%$F_{2k+3}$ as their nearest target, and thus, by the rules of the game, throw a
%pie at $F_{2k+2}$ or $F_{2k+3}$.  In this case, removing   $F_{2k+2}$ and
%$F_{2k+3}$ as possible targets forces these members to target the nearest
%neighbor among 
%$F_1,\dots,F_{2k+1}$.
%We can then again apply   the
%induction hypothesis to obtain a survivor among $F_1,\dots,F_{2k+1}$.
%Since $F_{2k+2}$ and $F_{2k+3}$ only target themselves, that person remains a
%survivor after adding these two members back in. 
%
%Thus, in either case, we have shown that there is a survivor in the given group of 
%$2(k+1)+1$ fraternity members. Hence $P(k+1)$ holds, and the induction step is
%complete.   
%
%\textbf{Conclusion:} By the principle of induction, $P(m)$ holds for all
%$m\in\NN$.
%}
\end{enumerate}


\section{~Fallacies and pitfalls }
\label{sec:Induction:FallaciesAndPitfalls} 

By now, induction proofs should feel routine to you, to the point that you
could almost do them in your sleep.  However, it is important not to become
complacent and careless, for example, by skipping seemingly minor details in
the write-up, omitting quantifiers, or neglecting to check conditions and
hypotheses. 

Below are some examples of false induction proofs that illustrate what can
happen when some minor details are left out.  In each case, the statement 
claimed is clearly nonsensical (e.g., that all numbers are equal), but the
induction argument sounds perfectly fine, and in some cases the errors are quite
subtle and hard to spot. Try to find them!

\begin{example}{fall1} Let us ``prove'' that for all $n\in\NN$,  

\[
\tag{$P(n)$} \sum_{i=1}^n i=\frac12(n+\frac12)^2
\]

\textbf{Proof:} We prove the claim by induction. 

\textbf{Base step:} When $n=1$, $P(n)$ holds.

\textbf{Induction step:} 
Let $k\in\NN$ and 
suppose $P(k)$ holds. Then
\begin{align*}
\sum_{i=1}^{k+1} i &=\sum_{i=1}^k i + (k+1)
\\
&=\frac12\left(k+\frac12\right)^2+ (k+1)
\quad\text{(by ind. hypothesis)}
\\
&=\frac12\left(k^2+k+\frac14+2k+2\right)
\quad\text{(by algebra)}
\\
&=\frac12\left(\left(k+1+\frac12\right)^2-3k-\frac94 +k+\frac14+2k+2\right)
\quad\text{(more algebra)}
\\
&=\frac12\left((k+1)+\frac12\right)^2
\quad\text{(simplifying).}
\end{align*}
Thus, $P(k+1)$ holds,  so the induction
step is complete.

\textbf{Conclusion:} By the principle of induction, $P(n)$ holds for all
$n\in\NN$.
\end{example}

%
%\analysis{Here there is no problem with the induction step, but the base case
%is not valid despite the claim that it is true in this case.
%Moral: Make sure to \emph{really} check the base case. Simply 
%stating that it is true doesn't make it true!}

%%%%%%%%%%%%%%%%%%%%%%%%%%%%%%%%%%%%%%%%%%%%%%%%%%%%%%%%%%%%%%%%%%%%%%%%%


\begin{example}{fall2}
Now we will ``prove'' that all real numbers are equal. To prove the claim, we will prove by induction that, for all $n\in\NN$,
the following statement holds:
\[
\text{For any real numbers $a_1,a_2,\dots, a_n$, we have $a_1=a_2=\dots=a_n$.}
\tag{$P(n)$}
\]
\textbf{Base step:} When $n=1$, the statement is trivially true, so $P(1)$
holds.  

\textbf{Induction step:} 
Let $k\in\NN$ be given and suppose 
$P(k)$ is true, i.e., that any $k$ real numbers must be equal.
We seek to show that $P(k+1)$  is true as well,
i.e., that any $k+1$ real numbers must also be equal.

Let $a_1,a_2,\dots,a_{k+1}$ be given real numbers.  Applying the induction
hypothesis to the first $k$ of these numbers, $a_1,a_2,\dots,a_k$, we obtain
\[
\tag{1}
a_1=a_2=\dots=a_k.
\]
Similarly, applying the induction hypothesis to the last $k$ of these numbers, 
$a_2,a_3,\dots,a_k,a_{k+1}$, we get
\[
\tag{2}
a_2=a_3=\dots=a_k=a_{k+1}.
\]
Combining (1) and (2) gives
\[
\tag{3}
a_1=a_2=\dots=a_k=a_{k+1},
\]
so the numbers $a_1,a_2,\dots,a_{k+1}$ are equal.
Thus, we have proved $P(k+1)$, and the induction step is complete. 

\textbf{Conclusion:} By the principle of induction, 
$P(n)$  is true for all $n\in\NN$. Thus, any $n$ real numbers must be equal.
\end{example}

%\analysis{The base step is valid, as is the induction step \emph{provided $k$
%is at least $2$.} However, when $k=1$, the induction step breaks down since in
%this case there is no overlap in the variables in (1) and (2), so one cannot
%``chain together'' these equalities. Formally, this means that in the
%implication chain $P(1)\implies P(2)\implies P(3)\implies P(4)\implies\cdots$, the
%first link (from $P(1)$ to $P(2)$) is broken, while all other links are valid.
%This single broken link is enough to render the induction argument invalid.} 
%


\begin{example}{fall3} 
Here is a ``proof'' that for every nonnegative integer $n$, 

\[ \tag{$P(n)$} 5n=0.  \]

\textbf{Proof:} We prove that $P(n)$ holds for all $n=0,1,2,\dots$, using
strong induction with the case $n=0$ as base case.

\textbf{Base step:} When $n=0$, $5n=5\cdot0=0$, so $P(n)$ holds in this case.

\textbf{Induction step:} 
Suppose $P(n)$ is true for all integers $n$ in the range $0\le n\le k$, i.e.,
that for all integers in this range $5n=0$.  We will show that $P(k+1)$ also holds, so that
\[ \tag{$P(k+1)$} 5(k+1)=0.  \]

Write $k+1=i+j$ with integers $i,j$ satisfying $0\le i,j\le k$. 
Applying the induction hypothesis to $i$ and $j$, we get $5i=0$ and $5j=0$.
Then
\[
5(k+1)=5(i+j)=5i+5j=0+0=0,
\]
proving $P(k+1)$.  Hence the induction step is complete.

\textbf{Conclusion:} By the principle of strong induction, $P(n)$ holds for all
nonnegative integers $n$.
\end{example}

%\analysis{The base step is valid, and the induction step is valid, too,
%\emph{provided $k$ is at least $1$.} However, the first $k$-value for which
%we need the induction step is $k=0$ (since $n=0$ is our base case). When
%$k=0$, the equation $k+1=i+j$ reduces to $1=i+j$, 
%and the constraints on $i,j$ become $0\le i,j\le 0$. Since there is no  
%choice of $i,j$ satisfying both these constraints  and the equation $i+j=1$,
%the argument breaks down in this case. (When $k\ge1$, there is no problem
%since choosing $i=1$ and $j=k$ we can satisfy both $i+j=k+1$ and 
%the constraints $0\le i,j\le k$.)}


\begin{example}{fall4} Let's ``prove'' that for every nonnegative integer $n$, 
\[
\tag{$P(n)$} 2^n=1
\]
\textbf{Proof:} We prove that $P(n)$ holds for all $n=0,1,2,\dots$, using
strong induction with the case $n=0$ as base case.

\textbf{Base step:} When $n=0$, $2^0=1$, so $P(0)$ holds. (Note: it is perfectly OK to begin with a base case of $n=0$.)

\textbf{Induction step:} 
Suppose $P(n)$ is true for all integers $n$ in the range $0\le n\le k$, i.e.,
assume that for all integers in this range $2^n=1$.
We will show that $P(k+1)$ also holds, i.e., 
\[
\tag{$P(k+1)$} 2^n=1
\]

We have
\begin{align*}
2^{k+1}&=\frac{2^{2k}}{2^{k-1}}
\quad~\text{(by algebra)}
\\
&=\frac{2^k\cdot 2^k}{2^{k-1}}
\quad\text{(by algebra)}
\\
&=\frac{1\cdot 1}{1}
\quad~~\text{(by strong ind. hypothesis applied to each term)}
\\
&=1
\qquad~~~\text{(simplifying),}
\end{align*}
proving $P(k+1)$. Hence the induction step is complete.

\textbf{Conclusion:} By the principle of strong induction, $P(n)$ holds for all
nonnegative integers $n$.
\end{example}

%\analysis{The base step is valid, and the induction step is valid, too,
%\emph{provided $k$ is at least $1$.} However, as in the previous example, 
%when $k=0$ (the first $k$-value we need in the induction step), something goes wrong:
%Namely, in this case, $k-1=0-1=-1$ is negative and hence out of range of 
%the induction hypothesis.}

\begin{example}{fall5}
We will ``prove'' that all positive integers are equal. To prove this claim, we will prove by induction that, for all $n\in\NN$,
the following statement holds:
\[
\text{For any $x,y\in\NN$, if $\max(x,y)=n$, then $x=y$.}
\tag{$P(n)$}
\]
(Here $\max(x,y)$ denotes the larger of the two numbers $x$ and $y$, or the
common value if both are equal.)

\textbf{Base step:} When $n=1$,
the condition in $P(1)$ becomes $\max(x,y)=1$. But this forces
$x=1$ and $y=1$, and hence $x=y$. 

\textbf{Induction step:} 
Let $k\in\NN$ be given and suppose $P(k)$ is true.  We seek to show that
$P(k+1)$ is true as well.

Let $x,y\in\NN$ such that $\max(x,y)=k+1$. Then
$\max(x-1,y-1)=\max(x,y)-1=(k+1)-1=k$. By the induction hypothesis, it follows
that $x-1=y-1$, and therefore  $x=y$. This proves $P(k+1)$, so the induction
step is complete.

\textbf{Conclusion:} By the principle of induction, 
$P(n)$  is true for all $n\in\NN$. In particular, since $\max(1,n)=n$ for any
positive integer $n$, it follows that $1=n$ for any positive integer $n$. 
Thus, all positive integers must be equal to $1$
\end{example}

%\analysis{The base step is valid, 
%but there is a problem with the induction step: In this step
%the induction hypothesis is applied with $x-1$ and $y-1$ in place
%of $x$ and $y$. However, this requires that $x-1$ and $y-1$ be positive
%integers, something that we are not assured.
%For example, if $x=1$, then $x-1=0$, so $x-1$ is out of range, and therefore we
%cannot apply the induction hypothesis with $x-1$ as the $x$-value. 
%This renders the induction step invalid \emph{for all values of $k$}.}








