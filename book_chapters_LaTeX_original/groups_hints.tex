\section{Hints for ``Abstract Groups: Definitions and Basic Properties'' exercises}
\label{sec:Groups:Hints}

\noindent Exercise \ref{exercise:groups:ident}: For the ``if'' part, assume that $g \circ h = h$, and use this to show that $g = e$. Multiply both sides of the assumed equation by $h^{-1}$. You will need to use associativity and properties of inverses and the identity to obtain the result.  For the ``only if'' part, assume that $g=e$ and use this fact to show that $g \circ h = h$.

\noindent Exercise \ref{exercise:groups:cayley}: Prove by contradiction. Suppose that for row ``$g$'', the entries in columns ``$h$'' and ``$h'$'' are the same, where $h \neq h'$.  Then what equation must be true? Show this equation leads to a contradiction.

\noindent Exercise \ref{exercise:groups:U(n)_abgroup}(b): Refer to Section~\ref{subsec:ModularArithmetic:MultInve}.

\noindent Exercise \ref{exercise:groups:group_abelian} In general, the way to prove statements like this is to multiply both sides of the equation by the same thing.  In this case, you may multiply by $ab$.  Some additional multiplications will give you the result $ba=ab$.

\noindent Exercise \ref{exercise:groups:90}: Use the fact that $a^k = a^l$ for $k \neq l$.  You may assume that $k < l$ in your proof.

\noindent Exercise \ref{exercise:groups:OrderEltCyclic}: Write $\bmod(m,n)$ as $m + kn$, where $k$ is an integer.

\bigskip

\textbf{Additional exercises:}

\noindent Exercise \ref{ex:groups:quat}: You may obtain 4 cyclic subgroups of order 2 (why?)  To look for more subgroups, suppose for instance there is a subgroup that contains both $i$ and $j$.  What other elements must it contain?  Do the same for $i$ and $k$, $j$ and $k$, etc.

\noindent Exercise \ref{ex:eoc:evenInv}: This is a counting argument. Prove by contradiction. Assume the contrary, and pair each group element with its inverse. The entire group is the union of these pairs, plus the identity. What does this tell you about the order of the group?

\noindent Exercise \ref{ex:eoc:abelian1}: Multiply the equation by some well-chosen inverses.

\noindent Exercise \ref{ex:eoc:abelian2}: You may use Exercise \ref{ex:eoc:abelian1}.

\noindent Exercise \ref{ex:groups:2elt}: In fact, such a group must have at most two elements. Do you see why?

