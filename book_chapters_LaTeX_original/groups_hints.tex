\section{Hints for ``Abstract Groups: Definitions and Basic Properties'' exercises}\label{sec:groups:hints}

\noindent Exercise \ref{exercise:groups:ident}: For the ``if'' part, assume that $g \circ h = h$, and use this to show that $g = e$. Multiply both sides of the assumed equation by $h^{-1}$. You will need to use associativity and properties of inverses and the identity to obtain the result.  For the ``only if'' part, assume that $g=e$ and use this fact to show that $g \circ h = h$.

\noindent Exercise \ref{exercise:groups:cayley}: Prove by contradiction. Suppose that for row ``$g$'', the entries in columns ``$h$'' and ``$h'$'' are the same, where $h \neq h'$.  Then what equation must be true? Show this equation leads to a contradiction.

\noindent Exercise \ref{exercise:groups:U(n)_abgroup}(b): Refer to Section~\ref{subsec:MultInve}.

\noindent Exercise \ref{exercise:groups:90}: Use the fact that $a^k = a^l$ for $k \neq l$.  You may assume that $k < l$ in your proof.

\noindent Exercise \ref{exercise:groups:OrderEltCyclic}: Write $\bmod(m,n)$ as $m + kn$, where $k$ is an integer.

\bigskip

\textbf{Additional exercises:}

\noindent Exercise \ref{ex:groups:quat}: You may obtain 4 cyclic subgroups of order 2 (why?)  To look for more subgroups, suppose for instance there is a subgroup that contains both $i$ and $j$.  What other elements must it contain?  Do the same for $i$ and $k$, $j$ and $k$, etc.

\noindent Exercise \ref{ex:groups:abelian_proof}: Consider Exercise~\ref{abelian2} above.

\noindent Exercise \ref{ex:groups:abelian_proof2}: In fact, such a group can have at most two elements.  Why?
