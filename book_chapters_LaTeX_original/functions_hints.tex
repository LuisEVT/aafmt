\section{Hints for ``Functions: basic concepts'' exercises}
\label{sec:Functions:Hints}

\noindent Exercise \ref{exercise:functions:funtable}(e): There is a formula of the form $f(x) = ax^2 + bx + c$

\noindent Exercise \ref{exercise:functions:45}: Can there be any elements in the codomain that are not in the range?.

\noindent Exercise \ref{exercise:functions:NxNBijection}(a): Consider the values $f(1,i)$  for $i=1,2,3,\ldots$. (b): Consider  the values $f(2,j)$ and $f(1,i)$.

\noindent Exercise \ref{exercise:functions:NxN}(a): Given any element $(i,j)$ of $\mathbb{Z} \times \mathbb{Z}$, set $i=m+n$ and $j=m+2n$ and solve for $m$ and $n$ in terms of $i$ and $j$.

\noindent Exercise \ref{exercise:functions:NxN}(b): Suppose that $g(m,n) = g(p,q)$. It follows that $(m + n, m + 2n) = (p + q, p + 2q)$. %This gives two separate equations:  $m+n=p+q$ and $m+2n = p+2q$.

\noindent Exercise \ref{exercise:functions:inverMod}: (c) 
Use Proposition~\ref{proposition:modular:mod_eq_solution}, and recall that $ax \equiv 1 \pmod{n}$ means the same thing as $a \odot x = 1$ for $a,x \in \mathbb{Z}_n$.  You may use this fact to find an inverse for $f_a$. (d) Use the fact that $a \odot x = 1$ has no solution to  show that $f_a$ is not onto, which implies that $f_a$ has no inverse.

\noindent Exercise \ref{exercise:functions:IdAInverse}: (a) Notice that $f(x)=x$ is the identify function when the set $A$ is equal to $\mathbb{R}$.  Think about how you would show that $f(x)$ is invertible in this case.  Then apply the same proof, replacing $x$ with $a$ and $f$ with
$Id_A$.  (b) Again, think of the case $f(x)=x$.  What is the inverse of this function? 

\noindent Exercise \ref{exercise:functions:InverseBijection2}: Given that $f$ is a bijection from $X$ to $Y$. We may define a function $g$ from $Y$ to $X$ as follows.  Given any $y \in Y$, since $f$ is onto there is at least one $x$ such that $f(x) = y$. Furthermore, since $f$ is one-to-one there is at most one $x$ such that $f(x)=y$. Putting these two facts together gives us that there is \emph{exactly} one $x$ such that $f(x)=y$.  We may define $g(y)$ as this unique $x$. It remains to show that for any $y \in Y$, $f(g(y))=y$; and for any $x \in X$, $g(f(x))=x$.

\noindent Exercise \ref{exercise:functions:InverseIdentityExers}: (a) Apply Definition  \ref{def:invfna} directly, replacing $f$ with
$g \circ f$ and $g$ with $g^{-1} \circ f^{-1}$. (b) Apply Definition  \ref{def:invfna} again, this time replacing $f$ with
$f^{-1}$. What should $g$ be replaced with?

