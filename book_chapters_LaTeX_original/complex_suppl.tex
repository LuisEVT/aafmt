\section{Solutions for``Complex Numbers''}
\noindent\textbf{\textit{ (Chapter \ref{complex})}}\bigskip

\noindent\textbf{Exercise \ref{exercise:complex:9}:} \\
b. $43-18i$\\
d. $a^2+b^2$\\
n. $i$\\
r. 0\\
\\
\noindent\textbf{Exercise \ref{exercise:complex:findk}:} \\
$z=3+i$\\
$z^{2}-6z+k=0$\\
$3^{2}+6i+i^{2}-6(3+i)+k=0$\\
$8-18+k=0$\\
$k=10$\\
\\
\textbf{Exercise \ref{exercise:complex:12}:}\\
1. 0\\
2. 0\\
3. 0\\
4. 1\\
5. $z^{-1}$\\
6. 0\\
7. $z \neq 0$ and $w \neq 0$ \\
8. False\\
9. $z \neq 0$ and $w \neq 0$ \\
10. Either $z=0$ or $w=0$\\
\\
\textbf{Exercise \ref{exercise:complex:tableentries}:}\\
The last column of the table is:\\
Additive identity: $(a+bi)+0=0+(a+bi)=a+bi$\\
Additive inverse : $(a+bi)+(-a-bi)=(-a-bi)+(a+bi)=0$\\
Associative law  : $(a+bi)+[(c+di)+(e+fi)]=[(a+bi)+(c+di)]+(e+fi)$\\
Commutative law: $(a+bi)+(c+di)=(c+di)+(a+bi)$\\
\\
\textbf{Exercise \ref{exercise:complex:14}:}\\
Real number 0 does not have multiplicative inverse because any number times 0 will be 0.\\
\\
\textbf{Exercise \ref{exercise:complex:15}:}\\
Multiplicative identity: $(a+bi)(1+0i)=(1+0i)(a+bi)=(a+bi)$\\
Multiplicative inverse: $$(a+bi)\frac{a-bi}{a^{2}+b^{2}}= \frac{a-bi}{a^{2}+b^{2}}(a+bi)=1$$\\
Show similar examples for associative law and commutative law.\\
\\
\textbf{Exercise \ref{exercise:complex:16}:}\\
Use FLOI for $[(a+bi)(c+di)](e+fi)$ and $(a+bi)[(c+di)(e+fi)]$ to show that they are equal.\\
\\
\textbf{Exercise \ref{exercise:complex:18}:}\\
a. $-i$\\
\\
b. $ (4-5i)-(\overline{4i-4})=8-i$\\
\\
c. $(9-i)(\overline{9-i})=82$\\
\\
e. $16i$\\
\\
f. $(\overline{\sqrt{3}-i})^{-1}=\displaystyle\frac{\sqrt{3}-i}{10}$\\
\\
g. $\overline{(\sqrt{3}-i)^{-1}}=\displaystyle\frac{\sqrt{3}-i}{10}$\newline
\\
h. $(\overline{(\overline{4-9i})^{-1}})^{-1}=4-9i$\\
\\
i. $(a+bi)(\overline{a+bi})=a^{2}+b^{2}$\\
\\
j. $(a+bi)+(\overline{a+bi})=2a$\\
\\
\textbf{Exercise \ref{exercise:complex:cxprops}:}\\
Let $z=a+bi$ and $w=c+di$ for the following problems:\\
b. $(\bar{z})(\bar{w})=(a-bi)(c-di)=...=(\overline{zw})$\\
\\
e. $z \bar{z}=(a+bi)(a-bi)=a^{2}-b^{2}i^{2}=...=\left|z\right|^{2}$\\
\\
f. $\left|zw\right|=\left|(a+bi)(c+di)\right|=...=\left|ac-bd+(ad+bc)i\right|= \ldots \\
=\sqrt{(a^{2}+b^{2})(c^{2}+d^{2})}=\left|z\right|\left|w\right|$\\
\\
g. $\left|z^{3}\right|=\left|z\cdot z \cdot z\right|=...=\left|z\right|\cdot\left|z\right|\cdot\left|z\right|=\left|z\right|^{3}$, and use result from problem (e) above.\\
\\
i. $\left|z^{-1}\right|=\left|\displaystyle\frac{a-bi}{a^{2}+b^{2}}\right|=...=\displaystyle\frac{1}{\left|z\right|}$\\
\\
j. $(\bar{z})^{-1}=(a-bi)^{-1}=...=\overline{(z^{-1})}$\\
\\
\textbf{Exercise \ref{exercise:complex:abs1}:}\\
Let $z=a+bi$ for the following calculations:\\
a. If $z$ is a pure real number $\rightarrow b=0 \rightarrow \bar{z}=a-0i=a=z$.\\
If $\bar{z}=z \rightarrow a-bi=a+bi \rightarrow ... \rightarrow b=0 \rightarrow $ z is a pure real number.\\
\\
b. If $z$ is pure imaginary $\rightarrow a=0 \rightarrow ... \rightarrow \bar{z}=-z$.\\
If $\bar{z}=-z \rightarrow (a-bi)=-(a+bi) \rightarrow ... \rightarrow a=0 \rightarrow $ z is pure imaginary.\\
\\
\textbf{Exercise \ref{exercise:complex:21}:}\\
\\
a. $2\cis(\displaystyle\frac{\pi}{6})=2\cos\frac{\pi}{6}+2i\sin\frac{\pi}{6}=...=\sqrt{3}+i$\\
\\
c. $-3$\\
\\
e. $\displaystyle\frac{\sqrt{2}}{2}-\frac{\sqrt{6}}{2}i$\\
\\
f. $-\displaystyle\frac{\sqrt{21}}{14}+\frac{\sqrt{7}}{14}i$\\
\\
g. $14i$\\
\\
\textbf{Exercise \ref{exercise:complex:22}:}\\
\\
a. $\sqrt{2}\cis\displaystyle\frac{7\pi}{4}$\\
\\
e. $2\sqrt{2}\cis\displaystyle\frac{5\pi}{4}$\\
\\
f. $2\cis\displaystyle\frac{\pi}{6}$\\
\\
g. $3\cis\displaystyle\frac{3\pi}{2}$\\
\\
i. $2\sqrt{3}\cis\displaystyle\frac{7\pi}{4}$\\
\\
k. $10\cis\displaystyle\frac{5\pi}{4}$\\
\\
\textbf{Exercise \ref{exercise:complex:23}:}\\
a. It's a disk (interior of the circle centered at the origin with radius 2).\\
\\
b. It's a circle centered at $(0,0)$ with radius 5.\\
\\
c. It's a circle centered at (0,1) with radius 2.\\
\\
d. It's a circle centered at (3,0) with radius 3.\\
\\
\textbf{Exercise \ref{exercise:complex:24}:}
\begin{align*}
z.w & = (r\cis\theta).(s\cis\phi)\\
&= r(\cos\theta+i\sin\theta).s(\cos\phi+i\sin\phi)\\
&= rs(\cos\theta+i\sin\theta)(\cos\phi+i\sin\phi)\\
&= ...\\
&= rs[\cos(\theta+\phi)+i\sin(\theta+\phi)]\\
&= rs\cis(\theta+\phi)
\end{align*}
\textbf{Exercise \ref{exercise:complex:polar_z_inv}:}\\
\\
a. $w=z^{-1}=...=\displaystyle\frac{1}{13}\cis\frac{9\pi}{7}$
Sum of the argument of $z$ and $w$ is $2\pi$.\\
\\
b. $w=\frac{8}{3}\cis(1.61\pi)$ 
Sum of the argument of $z$ and $w$ is $2\pi$.\\
\\
c. Let $z=r\cis\theta$ and $w=s\cis\phi$\\
$z^{-1}=...=\displaystyle\frac{1}{r}\cis(-\theta)=\frac{1}{r}\cis(2\pi-\theta)$\\
So if $w=z^{-1}$ then $s=\displaystyle\frac{1}{r}$ and $\phi=2\pi-\theta$.\\
Therefore, if $s=\displaystyle\frac{1}{r}$ and $\phi=2\pi-\theta$ we will have $z\cdot w=1$.\\
\\
\textbf{Exercise \ref{exercise:complex:25}:}\\
\\
a. -1\\
\\
b. $...=14\cdot \displaystyle\frac{1}{7}\cis\frac{10\pi}{5}=2$\\
\\
c. -6\\
\\
d. $...=42\sqrt{2}\cis\displaystyle\frac{13\pi}{60}$\\
\\
\textbf{Exercise \ref{exercise:complex:26}:}\\
\\
a. $...=\displaystyle\frac{5}{4}+\frac{5\sqrt{3}}{4}i$\\
\\
c. $...=8\cis\displaystyle\frac{\pi}{12}$\\
\\
e. $...=\displaystyle\frac{3}{2}\cis\displaystyle\frac{\pi}{6}$\\
\\
\textbf{Exercise \ref{exercise:complex:27}:}
\begin{align*}
[r\cis\theta]^{2} & = (r\cis\theta)(r\cis\theta)\\
& = rr\cis(\theta+\theta)\\
& = r^{2}\cis(2\theta)\\
\end{align*}
\textbf{Exercise \ref{exercise:complex:28}:}\\
Use the proposition and result from the previous Exercise, we'll have $[r\cis\theta]^{3}=r^{3}\cis(3\theta)$.\\
\\
\textbf{Exercise \ref{exercise:complex:31}:}\\
\\
c. $...=2^{5}\cis\displaystyle\frac{5\pi}{6}=...=-16\sqrt{3}+16i$\\
\\
e. $...=\displaystyle\frac{4\cis(7\pi)}{16}=-\frac{1}{4}$\\
\\
f. $...=2^{12}\cis(15\pi)=-2^{12}$\\
\\
\textbf{Exercise \ref{exercise:complex:cos form}:}\\
a. Use Moivre's Theorem: $[r\cis\theta]^{n}=r^{n}\cis(n\theta)$\\
$z^{3}=[r\cis\theta]^{3}=r^{3}\cis(3\theta)$. Therefore:
\begin{align*}
\cos(3\theta)+i\sin(3\theta) & =[\cos\theta+i\sin\theta]^{3}\\
& = ...\\
& = \cos^{3}\theta-3\sin^{2}\theta\cos\theta+i(3\cos^{2}\theta\sin\theta-\sin^{3}\theta)\\
& = \cos\theta(\cos^{2}\theta-3\sin^{2}\theta)+i\sin\theta(3\cos^{2}\theta-\sin^{2}\theta)\\
\end{align*}
Apply real part = real part and imaginary part = imaginary part, we have:\\
$\cos(3\theta)=\cos\theta(\cos^{2}\theta-3\sin^{2}\theta)$\\
$\sin(3\theta)=\sin\theta(3\cos^{2}\theta-\sin^{2}\theta)$\\
\\
b. Use result from (a) to prove $\cos(3\theta)=...=4\cos^{3}\theta-3\cos\theta $.\\
\\
\textbf{Exercise \ref{exercise:complex:cos form2}:}\\
b. Polar representation of $\bar{z}$ is $r\cis(-\theta)$ or $r\cis(2\pi-\theta)$.\\
\\
\textbf{Exercise \ref{exercise:complex:49}:}\\
a. $1, -\displaystyle\frac{1}{2}+\displaystyle\frac{\sqrt{3}}{2}i, \displaystyle\frac{1}{2}-\displaystyle\frac{\sqrt{3}}{2}i$\\
\\
