\section{Hints for ``Sigma Notation'' exercises}\label{sec:sigma:hints} 


\noindent Exercise \ref{exercise:Sigma:trace3}: Notice that the product ${AB}$ is in both terms.

\noindent Exercise \ref{exercise:Sigma:linalg}(b): Use one of the previous exercises.

\noindent Exercise \ref{exercise:Sigma:KLC}: There are two possibilities to consider, $i=j$ and $i \neq j$.

\noindent Exercise \ref{exercise:Sigma:split}(a): Hint: Make a table for all possible values of $i,j,k$.

\noindent Exercise \ref{exercise:Sigma:split}(b): Multiply the equation you found in (a) by $a_{ijk}$ and sum over all $i,j,k$.


\noindent Exercise \ref{exercise:Sigma:detTrans}: $\epsilon_{ijk}$ is the sign of the permutation $\left( \begin{smallmatrix}  1 & 2 & 3  \\ i & j & k  \end{smallmatrix} \right)$. Note that the inverse of this permutation is $ \left( \begin{smallmatrix}  i & j & k  \\ 1 & 2 & 3  \end{smallmatrix} \right)$.  (It doesn't matter what order the indices are written in the top row.) How are the signs of the two permutations related? 


\noindent Exercise \ref{exercise:Sigma:EqualZero}: Replace $\epsilon_{xy}$ with $-\epsilon_{yx}$, and show that the expression is equal to the negative of itself. (Alternatively, you can just verify the two cases:  $i=j=1$ and $i=j=2$.)


\noindent Exercise \ref{exercise:Sigma:last}: First show that:  $\sum_{k=1}^n 2^k = 2^{n+1} - 2$.
