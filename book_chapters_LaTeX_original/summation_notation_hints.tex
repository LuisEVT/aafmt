\section{Hints for ``Sigma Notation'' and ``Applications of Sigma Notation''  exercises}
\label{sec:SigmaNotation:Hints} 

\noindent Exercise \ref{exercise:SigmaNotation:nested2}(e): This is a more difficult one.  Exchange order of summation. You will need to use a summation formula from the next section.  The denominator factors as the difference of squares.  Part of the final answer will look like $1 + 1/2 + 1/3 + \ldots + 1/19$, which you can evaluate using a spreadsheet or some other method. 

\noindent Exercise \ref{exercise:SigmaApp:detId}: Both sides are 0 if any of the two indices $i,j,k$ are equal (show this). Then you only need to consider the three possible cases where $i,j,k$ are all unequal.

\noindent Exercise \ref{exercise:SigmaApp:nonzeroVec}(a): You will need to change around the indices in the formula from Exercise~\ref{exercise:SigmaApp:2epsIdent}. Make the following replacements: $k \rightarrow i, i \rightarrow j, j \rightarrow k$.  Then use the fact that $\epsilon_{ijk}= \epsilon_{jki}$ (see Exercise~\ref{exercise:SigmaApp:KLC}.  to obtain
\[\sum_i \epsilon_{ijk} \epsilon_{imn} = \delta_{jm} \delta_{kn} - \delta_{jn} \delta_{km}. \]
You may plug this form into the expression on the left-hand side. You then obtain 2 terms, which you can evaluate separately. Summing over a delta  eliminates one of its two indices: for example:
\[ \sum_{j,k,m,n} \delta_{jm} \delta_{kn}  y_j y_m z_k z_n  = \sum_{j,k} y_j y_j z_k z_k, \]
since the only $m$ term that contributes is $m=j$, and the only $n$ term that contributes is $n=k$. From there, it's a short hop to the expression with inner products.

In order to get the expression with $\sin \theta$, you will need the cosine formula for inner products (see Section~\ref{sec:SigmaApp:RotationMatrix3D}).

\noindent Exercise \ref{exercise:SigmaApp:sigmaAssoc}: Write matrices $G$ and $H$ from parts (b) and (c) in terms of $A,B$, and $C$.

\noindent Exercise \ref{exercise:SigmaApp:trace3}: Notice that the product ${AB}$ is in both terms. So for simplicity you can define $M := AB$, and use a previous result.

\noindent Exercise \ref{exercise:SigmaApp:linalg}(a) You don't need summation notation here, just use basic properties of inverses. (b): Use one of the previous exercises.

\noindent Exercise \ref{exercise:SigmaApp:KLC}: There are two possibilities to consider, $i=j$ and $i \neq j$.

\noindent Exercise \ref{exercise:SigmaApp:split}(a): Hint: Make a table for all possible values of $i,j,k$.

\noindent Exercise \ref{exercise:SigmaApp:split}(b): Multiply the equation you found in (a) by $a_{ijk}$ and sum over all $i,j,k$.


\noindent Exercise \ref{exercise:SigmaApp:detTrans}: Notice that $a_{1, \phi(1)} a_{2, \phi(2)} a_{3, \phi(3)}$ is equal to $a_{\phi^{-1}(1), 1} a_{\phi^{-1}(2), 2} a_{\phi^{-1}(3), 3}$, and that sign($\phi$) is equal to sign($\phi^{-1}$). 

\noindent Exercise \ref{exercise:SigmaApp:EqualZero}: Replace $\epsilon_{xy}$ with $-\epsilon_{yx}$, and show that the expression is equal to the negative of itself. (Alternatively, you can just verify the two cases:  $i=j=1$ and $i=j=2$.)


