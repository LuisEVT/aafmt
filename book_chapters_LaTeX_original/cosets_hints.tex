\section{Hints for ``Cosets'' exercises}
\label{sec:Cosets:Hints} 

\noindent Exercise \ref{exercise:Cosets:left_right_cosets}: 

For part (f), consider that any even permutation times $A_4$ will produce a set of even permutations with 12 elements (Why?), and there are exactly 12 even permutations in $S_4$.  Similar reasoning applies when you take any odd permutation times $A_4$.

For part (g), generalize your result in part (f).

For part (h), your task will be simplified if you notice that all elements in a given coset produce the same coset. So once you've found a coset, you don't need to do any work to find the cosets of elements in the coset that you've found.


\noindent Exercise \ref{exercise:Cosets:abelian_cosets}:  (\emph{Hint}: You're trying to show that the two sets $gH$ and $Hg$ are equal. One way to do this is to show every element of $gH$ is an element of $Hg$, and vice versa.)


\noindent Exercise \ref{exercise:Cosets:CosetEquiv}(a): The hypothesis $g_1 H = g_2 H$ implies that there exists $h \in H$ such that $g_1 h  = g_2 e$, where $e$ is the group identity. 

\noindent Exercise \ref{exercise:Cosets:CosetEquiv}(b):   $g_1^{-1} g_2 \in H$ means that $g_1^{-1} g_2 = h$ for some $h \in H$. 

\noindent Exercise \ref{exercise:Cosets:CosetEquiv}(c):   You need to show that $g_2 H \subset g_1 H$. From (3), deduce that $g_2 = g_1 h$ for some $h \in H$. Then, show that any element of the form $g_2 h'$ for $h' \in H$ can be expressed as $g_1 h''$ where $h'' \in H$. You should be able to express $h''$ in terms of $h$ and $h'$.

\noindent Exercise \ref{exercise:Cosets:CosetEquiv}(d):   You need to show that (4) implies $g_1H \subset g_2H$. It's enough to show that for any $h \in H$,  $g_1 h \in g_2 H$. To do this, express $g_1$ in terms of $g_2$. 

\noindent Exercise \ref{exercise:Cosets:CosetEquiv}(e):    Condition (2) implies that $g_1^{-1} g_2 = h$ for some $h \in H$.

\noindent Exercise \ref{exercise:Cosets:SL2_cosets}:  You may use the equivalence of conditions (3) and (2) in Proposition~\ref{proposition:Cosets:cosets_theorem_1}.  You will also need the following facts about determinants:  (a) $\det(AB) = \det(A)\det(B)$ and (b) $\det(A^{-1}) = 1/\det(A)$  (note that (b) follows from (a)).

\noindent Exercise \ref{exercise:Cosets:order5and7}:  Remember cyclic subgroups.

\noindent Exercise \ref{exercise:Cosets:phivals}:   For part (g), use the fact that the numbers less than $p^2$ that are \emph{not} relatively prime to $p^2$ are $p, 2p, 3p, \ldots (p-1)p$: how many numbers remain? For parts (h) and (i) use a similar logic.
 
\noindent Exercise \ref{exercise:Cosets:primeGroups}:
You may refer to Proposition~\ref{proposition:Groups:OrderEltCyclic}.

\noindent Exercise \ref{exercise:Cosets:SL2_normal}:
Look back at your work on Exercise~\ref{exercise:Cosets:SL2_cosets}.

\noindent Exercise \ref{exercise:Cosets:i_normal}:
Compute the left and right cosets.

\noindent Exercise \ref{exercise:Cosets:abelian_normal}: Use  Exercise~\ref{exercise:Cosets:abelian_cosets} earlier in this chapter.

\noindent Exercise \ref{exercise:Cosets:ghg-1_prove}: Let $H$ be a subgroup of the group $G$ that satisfies the property that  for any $g \in G$ and any $h \in H$, then $ghg^{-1}$ is also in $H$. Show that every right coset of $H$ in $G$ is also a left coset, and vice versa (and hence $H$ is a normal subgroup of $G$.

\noindent Exercise \ref{exercise:Cosets:normalk}:   Use part (a) and Definition~\ref{definition:Cosets:normal_alt}.

\noindent Exercise \ref{exercise:Cosets:normal_mult}(c):  We may write $x_1 = g_1 h_1$ and $x_2 = g_2 h_2$, so that $x_1 x_2 = g_1 h_1 g_2 h_2$.  Use part (b) with $h=h_1$, $g=g_2$.

\noindent Exercise \ref{exercise:Cosets:Sn_NotSimple}:  $S_n$ has a subgroup of index 2.  This shows $S_n$ is not simple  (why?).
\medskip

\textbf{Additional exercises}

\noindent Exercise \ref{eoc:Cosets:1}:  Define an equivalence relation on $G$ as follows: $g_1 \sim g_2$ if and only if either $g_1 = g_2$ or $g_1 = g_2^{-1}$. Prove that this is indeed an equivalence relation; and show that the equivalence class of $g$ has an odd number of elements if and only if $g = g^{-1}$. Use the partition of $G$ to show that there must be an even number of  equivalence classes with an odd number of elements (including the equivalence class of the identity).

