\section{Solutions for ``Functions''}
\noindent\textbf{\textit{ (Chapter \ref{functions}})}\bigskip

\textbf{Exercise \ref{exercise:functions:ExOrderPairs}:}\\
$A=\{a,b,c,d\}$, $B=\{1,3,5,7,9\}$\\
b. function from $A \implies B$\\
d. function from $A \implies B$\\
f. function from $A \implies B$\\
h. not a function from $A \implies B$\\
\\
\textbf{Exercise \ref{exercise:functions:23}:}\\
a. Domain = $\{2,4,6,8,10\}$\\
b. Range = $\{7,9,11,13,15\}$\\
c. $g(6)=11$\\
d. $g(7)$ is undefined\\
e. $g=\{(2,7),(4,9),(6,11),(8,13),(10,15)\}$\\
g. $g(n)=n+5$\\
\\
\textbf{Exercise \ref{exercise:functions:27}:}\\
b. $A=\{a,b\}$, $B=\{c,d\}$\\
$f=\{(a,c),(b,d)\}$ Range: $\{c,d\}$\\
$g=\{(a,d),(b,c)\}$ Range: $\{d,c\}$\\
$h=\{(a,c),(b,c)\}$ Range: $\{c,c\}$\\
$k=\{(a,d),(b,d)\}$ Range: $\{d,d\}$\\
\\
\textbf{Exercise \ref{exercise:functions:11Exers}:}\\
c. $h(x)=x^2$\\
$h(x)$ is not one-to-one because $f(-1)=f(1)=1$ while $-1\neq 1$\\
d. $i(x)=3x+2$\\
Let $x_1,x_2 \in R$. If $f(x_1)=f(x_2)$, then we have:
$3x_1+2=3x_2+2 \implies x_1=x_2$\\
$\implies i(x)$ is a one-to-one function.\\
\\
%\textbf{Exercise 37:}\\
%a. $f$ is one-to-one because if $x_1\neq x_2$ then we have $f(x_1)\neq f(x_2)$ (contrapositive).\\
%b. $g$ is not one-to-one because $g(2)=g(3)=d$ while $2\neq 3$.\\
%\\
\textbf{Exercise \ref{exercise:functions:40}:}\\
e. $g: Z_8 \implies Z_8$ with $g(x)=x\odot 2$\\
$g(x)$ is not one-to-one because we have $g(1)=g(5)=2 \pmod{8}$ while $1\neq 5$.\\
f. $g: Z_7 \implies Z_7$ with $g(x)=x\odot 2$\\
$g(x)$ is one-to-one function because if $x_1\neq x_2$, then we will have $g(x_1)\neq g(x_2)$.\\
\\
\textbf{Exercise \ref{exercise:functions:45}:}\\
If the function $f$ is onto then the range of $f$ equals the codomain of $f$. There is no element in the codomain that is not in the range.\\
\\
\textbf{Exercise \ref{exercise:functions:OntoExers}:}\\
c. $c(x)=x^2$\\
$c(x)$ is not onto because there is a number $-2 \in R$ in the codomain that doesn't map to any element in the domain.\\
d. $d(x)=3x+2$\\
$d(x)$ is onto because for any element $m$ in the codomain, we can find\\ $x=\displaystyle\frac{m-2}{3}$ in the domain that maps to $m$.\\
\\
\textbf{Exercise \ref{exercise:functions:OntoExers-pairs}:}\\
a. It's onto. For any element in the codomain, there is an element in the domain thap maps to it.\\
b. It's not onto. Element $\diamondsuit$ doesn't map to any element in the domain.\\
\\
\textbf{Exercise \ref{exercise:functions:BijectRtoRExer}:}\\
d. $d(x)=-15x-12$\\
$d(x)$ is one-to-one because if $d(x_1)=d(x_2)$ then $x_1=x_2$.\\
$d(x)$ is onto because for any element y in the codomain, we have $x=\displaystyle\frac{y+12}{-15}$ in the domain that maps to $y$.\\
$\implies d(x)$ is a bijection.\\
e. $e(x)=x^3$\\
Similarly, we have $e(x)$ is both one-to-one and onto $\implies e(x)$ is a bijection.\\
\\
%\textbf{Exercise 62:}\\
%$f: R\implies R$ by $f(x)=ax+b$ with $a,b \in R$\\
%a. When $a\neq 0$: $f(x)$ is a bijection.\\
%b. When $a=0$: $f(x)$ is not a bijection.\\
%\\
\textbf{Exercise \ref{exercise:functions:RealWorldCompositionExer}:}\\
b. husband $\circ$ mother : father\\
e. mother $\circ$ sister : mother\\
f. daughter $\circ$ sister : niece\\
\\
\textbf{Exercise \ref{exercise:functions:func_comp_assoc}:}\\
$(h\circ(g\circ f))(x)=h(g(y))=h(w)=z$\\
$((h\circ g)\circ f)(x)=...=z$\\
$\implies (h\circ(g\circ f))(x)=((h\circ g)\circ f)(x)$\\
\\
\textbf{Exercise \ref{exercise:functions:ComposeExers-form}:}\\
a. $f(x)=3x+1$ and $g(x)=x^2+2$\\
$(f\circ g)(x)=3x^2+7$\\
$(g\circ f)(x)=9x^2+6x+3$\\
b. $f(x)=3x+1$ and $g(x)=\displaystyle\frac{x-1}{3}$\\
$(f\circ g)(x)=x$\\
$(g\circ f)(x)=x$\\
c. $f(x)=ax+b$ and $g(x)=cx+d$\\
$(f\circ g)(x)=acx+ad+b$\\
$(g\circ f)(x)=cax+cb+d$\\
\\
\textbf{Exercise \ref{exercise:functions:ComposeExers-pairs}:}\\
a. $g\circ f=\{(1,\clubsuit),(2,\diamondsuit),(3,\heartsuit),(4,\spadesuit)\}$\\
b. $g\circ f=\{(1,\clubsuit),(2,\clubsuit),(3,\clubsuit),(4,\clubsuit)\}$\\
\\
\textbf{Exercise \ref{exercise:functions:BijectionComposeExer}:}\\
$f: A\implies B$ and $g: B\implies C$\\
a. $f$ and $g$ are bijection $\implies$  $f$ and $g$ are one-to-one $\implies g\circ f$ is one-to-one (previous Exercise).\\
$f$ and $g$ are onto $\implies g\circ f$ is onto.\\
$\implies g\circ f$ is bijection.\\
\\
\textbf{Exercise \ref{exercise:functions:88}:}\\
Prove part (b) of previous Example\\
For any $y\in Y$, from the supposition, we know that $\exists x\in X$ such that $g(y)=x$ and for that particular $x$, we also have $f(x)=y$.\\
So $g(f(x))=g(y)=x$ for all $x\in X$\\
\\
\textbf{Exercise \ref{exercise:functions:VerifyInverseExers}:}\\
a. We have $f(g(x))=x+6-6=x$ and $g(f(x))=...=x$\\
$\implies g$ is an inverse of $f$.\\
c. $f(g(x))=g(f(x))=x \implies g$ is an inverse of $f$.\\
\\
\textbf{Exercise \ref{exercise:functions:92}:}\\
$g$ has an inverse.\\
\\
\textbf{Exercise \ref{exercise:functions:IdAInverse}:}\\
a. $Id_A$ is one-to-one because if $Id_A(a_1)=Id_A(a_2)$ then $a_1=a_2$.\\
$Id_A$ is onto because for any $a\in$ codomain $A$, we can always have $a\in$ domain $A$ that maps to $a$.\\
$\implies Id_A$ is bijection $\implies Id_A$ is invertible (Prop. 89)\\
b. $Id_A$ is its own inverse.\\
\\
\textbf{Exercise \ref{exercise:functions:InverseIdentityExers}:}\\
a. $f: A\implies B$ and $g: B\implies C$ are bijection\\
$\implies g\circ f$ is bijection, $f$ has an inverse, and $g$ has an inverse.\\
Let $a\in A$, $b\in B$, $c\in C$ and $f(a)=b$ and $g(b)=c$\\
$\implies (g\circ f)(a)=c$\\
$\implies (g\circ f)^{-1}(c)=a$\\
We also have $(f^{-1}\circ g^{-1})(c)=f^{-1}(b)=a$\\
$\implies (g\circ f)^{-1}=f^{-1}\circ g^{-1}$\\
b. Let $x\in X$ and $y\in Y$\\
If $g$ is the inverse of $f$, we will have $f(x)=y$ and $g(y)=x$\\
$\implies (f\circ g)(y)=f(g(y))=f(x)=y=Idy$ and $(g\circ f)(x)=g(y)=x=Idx$\\
On the other hand, if we have $(f\circ g)(y_1)=Idy_1=y_1$ where $g(y_1)=x_1 \implies f(x_1)=y_1$.\\
And if $(g\circ f)(x)=Idx \implies (g\circ f)(x_1)=...=x_1$\\
So we have $g$ is the inverse of $f$ if and only if $f\circ g=Idy$ and $g\circ f=Idx$\\
c. $f: X\implies Y$ is a bijection $\implies f$ is both one-to-one and onto\\
Let $x\in X$ and $y\in Y$ where $f(x)=y$\\
$\implies f$ has an inverse $f^{-1}(y)=x$\\
$[f^{-1}(y)]^{-1}=y=f(x)$\\
Therefore $(f^{-1})^{-1}=f$.\\