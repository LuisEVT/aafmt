
\section{Solutions for  ``Abstract Groups''}
\noindent\textbf{\textit{ (Chapter \ref{Groups})}}\bigskip

\noindent\textbf{Exercise \ref{exercise:Groups:trivial}:}
%Prove that the trivial group is in fact a group according to Definition~\ref{definition:Groups:group_definition}.
\\
A trivial group $(T, \circ )$, containing only the element $e$, is a group $T$ under the law of composition because it satisfies the four group axioms: 

\begin{itemize}
\item
The set $T$ is closed under the law of composition because $e \circ e $ is equal to $e$ which is also in $T$.

\item
The element $e \in T$ is the identity element of $T$ since $e \circ e = e$.

\item
For each element $e \in T$, there exists an inverse element in $T$, because $e \circ e = e$.

\item
The law of composition is associative because $(e \circ e) \circ e = e \circ (e \circ e) = e$.
\end{itemize}
\medskip

\noindent\textbf{Exercise \ref{exercise:Groups:Rstar_group}:}
\begin{enumerate}[(a)]
\item
%Finish proving that $({\mathbb R}^{\ast}, \cdot)$ is a group.
$({\mathbb R}^{\ast}, \cdot)$ is a group.
\begin{itemize}
	\item
	${\mathbb R}^{\ast}$ is closed under multiplication. (by Example~\ref{example:Groups:Rstar_group}) 

	\item
	${\mathbb R}^{\ast}$ has 1 as the identity element. (by Properties of arithmetic operations~\ref{subsec:Preliminaries:OpsAndRels}(A))

	\item
	Each element has its own multiplicative inverse. (by  Properties of arithmetic operations\ref{subsec:Preliminaries:OpsAndRels}(C))

	\item
	Multiplication is associative. (by Properties of arithmetic operations~\ref{subsec:Preliminaries:OpsAndRels}(D))
	\end{itemize}
	
\item
%Either prove or disprove that $({\mathbb R}^{\ast}, +)$ is a group.
$({\mathbb R}^{\ast}, +)$ is not a group because it's not closed under addition. For example, we have $1 + (-1) = 0 \not\in {\mathbb R}^{\ast}$.

\item
%What is the order of $({\mathbb R}^{\ast}, \cdot)$?
$({\mathbb R}^{\ast}, \cdot)$ has infinite order.
\end{enumerate}

\noindent\textbf{Exercise \ref{exercise:Groups:C_star}:}
%Let ${\mathbb C}^{\ast} $\label{noteCstar} be the set of non-zero complex numbers. 
\begin{enumerate}[(a)]
\item
%Why is ${\mathbb C}^{\ast}$ not a group under the operation of complex addition?
${\mathbb C}^{\ast}$ is not a group under addition because it's not closed. For example, 
$3i + (-3i) = 0 \not\in {\mathbb C}^{\ast}$.

\item
%Prove ${\mathbb C}^{\ast}$ is a group under the operation of (complex) multiplication.
${\mathbb C}^{\ast}$ is a group under multiplication.
	
	\begin{itemize}
	\item
	${\mathbb C}^{\ast}$ is closed under multiplication. \quad \quad (by Excercise \ref{exercise:ComplexNumbers:15})

	\item
	$1 \in C$ is the identity element. \quad \quad (by Excercise \ref{exercise:ComplexNumbers:15})

	\item
	Each element in ${\mathbb C}^{\ast}$ has its own inverse. \quad \quad (by Excercise \ref{exercise:ComplexNumbers:15})

	\item
	Multiplication is associative. \quad \quad (by Excercise \ref{exercise:ComplexNumbers:15})
	\end{itemize}
	
\item
%What is $| ({\mathbb C}^{\ast}, \cdot) |$?
$|({\mathbb C}^{\ast}, \cdot)|$ has infinite order.

\item
%Is $({\mathbb C}^{\ast}, \cdot)$ an abelian group?  Justify your answer.
Yes, by Exercise \ref{exercise:ComplexNumbers:15} complex numbers are abelian (commutative).
\end{enumerate} 

\noindent\textbf{Exercise \ref{exercise:Groups:15}:}
\begin{enumerate}[(a)]
\item
%Why is it impossible for a set of complex numbers $S$ which has more than one element and  includes 0 to be a group under multiplication?  Why is the condition $|S|>1$ necessary?
Because 0 times anything is 0, it follows that 0 can only have an inverse if 0 is the multiplicative identity. But since 0 times anything is 0, 0 can only be the multiplicative identity if 0 is the only element. So $S$ cannot have more than 2 elements.
 $S = \{0\}$ is a group under multiplication. In this case, $0$ is the identity element.

\item
%Why is it impossible for a set of complex numbers $S$ that excludes 0 to be a group under addition?
Because $S$ doesn't have the identity element: for any elements $a,b \in S, a+b \neq a$ so there is no element of $S$ that can be the additive identity.
\end{enumerate}

\noindent\textbf{Exercise \ref{exercise:Groups:S_group}:}
%Finish the proof that $(S, \ast)$ is an abelian group.
\\
- Identity element:\\
Suppose $a,e\in S$: $a*e=a+e+ae=a$\\
$\implies e+ae=0$\\
$\implies e(1+a)=0$\\
$\implies e=0$ (because $S=R\setminus\{-1\}$)\\
$\implies e=0$ is the identity element.\\
- Inverse: Suppose $a\in S$: $a*a^{-1}=0$\\
$\implies a+a^{-1}+aa^{-1}=0$\\
$\implies a^{-1}=\displaystyle\frac{-a}{1+a}\neq -1$\\
$\implies$  each element has its own inverse.\\
- Associative:\\
$(a*b)*c=(a+b+ab)*c=(a+b+ab+c)+(a+b+ab)c$\\
$a*(b*c)=a*(b+c+bc)=(a+b+c+bc)+a(b+c+bc)=(a+b+ab+c)+(a+b+ab)c$\\
$\implies (a*b)*c=a*(b*c)$\\
$\implies (S,*)$ is an abelian group.\\
\\

\noindent\textbf{Exercise \ref{exercise:Groups:integer_coordinate_plane}:}
%Let $H = {\mathbb Z} \times {\mathbb Z}$ (all integer coordinate-pairs).
\begin{enumerate}[(a)]
\item
%Define a binary operation $\circ$ on $H$ by  $(a,b) \circ (c,d) = (a + c , b + d)$, for $(a,b), (c,d) \in H$.  This operation is in fact just coordinate-pair addition.  Is $(H , \circ)$ a group?  If so, is $(H , \circ)$ abelian?  Justify your answers.
	\begin{itemize}
	\item
	identity element: $(0, 0)$\\
	$(a, b) \circ (0, 0) = (a + 0, b + 0) = (a, b)$

	\item
	inverse: $(a, b) \circ (-a, -b) = (0, 0)$

	\item
	closed: $(a, b) \circ (c, d) = (a + c, b + d) \in {\mathbb Z} \times {\mathbb Z}$

	\item
	commutative: $(a, b) \circ (c, d) = (a + c, b + d) = (c + a, d + b) = (c, d) \circ (a, b)$

	\item
	associative:\\
	$[(a, b)\circ (c, d)]\circ (e, f) = (a + c, b + d)\circ (e, f) = (a + c + e, b + d + f)$\\
	$(a, b)\circ[(c, d)\circ (e, f)] = (a, b)\circ (c + e, d + f) = (a + c + e, b + d + f)$
	\end{itemize}

$\implies (H,\circ)$ is a group.\\
$\implies (H,\circ)$ is an abelian group because it has commutative property.

\item
%Define a binary operation $\circ$ on $H$ by $(a,b) \circ (c,d) = (ac , bd)$, for $(a,b), (c,d) \in H$.  This is just coordinate-pair multiplication.  Is $(H , \circ)$ a group?  If so, is $(H , \circ)$ abelian? Justify your answers.
Similar to part (a), we have $(H,\circ$ is an abelian group because it has identity element, inverse, and is closed, associative, and commutative.
\end{enumerate}

\noindent\textbf{Exercise \ref{exercise:Groups:19}:}
%Let $G = {\mathbb R}^{\ast} \times {\mathbb Z}$ (all pairs such that the first element is a nonzero real number, and the second is an integer) 
\begin{enumerate}[(a)]
\item
%Define a binary operation $\circ$ on $G$ by $(a,m) \circ (b,n) = (a+b, m+n)$.  Is $(G , \circ)$ a group?  If so, is $(G , \circ)$ abelian?  Justify your answers.
$(G, \circ)$ is not a group since it doesn't have identity element (since $ 0 \notin {\mathbb R}^*$). Alternatively, it is not closed under the operation.

\item
%Define a binary operation $\circ$ on $G$ by $(a,m) \circ (b,n) = (ab, mn)$.  Is $(G , \circ)$ a group?  If so, is $(G , \circ)$ abelian?  Justify your answers.
$(G, \circ)$ is not a group because it doesn't satisfy the inverse element requirement. For example, (1,2) has no inverse element.

\item
%Define a binary operation $\circ$ on $G$ by $(a,m) \circ (b,n) = (ab, m+n)$.  Is $(G , \circ)$ a group?  If so, is $(G , \circ)$ abelian?  Justify your answers.
$(G, \circ)$ is an abelian group. Similar to Exercise \ref{exercise:Groups:integer_coordinate_plane}, we see that $(G, \circ)$ satisfies all 4 requirements to be a group and it's also commutative.
\end{enumerate}

\noindent\textbf{Exercise \ref{exercise:Groups:prod_of_groups}:}
\begin{enumerate}[(a)]
\item
%Consider $(3,6)$ and $(2,4)$ as elements of ${\mathbb Z}_7 \times {\mathbb Z}_7$. Compute 
%$(3,6) \circ (2,4)$.
$(g_1 g_2, h_1 h_2) \in {\mathbb Z}_7 \times {\mathbb Z}_7$ (no $^\ast$ so include 0 and do addition) \\
$(3 + 2, 6 + 4) = (5, 10) = (5, 3)$

\item
%Consider $(3,6)$ and $(2,4)$ as elements of ${\mathbb R}^{\ast} \times {\mathbb Z}_{10}$. Compute $(3,6) \circ (2,4)$.
$(g_1 g_2, h_1 h_2) \in {\mathbb R}^{\ast} \times {\mathbb Z_{10}}$ ($^\ast$ so exclude 0 and do multiplication) \\
$(3 \cdot 2, 6 + 4) = (6, 10) = (6, 0)$

\item 
%Consider $(3,6)$ and $(2,4)$ as elements of ${\mathbb Q}^{\ast} \times {\mathbb Q}^{\ast}$. Compute $(3,6) \circ (2,4)$.
$(g_1 g_2, h_1 h_2) \in {\mathbb Q}^{\ast} \times {\mathbb Q}^{\ast}$ ($^\ast$ so exclude 0 and do multiplication) \\
$(3 \cdot 2, 6 \cdot 4) = (6, 24)$
\end{enumerate}

\noindent\textbf{Exercise \ref{exercise:Groups:prod_group}:}
%Show that the product of two groups is a group.
\begin{itemize}
\item
identity element: $(e_g, e_h) \circ (g, h) = (g, h) \circ (e_g, e_h) = (g, h)$

\item
inverse: $(g, h) \circ (g^{-1}, h^{-1}) = (g g^{-1}, h h^{-1}) = (e, e)$\\
Similarly, $(g^{-1}, h^{-1}) \circ (g, h)  = ( g^{-1} g,  h^{-1}h) = (e, e)$

\item
closure: Definition \ref{definition:Groups:defProductOfGroups}

\item
associative: $((g_1, h_1) \circ (g_2, h_2)) \circ (g_3, h_3) = (g_1 g_2, h_1 h_2) \circ (g_3, h_3)$\\
$= (g_1 g_2 g_3, h_1 h_2 h_3)$\\
$= (g_1 (g_2 g_3), h_1 (h_2 h_3))$\\
$= (g_1, h_1) \circ (g_2 g_3, h_2 h_3)$
\end{itemize}

\textbf{Exercise \ref{exercise:Groups:Z_2_n}:}\\
Let $x,y\in Z_2=\{0,1\}$\\
$\implies x\oplus y\in Z_2$ (because $Z_2$ closed under modular addition) (1)\\
Let $Z_2^n=\{(a_1,a_2,...,a_n):a_i\in Z_2\}$\\
Let $a=\{(a_1,a_2,...,a_n):a_i\in Z_2^n\}$ and $b=\{(b_1,b_2,...,b_n):b_i\in Z_2^n\}$\\
Then we have $a+b=(a_1,a_2,...,a_n)+(b_1,b_2,...,b_n)=(a_1+b_1,a_2+b_2,...,a_n+b_n)$\\
According to (1), we have $a_1+b_1\in Z_2$, $a_2+b_2\in Z_2$,...,$a_n+b_n\in Z_2$\\
$\implies a+b\in Z_2^n$\\
$\implies Z_2^n$ is closed under this operation.\\
(i) identity element is $(0,0,...,0)\in Z_2^n$\\
(ii) We have $(a_1,a_2,...,a_n)+(a_1,a_2,...,a_n)=(0,0,...,0)$\\
$\implies$  each element has its own inverse in $Z_2^n$\\
(iii) Associative: Let $a,b,c\in Z_2^n$, then we have:\\
$(a+b)+c =[(a_1,a_2,...,a_n)+(b_1,b_2,...,b_n)]+(c_1,c_2,...,c_n)$\\
$=(a_1+b_1,a_2+b_2,...,a_n+b_n)+(c_1,c_2,...,c_n)$\\
$=(a_1+b_1+c_1,a_2+b_2+c_2,...,a_n+b_n+c_n)$\\
$=(a_1,a_2,...,a_n)+(b_1+c_1,b_2+c_2,...,b_n+c_n)$\\
$=(a_1,a_2,...,a_n)+[(b_1,b_2,...,b_n)+(c_1,c_2,...,c_n)]$\\
$=a+(b+c)$\\
Therefore, $Z_2^n$ is a group under the defined operation.\\
\\


\noindent\textbf{Exercise \ref{exercise:Groups:ident}:}
%Given two elements $g,h$  of a group ($G,\circ$). Show that $g$ is the identity element of $G$ if and only if  $g \circ h = h$.
%\hyperref[sec:Groups:Hints]{(*Hint*)} 
\\
Given $g, h \in (G, \circ)$\\
($\implies$) Suppose $g = e$, which is the identity element.\\
Then, $g \circ h = h$, by definition of identity.\\
($\impliedby$) Suppose that $g \circ h = h$,\\
$(g \circ h) \circ h^{-1} = h \circ h^{-1}$ \quad right multiply by $h^{-1}$\\
$g \circ (h \circ h^{-1}) = h \circ h^{-1}$ \quad associativity\\
$g \circ e = e$ \quad definition of inverse \\
$g = e$ \quad ~~~ definition of identity.
\\

\noindent\textbf{Exercise \ref{exercise:Groups:cayley}:}
%Show that if $G$ is a group, then for every row of the Cayley table for $G$ no two entries are the same. Show also that for every column of the Cayley table no two entries are the same.
%\hyperref[sec:Groups:Hints]{(*Hint*)}  
\\
(\emph{Rows})  Prove by contradiction. Suppose that for row $g$,  the entries in columns $h$ and $h'$ are the same, where $h \neq h'$.\\

\noindent
$g \circ h = g \circ h'$\\
$g^{-1} \circ (g \circ h) = g^{-1} \circ (g \circ h)$ \quad left multiply by $g^{-1}$\\
$(g^{-1} \circ g) \circ h = (g^{-1} \circ g) \circ h'$ \quad associativity\\
$e \circ h = e \circ h'$ \quad def of inverse\\
$h = h'$ \quad def identity\\

\noindent
This contradicts the supposition that $h \neq h'$.  Therefore, the entries in columns $h$ and $h'$ must be different.\\
\\
(\emph{Columns}) Prove by contradiction. Suppose that for column $h$,  the entries in rows $g$ and $g'$ are the same, where $g \neq g'$.\\

\noindent
$g \circ h = g' \circ h$\\
$(g \circ h) \circ h^{-1} = (g' \circ h) \circ h^{-1}$ \quad right multiply by $h^{-1}$\\
$g \circ (h \circ h^{-1}) = g' \circ (h \circ h^{-1})$ \quad associativity\\
$g \circ e = g' \circ e$ \quad def of inverse\\
$g = g'$ \quad def identity\\

\noindent
This contradicts the supposition that $g \neq g'$.  Therefore, the entries in rows $g$ and $g'$ must be different.\\
\\

\noindent\textbf{Exercise \ref{exercise:Groups:Cayley_groups}:}
%For each of the following multiplication tables defined on the set $G = \{ a, b, c, d \}$ tell whether $(G, \circ)$ represents a group, and if so,  whether it is abelian.  Support your answer in each case.  Assume that the associative property holds in each case. \emph{Note} the identity is not always the first element listed!
%\begin{multicols}{2}
\begin{enumerate}[(a)]

\item
%\begin{center}
%\begin{tabular}{c|cccc}
%$\circ$ & $a$ & $b$ & $c$ & $d$ \\
%\hline
%$a$ & $a$ & $b$ & $c$ & $d$ \\
%$b$ & $b$ & $a$ & $d$ & $c$ \\
%$c$ & $c$ & $d$ & $a$ & $b$ \\
%$d$ & $d$ & $a$ & $b$ & $c$
%\end{tabular}
%\end{center}
%\smallskip
Since $a \circ a = a, b \circ a = b, c \circ a = c$, and $d \circ a = d$, then by definition of identity we know that $a$ is the identity element.  We see also that each element is it's own inverse by $a \circ a = a, b \circ b = a, c \circ c = a$, except this fails at $d \circ d = c$.  Since there cannot be two identity elements this is not a group.

\item
%\begin{center}
%\begin{tabular}{c|cccc}
%$\circ$ & $a$ & $b$ & $c$ & $d$ \\
%\hline
%$a$ & $b$ & $a$ & $d$ & $c$ \\
%$b$ & $a$ & $b$ & $c$ & $d$ \\
%$c$ & $d$ & $c$ & $b$ & $a$ \\
%$d$ & $c$ & $d$ & $a$ & $b$
%\end{tabular}
%\end{center}
%\smallskip
Since $a \circ b = a, b \circ b = b, c \circ b = c$, and $d \circ b = d$, then by definition of identity we know that $b$ is the identity element.  We see also that each element is it's own inverse by $a \circ a = b, b \circ b = b, c \circ c = b$, and $d \circ d = b$.  So this is a group.  It is an abelian group because all elements can be composed in either order and the solution is the same.

\item
%\begin{center}
%\begin{tabular}{c|cccc}
%$\circ$ & $a$ & $b$ & $c$ & $d$ \\
%\hline
%$a$ & $d$ & $c$ & $b$ & $a$ \\
%$b$ & $c$ & $d$ & $a$ & $b$ \\
%$c$ & $b$ & $c$ & $d$ & $a$ \\
%$d$ & $a$ & $b$ & $c$ & $d$
%\end{tabular}
%\end{center}
%\smallskip
Since $a \circ d = d \circ a = a, b \circ d = d \circ b = b, d \circ c = c$, and $d \circ d = d$, then by definition of identity the identity should be $d$, except that it fails at $c \circ d = a$ so the identity fails and this is not a group. 
\end{enumerate}
%\end{multicols}

\noindent\textbf{Exercise \ref{exercise:Groups:Cayley_groups2}:}
%For each of the following multiplication tables, fill in the blanks to make a Cayley table for a group.
\begin{multicols}{2}
\begin{enumerate}[(a)]

\item
%\begin{center}
%\begin{tabular}{c|cccc}
%$\circ$ & $a$ & $b$ & $c$ & $d$ \\
%\hline
%$a$ & $a$ & $b$ & $c$ & $\_$ \\
%$b$ & $\_$ & $a$ & $\_$ & $\_$ \\
%$c$ & $c$ & $\_$ & $a$ & $\_$ \\
%$d$ & $d$ & $\_$ & $\_$ & $\_$
%\end{tabular}
%\end{center}
%\smallskip
\begin{center}
		\begin{tabular}{c| c c c c }
			$\circ$ & a & b & c & d\\
			\hline
			a & a & b & c & \underline{d}\\
			b & \underline{b} & a & \underline{d} & \underline{c}\\
			c & c & \underline{d} & a & \underline{b}\\
			d & d & \underline{c} & \underline{b} & \underline{a}
		\end{tabular}
	\end{center}

\skipitems{1}

\item
%\begin{center}
%\begin{tabular}{c|cccc}
%$\circ$ & $a$ & $b$ & $c$ & $d$ \\
%\hline
%$a$ & $d$ & $\_$ & $\_$ & $\_$ \\
%$b$ & $\_$ & $d$ & $\_$ & $\_$ \\
%$c$ & $\_$ & $\_$ & $d$ & $\_$ \\
%$d$ & $\_$ & $\_$ & $\_$ & $d$
%\end{tabular}
%\end{center}
%\smallskip
\begin{center}
		\begin{tabular}{c| c c c c }
			$\circ$ & a & b & c & d\\
			\hline
			a & d & \underline{c} & \underline{b} & \underline{a}\\
			b &\underline{c} & d & \underline{a} & \underline{b}\\
			c & \underline{b} & \underline{a} & d & \underline{c}\\
			d & \underline{a} & \underline{b} & \underline{c} & d
		\end{tabular}
	\end{center}
	
\item
%\begin{center}
%\begin{tabular}{c|cccc}
%$\circ$ & $a$ & $b$ & $c$ & $d$ \\
%\hline
%$a$ & $a$ & $b$ & $\_$ & $d$ \\
%$b$ & $\_$ & $a$ & $\_$ & $\_$ \\
%$c$ & $c$ & $\_$ & $\_$ & $\_$ \\
%$d$ & $d$ & $\_$ & $\_$ & $\_$\\
%\end{tabular}
%\end{center}
%(There are two different ways to complete this one:  find both)
\begin{center}
		\begin{tabular}{c| c c c c }
			$\circ$ & a & b & c & d\\
			\hline
			a & a & b &\underline{c} & d\\
			b & \underline{b} & a & \underline{d} & \underline{c}\\	
			c & c & \underline{d} & \underline{a} & \underline{b}\\
			d & d & \underline{c} & \underline{b} & \underline{a}
		\end{tabular}
	\end{center}

	\begin{center}
		\begin{tabular}{c| c c c c }
			$\circ$ & a & b & c & d\\
			\hline
			a & a & b &\underline{c} & d\\
			b & \underline{b} & a & \underline{d} & \underline{c}\\	
			c & c & \underline{d} & \underline{b} & \underline{a}\\
			d & d & \underline{c} & \underline{a} & \underline{b}
		\end{tabular}
	\end{center}

\end{enumerate}
\end{multicols}

\noindent\textbf{Exercise \ref{exercise:Groups:Cayley_groups3}:}
%* Show that it is \emph{impossible} to complete the following Cayley tables to make a group.
%\begin{multicols}{2}
%\begin{enumerate}[(a)]
%
%\item
%\begin{center}
%\begin{tabular}{c|cccc}
%$\circ$ & $a$ & $b$ & $c$ & $d$ \\
%\hline
%$a$ & $\_$ & $\_$ & $\_$ & $\_$ \\
%$b$ & $b$ & $\_$ & $\_$ & $\_$ \\
%$c$ & $d$ & $\_$ & $\_$ & $\_$ \\
%$d$ & $c$ & $\_$ & $\_$ & $\_$
%\end{tabular}
%\end{center}
%
%\item
%\begin{center}
%\begin{tabular}{c|cccc}
%$\circ$ & $a$ & $b$ & $c$ & $d$ \\
%\hline
%$a$ & $a$ & $\_$ & $\_$ & $\_$ \\
%$b$ & $\_$ & $b$ & $\_$ & $\_$ \\
%$c$ & $\_$ & $\_$ & $\_$ & $\_$ \\
%$d$ & $\_$ & $\_$ & $\_$ & $\_$
%\end{tabular}
%\end{center}
%
%\item
%\begin{center}
%\begin{tabular}{c|cccc}
%$\circ$ & $a$ & $b$ & $c$ & $d$ \\
%\hline
%$a$ & $a$ & $\_$ & $\_$ & $\_$ \\
%$b$ & $\_$ & $c$ & $\_$ & $\_$ \\
%$c$ & $\_$ & $\_$ & $b$ & $\_$ \\
%$d$ & $\_$ & $\_$ & $\_$ & $\_$
%\end{tabular}
%\end{center}
%
%\item
%\begin{center}
%\begin{tabular}{c|cccc}
%$\circ$ & $a$ & $b$ & $c$ & $d$ \\
%\hline
%$a$ & $b$ & $\_$ & $\_$ & $\_$ \\
%$b$ & $\_$ & $c$ & $\_$ & $\_$ \\
%$c$ & $\_$ & $\_$ & $d$ & $\_$ \\
%$d$ & $\_$ & $\_$ & $\_$ & $\_$
%\end{tabular}
%\end{center}
%
%
%\end{enumerate}
%\end{multicols}
\begin{enumerate}[(a)]
\item
We have $b\circ a=b\implies a=e$. However, $c\circ a=d\implies a\neq e$.
So we can't fill the table to make a group.

\item
 We have $a\circ a=a\implies a=e$ and $b\circ b=b\implies b=e$\\
$\implies$  can't fill the table to make a group because there are 2 different $e$.

\item
There is no identity element.
\end{enumerate}

\noindent\textbf{Exercise \ref{exercise:Groups:27}:}
%Prove that the Cayley table in Table~\ref{CayleyU8} represents a group. (Note that associativity holds because we already know that modular multiplication is associative.)
\begin{itemize}
\item
identity is 1

\item
each element is its own inverse

\item
closed

\item
commutative

\item
associative

\end{itemize}
\noindent
$\implies$  it's a group.
\\

\noindent\textbf{Exercise \ref{exercise:Groups:28}:}
%Is the group in Table ~\ref{CayleyU8} abelian?  Justify your answer.
\\
It's abelian because $a\odot b=b\odot a$ for all elements in the group.
\\

\noindent\textbf{Exercise \ref{exercise:Groups:U(n)_abgroup}:}
%In this exercise, we prove that $U(n)$ is a group under multiplication mod $n$ for any $n$. We know that modular multiplication is associative, so it remains to show the closure and inverse properties.  
\begin{enumerate}[(a)]
\item
%Fill in the blanks to show that $U(n)$ is closed under modular multiplication:
%
%Let $k,m$ be arbitrary elements of  $U(n)$. It follows that both $k$ and \underline{$~<1>~$} are relatively prime to \underline{$~<2>~$}. So neither $k$ nor \underline{$~<3>~$} has any prime factors in common with \underline{$~<4>~$}. It follows that the product \underline{$~<5>~$} also has no prime factors in common with \underline{$~<6>~$}. Furthermore, the remainder of \underline{$~<7>~$} under division by \underline{$~<8>~$} also has no prime factors in common with \underline{$~<9>~$}. Therefore the product of \underline{$~<10>~$} and \underline{$~<11>~$} under modular multiplication is also an element of \underline{$~<12>~$}, so \underline{$~<13>~$} is closed under modular multiplication.
	\begin{multicols}{4}
	\begin{enumerate}[1.]
	\item
	$m$
	
	\item
	$n$
	
	\item
	$m$
	
	\item
	$n$
	
	\item
	$km$
	
	\item
	$n$
	
	\item
	$km$
	
	\item
	$n$
	
	\item
	$n$
	
	\item
	$k$
	
	\item
	$m$
	
	\item
	$U(n)$
	
	\item
	$U(n)$
	\end{enumerate}
	\end{multicols}
	
\item
%It remains to show that $U(n)$ is closed under inverse. Suppose that $m \in U(n)$ and $x$ is the inverse of $m$. What modular equation must $x$ satisfy?
%\hyperref[sec:Groups:Hints]{(*Hint*)}
$xm\equiv 1 \pmod{n}$ (or $mx\equiv 1 \pmod{n}$)

\item 
%Show that the equation in $x$ that you wrote in part (b) has a solution as long as $m$ is relatively prime to $n$.
$m$ is relative prime to $n\to$ gcd of $m$ and $n$ is 1.\\
1 is a multiple of 1, so a previous theorem from the Modular Arithmetic chapter implies  there is a solution.
\end{enumerate}

\noindent\textbf{Exercise \ref{exercise:Groups:UnAbel}:}
%Show that $U(n)$ is abelian.
\\
$U(n)$ is abelian\\
Let $x, y \in U(n)$, we have $x \odot y \equiv a \pmod{n}$\\
$\implies xy=kn + a \implies yx = kn + a \implies y \odot x \equiv a \pmod{n}$\\
$\implies x \odot y = y \odot x$
\\

\noindent\textbf{Exercise \ref{exercise:Groups:M2x2_group}:}
%We use  ${\mathbb M}_2 ( {\mathbb C})$\label{notematrices} to denote the set of all $2 \times 2$ matrices with complex entries.  That is
%\[
%{\mathbb M}_2 ( {\mathbb C}) = \left\{ 
%\begin{pmatrix}
%a & b \\
%c & d
%\end{pmatrix} ~|~ a,b,c,d \in {\mathbb C} \right\} \]
%
\begin{enumerate}[(a)]
\item
%Show that ${\mathbb M}_2 ( {\mathbb C})$ a group under matrix addition. Is it abelian? If so, prove it: if not, find a counterexample.
Suppose $a, b, c, d, e, f, g, h \in {\mathbb C}$ then, $
\begin{pmatrix}
a & b \\
c & d
\end{pmatrix} + 
\begin{pmatrix}
e & f \\
g & h
\end{pmatrix} = \begin{pmatrix}
a + e & b + f \\
c + g & d + h
\end{pmatrix}$

	\begin{itemize}
	\item
	Since $a + e, b + f, c + g, d + h, e + a, f + b, g + c, h + d \in {\mathbb C}$ this proves closure.
	
	\item
	Since $a + e =  e + a, b + f = f + b, c + g = g + c$, and $d + h = h + d$ this proves abelian.

	\item
	Identity:
	$
	\begin{pmatrix}
	a & b \\
	c & d
	\end{pmatrix} + 
	\begin{pmatrix}
	0 & 0 \\
	0 & 0
	\end{pmatrix} = \begin{pmatrix}
	0 & 0 \\
	0 & 0
	\end{pmatrix}+ 
	\begin{pmatrix}
	a & b \\
	c & d
	\end{pmatrix}  = \begin{pmatrix}
	a & b \\
	c & d
	\end{pmatrix}$
	
	This proves the matrix 
	$\begin{pmatrix}
	0 & 0 \\ 
	0 & 0 \end{pmatrix}$ 
	is the identity element under matrix addition.

	\item
	Inverse: Suppose $a, b, c, d, -a, -b, -c, -d \in {\mathbb C}$, then 
	\\
	$\begin{pmatrix}
	a & b \\
	c & d
	\end{pmatrix} + 
	\begin{pmatrix}
	-a & -b \\
	-c & -d
	\end{pmatrix} = \begin{pmatrix}
	-a & -b \\
	-c & -d
	\end{pmatrix}+ 
	\begin{pmatrix}
	a & b \\
	c & d
	\end{pmatrix}  = \begin{pmatrix}
	0 & 0 \\
	0 & 0
	\end{pmatrix}$
	
	This proves that every element has an inverse by definition.

	\item
	
	Associative: Suppose $a, b, c, d, e, f, g, h, i, j, k, l \in {\mathbb C}$, then 
	\\
	$(\begin{pmatrix}
	a & b \\
	c & d
	\end{pmatrix} + 
	\begin{pmatrix}
	e & f \\
	g & h
	\end{pmatrix}) + 
	\begin{pmatrix}
	i & j \\
	k & l
	\end{pmatrix} = 
	\begin{pmatrix}
	a & b \\
	c & d
	\end{pmatrix} + 
	(\begin{pmatrix}
	e & f \\
	g & h
	\end{pmatrix} + 
	\begin{pmatrix}
	i & j \\
	k & l
	\end{pmatrix}) = 
	\begin{pmatrix}
	a + e + i & b + f + j \\
	c + g + k & d + h + l
	\end{pmatrix})$
	
	This proves the associative property.
	\end{itemize}

Therefore this set, ${\mathbb M}_2 ( {\mathbb C})$ is a group and abelian since it is closed under matrix addition, has an identity element, has an inverse element, is associative, and commutative.

\item
%What is the order of this group?
The group is infinite by Definition \ref{group_order} and the fact that ${\mathbb C}$ is an infinite set.

\item
%Is ${\mathbb M}_2 ( {\mathbb C})$ a group under matrix multiplication? Is it abelian? Justify your answers.
We may prove by contradiction, using a specific example. Suppose $({\mathbb M}_2 ( {\mathbb C}),\cdot)$ is a group. Then:
	$$\begin{pmatrix}
	1 & 1 \\
	1 & 1
	\end{pmatrix}   
	\begin{pmatrix}
	0 & 0 \\
	0 & 0
	\end{pmatrix}  
	= 
	\begin{pmatrix}
	0 & 0 \\
	0 & 0
	\end{pmatrix}.$$
So by a previous proposition we have that 
$(\begin{pmatrix}
	1 & 1 \\
	1 & 1
	\end{pmatrix} $
must be the identity. On the other hand,
	$$\begin{pmatrix}
	1 & 1 \\
	1 & 1
	\end{pmatrix}   
	\begin{pmatrix}
	1 & 0 \\
	0 & 1
	\end{pmatrix}  
	= 
	\begin{pmatrix}
	1 & 0 \\
	0 & 1
	\end{pmatrix}.$$
So 
$\begin{pmatrix}
	1 & 1 \\
	1 & 1
	\end{pmatrix} $
  is not the identity.   This contradiction  shows that  $({\mathbb M}_2 ( {\mathbb C}),\cdot)$ is not a group.
\end{enumerate}

\noindent\textbf{Exercise \ref{exercise:Groups:GL2_group}:}
\begin{enumerate}[(a)]
\item
%Show that for any matrix $A \in GL_2(\mathbb{C})$, the matrix
%\[
%B =
%\frac{1}{\det(A)}\begin{pmatrix}
%d & -b \\
%-c & a
%\end{pmatrix} 
%\]
% satisfies
%\[ AB = BA = I.\]
\[
AB =
\begin{pmatrix}
a & b \\
c & d
\end{pmatrix} \left(\frac{1}{ad-bc}\begin{pmatrix}
d & -b \\
-c & a
\end{pmatrix}\right)
=
\frac{1}{ad-bc}\begin{pmatrix}
ad - bc & -ab + ba \\
cd - cd & -cb + da
\end{pmatrix}
\]
\\
\[
=
\frac{1}{ad-bc}\begin{pmatrix}
ad - bc & 0 \\
0 & -cb + da
\end{pmatrix}
=
\begin{pmatrix}
1 & 0 \\
0 & 1
\end{pmatrix}
=
I
\]
\\
\[
BA =
\left(\frac{1}{ad-bc}\begin{pmatrix}
d & -b \\
-c & a
\end{pmatrix}\right)\begin{pmatrix}
a & b \\
c & d
\end{pmatrix} 
=
\frac{1}{ad-bc}\begin{pmatrix}
da - bc & db - bd \\
-ca + ac & -bc + ad
\end{pmatrix}
\]
\\
\[
=
\frac{1}{ad-bc}\begin{pmatrix}
ad - bc & 0 \\
0 & -cb + da
\end{pmatrix}
=
\begin{pmatrix}
1 & 0 \\
0 & 1
\end{pmatrix}
=
I
\]

\item
%Using  Exercise~\ref{exercise:Groups:detMult}, show that $GL_2(\mathbb{C})$ is closed under matrix multiplication.
$
\begin{pmatrix}
a & b \\
c & d
\end{pmatrix}$
 is invertible if and only if $ad-bc=0$.  So $A,B \in  GL_2(\mathbb{C})$ iff $A,B$ both have inverses. Then it can be shown that $(AB)^{-1} = B^{-1}A^{-1}$, so $AB$ is also invertible, which means that $AB$ is also in $GL_2(\mathbb{C})$.

\item
%Show that  matrix multiplication in $GL_2(\mathbb{C})$ is associative.
Matrix multiplication is known to be associative
\end{enumerate}

\noindent\textbf{Exercise \ref{exercise:Groups:inverse_unique}:}
%Fill in the blanks to complete the following proof of Proposition~\ref{proposition:Groups:inv_unique}.
\begin{enumerate}[(a)]
\item
%By the definition of inverse, if $g'$ is an inverse of an element $g$ in a group $G$, then 
%\noindent
%$g \cdot \underline{~<1>~} = g' \cdot \underline{~<2>~} = e$.
	\begin{multicols}{2}
	\begin{enumerate}[1.]
	\item
	$g'$
	
	\item
	$g$
	\end{enumerate}
	\end{multicols}
	
\item
%Similarly, if $g''$ is an inverse of $g$ then  $g \cdot \underline{~<3>~} = \underline{~<4>~} \cdot g = e$. 
	\begin{multicols}{2}
	\begin{enumerate}[1.]
	\skipitems{2}
	\item
	$g''$
	
	\item
	$g''$
	\end{enumerate}
	\end{multicols}
	
\item
%We may show that $g' = g''$ as follows:
%\begin{align*}
%g' & = g' \cdot \underline{~<5>~}  &\text{ (definition of identity) } \\
%    & = g' \cdot (\underline{~<6>~} \cdot g'') &\text{ (part b above, def. of inverse) } \\
%    & = (g' \cdot g) \cdot \underline{~<7>~}  &\text{ (associative property of group G) } \\
%    & = \underline{~<8>~} \cdot g'' &\text{ (part a above, def. of inverse) } \\
%    & = g'' &\text{ (def. of identity) }
%\end{align*} 
	\begin{multicols}{2}
	\begin{enumerate}[1.]
	\skipitems{4}
	\item
	$e$
	
	\item
	$g$
	
	\item
	$g''$
	
	\item
	$e$
	\end{enumerate}
	\end{multicols}
\end{enumerate}

\noindent\textbf{Exercise \ref{exercise:Groups:39}:}\\
\begin{enumerate}[(a)]
\item
%Consider the group ${\mathbb C}^{\ast}$, and let  $a = 5 + 3i  \in {\mathbb C}^{\ast}$.  What is $a^{-1}$? 
$a^{-1}=\displaystyle\frac{5-3i}{34}$

\skipitems{1}
 
\item
%Consider the group defined by the set $G = {\mathbb R}^{\ast} \times {\mathbb Z}$ and the operation $(a,m) \circ (b,n) = (ab, m+n)$.  What is $(3,2)^{-1}$?
 identity is $(1,0)$\\
$(3,2)\circ(x,y)=(1,0)\implies y=-2$ and $x=\displaystyle\frac{1}{3}$

\item
%Consider the group $U(12)$.  What is $5^{-1}$?
$5x \equiv 1 \pmod{12}$\\
$5x \equiv 12k + 1$ for k $\in \mathbb{ Z}$\\
$x = 5$

\item
%Consider the group $GL_2({\mathbb R})$.  What is 
%$\begin{pmatrix}
%4 & 3 \\
%3 & 2
%\end{pmatrix}^{-1}$?
$(ad - bc)
\begin{pmatrix}
2 &-3 \\
-3 & 4
\end{pmatrix}\\
$= (8-9)
$\begin{pmatrix}
2 &-3 \\
-3 & 4
\end{pmatrix}$\\
$= \begin{pmatrix}
-2 & 3 \\
3 & -4
\end{pmatrix}$
\end{enumerate}

\noindent\textbf{Exercise \ref{exercise:Groups:42}:}\\
%Fill in the blanks to complete the proof of Proposition~\ref{proposition:Groups:group_inv_reverse}
%\begin{align*}
%(b^{-1}a^{-1})(ab) & = b^{-1}(a^{-1}a)b  &(\_\_\_\_\_\_\_\_\_\_\_\_\_\_\_\_\_\_\_\_\_) \\
% & = b^{-1}eb  &(\_\_\_\_\_\_\_\_\_\_\_\_\_\_\_\_\_\_\_\_\_) \\
% & = b^{-1}b  &(\_\_\_\_\_\_\_\_\_\_\_\_\_\_\_\_\_\_\_\_\_) \\
% & = e.  &(\_\_\_\_\_\_\_\_\_\_\_\_\_\_\_\_\_\_\_\_\_)
%\end{align*}
\begin{multicols}{2}
\begin{enumerate}
\item
Associative

\item
Definition  of  inverse

\item
Definition of  identity

\item
Definition  of  inverse
\end{enumerate}
\end{multicols}

\noindent\textbf{Exercise \ref{exercise:Groups:44}:}
%\begin{align*}
%a(a^{-1}(a^{-1})^{-1}) & = ae &\mbox{(multiplication by}~a)\\
%(aa^{-1})(a^{-1})^{-1} & = ae   &(\_\_\_\_\_\_\_\_\_\_\_\_\_\_\_\_\_\_\_\_\_) \\
%e(a^{-1})^{-1} & = ae   &(\_\_\_\_\_\_\_\_\_\_\_\_\_\_\_\_\_\_\_\_\_) \\
%(a^{-1})^{-1} &= a.  &(\_\_\_\_\_\_\_\_\_\_\_\_\_\_\_\_\_\_\_\_\_)
%\end{align*}
\begin{multicols}{2}
\begin{enumerate}
\item
def. of inverse

\item
associative

\item
def. of inverse

\item
def. of identity
\end{enumerate}
\end{multicols}

\noindent\textbf{Exercise \ref{exercise:Groups:group_abelian}:}
%Given a group $G$ and $a, b \in G$, prove that $G$ is abelian if and only if $(ab)^{-1} =a^{-1}b^{-1}$ for all $a,b$ in $G$.
\\
First, assume $G$ is abelian.\\
Given $a,b \in G$ then $a^{-1},b^{-1}, \in G$\quad (by the inverse property)\\
so $a^{-1}b^{-1} = b^{-1}a^{-1}$\quad (by the abelian property)\\
However, we've shown that $(ab)^{-1} = b^{-1}a^{-1}$ \quad (previous proposition)\\
which implies $a^{-1}b^{-1} = (ab)^{-1}$ for all $a,b \in G$ , \quad (by substitution)\\
which was the statement to be proved.\\
\\
Next, assume that $a^{-1}b^{-1} = (ab)^{-1}$ for all $a,b \in G$.\\
Given $c,d \in G$, let $a=c^{-1}$ and $b=d^{-1}$,\\
then $(c^{-1})^{-1}(d^{-1})^{-1} = (a)^{-1}$. \quad (substitution)\\
Thus $cd = (ab)^{-1}$, \quad (We've shown $(a^{-1})^{-1} = a$)\\
and $cd = b^{-1}(a^{-1}$ \quad (Using $(ab)^{-1} = b^{-1}a^{-1}$)\\
which implies $cd = dc$ for all $c,d \in G$, \quad (by substitution)\\
which was the statement to be proved.
\\

\noindent\textbf{Exercise \ref{exercise:Groups:48}:} 
\begin{enumerate}[(a)]
\item 
%Complete the proof that $ax=b$ has a unique solution by filling in the blanks:
%
%Suppose that $ax = b$. First we must show that such an $x$ exists. Since $a \in G$ and $G$ is a group, it follows that $a^{-1}$ exists.
%Multiplying both sides of $ax = b$ on the left by $a^{-1}$, we have 
%
%\begin{align*}
%a^{-1}(ax) & = a^{-1}b &\mbox{(left multiplication by $a^{-1}$)}\\
%(a^{-1}a)x & = a^{-1}b  &(\_\_\_\_\_\_\_\_\_\_\_\_\_\_\_\_\_\_\_) \\
%ex & = a^{-1}b  &(\_\_\_\_\_\_\_\_\_\_\_\_\_\_\_\_\_\_\_) \\
%x & = a^{-1}b.  &(\_\_\_\_\_\_\_\_\_\_\_\_\_\_\_\_\_\_\_)  
%\end{align*} 
% 
%We have thus shown that $ax = b$ implies $x  = a^{-1}b$, so $ax = b$ can have at most one solution. We may also verify that $x  = a^{-1}b$ is indeed a solution:
%
%\begin{align*}
%a(a^{-1}b) & =  (aa^{-1})b  &(\_\_\_\_\_\_\_\_\_\_\_\_\_\_\_\_\_\_\_) \\
%  & = eb  &(\_\_\_\_\_\_\_\_\_\_\_\_\_\_\_\_\_\_\_) \\
% & =b.  &(\_\_\_\_\_\_\_\_\_\_\_\_\_\_\_\_\_\_\_)  
%\end{align*} 
%
%This completes the proof that the solution both exists, and is unique.
associative;  	def. of inverse;  def. of identity; associative; def. of inverse; def. of identity
\item
%Prove now the existence and uniqueness of the solution of $xa = b$ (similar to part (a)).
Suppose that $xa = b$. Since, $a \in G$ and $G$ is a group, it follow that $a^{-1}$ exists.\\
\begin{align*}
xa(a^{-1} &= b(a^{-1})	& (\text{right multiplication by $a^{-1}$})\\
x(aa^{-1} &= b(a^{-1})	& (\text{associative})\\
xe &= b(a^{-1})	& (\text{def. of inverse})\\
x &= b(a^{-1})	& (\text{def of identity})
\end{align*}
As above, this shows that $b (a^{-1})$ is the only possible solution. As above, we may verify that $b (a^{-1})$ is indeed a solution, which proves existence.
\end{enumerate}

\noindent\textbf{Exercise \ref{exercise:Groups:49}:}
%Given $a, b \in {\mathbb C}^{\ast}$, where $a = 3 - 3i$ and $b = 2 + 12i$; solve for $x$ in each of the following equations.
%\[
%\textrm{(a)}~~ax = b \qquad \qquad
%\textrm{(b)}~~xa = b\qquad\qquad
%\textrm{(c)}~~bx = a\qquad\qquad
%\textrm{(d)}~~xb = a.\qquad\qquad
%\]
\begin{enumerate}[(a)]
\item
$ax=b$\\
$a^{-1}ax=a^{-1}b$\\
$x=a^{-1}b$\\
$x = -\frac{5}{3} + \frac{7}{3}i$

\skipitems{1}

\item
$bx = a$\\ 
$b^{-1}bx = b^{-1}a$\\
$x = b^{-1}a$\\
$x = \frac{1}{3 - 3i}(2 + 12i)$\\
$x = \frac{2 + 12i}{3 - 3i}$\\
$x = \frac{(2 \times 3 + 12 \times (-3)) + (12 \times 3 - 2 \times (-3))i}{3^2 + (-3)^2}$\\
$x = \frac{-30 + 42i}{18}$\\
$x = \frac{-5 + 7i}{3}$
\end{enumerate}

\noindent\textbf{Exercise \ref{exercise:Groups:51}:}
%Given $\rho, \mu \in S_8$, where $\rho = (532)(164)$ and $\mu = (18753)(26)$; solve for $x$ in each of the following equations.
%\[
%\textrm{(a)}~~\rho x = \mu \qquad \qquad
%\textrm{(b)}~~x \rho = \mu\qquad\qquad
%\textrm{(c)}~~\mu x = \rho\qquad\qquad
%\textrm{(d)}~~x \mu = \rho.\qquad\qquad
%\]
\begin{enumerate}[(a)]
\item
$\rho x=\mu$\\
$\implies x=\rho^{-1}\mu=(18752)(346)$\\

\item
$x \rho = \mu$\\
$x=\mu\rho^{-1}=(142)(5687)$

\item
$\mu x = \rho$\\
$\mu^{-1}(\mu x) = \mu^{-1}\rho$\\
$(\mu^{-1}\mu)x = \mu^{-1}\rho$\\
$ex = \mu^{-1}\rho$\\
$x = \mu^{-1}\rho$\\
\\
$x = ((18753)(26))^{-1}((532)(164))$\\
$x = (26)^{-1}(18753)^{-1}(532)(164)$\\
$x = (26)(13578)(532)(164)$\\
$x = (1278)(364)$
\end{enumerate}

\noindent\textbf{Exercise \ref{exercise:Groups:52}:}
%Given the group $U(9)$, solve for $x$ in each of the following equations.
%\[
%\textrm{(a)}~~5x=8 \qquad \qquad
%\textrm{(b)}~~x5=8\qquad\qquad
%\textrm{(c)}~~8x=5\qquad\qquad
%\textrm{(d)}~~x8=5.\qquad\qquad
%\]
\begin{enumerate}[(a)]
\item
$5x=8$\\
$5x \equiv 8 \pmod{9}$
$\implies 5x = 9k + 8$ \quad \quad (mod equiv)\\
$x = \frac{9k + 8}{5} = 2k + 1 + \frac{-k+3}{5}$ \quad \quad (by algebra)\\
Since $x$ must be an integer $-k+3$ must be a multiple of 5,or $k = 5n + 3$. Substituting back in gives $x = 10n+6+1-n$, or $x = 7+9n$.  since $x \in U(9)$, we have $x=7$.

\skipitems{1}


\item
$8x=5$\\
$8x \equiv 5 \pmod{9}$
$\implies 8x = 9k + 5$ \quad \quad (mod equiv)\\
$x = \frac{9k + 5}{8} = k + \frac{k+5}{8}$ \quad \quad (by algebra)\\
Since $x$ must be an integer $k + 5$ must be a multiple of 8, or $k = -5 + 8n$.\\
Plugging back in gives $x = -5 + 8n +n=-5 + 9n$. Since $x \in U(9)$, this implies $x=4$.
\end{enumerate}

\noindent\textbf{Exercise \ref{exercise:Groups:53}:}
%Given $A, B \in GL_2({\mathbb R})$, where
%\[
%A = \begin{pmatrix}
%6 & 5 \\
%4 & 4
%\end{pmatrix} \mbox{ and }
%B = \begin{pmatrix}
%-2 & -1 \\
%7 & 4
%\end{pmatrix} \]
%
%Solve for $X$ in each of the following equations.
%\[
%\textrm{(a)}~~AX=B \qquad \quad
%\textrm{(b)}~~XA=B\qquad\quad
%\textrm{(c)}~~BX=A\qquad\quad
%\textrm{(d)}~~XB=A.\qquad\quad
%\]
\begin{enumerate}[(a)]
\item
$AX=B$\\
$X=A^{-1}B=\begin{pmatrix}
-10.75 & -6\\
12.5 & 7
\end{pmatrix}$ = $\begin{pmatrix}
\frac{-43}{4} & -6\\
\frac{25}{2} & 7
\end{pmatrix}$\\

\skipitems{1}

\item
$BX=A$\\
$X=B^{-1}A=\begin{pmatrix}
-28 & -24\\
50 & 43
\end{pmatrix}$\\
\end{enumerate}
a. $X=A^{-1}B=\begin{pmatrix}
-10.75 & -6\\
12.5 & 7
\end{pmatrix}$\\
b. $X=BA^{-1}=\begin{pmatrix}
-1 & 1\\
3 & -2.75
\end{pmatrix}$\\
\\

\noindent\textbf{Exercise \ref{exercise:Groups:54}:}
%Given a group $G$ and $a, b \in G$, prove that if $G$ is abelian, then any solution of $ax = b$ is also a solution of $xa = b$ (and vice versa). Given a group $G$  that is \emph{not} abelian, show that it is always possible to find an equation of the form $ax = b$ which has a solution that is \emph{not} a solution to $xa = b$.
\\
\noindent Proof (1):
Let $c$ be a solution to $ax=b$.  This implies $ac=b$ Since $G$ is abelian, we also have $ca=b$. Thus $c$ is also a solution to $xa=b$.\\
Similarly, let $c$ be a solution to $xa=b$.  This implies $ca=b$ Since $G$ is abelian, we also have $ac=b$. Thus $c$ is also a solution to $ax=b$.\\

\noindent Proof (2):
Suppose $G$ is not abelian, then $\exists a, c \in G$ such that $ac \neq ca$.\\
Let $b = ac$.  Then $c$ is a solution to $ax=b$. Furthermore,$ca \neq b$, so $c$ is not a solution to $xa = b$. Thus $ax=b$ has a solution that is not a solution of $xa=b$. 
\\

\noindent\textbf{Exercise \ref{exercise:Groups:56}:}
\begin{enumerate}[(a)]
\item
%To prove  Proposition~\ref{proposition:Groups:cancel}, we need to prove both that   $ba = ca$ implies $b = c$,  and that $ab = ac$ implies $b = c$. Why do these two statements require two different proofs?
Because $ab\neq ba$ in most of groups.\\
$ba=ab$ only in abelian group.

\item
%Prove Proposition~\ref{proposition:Groups:cancel}.
$ba=ca$ \quad \quad [Given]\\
$(ba)a^{-1}=(ca)a^{-1}$ \quad \quad [Substitution] \\
$b(aa^{-1})=c(aa^{-1})$ \quad \quad [Associative] \\
$be=ce$ \quad \quad \quad [Inverse]\\
$b=c$ \quad \quad \quad [Identity]\\
The proof with $ab=ac$ is similar.
\end{enumerate}

\noindent\textbf{Exercise \ref{exercise:Groups:57}:}
%Using Definition~\ref{definition:Groups:DefGroupExponents}, prove that 
%\[ (g^n)^{-1} = g^{-n},\]
%i.e. the inverse of $g^n$ is equal to $g^{-n}$ for any group element $g$ and for any natural number $n$.
\\
By Definition \ref{definition:Groups:DefGroupExponents},
\[
g^n =  \underbrace{g \cdot g \cdots g}_{n \; {\rm times}}
\  \text{and}\  
g^{-n} =  \underbrace{g^{-1} \cdot g^{-1} \cdots g^{-1}}_{n \; {\rm times}}
\]
so
\[ g^n g^{-n} = (\underbrace{g \cdot g \cdots g}_{n \; {\rm times}}) (\underbrace{g^{-1} \cdot g^{-1} \cdots g^{-1}}_{n \; {\rm times}})\]
Using associativity, we have
\[ g^n g^{-n} = (\underbrace{g \cdot g \cdots g}_{n-1 \; {\rm times}})(gg^{-1}) (\underbrace{g^{-1} \cdot g^{-1} \cdots g^{-1}}_{n-1 \; {\rm times}}),\]
and using inverse and identity properties we obtain
\[ g^n g^{-n} = (\underbrace{g \cdot g \cdots g}_{n-1 \; {\rm times}}) (\underbrace{g^{-1} \cdot g^{-1} \cdots g^{-1}}_{n-1 \; {\rm times}}).\]
Using the same procedure, we can show
\[ g^n g^{-n} = (\underbrace{g \cdot g \cdots g}_{n-2 \; {\rm times}}) (\underbrace{g^{-1} \cdot g^{-1} \cdots g^{-1}}_{n-2 \; {\rm times}}).\]
We may continue to work our way down from $n-2$ to $n-3$ to $\ldots$ to 0.  At the end we obtain $g^n g^{-n} = e$.  \\

\noindent Following the same procedure beginning with $g^{-n} g^n$ gives $g^{-n} g^n = e$.
It follows that $g^{-n} = (g^n)^{-1}$, because it satisfies the inverse conditions.
\\

\noindent\textbf{Exercise \ref{exercise:Groups:59}:}
%Prove parts (2) and (3) of Proposition~\ref{proposition:Groups:exponent_laws}.
\\
Part 2\\
There are 4 cases (a) $m, n \geq 0$, (b) $m, n < 0$, (c) $m \geq 0, n < 0$, and (d) $m < 0, n \geq 0$.\\
\begin{itemize}
\item
Case (a): using Definition \ref{definition:Groups:DefGroupExponents}, we have\\
\[
(g^m)^n = \underbrace{ \underbrace{g \cdot g \cdots g}_{m \; {\rm times}}\underbrace{g \cdot g \cdots g}_{m \; {\rm times}}\underbrace{g \cdot g \cdots g}_{m \; {\rm times}}}_{n \: {\rm times}}
\]
and
\[
g^{m \cdot n} = \underbrace{g \cdot g \cdots g}_{m \cdot n \; {\rm times}}.
\]
Since the right hand sides of these expressions are equal, then so are the left hand sides; so $(g^m)^n = g^{m \cdot n}$\\

\item
Case (b): using Definition  \ref{definition:Groups:DefGroupExponents}, we have\\
\begin{align*}
(g^m)^n &= (g^{-{|m|})^{-|n|}}  \qquad \text{[since }m,n < 0, m=-|m| \text{ and } n=-|n|\\
&= ( \underbrace{g^{-1} \cdot g^{-1} \cdots g^{-1}}_{|m| \; {\rm times}})^{-|n|}\\
&= (( \underbrace{g^{-1} \cdot g^{-1} \cdots g^{-1}}_{|m| \; {\rm times}})^{-1})^{|n|}\\
&= (( \underbrace{g \cdot g \cdots g}_{|m| \; {\rm times}})^{|n|}\\
& = g^{|m||n|}\\
&=g^{mn}
\end{align*}

\item
Case (c): we have\\
\begin{align*}
(g^m)^n &= {\underbrace{(g \cdot g \cdots g)}_{m \; {\rm times}}}^{-|n|}  \quad \quad m \text{ is positive, } n \text{ is negative}\\
&= {\underbrace{((g \cdot g \cdots g)}_{m \; {\rm times}}}^{-1})^{|n|}  \quad \quad \text{Definition  \ref{definition:Groups:DefGroupExponents}}\\
& = {\underbrace{(g^{-1} \cdot g^{-1}  \cdots g^{-1})}_{m \; {\rm times}}}^{|n|}  \quad \quad \text{ proved previously}\\
&= \underbrace{ \underbrace{(g^{-1} \cdot g^{-1} \cdots g^{-1})}_{m \; {\rm times}}\underbrace{(g^{-1} \cdot g^{-1} \cdots g^{-1})}_{m \; {\rm times}}\underbrace{(g^{-1} \cdot g^{-1} \cdots g^{-1})}_{m \; {\rm times}}}_{|n| \: {\rm times}}\\
 &= g^{-(m  |n|)} = g^{m \cdot (-|n|)} = g^{mn}.
\end{align*}

\item
Case (d): using Definition \ref{definition:Groups:DefGroupExponents}, we have\\
\[
(g^m)^n = \underbrace{ \underbrace{g^{-1} \cdot g^{-1} \cdots g^{-1}}_{-|m| \; {\rm times}}\underbrace{g^{-1} \cdot g^{-1} \cdots g^{-1}}_{-|m| \; {\rm times}}\underbrace{g^{-1} \cdot g^{-1} \cdots g^{-1}}_{-|m| \; {\rm times}}}_{n \: {\rm times}}
\]
and
\[
g^{m  n} = {\underbrace{g^{-1} \cdot g^{-1}  \cdots g^{-1}}_{-|mn| \; {\rm times}}}
\]
Since the right hand sides of these expressions are equal, then so are the left hand sides; so $(g^m)^n = g^{m  n}$\\
\end{itemize}


Part 3\\
\begin{align*}
(h^{-1}g^{-1})^{-n}&= ((gh)^{-1})^{-n} \quad \text{(by inverse product formula))}\\
&=(gh)^{(-1)(-n)} \quad \text{(by part (2) of the proposition) }\\
&=(gh)^n.
\end{align*}

\noindent\textbf{Exercise \ref{exercise:Groups:63}:}\\
%Given any fixed integer $m$,  prove that
% \[m{\mathbb Z} = \{ \ldots, -2m, -m,  0, m, 2m, \ldots \}\] 
%is a subgroup of ${\mathbb Z}$ under the operation of addition.
We need to show that: (a) $({\mathbb Z}, +)$ is a group, (b) $2{\mathbb Z} \subset {\mathbb Z}$, and (c) $(2{\mathbb Z}, +)$ is a group.

\begin{enumerate}[(a)]
\item
We proved that $({\mathbb Z}, +)$was a group in Chapter 1 and 2.

\item
Any element $a \in m{\mathbb Z}$ can be written as $a = mn$, where $n \in {\mathbb Z}$, hence $a \in {\mathbb Z}$ also.

\item
	\begin{itemize}
	\item
	Closure: Given $x, y \in m{\mathbb Z}$, it follows that $x = ma$ and $y = mb$ for some $a, b \in {\mathbb Z}$.\\
	Therefore, $x + y = ma + mb = m(a + b)$.\\
	Since ${\mathbb Z}$ is closed under +, it follows that $(a + b) \in {\mathbb Z}$, so $m(a + b) \in m{\mathbb Z}$. Since $x$ and $y$ are arbitrary, it follows that $m{\mathbb Z}$ is closed under addition.

	\item
	Identity: $0 \in m{\mathbb Z}$, since $m \cdot 0 = 0$ and for any $x \in m{\mathbb Z}, 0 + x = x + 0 = x$.\\
	Hence, $m{\mathbb Z}$ has an identity under addition, namely 0.

	\item
	Inverse: Given $x \in m{\mathbb Z}$, where $x = mn$, \\
	$-x = -(mn) = m(-n)$ \quad \quad (associative, commutative of ${\mathbb Z}$ under multiplication)\\
	and since$-n \in {\mathbb Z}$ \quad \quad (closure of ${\mathbb Z}$ under multiplication)\\
	it follows that $-x \in m{\mathbb Z}$. \\
	Since $-x + x = x + (-x) = 0$ it follows $\forall x \in m{\mathbb Z}$, $\exists x ^{-1} \in m{\mathbb Z}$, namely $x^{-1} = x$.

	\item
	Associative: Suppose $w, x, y \in m{\mathbb Z}$. Then $w, x, y$ are integers, and $w + (x + y) = (w + x) + y$ by the associativity of $({\mathbb Z}, +)$.\\
	Hence $m{\mathbb Z}$ is associative under addition.\\

	\end{itemize}
\end{enumerate}

\noindent\textbf{Exercise \ref{exercise:Groups:64}:}
%Prove or disprove:
\begin{enumerate}[(a)]
\item
%$GL_2({\mathbb R})$ is a subgroup of ${\mathbb M}_2 ( {\mathbb R})$.
$GL_2({\mathbb R})$ is group of all invertible 2x2 matrices with multiplication operation.\\
$M_2({\mathbb R})$ is group of all 2x2 matrices with addition operation.\\
$\implies GL_2({\mathbb R})$ is not a subgroup of $M_2({\mathbb R})$ because they have different group operations.

\item
%$U(n)$ is  a subgroup of ${\mathbb Z}_n$.
$U(n)$ is not a subgroup of ${\mathbb Z}_n$ because they have different group operations.\\
$U(n)$ is a group under multiplication, while ${\mathbb Z}_n$ is a group under addition.
\end{enumerate}

\noindent\textbf{Exercise \ref{exercise:Groups:65}:}
%Prove the following: Suppose $G$ is a group with identity element $e$, and let $H$ be a subgroup of $G$ with identity element $f$.  Then $e=f$.
\\
Proof:  \\
Suppose $G$ is a group with the identity element $e$, and $H$ is a subgroup of $G$, with an identity element of $f$.\\
Since $f \in G$ and $e$ is the identity of $G$, it follows that $fe=f$. Furthermore, since $f \in H$ and $f$ is the identity of $H$, we must have  $ff=f$. It follows that $fe = ff$.  Since $f \in G$, 
$f$ must have an inverse in $G$.  Multiply both sides on the left by $f^{-1}$ gives
$f^{-1}(fe) = f^{-1}(ff)$. Using the associative rule, inverse property, and identity property gives us $e=f$. 
\\

\noindent\textbf{Exercise \ref{exercise:Groups:T_subgroup}:}
%The set ${\mathbb T}$ is defined as the subset of  ${\mathbb C}$ whose elements all have a modulus of $1$; that is
%\[
%{\mathbb T} = \{c \in {\mathbb C} :  | c | =1 \} \]

\begin{enumerate}[(a)]
\item
%Using Proposition \ref{proposition:Groups:subgroup_prove} above, prove that ${\mathbb T}$ is a subgroup of ${\mathbb C}^{\ast}$.
	\begin{enumerate}[(i)]
	\item
	1 is in $T$ because $|1|=1$
	
	\item
	Let $c_1,c_2\in T\implies |c_1|=|c_2|=1$\\
	$\implies |c_1c_2|=|c_1||c_2|=1$\\
	$\implies c_1c_2\in T$
	
	\item
	Let $c\in T \implies |c^{-1}|=\displaystyle\frac{1}{|c|}=1$\\
	$\implies c^{-1}\in T$\\
	So $T$ is a subgroup of $C^*$
	\end{enumerate}
	
\item
%What is $| {\mathbb T} |$?
There are infinite $\mathbb{ T}$ with modulus 1.

\item
%Prove or disprove that ${\mathbb T}$ is commutative.
All elements of ${\mathbb T}$  are elements of ${\mathbb C}$.  Since all elements of ${\mathbb C}$ commute, it follows that all elements of ${\mathbb T}$ commute, hence ${\mathbb T}$ is abelian.\\
\end{enumerate}

\noindent\textbf{Exercise \ref{exercise:Groups:67}:}
%Let  $H_4 = \{ 1, -1, i, -i \}$, (these are the  fourth roots of unity, which we studied in Section~\ref{subsec:ComplexNumbers:ComplexRoots:RootsOfUnity}).  
\begin{enumerate}[(a)]
\item
%Using Proposition \ref{proposition:Groups:subgroup_prove}, prove that $H_4$ is a subgroup of ${\mathbb T}$. (\emph{Note} you should first verify that $H_4$ is a subset of ${\mathbb T}$.)
$H$ is a subset of $T$ since $|1|=|i|=|-1|=|-i|=1$. To show $H$ is a subgroup of $T$, make
a Cayley table to show $H$ is closed under complex multiplication. 

\item  
%What is $| H_4 |$?
$|H|=4$

\item
%Prove or disprove that $H_4$ is commutative.
This can be proved from the Cayley table.
\end{enumerate}

\noindent\textbf{Exercise \ref{exercise:Groups:69}:}
%Let ${\mathbb Q}^*\label{noteQstar}$ be defined in the following way:
%\[
%{\mathbb Q}^* = \{ p/q : p, q \mbox{ are nonzero integers } \}
%\]
%
%In other words ${\mathbb Q}^*$ is the set of non-zero rational numbers (${\mathbb Q}^* = {\mathbb Q} \setminus 0$).

\begin{enumerate}[(a)]
\item
%Using Proposition \ref{proposition:Groups:subgroup_prove}, prove that ${\mathbb Q}^*$ is a subgroup of ${\mathbb R}^*$.
${\mathbb Q}^*$ is a subgroup of ${\mathbb R}^*$ because:
	\begin{enumerate}[(i)]
	\item
	$1=\displaystyle\frac{1}{1}\in {\mathbb Q}^*$
	
	\item
	Let $\displaystyle\frac{p}{q},\frac{m}{n}\in {\mathbb Q}^*\implies \frac{p}{q}.\frac{m}{n}=\frac{pm}{qn}\in {\mathbb Q}^*$
	
	\item
	$\left(\displaystyle\frac{p}{q}\right)^{-1}=\displaystyle{\frac{q}{p}}\in {\mathbb Q}^*$
	
	\item
	${\mathbb Q}^*$ is associative since it's a subset of ${\mathbb R}^*$
	\end{enumerate}
	
\item
%Prove or disprove that ${\mathbb Q}^*$ is commutative.
${\mathbb Q}^*$ is a subset of ${\mathbb R}^*$, which is abelian, so ${\mathbb Q}^*$ is abelian.
\end{enumerate}

\noindent\textbf{Exercise \ref{exercise:Groups:70}:}
%We define $SL_2( {\mathbb R})$\label{speciallinear} to be the set of $2 \times 2$ matrices of determinant one; that is, a matrix
%\[
%A =
%\begin{pmatrix}
%a & b \\
%c & d
%\end{pmatrix}
%\]
%is in $SL_2( {\mathbb R})$ exactly when $ad - bc = 1$.  We call this the \term{Special Linear Group} \index{Group!special
%linear $(SL_n(\mathbb{R}))$}\label{speciallinear}.

\begin{enumerate}[(a)]
\item
%Using Proposition \ref{proposition:Groups:subgroup_prove}, prove that $SL_2( {\mathbb R})$ is a subgroup of $GL_2( {\mathbb R})$.
This follows from:
$$\det(AB) = \det(A) \det(B).$$
If $A,B \in  SL_2({\mathbb R})$ then $\det(A) = \det(B) = 1$ so $\det(AB) = (1)(1) = 1$, which implies $ab \in SL_2({\mathbb R})$.

On the other hand, $A A^{-1} = I$ so $\det(A)\det(A^{-1}) = \det(AA^{-1}) = det(I) = 1.$ If $A \in  SL_2({\mathbb R})$ then $\det(A)=1$, and the equation implies $\det(A^{-1}) = 1$, so If $A^{-1} \in  SL_2({\mathbb R})$. Thus both conditions of Proposition \ref{proposition:Groups:subgroup_prove} are satisfied, and $SL_2({\mathbb R})$ is a subgroup of $GL_2({\mathbb R})$.

\item
%Prove or disprove that $SL_2( {\mathbb R})$ is commutative.
Not commutative: there are many counterexamples.
\end{enumerate}

\noindent\textbf{Exercise \ref{exercise:Groups:73}:}
%Use Proposition \ref{proposition:Groups:subgroup_prove_2} to reprove the following:
\begin{enumerate}[(a)]
\item
%${\mathbb Q}^*$ is a subgroup of ${\mathbb R}^*$
$Q^*$ is a subgroup of $R^*$ (Proposition 69)\\
\end{enumerate}

\noindent\textbf{Exercise \ref{exercise:Groups:gen_3_Rstar}:}
%List the set $\langle 3 \rangle$ for $3 \in {\mathbb R}^*$.
\begin{align*}
\langle 3 \rangle &= \{ \ldots, 3^{-3}, 3^{-2}, 3^{-1}, 1, 3, 3^2, 3^3, \ldots \}\\
&= \{ \ldots, \frac{1}{27}, \frac{1}{9}, \frac{1}{3}, 1, 3, 9, 27, \ldots \}
\end{align*}

\noindent\textbf{Exercise \ref{exercise:Groups:Z_construct_-1}:}\\
%Show that $-1$ is a generator of $\mathbb Z$; that is that $\mathbb Z = \langle -1 \rangle$.
$\langle-1\rangle=\{...,-2,-1,0,1,-1,-2\}=\{n(-1):n\in Z\}=Z$\\

\noindent\textbf{Exercise \ref{exercise:Groups:Z6_construct_5}:}\\
%In the group ${\mathbb Z}_6$, show that $\langle 5 \rangle = {\mathbb Z}_6$.
$\langle 5\rangle=Z_6=\{0,1,2,3,4,5\}$\\

\noindent\textbf{Exercise \ref{exercise:Groups:82}:}\\
%Given a group $G$, suppose that $G = \langle a \rangle$.  Prove that $G = \langle a^{-1} \rangle$.
We have:
\begin{align*}
\langle a^{-1}\rangle &= \{ \ldots (a^{-1})^{-2},(a^{-1})^{-1},e,(a^{-1}),(a^{-1})^2 \ldots\} \\
& = \{ \ldots (a^{2}),a,e,a^{-1},a^{-2} \ldots\} \qquad \text{(exponent rules)}\\
& = \{ \ldots (a^{-2}),a^{-1},e,a,a^1 \ldots\} \qquad \text{(list elements in reverse order)} \\
& = \langle a\rangle \qquad \text{(by definition of $\langle a\rangle$)}.
\end{align*}
So if $G=\langle a\rangle$ then $G=\langle a^{-1}\rangle$ by substitution.
\\

\noindent\textbf{Exercise \ref{exercise:Groups:U9_construct}:}\\
%Find any other generators of $U(9)$ if they exist (say so if no others exist).
$U(9)=\{1,2,4,5,7,8\}$\\
$\langle 5\rangle=\{5^n\pmod{9}:n\in Z\}=U(9)$\\
\\

\noindent\textbf{Exercise \ref{exercise:Groups:subgroup_3Z}:}\\
%Prove that $4 {\mathbb Z}$ is a subgroup of $\mathbb Z$.
$4Z\neq\emptyset$ because $0,4,8,...\in 4\mathbb{Z}$\\
Let $a,b\in Z\implies a=4m$ and $b=4n$\\
$\implies b^{-1}=-4n$\\
$\implies a+b^{-1}=4m-4n=4(m-n)\in 4\mathbb{Z}$\\
$\implies 4Z$ is a subgroup of $\mathbb{Z}$\\
\\

\noindent\textbf{Exercise \ref{exercise:Groups:87}:}
%Let $H = \{ 2^n : n \in {\mathbb Z} \} = \langle 2 \rangle$ under multiplication.
\begin{enumerate}[(a)]
\item
%List the elements in $H$
$H=\{...\displaystyle\frac{1}{4},\frac{1}{2},1,2,4,...\}$

\item
%Show that $H \subset {\mathbb Q}^*$.
Let $x\in H\implies x=\displaystyle\frac{2^n}{1}(n\in {\mathbb Z})\implies x\in {\mathbb Q}^*$\\
$\implies H\subset {\mathbb Q}^*$

\item
%Show that $H$ is closed under multiplication.
Let $2^k,2^l\in H\implies 2^k2^l=2^{k+l}\in H$\\
$\implies H$ is closed under multiplication.

\item
%Show that $H$ is closed under inverse.
Let $2^k\in H\implies (2^k)^{-1}=2^{-k}\in H$\\
$\implies H$ is closed under inverse.

\item
%Is $H$ a subgroup of ${\mathbb Q}^*$? \emph{Explain} your answer.
$H$ is a subgroup of ${\mathbb Q}^*$ because $H$ has the identity element and is closed under multiplication and inverse.
\end{enumerate}

\noindent\textbf{Exercise \ref{exercise:Groups:90}:}\\
%Let $G$ be a finite group, and let $a \in G$ where $a \neq e$. Show there exists a natural number $m>0$ such that $a^m = e$.
%\hyperref[sec:Groups:Hints]{(*Hint*)}
$a\in G$ and $a\neq e$\\
$G=\{...,a^{-2},a^{-1},a,a^1,a^2,a^3,...,a^n\}$\\
Since $G$ is a finite group, elements can't all be different.\\
$\implies$  We can find $k,l\in {\mathbb Z}$ and $k>l$ such that $a^k=a^l$\\
$\implies a^k(a^l)^{-1}=a^l(a^l)^{-1}$\\
$\implies a^{k-l}=e$\\
Let $m=k-l\in {\mathbb Z}\to$ there is $m>0$ such that $a^m=e$\\

\noindent\textbf{Exercise \ref{exercise:Groups:Z6_orders}:}
\begin{enumerate}[(a)]
\item
%In the group ${\mathbb Z}_6$, What is $| 1 |$? What is $| 5 |$?
$|1|=6$\\
$|5|=6$

\item
% Given any group $G$, If $e$ is the identity element of $G$ then what is $|e|$?
 $|e|=1$
\end{enumerate}

\noindent\textbf{Exercise \ref{exercise:Groups:U9_orders}:}\\
%Find the order of each element of $U(9)$. Find also the cyclic subgroup generated by each element.
$U(9)=\{1,2,4,5,7,8\}$\\
$|1|=1$\\
$|2|=6$ because $2^6\equiv 1\pmod{9}$\\
$|4|=3$ because $4^3\equiv 1\pmod{9}$\\
$|5|=6$ because $5^6\equiv 1\pmod{9}$\\
$|7|=3$ because $7^3\equiv 1\pmod{9}$\\
$|8|=2$ because $8^2\equiv 1\pmod{9}$\\
Cyclic subgroups:\\
$\langle1\rangle=\{1\}$\\
$\langle2\rangle=\{2,4,8,7,5,1\}=U(9)$\\
$\langle4\rangle=\{4,7,1\}$\\
$\langle5\rangle=\{5,7,8,4,2,1\}=U(9)$\\
$\langle7\rangle=\{7,4,1\}$\\
$\langle8\rangle=\{8,1\}$\\
\\

\noindent\textbf{Exercise \ref{exercise:Groups:finite_inv}:}\\
%Let $G$ be a finite group, and let $a \in G$ where $|a|=n$. Show that $(a^{-1})^n = e$.
We are given that $|a|=n$, which implies that $a^n=e$. By a previous proposition and commutativaty of multiplication, $(a^{-1})^n = a^{(-1)(n)} = a^{(n)(-1)} = (a^n)^{-1} = e^{-1} = e$. 

Alternative proof (more formal)

\begin{align*}
(a^{-1})^n &= a^{-n} \qquad \textrm{ [Definition \ref{definition:Groups:DefGroupExponents}]}\\
(a^n)^{-1} &= a^{-n} \qquad \textrm{[Proposition  \ref{proposition:Groups:exponent_laws}]}\\
(a^{-1})^n &= (a^n)^{-1} \qquad \textrm{ [substitution]}\\ 
a^n &= e \qquad \textrm{  [ def of  |a|]} \\
(a^{-1})^n &= e^{-1} = e  \qquad \textrm{[substitution and def. of identity]}
\end{align*}

\noindent\textbf{Exercise \ref{exercise:Groups:OrderEltCyclic}:}
%In the following exercises, $G$ is a finite group, and  $a \in G$ where $|a|=n$. 
\begin{enumerate}[(a)]
\item
%Show that for any integer $m \in \mathbb{Z}$, $a^m = a^{\bmod(m,n)}$.
%\hyperref[sec:Groups:Hints]{(*Hint*)} 
By the definition of $\bmod(m,n), m = \bmod(m,n) + kn$ for some integer $k$.  So
$$a^m = a^{\bmod(m,n) + kn} =  a^{\bmod(m,n)}  a^{kn} =  a^{\bmod(m,n)}  (a^n)^k
= a^{\bmod(m,n)}  e^k = a^{\bmod(m,n)}.$$

\item
%Let $A = \{e,a,a^2,\ldots,a^{n-1}\}$.  Show that $\langle a \rangle \subset A$ and $A \subset \langle a \rangle$.
$A \subset \langle a \rangle$ by the definition of $\langle a \rangle$. On the other hand, any element of $\langle a \rangle$ is equal to $a^m$ for some integer $m$.  By part (a), $a^m= a^{\bmod(m,n)}$. Since $0 \le \bmod(m,n) \le n-1$, it follows that  $a^{\bmod(m,n)} \in A$ and hence $a^m \in A$.

\item 
%Prove that $| \langle a \rangle | = |A|$. (Note that $|A|$ is the number of elements in $A$.)
Part (b) shows that $A = \langle a \rangle$, so they have the same number of elements.

\item
%If $m,k \in \mathbb{Z}_n$ and $m \ne k$, show that $a^m \ne a^k$.
Given $k,m \in \mathbb{Z}_n$ where $k \neq m$. Without loss of generality we may suppose that $m < k$. Then $a^k = a^ma^{k-m}$.  since $1 \le m < k \le n-1$, it follows that $0<k-m < n$.  Since $n$ is the smallest positive integer such that $a^n=1$, it follows that $a^{k-m} \neq 1$.  It follows from a previous proposition that $a^m \neq a^k$.

\item
%Prove that $|a| =|A|$.  (This together with part (c) implies that $|a| =  | \langle a \rangle |$.)
Part (d) shows there are $n$ distinct elements of $A$. So $|A|=n$, and our supposition at the beginning was that $|a|=n$ so $|A|=|a|$.
\end{enumerate}

\noindent\textbf{Exercise \ref{exercise:Groups:97}:}\\
%Let $G$ be a finite group, and let $a \in G$ such that $|a| = n$ for $n>0$. Show that there exists a natural number $m$ such that $a^{-1} = a^m$, and express $m$ in terms of $n$.
$|a|= n \implies a^n = e \implies a^na^{-1}=ea^{-1}\implies a^{n-1}=a^{-1}$\\
Let $m=n-1 \implies a^{-1}=a^m$\\
\\

\noindent\textbf{Exercise \ref{exercise:Groups:not_cyclic}:}
\begin{enumerate}[(a)]
\item
%Show that ${\mathbb Q}$ is not cyclic. 
Proof:
Suppose that $a$ is a generator of $\mathbb{Q}$, i.e. $\langle a \rangle = \mathbb{Q}$.  Then since $a/2 \in  \mathbb{Q}$, it follows that  $a/2 \in \langle a \rangle$. This means $a/2 =ka$ for some integer k.  Rearranging this equation, we find that $a(1/2-k)=0$.  By the zero-divisor property of rational numbers, this implies that either $a=0$ or $1/2-k = 0$ (or both). But $a$ cannot be 0, because $\langle \frac{0}{1} \rangle = \langle 0 \rangle = 0$.  Also, $1/2-k \neq 0$, since $k$ is an integer.  This contradiction shows that  our assumption that ${\mathbb Q} = \langle a \rangle$ is false. Therefore ${\mathbb Q} \neq \langle a \rangle$ for any rational number $a$, and ${\mathbb Q}$ is not cyclic.

\item
%Show that ${\mathbb C}$ is not cyclic.
\end{enumerate}

 
\subsection{Subgroups of cyclic groups}
\subsection{Answers for additional group and subgroup exercises}
%\textbf{Exercise \ref{ex:eoc:9}:}\\
%Let $A=\begin{pmatrix}
%x & y\\
%u & v
%\end{pmatrix}$ and $B=\begin{pmatrix}
%m & n\\
%o & p
%\end{pmatrix}$\\
%We have $\det(A)=xv-yu$ and $\det(B)=mp-on$\\
%We also have $AB=\begin{pmatrix}
%xm+yo & xn+yp\\
%um+vn & un+vp
%\end{pmatrix}$\\
%So we have $\det(AB)=...=xmvp+youn-umyp-voxn=\det(A).\det(B)$ in $GL_2(R)$.\\
%$A\in GL_2(R) \to$ A is invertible $\implies \det(A)\neq 0$\\
%$B\in GL_2(R) \to$ B is invertible $\implies \det(B)\neq 0$\\
%$\det(AB)=\det(A).\det(B)\implies \det(AB)\neq 0$\\
%$\implies AB$ is invertible $\implies AB\in GL_2(R)$\\
%So $GL_2(R)$ is closed under binary operation.\\
%\\
\textbf{Exercise \ref{ex:eoc:11}:}\\
Let $G$ is a group with 6 elements. We have:\\
$e\in G$\\
if $a\in G\implies a^{-1}\in G$\\
if $b\in G\implies b^{-1}\in G$\\
we need $ab\in G$ and $ba\in G$ to make G a group.\\
$\implies G$ must contains $a,b,e,a^{-1},b^{-1},ab,ba$\\
However, because $G$ has only 6 elements, we can make $ab=ba$. But then we also need $(ab)^{-1}\in G\implies G$ must have more than 6 elements.\\
Therefore, we disprove that every group containing 6 elements is abelian.\\
(Or you can give a specific counterexample).\\
\\
\textbf{Exercise \ref{ex:eoc:15}:}\\
$(g_1g_2...g_n)(g_n^{-1}g_{n-1}^{-1}...g_1^{-1}=g_1g_2...(g_ng_n^{-1})g_{n-1}^{-1}...g_1^{-1}$\\
$=g_1g_2...(g_{n-1}eg_{n-1}^{-1})...g_1^{-1}$\\
$=g_1g_2...(g_{n-1}g_{n-1}^{-1})...g_1^{-1}$\\
$=...$\\
$=g_1g_1^{-1}=e$\\
Similarly, we have $(g_n^{-1}g_{n-1}^{-1}...g_1^{-1})(g_1g_2...g_n)=e$\\
Therefore $g_n^{-1}g_{n-1}^{-1}...g_1^{-1}$ is the inverse of $g_1g_2...g_n$\\
\\
\textbf{Exercise \ref{ex:eoc:31}:}\\
Given that $H$ and $K$ are subgroups of $(G,\circ)$\\
(i) Let $x,y\in H\cap K \implies x,y\in H$ and $x,y\in K$\\
$\implies x\circ y\in H$ and $x\circ y\in K$ (because $H$ and $K$ are subgroups)\\
$\implies x\circ y\in H\cap K$\\
$\implies H\cap K$ is closed under $\circ$\\
(ii) $H$ and $K$ are subgroups $\implies e\in H$ and $e\in K$ by def. of subgroup\\
$\implies e\in H\cap K$\\ (by def. of intersection)\\
(iii) Let $x\in H\cap K \implies x\in H$ and $x\in K$ (def. of intersection)\\
$\implies x^{-1}\in H$ and $x^{-1}\in K$\\
$\implies x^{-1}\in H\cap K$\\
$\implies H\cap K$ is closed under inverse.\\
Therefore, the intersection of two subgroups of a group $G$ is also a subgroup of $G$.\\