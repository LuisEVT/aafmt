\chapter*{Forward}

\emph{To the student:}
\medskip

Many students start out liking math. Some like it well enough that they even want to teach it. However, when they reach advanced math classes (such as abstract algebra), they feel bewildered and frustrated. Their textbooks talk about mathematical concepts they've never heard of before, which have various properties which come from who knows where. In lectures, the prof. pronounces oracles (a.k.a “theorems”) and utters long incantations called ``proofs'' , but it's hard to see the point of either. 
\medskip

If the above paragraph describes you, then this book is meant for you!
\medskip

There's a good reason why higher math classes are bewildering for most students. I believe that we math instructors tend to take too much for granted.\footnote{My dad always says that math is excruciatingly difficult to learn, but once you get  it it's excruciatingly difficult to see why others can't understand it like you do.} We've forgotten that we are able to understand abstractions because we have concrete \emph{examples} in the back of our minds that we keep referring back to, consciously or subconsciously. These examples enable us to fit abstract ideas in with specific behaviors and patterns that we're very familiar with. But students who don't have a firm hold on the examples have nothing to hold on to, and are left grasping (and gasping) for air. 

To be sure, most students have previously been exposed to various important examples that (historically) gave rise to abstract algebra. These examples include the complex numbers, integers mod $n$, symmetries, and so on. They can give definitions and do some computations according to the rules. But they haven't been given a chance to \emph{internalize} these examples.  They can kind of follow along, but they don't really ``speak the language".

Our hope is that after reading this book students will be able to say, ``I've seen complex numbers, permutations and other such things before, but now I understand what makes them tick. I also see they have some very deep similarities with each other, and with other mathematical objects that I'm familiar with.'' 

This is actually a very good time to be learning abstract algebra. Long the province of specialists and puzzle enthusiasts, abstract algebra has recently made  impressive showings on the center stage of modern  science and technology. Two areas where abstract algebra has made strong contributions stand out particularly: information processing and physics. Coding of information is at the heart of information technology, and abstract algebra provides all of the methods of choice for information coding that is both reliable (impervious to errors) and private.  
On the other hand,  many of the recent advances in  physics are due to deeper understanding of physical symmetries and the groups that produce them. We try as much as possible to make connections with these two areas, and hope to do so increasingly in future editions. 

Enjoy the book, and send us your comments!
\medskip

\noindent
\emph{To the instructor}
\smallskip

This book  is not intended for budding mathematicians. It was created for a math program in which most of the students in upper-level math classes are planning to become secondary school teachers. For such students, conventional abstract algebra texts are practically  incomprehensible, both in style and in content. Faced with this situation, we decided to create a book that our students could actually read for themselves. In this way we have been able to dedicate class time to problem-solving and personal  interaction rather than rehashing the same material in lecture format.

Some instructors may feel that this book doesn't cover enough of the theory, and admittedly it falls short of the typical syllabus. But in far too many abstract algebra classes, the syllabus is covered and the students retain nothing. We feel it is much better to cover less but have the material stick. We have also avoided an ``abstract'' treatment and instead used specific examples (complex numbers, modular arithmetic, permutations, and so on) as stepping stones to general principles. The unhappy fact is that many students at this level haven't yet mastered these important basic examples, and it is useless to expect them to grasp abstractifications of what they don’t understand in the first place. Furthermore, these are the just the basic examples that will be most useful to them in their future career as high school teachers.

The book is highly modular, and chapters may be readily omitted if students are already familiar with the material. Some chapters  (``Preliminaries'' and ``Sigma Notation'') are remedial. Other chapters cover topics that are often covered in courses in discrete mathematics, such as sets, functions, and equivalence classes. (Much of this material is taken from the Morris' book, with some amplifications.) We have found from experience that students need this re-exposure in order to gain the necessary facility with these concepts, on which so much of the rest of the book is based.

Whenever possible we have introduced applications, which may be omitted at the instructor's discretion. However, we feel that it is critically important for preparing secondary teachers to be familiar with these applications. They will remember these long after they have forgotten proofs they have learned, and they may even be able to convey some of these ideas to their own students. 

\noindent
\emph{Additional resources}
\smallskip

This is the Information Age, and a mere textbook is somewhat limited in its ability to convey information.  Accordingly, as we continue to use the book in our classes, we are continuing to build an ``ecosystem''  to support the book's use:
\begin{itemize}
\item
The latest version of the book (and any accessory materials) may be found at the TAMU-CT Mathematics Department web page (go to 
\url{www.tamuct.edu} and search for ``Mathematics Resources'').
\item
An electronic version of the book is available online: the link may be found on the ``Mathematics Resources'' page cited above.
\item
 A comprehensive set of short video presentations of the book's content may be found on YouTube (search for the book's title). 
\item
An ``Instructor's Supplement'' is available upon request: email the editor (C.T.) from a verifiable faculty email address.
\item
Any instructor wishing to customize the material or extract certain portions may email the editor (C.T.)  to request the LaTeX source code. We have received freely from others, so we are happy to freely give.
\end{itemize}


\medskip

\noindent
\emph{Acknowledgements}
\smallskip

In our preparation of this text, we were fortunate to find via the web  some extraordinarily generous authors (Tom Judson, Dave Witte Morris and Joy Morris, A. J. Hildebrand) who freely shared their material with us. Thanks to them, we were able to put the first version of this  textbook together within the span of a single semester (not that we're finished -- this is a living book, not a dead volume). We hope that other instructors will similarly benefit from the material offered here.  

Several Master's students at Texas A\&M-Central Texas have made contributions to the book as part of a projects course or thesis. The original version was Justin Hill's Master's thesis. Holly Webb wrote the Group Actions chapter and David Weathers and Rachel McCoy wrote parts of ``In the Beginning'', ``A Sticky Problem'', "Sigma Notation'' and ``Polynomials``  as part of their coursework. Others have made contributions to both content and format. Johnny Watts our ``math technician'' helped with technical details. Khoi Tran contributed his excellent exercise solutions. 

Our very special thanks to Meghan DeWitt for her thorough critical reading of the book and her incisive comments. The book has greatly benefitted from her numerous suggestions.
\medskip 

Above all we want to acknowledge the One to whom all credit is ultimately due. ``Unless the LORD builds the house, the builders labor in vain. Unless the LORD keeps the city, the watchman is wakeful in vain.  It is vanity to rise up early, stay up late, and eat the bread of sorrows, for He gives sleep to those He loves."  (Psalm 127:1-2)

