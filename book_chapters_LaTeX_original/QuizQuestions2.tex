\chap{Practice Test Questions, Number Theory (Math 301), Fall 2015}{QuizPractice}

\begin{enumerate}[(1)]
\item
Simplify:
$\displaystyle{ \left(\frac{(x+y)^{x+y}(x-y)^{x-y}}{(x^2 - y^2)^x}\right)}$
\item
Simplify:
$ \displaystyle{\left( \frac{6^6}{2^2 3^3} +  \frac{2^8 3^6}{6^3}\right)^{1/2}} $
\item
Simplify:   $ \displaystyle{\frac{(a+b)(b+c) + (a-b)(b-c)}{a+c} }$
\item
Given the expression:  $( a(bc - cb) + (ac - ca)b) + c(ab - ba)$:
\begin{enumerate}[(a)]
\item
Simplify, using the associative property ONLY.
\item
Simplify, using the associative and distributive properties ONLY.
\item
Simplify, using associative, distributive, and commutative properties.
\end{enumerate}

\item
Give an example (using actual numbers) to show that division is not associative.
\item
Suppose $ab>cb, b < 0,$ and $c<0$.  For each of the following statements, either prove that it is always true, or give an example to show that
it is not always true:
\begin{enumerate}[(a)]
\item
$a > b$ \qquad 
\item
$a < 0$.
\item
$b < c$ \qquad 
\item
$a < c$ \qquad 
\end{enumerate}
\item
Evaluate: $(\sqrt{6}+3\sqrt{2}i)^6/36$.
%%%\item
%%%Prove that  $(z + \bar{z})(z - \bar{z})$ is real for any complex number $z$
\item
Suppose that $z$ is a complex number such that $z^{-1} = \bar{z}$.
\begin{enumerate}[(a)]
\item
 Find the modulus of $z$.
\item
How many solutions does this equation have?
\end{enumerate}
	
\item
Find all solutions to:  $\displaystyle{\frac{\bar{z}^4}{z} = 8i.}$
(\emph{Hint}:: Use polar form.)
%%%\item
%%%Find all solutions to:  $\bar{z}^3(z^2) = -32i.$
%%%\item
%%%Evaluate:  $\displaystyle{\frac{ \overline{3 + 8i} }{7 + 6i}}$.
\item
Evaluate:  $\displaystyle{( \overline{4 -7i} ) \cdot (\overline{3 + 3i})^{-1}}$.
\item
$z$ and $w$ are complex numbers. $z$ has modulus 7 and complex argument $\pi/9$, while $w$ has modulus $\sqrt{7}$ and argument $\pi/6$.  What are the modulus and argument of $z^3 w^{-4}$?
\item
December 25, 2015 is on a Friday.  What day of the week is December 1, 2018?  (Note: 2016 has 366 days).
\item
Find all fifth roots of $-1-i$.
%%%\item
%%%Given the expression:~
%%% $(((a-b)+b)+b)(a-b) + b^2$
%%%\begin{enumerate}[(a)]
%%%\item
%%%Simplify the expression using the associative law ONLY.
%%%\item
%%%Simplify the expression using the associative and distributive laws ONLY.
%%%\item
%%%Simplify the expression using the associative, distributive, and commutative laws.
%%%\end{enumerate}
%%%\item
%%%A cubic polynomial of the form $x^3 + ax^2 + bx + c$  ($a,b,c$ are real)  has roots $11$ and $3-i$.  Find $a,b,c$.
%%%\item
%%%Find all $6^{\text{th}}$ roots of  $8i$.
\item
Find all solutions to:  $228 x - 104 \equiv 777 \text{(mod 56)}$
\item
A polynomial of the form $x^4 + a_3x^3 + a_2x^2 + a_1x + a_0$  ($a_0,a_1,a_2,a_3$ are real)  has roots $3+2i$ and $1-i$.  Find $a_0,a_1,a_2,a_3$.
\item
Find all solutions to:  $ 470x - 120 \equiv 852 \text{(mod 93)}$
%%%\item
%%%Compute:  mod($30!,19$).  (Note: $30!$ means $1 \cdot 2 \cdot 3 \cdot \ldots \cdot 30$.)
\item
Compute mod( $3^{100}$,5). 

%\begin{enumerate}[(1)]
%\item
%Show that if $m$ is odd, then $\mod(m^2,8) = 1$.
%\item
%If$\mod(n,8)=1$ and $n = (x+y)(x-y)$, show that  $x$ must be odd and $y$ is divisible by 4.
%\item
%If$\mod(n,8)=7$ and $n = (x+y)(x-y)$, show that  $y$ must be odd.
%\item
%If$\mod(n,8)=5$ and $n = (x+y)(x-y)$, show that  $\mod(y,4) = 2$.
%\item
%If$\mod(n,8)=3$ and $n = (x+y)(x-y)$, show that  $y$ is odd.
%\end{enumerate}
%
%Instructions:  You may use a basic calculator that does addition, multiplication, division, and subtraction. No other helps
%\bigskip

\item
Evaluate: 
\begin{enumerate}[(a)]
\item
 gcd(111,507) 
%%%\item
%%%gcd(182,367) 
\item
gcd(39,409)
\end{enumerate}

\item
Find values of $m$ and $n$ that solve the following equations:
\begin{enumerate}[(a)]
\item
$88m + 97n = 19$
%%%\item
%%% 411m + 312n = 41 
\item
105m + 75n = 225
\end{enumerate}

\item
Perform the following matrix multiplication mod 44. Simplify before multiplying

$$\left(
\begin{array}{cc}
444 & 486 \\
890 & 606
\end{array}
\right)
\left(
\begin{array}{cc}
1103 & 2200 \\
990 & 133
\end{array}
\right)$$

%%%\item
%%%Perform the following matrix multiplication mod 37. Simplify before multiplying
%%%
%%%$$\left(
%%%\begin{array}{cc}
%%%409 & 372 \\
%%%743 & 189
%%%\end{array}
%%%\right)
%%%\left(
%%%\begin{array}{cc}
%%%-105 & 410 \\
%%% -300& -225
%%%\end{array}
%%%\right)$$

\end{enumerate}
