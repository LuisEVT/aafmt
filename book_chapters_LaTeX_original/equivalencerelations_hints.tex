\section{Hints for ``Equivalence Relations and Equivalence Classes'' exercises}
\label{sec:EquivalenceRelations:Hints} 

\noindent Exercise \ref{exercise:EquivalenceRelations:7}(a): There are two.~~(b): There are four.~~(c): There are four.~~(d): There are sixteen.~~(e): The answer is \emph{bigger} than 500!

\noindent Exercise \ref{exercise:EquivalenceRelations:RelGraphR2}(a.ii): The graph consists of shaded areas, not points or lines.

\noindent Exercise \ref{exercise:EquivalenceRelations:RelationDef}(c): There are 9.

\noindent Exercise \ref{exercise:EquivalenceRelations:17}(a): $\leq$ is \emph{not} symmetric -- you may show this by giving a counterexample.

\noindent Exercise \ref{exercise:EquivalenceRelations:17}(c): The ``Is it transitive?'' question amounts to answering the following:  Given $z_1 \sim z_2$ and $z_2 \sim z_3$.  Is it always true that $z_1 \sim z_3$?  If yes, prove it; and if no, give a counterexample. 

\noindent Exercise \ref{exercise:EquivalenceRelations:trickyTransitive}(a): There are actually three counterexamples, you only need to find one.

\noindent Exercise \ref{exercise:EquivalenceRelations:trickyTransitive}(b): Give a specific example where $a \sim b$ and $b \sim c$ but $a \not\sim c$.  In other words, $(a,b)$ and $(b,c)$ are elements of $R_{\sim}$, but $(a,c)$ is not in  $R_{\sim}$. It is not necessary for $a,b,$ and $c$ to be distinct.

\noindent Exercise \ref{exercise:EquivalenceRelations:trickyTransitive}(c): Explain why it is impossible to find a counterexample.

\noindent Exercise \ref{exercise:EquivalenceRelations:EquivRelShowEx}(b): You may assume (without proof) that the negative of any integer is an integer, and that the sum of any two integers is an integer. For transitivity, notice that $x - z = (x - y) + (y - z)$.

\noindent Exercise \ref{exercise:EquivalenceRelations:EquivRelShowEx}(c): This is similar to the proof in Example \ref{example:EquivalenceRelations:NxNEquivRelEg}, but with multiplication in place of addition.

