\section{Hints for ``Equivalence Relations and Equivalence Classes'' exercises}\label{secEqRelChapHints} 

\noindent Exercise \ref{exercise:EquivalenceRelationsChap:7}(a): There are two.~~(b): There are four.~~(c): There are four.~~(d): There are sixteen.~~(e): The answer is \emph{bigger} than 500!

\noindent Exercise \ref{exercise:EquivalenceRelationsChap:RelationDef}(c): There are 9.

\noindent Exercise \ref{exercise:EquivalenceRelationsChap:17}(a): $\leq$ is \emph{not} symmetric -- you may show this by giving a counterexample.

\noindent Exercise \ref{exercise:EquivalenceRelationsChap:trickyTransitive}(a): Give a specific example where $a \sim b$ and $b \sim c$ but $a \not\sim c$.  In other words, $(a,b)$ and $(b,c)$ are elements of $R_{\sim}$, but $(a,c)$ is not in  $R_{\sim}$. It is not necessary for $a,b,$ and $c$ to be distinct.

\noindent Exercise \ref{exercise:EquivalenceRelationsChap:trickyTransitive}(b): Explain why it is impossible to find a counterexample.

\noindent Exercise \ref{exercise:EquivalenceRelationsChap:EquivRelShowEx}(b): You may assume (without proof) that the negative of any integer is an integer, and that the sum of any two integers is an integer. For transitivity, notice that $x - z = (x - y) + (y - z)$.

\noindent Exercise \ref{exercise:EquivalenceRelationsChap:EquivRelShowEx}(c): This is similar to the proof in Example \ref{NxNEquivRelEg}, but with multiplication in place of addition.

\noindent Exercise \ref{exercise:EquivalenceRelationsChap:67}(a): What is the equation of a circle?~~(d): Use (b) and (c).
