\section{Study guide  for ``Modular Arithmetic''  chapter}
\label{sec:ModularArithmetic:StudyGuide} 


\subsection*{Section \ref{sec:ModularArithmetic:Introduction}, Introductory examples}
\subsubsection*{Concepts:}
\begin{enumerate}
\item 
Modular arithmetic
\item
Modulus
\end{enumerate}

\subsubsection*{Competencies}
\begin{enumerate}
\item
Be able to give the modulus involved in a practical problem involving ``cycles''. (\ref{exercise:ModularArithmetic:modulus1}) 
\end{enumerate}


\subsection*{Section \ref{sec:ModularArithmetic:ModEquiv}, Modular equivalence and modular arithmetic}
\subsubsection*{Concepts:}
\begin{enumerate}
\item 
Net displacement
\item 
Modular equivalence: two numbers are equivalent mod $m$ if they have the same remainder under division by $m$.
\item 
Modular equivalence (alternative formulation): Given $a, b, m \in \mathbb{Z}, \text{then}\\ a \equiv b(\text{mod}\, m) \mbox{ iff }\, m\mid(a-b)$
\item
Integers modulo $m$  (these are the possible remainders of integers under division by $m$)
\end{enumerate}

\subsubsection*{Notation}
\begin{enumerate}
\item
$\in$  means `contained in' or `elements of'
\item 
$\equiv$  means modular equivalence, similar to equality, but not quite the same
\item 
$\,\mid\,$  means `divides'
\end{enumerate}


\subsubsection*{Competencies}
\begin{enumerate}
\item
Determine whether or not two integers are equivalent modulo a given base. (\ref{exercise:ModularArithmetic:22}) 
\end{enumerate}


\subsection*{Section \ref{sec:ModularArithmetic:ModularEquations}, Modular equations}
\subsubsection*{Concepts:}
\begin{enumerate}
\item 
Application of modular arithmetic to UPC and ISBN codes
\item 
Transposition errors in scanning codes
\item
Solving modular equations
\end{enumerate}

\subsubsection*{Key Formulas}
\begin{enumerate}
\item
Inner product of two tuples: $(d_{1}, d_{2}, \dots, d_{k}) \cdot (w_{1}, w_{2}, \dots, w_{k}) = d_{1} w_{1} + d_{2}w_{2} + \cdots + d_{k}w_{k}$
\item    
UPC check formula: $(d_{1}, d_{2}, d_{3}, d_{4}, \dots, d_{12}) \cdot (3, 1, 3, 1, \dots, 1) \equiv 0 (\text{mod}\, 10)$
\item 
ISBN formula: $(d_{1}, d_{2}, \dots, d_{10}) \cdot (1, 2, \dots, 10) \equiv 0(\text{mod}\, 11)$

\indent{(note  $d_{10}$ might have to be a 10 to make the inner product 0, `X' is used to represent 10).}

\end{enumerate}

\subsubsection*{Competencies}
\begin{enumerate}
\item
Be able to validate UPC codes and find errors. (\ref{exercise:ModularArithmetic:UPCSymbols}) 
\item
Be able to validate ISBN codes and find errors. (\ref{exercise:ModularArithmetic:ISBNCodes}) 
\item
Be able to solve modular equations with small coefficients using trial and error. (\ref{exercise:ModularArithmetic:mod_eq_1}, \ref{exercise:ModularArithmetic:mod_eq_2})
\item
In modular equations, replace coefficients with their remainders before solving. (Example \ref{example:ModularArithmetic:speed_up1}, \ref{exercise:ModularArithmetic:modeq3})
\end{enumerate}


\subsection*{Section \ref{sec:ModularArithmetic:IntegerModn}, The integers mod $n$ (also known as ${\mathbb Z}_{n}$)}
\subsubsection*{Concepts:}
\begin{enumerate}
\item 
Modular addition and multiplication
\item
Cayley tables for addition and multiplication in ${\mathbb Z}_{n}$
\item
Closure properties of ${\mathbb Z}_{n}$
\item
Additive \& multiplicative identities and inverses in ${\mathbb Z}_{n}$
\item
Commutative, associative, \& distributive properties in ${\mathbb Z}_{n}$
\item
Definition of a group (a set with an operation that is closed, associative, has an identity, and all set elements have inverses)
\end{enumerate}

\subsubsection*{Key Formulas}
\begin{enumerate}
\item
Modular addition: $a, b \in {\mathbb Z}_{n}\, \text{then}\, a \oplus b = r\,\mbox{ iff }\, a + b = r + sn \, \text{and} \, r \in {\mathbb Z}_{n}$
\item
Modular multiplication: $a \odot b = r \, \mbox{ iff } \, a \cdot b = r + sn \, \text{and} \, r \in {\mathbb Z}_{n}$

\indent{(note that  = is used rather than $\equiv$ in modular addition and multiplication equations, since $a \oplus b$ is defined as equal to the remainder for modular addition and modular multiplication.)}
\end{enumerate}

\subsubsection*{Competencies}
\begin{enumerate}
\item
Be able to draw ``commutative diagrams'' that relate arithmetic in ${\mathbb Z}$ to arithmetic in ${\mathbb Z}_{n}$. (\ref{exercise:ModularArithmetic:diagram}) 
\item
Prove modular equivalence between arithmetic expressions involving integers and modular arithmetic expressions involving the integers' remainders. (\ref{exercise:ModularArithmetic:ModPower}, \ref{exercise:ModularArithmetic:ops}) 
\item
Simplify expressions mod $n$ by replacing terms in the expression with their remainders. (\ref{exercise:ModularArithmetic:prove})
\item
Know how to tell whether a set is closed under a certain arithmetic operation. (\ref{exercise:ModularArithmetic:53})
\item
Create tables for addition and multiplication mod $n$.
\item
Be able to find multiplicative inverses of elements in ${\mathbb Z}_{n}$, or prove they have none. (\ref{exercise:ModularArithmetic:60})
\item
Know the group properties by memory. (Definition \ref{definition:ModularArithmetic:group})
\item
Be able to show if elements of a given ${\mathbb Z}_{n}$ are a group or not. (\ref{exercise:ModularArithmetic:64})
\end{enumerate}


\subsection*{Section \ref{sec:ModularArithmetic:ModularDivision}, Modular division}
\subsubsection*{Concepts:}
\begin{enumerate}
\item 
Greatest common divisors (gcd)
\item 
Euclidean algorithm for finding gcd
\item 
Computing gcd using spreadsheets
\item 
Diophantine equations: $a \cdot m + b \cdot n = c$, where $a, b, c$ are integers, and $m$ and $n$ are assumed to have integer values.
\item 
Multiplicative inverse for modular arithmetic: If $a \in \mathbb{Z}_n$, then $x \in  \mathbb{Z}_n$ is the multiplicative inverse of $a$ in $\mathbb{Z}_n$ if $a \odot x = 1$.
\item 
Chinese remainder theorem
\end{enumerate}

\subsubsection*{Key Formulas}
\begin{enumerate}
\item
Euclidean algorithm formulas: $a = b \cdot q_{1} + r_{1}, b = r_{1} \cdot q_{2} + r_{2}, \\ r_{1} = r_{2} \cdot q_{3} + r_{3}, \dots$
\end{enumerate}

\subsubsection*{Competencies}
\begin{enumerate}
\item
Be able to find the greatest common divisor using the Euclidean algorithm. (\ref{exercise:ModularArithmetic:gcd})
\item
Be able to find all integer solutions to a Diophantine equation. (\ref{exercise:ModularArithmetic:dio1})
\item
Know the four group properties by heart (closure, identity, inverse, associative) and be able to tell from a Cayley table whether or not a certain set with a given operation is a group.  (\ref{exercise:ModularArithmetic:90})
\item
Solve pairs of congruences or show they have no common solution.(\ref{exercise:ModularArithmetic:91})
\end{enumerate}
