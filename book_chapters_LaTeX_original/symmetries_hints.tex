\section{Hints for ``Symmetries of Plane Figures'' exercises}\label{sec:Symmetry:Hints}

\noindent Exercise \ref{exercise:symmetries:mercedes}: The rearrangement that doesn't move anything is still considered to be a symmetry: for obvious reasons, it is called the \term{identity}).

\noindent Exercise \ref{exercise:symmetries:prop_proof}: The proof is very similar to part (ii) of the same proposition.

\noindent Exercise \ref{exercise:symmetries:33}: The proof is very similar to the previous proof.

\noindent Exercise \ref{exercise:symmetries:36}: Look at Figure~\ref{D4} for some ideas.

\noindent Exercise \ref{exercise:symmetries:PentagonRefl}(a): Look at Figure~\ref{types} for some ideas.

\noindent Exercise \ref{exercise:symmetries:HexagonRefl}(a): Look at Figure~\ref{types} for some ideas.

\noindent Exercise \ref{exercise:symmetries:41}(b): If $\mu$ is a reflection, then what is $\mu \compose \mu$?
