\section{Hints for ``Complex Numbers'' exercises}
\label{sec:ComplexNumbers:Hints} 


\noindent Exercise \ref{exercise:ComplexNumbers:m2even}(b)
Start your proof this way: ``Given that $m$ is an integer and $m^2$ is even. Suppose that $m$ is odd. Then $\ldots$'' (complete the proof by obtaining a contradiction. You should make use of part (a) in your proof.

\noindent Exercise \ref{exercise:ComplexNumbers:m2even}(c)
The proof is similar to that in (b).  What modifications do you need to make?

\noindent Exercise \ref{exercise:ComplexNumbers:6}(a)
 Start out your proof this way: ``Let $x$ be the cube root of 2.  Then $x$ satisfies the equation $x^3 = 2$.'' For the rest of the proof, follow closely the proof of Proposition~\ref{proposition:ComplexNumbers:irrational}. (Or use the statement$-$reason format, if you prefer.

\noindent Exercise \ref{exercise:ComplexNumbers:2}
Since $3 | n$, it follows that $n=3j$ for some integer $j$. Obtain a similar equation from $4|m$, and multiply your equations together.


\noindent Exercise \ref{exercise:ComplexNumbers:3}
Since $n | 4m$, it follows that $4m=n\cdot j$ for some integer $j$. Since $12 | n$, then what can you substitute for $n$?

\noindent Exercise \ref{exercise:ComplexNumbers:root3}
Try using contradiction. Suppose $n$ is even, so that $n = 2k$ for some integer $k$.

\noindent Exercise \ref{exercise:ComplexNumbers:complex_mult_inv}
To show $zz^{-1}=1$, rewrite $z^{-1}$ as $(a-bi) \cdot \frac{1}{a^{2}+b^{2}}$. This is justified by the distributive law.  Remember also that showing $z^{-1}z=1$ requires its own proof.


\noindent Exercise \ref{exercise:ComplexNumbers:9}(i)
In the answer $x + yi$, $x$ and $y$ both turn out to be integers!

\noindent Exercise \ref{exercise:ComplexNumbers:9}(n)
Yes, you can do it! Find the first few powers of $i$, and see the pattern.

\noindent Exercise \ref{exercise:ComplexNumbers:9}(o)
It's easiest to compute $(1+i)^2 \cdot (1+i)^2$.

\noindent Exercise \ref{exercise:ComplexNumbers:10}
If you have trouble with this one, do some examples.

\noindent Exercise \ref{exercise:ComplexNumbers:cxprops}(f)
 Use part (e).

\noindent Exercise \ref{exercise:ComplexNumbers:cxprops}(g) and (h)
See Exercise~\ref{exercise:ComplexNumbers:complex_mult_inv}.


\noindent Exercise \ref{exercise:ComplexNumbers:abs2}
Use part (b) of the previous exercise, plus
some of the results from Exercise~\ref{exercise:ComplexNumbers:cxprops}.


\noindent Exercise \ref{exercise:ComplexNumbers:abs3}(a)
Use the formula $|w|^2 = w\cdot \overline{w}$.  (d) This one requires calculus.

\noindent Exercise \ref{exercise:ComplexNumbers:29}
Just make minor changes to the previous exercise.

\noindent Exercise \ref{exercise:ComplexNumbers:cos form}(a) Replace $\cis \ldots$ with $\cos \ldots + i \sin \ldots$ on both sides of the de Moivre equation.  Then do the algebra on the right-hand side, and separate the real and imaginary parts. Recall that two complex numbers are equal iff their real parts and imaginary parts are equal separately.
(b)
Use a basic identity involving cosine and sine.

\noindent Exercise \ref{exercise:ComplexNumbers:42}(b)
What left shifts will change a cosine curve into a sine curve?

\noindent Exercise \ref{exercise:ComplexNumbers:43}
 Use Proposition~\ref{proposition:ComplexNumbers:polar_mult} to evaluate $\cis\theta \cdot \cis( t)$, and recall that $\cis(\alpha)$ means the same as as $\cos(\alpha) + i \sin(\alpha)$.


\noindent Exercise \ref{exercise:ComplexNumbers:50}(a)
We have already shown  in Proposition~\ref{proposition:ComplexNumbers:nonzero_complex_product} that the product of two nonzero complex numbers is never equal to 0. Use this to show that the product of four nonzero complex numbers is never equal to 0.


\noindent Exercise \ref{exercise:ComplexNumbers:50}(b)
Multiply out the inequality that you proved in (a).

\noindent Exercise \ref{exercise:ComplexNumbers:51}(a)
It's easier to multiply the numbers in polar form, you don't have to convert to Cartesian.  Note that $\cis\left(\frac{4\pi}{3}\right)$ is the complex conjugate of $\cis\left(\frac{2\pi}{3}\right)$.

\noindent Exercise \ref{exercise:ComplexNumbers:53}
Besides showing that all sides are equal, you also have to show that they are perpendicular.

\noindent Exercise \ref{exercise:ComplexNumbers:54}
Draw lines from each point to the origin, forming 6 triangles. What can you say about these triangles?

\noindent Exercise \ref{exercise:ComplexNumbers:57}(c)
Note that $OA = |z|, OC = |w|.$ and $AC = |z-w|$.

\noindent Exercise \ref{exercise:ComplexNumbers:58}(c)
Use your answer to part (b).

\noindent Exercise \ref{exercise:ComplexNumbers:60}(c)
 To find the polar form of this number, try squaring it.

\noindent Exercise \ref{exercise:ComplexNumbers:62}(b)
If $r$ is a solution to the above equation, then $z-r$ divides  $z^4 + a_{3}z^{3} + a_{2} z^{2}+ a_1 z  + a_0$.

\noindent Exercise \ref{exercise:ComplexNumbers:conjugate_root2}
 Let $M$ be the number of distinct solutions with positive imaginary part. Then how many distinct solutions are there with negative imaginary part? And how many non-real solutions are there altogether?

\noindent Exercise \ref{exercise:ComplexNumbers:mandel2}
 You may show that $(|z+w|)^2 \le (|z| + |w|)^2$.  When you take the square, use the identity that expresses $|z|^2$ in terms of $z$ and its complex conjugate.  After simplification, use polar form.  