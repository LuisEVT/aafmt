\chap{Practice Test Questions}{TestPractice2}


\begin{exercise}{}
Let $S$ be a set with three elements.
\begin{enumerate}[(a)]
\item
How many subsets does it have?
\item
How many proper subsets does it have?
\item
How many nonempty subsets does it have?
\item
How many nonempty proper subsets does it have?
\end{enumerate}
\end{exercise}

\begin{exercise}{7}
What is the smallest number of elements a set must have in order to have 20 proper subsets?
\end{exercise}

\begin{exercise}{12}
\begin{enumerate}[(a)]
\item
Find three sets $A_1, A_2, A_3$ which are \emph{not} disjoint such that $A_{1} \cup A_2 \cup A_3 = {\mathbb R}$ and $A_{1} \cap A_2 \cap A_3 = \emptyset$
\end{enumerate}
\end{exercise}

\begin{exercise}{20}
Let ${\mathbb N}$ be the universal set and suppose that
\begin{align*}
A = \{ x \in {\mathbb N} : x \text{ is divisible by 5}\} \\ 
B = \{ x \in {\mathbb N} : x \text{ is divisible by 7}\} \\ 
C = \{ x \in {\mathbb N} : x \text{ is greater than 50}\} \\
D = \{\text{the even natural numbers}\}
\end{align*} 

\noindent
Specify each of the following sets. You may specify a set either by describing a property, by enumerating the elements, or as one of the four sets $A, B, C, D$:
\begin{enumerate}[(a)]
\item
$(A \cap B) \setminus C$
\item
$A \cap B \cap C \cap D$
\item
$A \cup B \cup C \cup D$

\end{enumerate}
\end{exercise}

\begin{exercise}{cap_group}
Given a set $A$, let $G$ be the set of all subsets of $A$. 
\begin{enumerate}[(a)]
\item
Does the set $G$  with the operation $\cap$ have the closure property? \emph{Justify} your answer.
\item
Does the set $G$  with the operation $\cap$ have an identity? If so, what is it? Which part of  Proposition~\ref{proposition:Sets:sets_theorem_set_ops} enabled you to draw this conclusion?
\item
Is the operation $\cap$ defined on the set $G$ associative? Which part of  Proposition~\ref{proposition:Sets:sets_theorem_set_ops} enabled you to draw this conclusion?
\item
Is the operation $\cap$ defined on the set $G$ commutative? Which part of  Proposition~\ref{proposition:Sets:sets_theorem_set_ops} enabled you to draw this conclusion?
\item
Does each element of $G$ have a unique inverse under the operation $\cap$? If so, which part of  Proposition~\ref{proposition:Sets:sets_theorem_set_ops} enabled you to draw this conclusion? If not, provide a counterexample.
\item
Is the set $G$ a group under the $\cap$ operation?  \emph{Justify} your answer.
\end{enumerate}
\end{exercise} 

\begin{exercise}{}
List the elements of the set
$\{y,g,Y,G\} \times \{y,g,Y,G\}$
\end{exercise}

\begin{exercise}{funtable}
Here is a function~$f$ given by a table of values.

\begin{center}
\begin{tabular}{c|c}
$x$ & $f(x)$ \\ \hline

1 & 2 \\
2 & -2 \\
3 & -6 \\
4 & -10 \\
5 & -14 \\
\end{tabular}
\end{center}

\begin{enumerate}[(a)]
\item  \label{FunctionByTableEx-domain}
What is the domain of~$f$?
\item \label{FunctionByTableEx-range}
What is the range of~$f$? 
\item  \label{FunctionByTableEx-f(3)}
What is $f(3)$?
\item  \label{FunctionByTableEx-pairs}
Represent $f$ as a set of ordered pairs.
\item  \label{FunctionByTableEx-formula}
Find a formula to represent~$f$.
\hyperref[sec:Functions:Hints]{(*Hint*)}
\end{enumerate}
\end{exercise}

\begin{exercise}{ExOrderPairs}
Let  $A = \{ \var{a}, \var{b}, \var{c}, \var{d} \}$ and  $B = \{1,3,5,7,9\}$.
Which of the following sets of ordered pairs represent functions from~$A$ to~$B$?

\begin{multicols}{2}
\begin{enumerate}[a.]
\item \label{WhichAreFuncsEx-a1b3c5d7e9a11}
$\{ (\var{a}, 1), (\var{b}, 3), (\var{c}, 5), (\var{d}, 7), (\var{a},9) \}$
\item \label{WhichAreFuncsEx-a1b3c5e7}
$\{ (\var{a}, 1), (\var{b}, 3), (\var{c}, 5) \}$
\item \label{WhichAreFuncsEx-a1b1c1d1e1}
$\{ (\var{a}, 1), (\var{b}, 1), (\var{c}, 1), (\var{d}, 1) \}$
\end{enumerate}
\end{multicols}
\end{exercise}

\begin{exercise}{27}
 For the given sets $A$ and~$B$:
\begin{enumerate}[(a)] 
\item Write each function from~$A$ to~$B$ as a set of ordered pairs. (It turns out that if $|A| = m$
and $|B| = n$, then the number of functions from~$A$ to~$B$ is~$n^m$. Do you see why?)
\item Find the range of each function.
\end{enumerate}
\begin{enumerate}[i.]
\item \label{FunctionsChapExers-FindAll-ab,cde}
$A = \{\var{a},\var{b}\}$, $B = \{\var{c},\var{d},\var{e}\}$
%\item $A = \{\var{a},\var{b},\var{c}\}$, $B = \{\var{d},\var{e}\}$
\end{enumerate}
\end{exercise}

\begin{exercise}{11Exers}
Either prove the function is one-to-one, or prove that it is not.
\begin{multicols}{2}
 \begin{enumerate}[(a)]
\item \label{11Exers-formula-f}
 $f \colon {\mathbb R} \to {\mathbb R}$ defined by $f(x) = x^3 - x$.
\item \label{11Exers-formula-g}
 $g \colon {\mathbb R} \to {\mathbb R}$ defined by $g(x) = x^3$.
\item \label{modular_g}
 $g \colon {\mathbb Z}_6 \to {\mathbb Z}_6$ defined by $g(x)= (x \oplus 2) \odot 2$ .
\item \label{modular_g}
 $g \colon {\mathbb Z}_7 \to {\mathbb Z}_7$ defined by $g(x)= (x \oplus 2) \odot 2$ .
\end{enumerate}
\end{multicols}
\end{exercise}

\begin{exercise}{11Exers}
Either prove the function is onto, or prove that it is not.
\begin{multicols}{2}
 \begin{enumerate}[(a)]
\item \label{11Exers-formula-f}
 $f \colon {\mathbb R} \to {\mathbb R}$ defined by $f(x) = x^3 - x$.
\item \label{11Exers-formula-g}
 $g \colon {\mathbb R} \to {\mathbb R}$ defined by $g(x) = x^3$.
\item \label{modular_g}
 $g \colon {\mathbb Z}_6 \to {\mathbb Z}_6$ defined by $g(x)= (x \oplus 2) \odot 2$ .
\item \label{modular_g}
 $g \colon {\mathbb Z}_7 \to {\mathbb Z}_7$ defined by $g(x)= (x \oplus 2) \odot 2$ .
\end{enumerate}
\end{multicols}
\end{exercise}

\begin{exercise}{LinearWhenBijectionExer}
Let $a,b \in \mathbb{R}$, and define $f \colon \mathbb{R} \to \mathbb{R}$ by $f(x) = a x + b$. 
\begin{enumerate}[(a)]
\item \label{LinearWhenBijectionExer-not0}
Show that if $a \neq 0$, then $f$ is a bijection.
\item \label{LinearWhenBijectionExer-0}
Show that if $a = 0$, then $f$ is \emph{not} a bijection.
\end{enumerate}
\end{exercise}

\begin{exercise}{} 
For each function, either prove that it is a bijection, or prove that it is not.
\begin{enumerate}[(a)]
\item \label{modular9}
 $g \colon {\mathbb Z}_9 \to {\mathbb Z}_9$ defined by $g(x)= (x \odot 3) \oplus  (x \odot 4)$ .
\item \label{modular_m6}
 $g \colon {\mathbb Z}_7 \to {\mathbb Z}_7$ defined by $g(x) = (x \odot 4) \oplus (x \odot 4) $ .
 \end{enumerate}
\end{exercise}

\begin{exercise}{ComposeExers-form} 
 The formulas define functions $f$ and~$g$ from~$\mathbb{R}$ to~$\mathbb{R}$. Find formulas for $(f \compose g)(x)$ and $(g \compose f)(x)$.
\begin{enumerate}[(a)]
\item \label{ComposeExers-form-(ax+b)(cx+d)}
 $f(x) = ax + b$ and $g(x) = c x + d$ (where $a,b,c,d \in \real$)
\item \label{ComposeExers-form-(|x|)(-x)}
 $f(x) = |x|$ and $g(x) = -x$ 
\end{enumerate}
\end{exercise}

 \begin{exercise}{VerifyInverseExers}
 In each case, determine whether $g$ is an inverse of~$f$.
 \begin{enumerate}[(a)]
 \item \label{VerifyInverseExers-(x^2)}
$f \colon \real^+ \to \real^+$ is defined by $f(x) =2x^2$ and 
 \\ $g \colon \real^+ \to \real^+$ is defined by $g(y) = \sqrt{y}/2$.
 \item \label{VerifyInverseExers-(sqrt(x+1)-1)}
$f \colon \real^+ \to \real^+$ is defined by $f(x) = \sqrt{x+1} - 1$ and 
 \\ $g \colon \real^+ \to \real^+$ is defined by $g(y) = y^2 + 2y$.
 \end{enumerate}
 \end{exercise}

\begin{exer} \label{DrawBinRelExer}
Let $A =  \{-2,-1,0,1,2\}$ Draw a digraph for each of the following binary relations on~$A$: 
 \begin{enumerate}[(a)]
 \item \label{DrawBinRelExer-married}
 $ R_c = \{\, (x,y) \mid  (x-y)^2 < 2 \,\} .$
  \item \label{DrawBinRelExer-lived}
$ R_d = \{\, (x,y) \mid  x\equiv y \pmod{2} \,\} .$
 \end{enumerate}
 \end{exer}

\begin{exercise}{17}
For each of the following, explain your answers. 
\begin{enumerate}[(a)]
\item Define the relation $\rel$ on $\mathbb{Z}$ as follows: $ a \rel b$ iff $|a - b|< 4$. Is $\rel$ transitive? Is it reflexive? Is it symmetric?
\end{enumerate}
\end{exercise}

\begin{exer} \label{BinRelSomePropsEx}
Find binary relations on $\{1,2,3,4\}$ that meet each of the following conditions 
(Express each relation as a set of ordered pairs, and draw the corresponding digraph.)
\begin{enumerate}[(a)]
\item \label{BinRelSomePropsEx-transandsymm}
transitive and symmetric, but not reflexive.
\end{enumerate}
\end{exer}

\begin{exercise}{EquivClassEasyEx}
Let $B = \{1,2,3,4,5\}$ and 
	$$S = \left\{ (1,1),\, (1,4),\, (2,2),\, (2,3),\, (3.2),\, 
		(3,3),\, (4,1),\, (4,4),\, (5,5)
		 \right\} .$$
Assume (without proof) that $S$ is an equivalence relation on~$B$. Find the equivalence class of each element of~$B$.
\end{exercise}

\begin{exercise}{}
Let $C = \{1,2,3,4,5,6\}$ and define $\rel_C$ by 
\[ x \rel_C y \iff x + y \text{ is divisible by 3.} \]
Draw the arrow diagram for $\rel_C$, and prove that it is an equivalence relation.
\end{exercise}

\begin{exercise}{Mod3TablesEx}
Make tables that show the results of:
\begin{enumerate}[(a)]
\item \label{Mod3TablesEx-multiplication}
multiplication modulo~$5$.
\item \label{Mod3TablesEx-subtraction}
subtraction modulo~$5$ (For $\class{a} - \class{b}$,  put the result in row $\class{a}$ and column; $\class{b}$.)
\end{enumerate}
For both (a) and (b), all table entries should be  either $\class{0} \ldots \class{4}$.
\end{exercise}

\begin{exercise}{ModArithEx2}  
Find $x,y \in \integer_{51}$ such that $x \neq \class{0}$ and $y \neq \class{0}$, but $x \cdot y = \class{0}$.
\end{exercise}

\begin{exercise}{WellDefEx}
 Show that$f(x)=x^2$ provides a well-defined function from~$\integer_5$ to~$\integer_5$. That is, show that if $a,b \in \integer$, such that 
\[ [a]_5 = [b]_5, \text{ then } [ a^2]_7 = [ b^2]_7.\]
\end{exercise}

\begin{exercise}{57}
Show that there is a well-defined function 
$f \colon \integer_4 \to \integer_{12}$, given by \\
$ f \bigl( [a]_4 \bigr) = [a]_{12}$. 
That is, show that if $[a]_{12} = [b]_{12}$, then $f \bigl( [a]_4 \bigr) = f \bigl( [b]_4 \bigr)$.
\end{exercise}

\begin{exercise}{66}
Let $f\colon \ZZ_8 \to \ZZ_8$ be defined by $f(x) =  x^2 $. Define a relation $\sim$ on $\ZZ_8$ by: $n \sim m$ iff $f(n) = f(m)$.
\begin{enumerate}[(a)]
\item
Show that $\sim$ is an equivalence relation: that is, show that $\sim$ is reflexive, symmetric, and transitive.
\item
According to Proposition~\ref{EquivRel->Part}, this equivalence relation produces a partition on  $\ZZ_8$. List the sets in the partition.
\end{enumerate}
\end{exercise}

\begin{exercise}{16}
With reference to a hexagon with vertices labeled $A,B,C,D,E,F$ counterclockwise, for the symmetries $f$ and $g$:
\begin{enumerate}[(i)]
\item
Write the symmetries $f$ and $g$ in tableau form.
\item
Compute $f \compose g$ and $g \compose f$, expressing your answers in tableau form.
\item 
Describe the symmetries that correspond to $f \compose g$ and $g \compose f$, respectively.
\end{enumerate}
\medskip
$f=$rotation by $ 180^\circ, g=$reflection across the line $CF$
\end{exercise}

\bigskip
\begin{exercise}{}
\begin{enumerate}[(a)]
\item
Write the symmetries of a square (vertices labeled $A,B,C,D$) in tableau form.
\item
Write the Cayley table for the symmetries of a square. You may use letters to represent each symmetry. 
\item
List the inverses of each symmetry of the symmetries of a square.
\end{enumerate}
\end{exercise}

