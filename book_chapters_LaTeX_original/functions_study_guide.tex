\section{Study guide  for ``Functions: Basic Concepts''  chapter}
\label{sec:functions:study} 
\subsection*{Section \ref{cartesian}, The Cartesian product: a different type of set operation}
\subsubsection*{Concepts:}
\begin{enumerate}
\item 
Ordered pairs $(x, y)$
\item
Cartesian product of sets: the set of all ordered pairs
\item
Order of a set $S$ (i.e. number elements), denoted by $\mid S\mid $.
\end{enumerate}

\subsubsection*{Key Formulas}
\begin{enumerate}
\item
Equality of ordered pairs: $(x_{1}, y_{1}) = (x_{2}, y_{2})~ \mbox{ iff }  x_{1} = x_{2} \mbox{ and } y_{1} = y_{2}.$
\item
Cartesian product: $A \times B = \{ (a, b) \, | \, a \in A, b \in B \}$  (Definition \ref{cartesianprod})
\item
Order of a Cartesian product: Given any sets $A$ and $B$, then: \\ $\mid A \times B \mid = \mid A \mid \cdot \mid B \mid. $ (Proposition \ref{proposition:functions:prop1})
\end{enumerate}

\subsubsection*{Competencies}
\begin{enumerate}
\item
Given a pair of finite sets, list the elements of the Cartesian product. (Example \ref{example:functions:4}, \ref{exercise:functions:6}, \ref{exercise:functions:7})
\item
Determine the number of elements in a Cartesian product. (\ref{exercise:functions:9})
\end{enumerate}


\subsection*{Section \ref{intro}, Introduction to functions}
\subsubsection*{Concepts:}
\begin{enumerate}
\item A function accepts inputs, and provides a single output for each input.
\item 
Domain \& codomain of functions (``inputs'' and ``possible outputs'' of the function). 
\item
Range of a function (range are the ``actual outputs''; range is contained in any possible codomain)
\item
Image of an element of the codomain:  $f(a)$ is the image of $a$ under the function $f$.
\item
Arrow diagrams representing functions
\item
``Official'' definition of a function as a subset of the Cartesian product of domain and codomain (Definition \ref{functionDef})
\end{enumerate}

\subsubsection*{Competencies}
\begin{enumerate}
\item
Be able to give the domain, range, $f(x)$, the set of ordered pairs, and write a formula to represent a function. (\ref{exercise:functions:funtable}, \ref{exercise:functions:23})
\item
Be able to represent a function using: a formula; a set of ordered pairs; a 2-column table; an arrow diagram.
\item
Know if a set of ordered pairs represents a function.  (\ref{exercise:functions:ExOrderPairs}, \ref{exercise:functions:16})
\end{enumerate}


\subsection*{Section \ref{onetoone}, One-to-one functions}
\subsubsection*{Concepts:}
\begin{enumerate}
\item 
One-to-one functions (injective): each element of the range is the image of a \emph{unique} element of the domain.
\item
Contrapositive of a statement: the contrapositive of a statement of the form ``If $A$ then $B$'' is, ``If not $B$ then not $A$''.  The contrapositive is logically equivalent to the original statement.
\end{enumerate}

\subsubsection*{Competencies}
\begin{enumerate}
\item
Be able to identify one-to-one functions.  (\ref{exercise:functions:atomic}, \ref{exercise:functions:arrow})
\item
Be able to use the horizontal line test on real-valued functions to determine one-to-oneness.  (\ref{exercise:functions:horizontal}, \ref{exercise:functions:horizontal2})
\item
Prove whether functions are one-to-one or not. (\ref{exercise:functions:11Exers}, \ref{exercise:functions:40})
\end{enumerate}


\subsection*{Section \ref{onto}, Onto functions}
\subsubsection*{Concepts:}
\begin{enumerate}
\item 
Onto functions (surjective): each element of the codomain is the image of \emph{at least} one element of the domain.
\item
Onto proofs
\item
Horizontal line test to show onto-ness (applies only to real-valued functions)
\end{enumerate}

\subsubsection*{Competencies}
\begin{enumerate}
\item
Be able to identify onto functions. (\ref{exercise:functions:atomic_onto})
\item
Be able to use the horizontal line test for real-valued onto functions. (\ref{exercise:functions:horizontal_onto})
\item
Prove whether a function is onto or not.  (\ref{exercise:functions:OntoExers}, \ref{exercise:functions:OntoExers2}, \ref{exercise:functions:OntoExers3})
\end{enumerate}


\subsection*{Section \ref{composition}, Composition of functions}
\subsubsection*{Concepts:}
\begin{enumerate}
\item 
Composition of two functions: apply the second function to the output of the first function.  \emph{Note:} functions are applied \emph{right to left}.
\item
Proofs involving function composition
\end{enumerate}

\subsubsection*{Competencies}
\begin{enumerate}
\item
Be able to draw arrow diagrams of function compositions (Figure \ref{arrowcomposefig})
\item
Be able to compute the composition of two functions. (\ref{exercise:functions:RealWorldCompositionExer}, \ref{exercise:functions:ComposeExers-form})
\item
1-1 and onto proofs of compositions of functions, based on the 1-1 and onto properties of the functions being composed. (\ref{exercise:functions:CompositionTheoryExers}-\ref{exercise:functions:InverseMakesBijectionExer})
\end{enumerate}


\subsection*{Section \ref{inverse}, Inverse functions}
\subsubsection*{Concepts:}
\begin{enumerate}
\item 
Inverse functions: the functions $f \colon X \to Y$  and $g \colon Y \to X$ are inverses of each other \mbox{ iff } $g(f(x)) = x$ for all $x \in X$, and $f(g(y)) = y$ for all $y \in Y$. (Definition \ref{def:invfna})
\item
A function has an inverse \mbox{ iff } it is a bijection (both 1-1 and onto). (Theorem \ref{InverseBijection})
\item
Identity map: $\Id_{A} \colon A \to A$ by $\Id_{A}(a) = a$ for every $a \in A$. (Definition \ref{identityMap})
\item
$f \colon X \to Y$ and $g \colon Y \to X$ are inverses of each other  $\mbox{ iff } f \compose g = \Id_{Y} \text{  and  }g \compose f = \Id_{X}$. (\ref{exercise:functions:InverseIdentityExers})
\item
Inverse of compositions: if $f \colon X \to Y$ and $g \colon Y \to Z$ both have inverses, then so does $g \compose f$  and 
$(g \compose f)^{-1} = f^{-1} \compose g^{-1}$. (\ref{exercise:functions:InverseIdentityExers})
\end{enumerate}

\subsubsection*{Competencies}
\begin{enumerate}
\item
Determine whether or not $g$ is an inverse of $f$. (\ref{exercise:functions:VerifyInverseExers})
\item
Prove that the invertible functions must be bijections. (Theorem \ref{InverseBijection}, \ref{exercise:functions:InverseBijection2}, \ref{exercise:functions:InverseUniqueExers})
\item
Show that $\Id_{A}$ is invertible and find the inverse. (\ref{exercise:functions:IdAInverse})
\item
Prove facts about inverse of compositions and inverse of inverse functions (\ref{exercise:functions:InverseIdentityExers}a, b, c)
\end{enumerate}


\subsection*{Section \ref{functions_group}, Do functions from $A$ to $B$ form a group?}
\subsubsection*{Concepts:}
\begin{enumerate}
\item 
Abelian group (same as commutative group)
\end{enumerate}

\subsubsection*{Competencies}
\begin{enumerate}
\item
Be able to determine whether particular sets of functions form groups under composition. (\ref{exercise:functions:GpCompFun})
\item
Be able to prove whether or not a particular group of functions is abelian or not. (\ref{exercise:functions:abelian_ex})
\end{enumerate}
