\chap{Preliminaries}{Preliminaries}

\section{In the Beginning\quad
\sectionvideohref{tepwsFnVAqQ&list=PL2uooHqQ6T7PW5na4EX8rQX2WvBBdM8Qo&index=2}}
\label{sec:Preliminaries:InTheBeginning}

\begin{verse}
Let's start at the very beginning\\
A very good place to start\\
When you read you begin with A B C\\
When you sing you begin with Do Re Me
\end{verse}

(Oscar Hammerstein, \emph{The Sound of Music})

\begin{quote}
God made the integers; all else is the work of man.
 (Leopold Kronecker, German mathematician, 1886)
\end{quote}
\bigskip

If Maria had been more mathematically inclined, she might have continued:  ``When you count, you begin with 1 2 3''. Ordinarily we think of the ``counting numbers''  (which mathematicians call the \term{natural numbers} or \term{positive integers}) as the ``very beginning'' of math. 

It's true that when we learn math in school, we begin with the counting numbers. But do we really start at the ``very beginning''?   How do we know that $1 + 1 = 2$? How do we know that the methods we learned to add,  multiply, divide, and subtract will always work? We've been taught how to factor integers into prime factors. But how do we know this always works?

 Mathematicians  are the ultimate skeptics: they won't take ``Everyone knows'' or ``It's obvious''  as valid reasons. They keep asking ``why'', breaking things down into the most basic assumptions possible. The very basic assumptions they end up with are called \term{axioms}\index{Axiom}. They then take these axioms and play with them like building blocks. The arguments that they build with these axioms are called \term{proofs}, and the conclusions of these proofs are called \term{propositions}\index{Proposition!mathematical} or \term{theorems}\index{Theorem}.  

The mathematician's path is not an easy one. It is exceedingly  difficult to push things back to their foundations. For example, arithmetic was used for thousands of years before a set of simple axioms was finally developed (you may look up ``Peano axioms'' on the web).\footnote{The same is true for calculus. Newton and Leibniz first developed calculus around 1670, but it wasn't made rigorous until 150 years later.}
Since this is an elementary book, we are not going to try to meet rigorous mathematical standards. Instead, we'll lean heavily on examples, including the integers, rationals, and real numbers. Once you are really proficient with different examples, then it will be easier to follow more advanced ideas.\footnote{Historically, mathematics has usually progressed this way: examples first, 
and axioms later after the examples are well-understood.}

This text is loaded with proofs, which are as unavoidable in abstract mathematics as they are intimidating to many students. We try to ``tone things down'' as much as possible. For example, we will take as ``fact'' many of the things that you learned in high school and  college algebra--even though you've never seen proofs of these ``facts''.  In the next section  we remind you of some of these ``facts''.  When writing proofs or doing exercise feel free to use any of these facts.  If you have to give a reason, you can just say  ``basic algebra''.\index{Algebra!high school and college}

We close this prologue with the assurance that abstract algebra is a beautiful subject that brings amazing insights into the nature of numbers, and the nature of Nature itself. Furthermore, engineers and technologists are finding more and more practical applications, as we shall see in some of the later chapters.

The original version of this chapter was written by David Weathers. 

\section{Integers, rational numbers, real numbers}
\label{sec:Preliminaries:IntegersRationalRealNumbers}


We assume that you have already been introduced to the following number systems: integers, rational numbers, and real numbers.  These number systems possess the well-known arithmetic operations of addition, subtraction, multiplication, and division. The following statements hold for all of these number systems. 

\begin{warn}
There are number systems for which the following properties do NOT hold (as we shall see later). So they may be safely assumed ONLY for integers, rational numbers, and real numbers.
\end{warn}

\subsection{Properties of arithmetic operations}
\label{subsec:Preliminaries:OpsAndRels}

We assume the following properties of arithmetic operations on the integers, rational numbers, and real numbers. In the following list of properties, $a,b,c$ are arbitrary numbers (integers, rational, or real), unless otherwise specified. We use the notation $a \cdot b$ to denote the product of $a$ and $b$ (i.e. $a$ multipliled by $b$).

\begin{enumerate}[(A)]
\item
\term*{Additive identity}\index{Identity!in ordinary arithmetic}:   $0+a =a, a+0=a$.
\item
\term*{Multiplicative  identity}\index{Identity!in ordinary arithmetic}:   $1 \cdot a =a, a \cdot 1=a$.
\item
\term*{Additive inverse}\index{Inverse!in ordinary arithmetic}. For every number $a$ there is a unique number denoted $-a$ such that $a + \,\,-a=0$ and $-a + a = 0$.   Note that $a +\,\,-b$ is usually written as $a-b$.
\item
\term*{Multiplicative inverse} (**\emph{real and rational numbers only}**)   For every nonzero real or rational number $a$ there is a unique number $1/a$ such that $a \cdot 1/a = 1$ and $1/a \cdot a = 1$.  
\item
\term*{Addition is associative}\index{Associative!in ordinary arithmetic}: $(a+b)+c = a+(b+c)$.  (Note that the parentheses indicate which operation is performed first: for example, in $(a+b)+c$ the $a+b$ is done first, and then $c$ is added to the result. 
\item
\term*{Multiplication is associative } $(a\cdot b) \cdot c= a\cdot(b\cdot c)$  (Same comment applies as in previous property.)
\item
\term*{Addition is commutative }\index{Commutative!in ordinary arithmetic}: $a+b = b+a$  (Be careful about this one!  It's easy to take for granted.  We will see that in some number systems, it's not true.)
\item
\term*{Multiplication is commutative }: $a\cdot b = b\cdot a$  (Same comment applies as in previous property.)
\item
\term*{Multiplication distributes over addition}\index{Distributive!in ordinary arithmetic}:   $a\cdot (b + c) = (a\cdot b) + (a \cdot c)$ and $(a+b)\cdot c = (a \cdot c) + (b \cdot c)$.  (Technically, these are called the  \emph{left distributive} and \emph{right distributive} properties respectively.
\item \term*{Zero divisor property}\index{Zero divisors!in ordinary arithmetic}
$a \cdot 0 = 0$ and $0 \cdot a = 0$.
\end {enumerate}


\begin{exercise}{1}
\begin{enumerate}[(a)]
\item
For each of the properties (D,E,F,G,H)  above, give a specific equation (with actual numbers) that illustrates the property.
For example, for property (E) a specific example would be  $(3+5)+4 = 8 + 4 = 12$ is equal to 3+(5+4) = 3 + 9 = 12. 
\item
Give a specific example  that shows that subtraction  is \emph{not} commutative 
.\item
Give a specific example  that shows that division is \emph{not} associative. 
\end{enumerate}
\end{exercise}

\begin{exercise}{}
Which of the above properties must be used to prove each of the following statements? (Note each statement may require more than one property)
\begin{enumerate}[(a)]
\item
$(x+y)+(z+w) = (z+w)+(x+y)$
\item
$(x \cdot y) \cdot z = ( z \cdot  x) \cdot y$
\item
$(a\cdot x + a \cdot y) + a \cdot z = a \cdot ( (x+y) + z)$
\item
$((a \cdot b) \cdot c + b \cdot c) + c \cdot a = c \cdot ((a+b) + a \cdot b)$
\end{enumerate}
\end{exercise} 

Note that the associative property allows us to write expressions without putting in so many parentheses.  So instead of writing $(a+b)+c$, we may simply write $a+b+c$. By the same reasoning, we can remove parentheses from any expression that involves only addition, or any expression that involves only multiplication: so for instance, $(a \cdot (b \cdot c) \cdot d) \cdot e = a \cdot b \cdot c \cdot d \cdot e$. Using the associative and distributive property, it is possible to write any arithmetic expression without parentheses. So for example, $(a \cdot b) \cdot (c + d)$ can be written as $a \cdot b \cdot c + a \cdot b \cdot d$.  (Remember that according to operator precedence rules, multiplication is always performed before addition: thus $3 \cdot 4 + 2$ is evaluated by first taking $3 \cdot 4$ and then adding 2.)

There properties can be used to prove arithmetic statements that ordinarily we take for granted.  For example, we automatically replace $-1 \cdot a$ with $-a$, but this really needs to be justified. In fact, this requires one of the other properties in the above list:

\begin{exercise}{}
Show that $a + (-1 \cdot a) = 0 and (-1 \cdot a)=0$  (this is the same thing as showing that  $-1 \cdot a$ is the additive inverse of $a$, or $-1 \cdot a = -a$).  Which of the above properties did you use?
\end{exercise}

\begin{exercise}{}
Rewrite the following expressions without any parentheses and simplify as much as possible, but \emph{without using the commuitative property}.
\begin{enumerate}[(a)]
\item
$(((x + y) + (y+z))\cdot w) - 2y \cdot w$
\item
$0.5 \cdot ( (x+y) + (y + z) + (z + x))$
\item
$((((((a+b)+c) \cdot d)+ e) \cdot f) + g) + h$
\end{enumerate}
\end{exercise}

\begin{exercise}{}
For parts (a--c) of the preceding exercise, now apply the commutative property to the results to simplify the expressions as much as possible.
\end{exercise}

\begin{exercise}{abCommDistrib}
Given the expression:~
 $(((a-b)+b)+b)(a-b) + b^2$
\begin{enumerate}[(a)]
\item
Simplify the expression \emph{without using distributive or commutative property}.
\item
Simplify the expression\emph{without using the commutative property}.
\item
Simplify the expression using all laws.
\end{enumerate}
\end{exercise}

\begin{exercise}{pqrCommDistrib}
Given the expression:~
 $(r+p)(s+q) - (p+s)(q+r)$
\begin{enumerate}[(a)]
\item
Simplify the expression \emph{without using distributive or commutative property}.
\item
Simplify the expression \emph{without using the commutative property}.
\item
Simplify the expression using all laws.
\end{enumerate}
\end{exercise}

%\begin{exercise}{}
%Evaluate the following expressions by hand (no calculators!).
%\begin{enumerate}[(a)]
%\item
%$3 \cdot 4 + 5 + 6 \cdot 2$
%\item
%$3 + 4 \cdot 5 \cdot 6 + 2$
%\item
%$1 + 2 \cdot 3 + 3 \cdot 4 \cdot 5 + 5 \cdot 6 \cdot 7 \cdot 2$
%\end{enumerate}
%\end{exercise}


\subsection{Order relations}
\label{subsec:Preliminaries:OrderRelations}

We also have \term*{order relations}\index{Order relation!in ordinary arithmetic} on the real, rational, and integer number systems, which are expressed by the terms `greater than'  and 'less than' with corresponding symbols $>$ and $<$. If $a$ and $b$ are numbers, then  the mathematical statement `$a > b$' is logically identical to the statement  `$b<a$' (another way of saying this is: $a > b$ if and only if $b < a$). \term{Positive numbers} are defined to be those numbers greater than the additive identity 0, and \term{negative numbers} are defined to be those that  are less than 0. We assume the following properties of the order relation on the integers, rational numbers, and real numbers:

\begin{enumerate}[(A)] 
\item
The multiplicative identity $1$ is positive.
\item
Given two numbers, exactly one of these three are true: either the first number is greater than the second, or the second number is greater than the first, or the two numbers are equal.
\item
The sum of two  positive numbers  is positive. The sum of two negative numbers is negative.
\item
The product of two  positive or two negative numbers is positive. The product of a positive and negative number is negative.
\end{enumerate}

\begin{exercise}{3-a}
Using the above properties, show that $1+1$, $1+1+1$, and $1+1+1+1$ are all positive. (It can be shown by induction that the sum of any number of copies of 1 must be positive. The set $\{1, 1+1, 1+1+1, \ldots\}$ is called the \term{set of positive integers}.)
\end{exercise}


\begin{exercise}{3}
Suppose $a > b$,  $b \ge 0$ and $ab = 0$ (note that `$b \ge 0$' means that either $b>0$ or $b=0$).  What can you conclude about the values of $a$ and $b$? Use one (or more)  of the properties we have mentioned to justify your answer.
\end{exercise}

\begin{exercise}{}
Suppose $ab>cb, b < 0,$ and $c<0$.  For each of the following statements, either prove that it is always true, or give an example to show that
it is not always true:
\begin{enumerate}[(a)]
\item
$a > b$ \qquad 
\item
$a < 0$.
\item
$b < c$ \qquad 
\item
$a < c$ \qquad 
\end{enumerate}
\end{exercise}

Besides these order properties, there is a special order property that applies only to integers. This property is called the \term{principle of well-ordering}, and may be stated as a proposition as follows:

\begin{prop}{WellOrderPrinciple}\emph{(Well-ordering principle)}
Any set of positive integers  has a smallest element.
\end{prop}

\noindent
This may seem obvious, but in mathematics we have to do our best not to take anything for granted.  Sometimes the most ``obvious'' statements are the most difficult to prove. In this case, the well-ordering principle can be proved from the principle of mathematical induction (see Chapter~\ref{Induction}). The proof is beyond the scope of this course.\footnote{It is also possible to prove the principle of mathematical induction from well-ordering principle--it's a matter of personal preference which is taken as an axiom, and which is taken as a consequence.}


\subsection {Manipulating equations and inequalities}
\label{subsec:Preliminaries:EqsAndIneqs}

Following are some common rules for manipulating equations and inequalities. Notice there are two types of inequalities:  \term{strict inequalities} (that use the $>$ or $<$ symbols) and \term{nonstrict inequalities} (that use the $\ge$ or $\le$ symbols).\index{Inequality!strict}\index{Inequality!nonstrict} 

\begin{enumerate}[(A)]
\item
\term*{Substitution}\index{Substitution!in ordinary arithmetic}: If two quantities are equal then one can be substituted for the other in any true equation or inequality and the result will still be true. 
\item
\term*{Balanced operations}\index{Balanced operations!in ordinary arithmetic}: Given an equation, one can perform the same operation to both sides of the equation and maintain equality.  The same is true for inequalities for the operation of addition, and for multiplication or division by a \emph{positive} number.
\item
\term*{Inequality reversal:}\index{Inequality reversal!in ordinary arithmetic}
Multiplying or dividing an inequality by a negative value will reverse the inequality symbol.
\item
\term*{Fractions in lowest terms:}\index{Fractions!in ordinary arithmetic}
The ratio of two integers can always be reduced to lowest terms, so that the numerator and denominator have no common factors.
\end {enumerate}

\begin{exercise}{4}
Give specific examples for statements (A--D)   given above. You may use either numbers or variables (or both) in your examples.. For (A) and (B), give one example for each of the following cases: (i) equality, (ii) strict inequality, (iii) nonstrict inequality.
\end{exercise}

\begin{exercise}{}
Parts (a-f) of this exercise give a sequence of successive steps in a proof of an important arithmetic fact. For each of the steps, give either an arithmetic operation property (from Section~\ref{subsec:Preliminaries:OpsAndRels}) or an equation manipulation rule (from Section~\ref{subsec:Preliminaries:EqsAndIneqs}) which justifies the step.
\begin{enumerate}[(a)]
\item
$1 -1 = 0$
\item
$(1-1)\cdot a = 0 \cdot a$
\item
$(1-1)\cdot a = 0$
\item
$1 \cdot a + (-1) \cdot a = 0$
\item
$a + (-1) \cdot a = 0$
\item
$(-1) \cdot a$ is the additive inverse of $a$.
\end{enumerate}
\end{exercise}

As a result of the previous exercise, we have a proof of the following proposition:

\begin{prop}{minusOne}
For any integer, rational, or real number $a$ the following equation holds: $ -a =  (-1) \cdot a $. 
\end{prop}
This proposition may seem way too obvious to you, but it's actually saying something very significant. ``$-a$'' denotes the additive inverse of $a$, while ``$(-1) \cdot a$'' denotes the additive inverse of 1 times the number $a$. There is no a priori reason why theses two things should be the same. Try to think back to when you first learned this arithmetic stuff--at that time, it probably wasn't as obvious as it seems now. The exercise shows that it actually follows from even more basic facts about arithmetic.


The following exercise walks you through a proof of another important fact.

\begin{exercise}{}
For each step in the following argument, give either an arithmetic operation property (from Section~\ref{subsec:Preliminaries:OpsAndRels}) or an equation manipulation rule (from Section~\ref{subsec:Preliminaries:EqsAndIneqs}) which justifies the step.

\noindent
We first suppose that $a>b$ and $c>d$. 
\begin{enumerate}[(a)]
\item
$a-b>0$ and $c-d>0$
\item
$(a-b) + (c-d) > 0$
\item
$a + (-b+c) - d > 0$
\item
$a + (c-b) - d > 0$
\item
$(a+c) + (-b + -d) > 0$
\item
$((a+c) + (-b + -d)) + (b+d) > b+d$
\item
$(a+c) + ((-b + -d) + (b+d)) > b+d$
\item
$(a+c) + ((-b + -d) + (d+b)) > b+d$
\item
$(a+c) + (-b + ((-d + d) +b)) > b+d$
\item
$(a+c) + (-b + (0 +b)) > b+d$
\item
$(a+c) + (-b + b) > b+d$
\item
$(a+c)  + 0 > b+d$
\item
$a+c > b+d$
\end{enumerate}
\end{exercise}

The preceding exercise gives us a proof of the following proposition, which we will need later in the book.

\begin{prop}{sumIneq}
Let $a,b,c,d$ be integer, rational, or real numbers such that $a>b$ and $c>d$.  It follows that  $a+c > b+d$.
\end{prop}

%\section{Number Theory Rules}
%
%Given that a prime number $p$ evenly divides into a number $q$, it is true that the prime number $p$ must divide one of the factors of $q$.
%
%Given that the product of two numbers is equal to 0, then it is true that one of the numbers must be 0.


Finally, we're going to prove is that $-1$ is negative.  At this point you may be thinking, `` Duh,  it's got a minus sign, so of course it's negative!'' But if you look back in Section~\ref{subsec:Preliminaries:OpsAndRels} property (B), you'll see that the minus sign on $-1$ just means that it's the additive inverse of the multiplicative identity 1. On the other hand, negative numbers were defined in  Section~\ref{subsec:Preliminaries:OrderRelations} as numbers that are less than the additive identity 0.  Just because we've decided to write the additive inverse of 1 as $-1$, doesn't mean that we can automatically assume that $-1<0$.  Remember, be skeptical!

\begin{prop}{minusOne2}
$-1 < 0$
\end{prop}

\begin{proof}
This will be our first exposure to a proof technique called \term{proof by contradiction}. We'll make use of this technique throughout the book. In this case, the idea goes as follows.  There's no way that $-1$ could be positive, because if it were then $1 + (-1)$ would also have to be positive, which it isn't because we know it's 0.  There's also no way that $-1$ could be 0, because if it were we'd have $-1=0$, and adding 1 to both sides gives $0 = 1$, which is false because 1 is positive and 0 isn't.  Since -1 isn't positive and it isn't equal to 0, the only option left is that it's negative. This is the gist of the argument, but we have to write it out more carefully to satisfy those nit-picking mathematicians. Every step in our argument must have a solid reason. 

So here goes the formal proof. We'll give a logical sequence of mathematical statements, followed by a reason that justifies each statement--this is called \term*{statement-reason format}\index{Proof!statement-reason format}\index{Statement-reason!proof}.

\noindent
First we show that $-1 > 0$ is false:

\begin{tabular}{l| l}
Statement& Reason\\
\hline
Suppose $-1>0$ . & Proof by contradiction: supposing the opposite\\
$1>0$ &  Prop. (A) in Section~\ref{subsec:Preliminaries:OrderRelations}\\
$1 + (-1)> 0$. & Prop. (C) in  Section~\ref{subsec:Preliminaries:OrderRelations}\\
$1 + (-1) = 0$ & Prop. (B) in Section~\ref{subsec:Preliminaries:OpsAndRels}\\
Contradiction is acheived&  The last 2 statements contradict\\
$-1>0 $ is false& The supposition must be false
\end{tabular}

\noindent
Next, we show that $-1=0$ is false:

\begin{tabular}{l| l}
Statement& Reason\\
\hline
Suppose $-1=0$ & Proof by contradiction: supposing the opposite\\
$1 + (-1) = 1 + 0$ & Follows from previous statement by substitution\\
$0=1$   & Props. (A) and (B) in Section~\ref{subsec:Preliminaries:OpsAndRels}\\
$0>0$  & Prop. (A)  in  Section~\ref{subsec:Preliminaries:OrderRelations}\\
Contradiction is achieved & $0>0$ contradicts Prop. (B)  in  Section~\ref{subsec:Preliminaries:OrderRelations}\\
$-1 = 0$ is false &The supposition must be false
\end{tabular}

\noindent
According to   Property (B) in  Section~\ref{subsec:Preliminaries:OrderRelations}, there are three  possibilities: either $-1>0, -1=0,$ or $-1<0$.  We have eliminated the first two possibilities.  So the third possibility must be true:  $-1<0$. This completes the proof.

\end{proof} 
\footnote{The '$\square$' symbol will be used to indicate the end of a proof. In other words: Ta-daa!}

\begin{exercise}{}
Using Proposition~\ref{proposition:Preliminaries:minusOne}  Proposition~\ref{proposition:Preliminaries:minusOne2}, and one of the order relation properties, show that the additive inverse of any positive number is negative.
\end{exercise}


\subsection {Exponentiation (VERY important)}
\label{subsec:Preliminaries:Exponentiation}

Exponentiation is one of the key tools of abstract algebra. It is \emph{essential} that you know your exponent rules inside and out!  

\begin{enumerate}[(I)]
\item
Any nonzero number raised to the power of 0 is equal to 1.
\footnote{ Technically $0^0$ is undefined, although often it is taken to be 1. Try it on your calculator!}
\item
A number raised to the sum of two exponents  is the product of the same number raised to each individual exponent.
\item
A number raised to the power which is then raised to another power is equal to the same number raised to the product of the two powers.
\item
The reciprocal of a number raised to a positive power is the same number raised to the negative of that power.
\item
Taking the  product of two numbers  and raising to a given power is the same as taking the powers of the two numbers separately, then multiplying the results.
\end{enumerate}

\begin{exercise}{5}
For each of the above items (I--V),  give a general equation (using variables) that expresses the rule.  For example one possible answer to (II) is:  $x^{y+z} = x^y \cdot x^z$ .
\end{exercise}
\begin{exercise}{6}
Write an equation that shows another way to express a number raised to a power that is the difference of two numbers.
\end{exercise}

\section{Test yourself}
\label{sec:Preliminaries:TestYourself}
Test yourself with the following exercises. If you feel totally lost, I strongly recommend that you improve your basic algebra skills before continuing with this course. Trying to do higher math without a confident mastery of basic algebra is like trying to play baseball without knowing how to throw and catch.

\begin{exercise}{TestYourselfQ1}
Simplify the following expressions. Factor whenever possible
\begin{multicols}{2}
\begin{enumerate}[(a)]
\item
$ 2^4 4^2$
\item
$ \dfrac{3^9}{9^3}$
\item
$\left( \dfrac{5}{9} \right)^7 \left( \dfrac{9}{5} \right)^6$
\item
$\dfrac{a^5}{a^7} \, \cdot \, \dfrac{a^3}{a}$
\item
$x(y-1) - y(x-1)$
\end{enumerate}
\end{multicols}
\end{exercise}



\begin{exercise}{TestYourselfQ2}
Same instructions as the previous exercise. These examples are  harder. (\emph{Hint}: It's usually best to make the base of an exponent as simple as possible. Notice for instance that $4^7 = (2^2)^7 = 2^{14}$.)
\begin{multicols}{2}
\begin{enumerate}[(a)]
\item
$6^{1/2\cdot}2^{1/6}\cdot3^{3/2}\cdot2^{1/3}$
\item
$(9^3)(4^7)\left(\frac{1}{2}\right)^8\left(\frac{1}{12}\right)^6$
\item
$4^5 \cdot 2^3 \cdot \left(\frac{1}{2}\right)^5 \cdot \left( \frac{1}{4} \right) ^3$
\item
$2^3 \cdot 3^4 \cdot 4^5 \cdot 2^{-5} \cdot 3^{-4} \cdot 4^{-3}$
\item
$\dfrac{x(x-3)+3(3-x)}{(x-3)^2}$
\end{enumerate}
\end{multicols}
\end{exercise}


\begin{exercise}{TestYourselfQ3}
Same instructions as the previous exercise. These examples are even harder. (\emph{Hint}: Each answer is a single term, there are no sums or differences of terms.)
\begin{multicols}{2}
\begin{enumerate}[(a)]
\item
$\dfrac{a^5 +a^3 - 2a^4}{(a-1)^2}$
\item
$a^x b^{3x}(ab)^{-2x}(a^2 b)^{x/2}$
\item
$(x+y^{-1})^{-2}(xy+1)^2$
\item
$\dfrac{(3^x+9^x)(1-3^x)}{1-9^x})$
\item
$\dfrac{3x^2 - x}{x-1} + \dfrac{2x}{1-x}$
\item
$\displaystyle{ \left(\frac{(x+y)^{x+y}(x-y)^{x-y}}{(x^2 - y^2)^x}\right)}$
\item
$ \displaystyle{\left( \frac{6^6}{2^2 3^3} +  \frac{2^8 3^6}{6^3}\right)^{1/2}} $
\item
$ \displaystyle{\frac{(a+b)(b+c) + (a-b)(b-c)}{a+c} }$

\end{enumerate}
\end{multicols}
\end{exercise}

\begin{exercise}{TestYourselfQ4}
Find ALL real solutions to the following equations. 
\begin{multicols}{2}
\begin{enumerate}[(a)]
\item
$x^2 = 5x$
\item
$(x - \sqrt{7})(x+\sqrt{7}) = 2$
\item
$2^{4+x} = 4(2^{2x})$
\item
$3^{-x} = 3(3^{2x})$
\item
$16^5 = x^4$
\item
$\dfrac{1}{1 + 1/x} -1= -1/10$
\end{enumerate}
\end{multicols}
\end{exercise}

\begin{exercise}{TestYourselfQ5}
(\emph{Challenge problems})  These problems come from Chinese high school math web sites (thanks to J. L. Thron)
\begin{enumerate}[(a)]
\item
Simplify: $\frac{2^{n+4} - 2(2^n)}{2(2^{n+3})}$
\item
Given $m = 7^9$ and $n=9^7$, express $63^{63}$ in terms of $m$ and $n$.
\item
Given $2^x3^y = 10$ and $2^y3^x = 15$, find $x$ and $y$.
\item
Show that the following expression always has real roots:  $(x-3)(x-2) = a(a+1)$, where $a$ is any real number.
\item
If $3x-5y+3=0$, find $\displaystyle \frac{8^{x+2}}{32^y}$.
\item
Solve for $x$:  $(6x+7)^2(3x+4)(x+1) = 6$  (multiply to obtain two quadratic terms, then substitute)
\item
Solve for $x$: $9^x + 12^x=16^x$.  (divide the equation by one of the terms)
\item
Solve for $x$:  $\frac{x+1}{x+2} + \frac{x+8}{x+9} =\frac{x+2}{x+3} + \frac{x+7}{x+8}$. (Simplify the numerators in each fraction)
\item
Given that $m=2019^2 + 2020^2$, evaluate $\sqrt{2m-1}$. (use the fact that 2020 = 2019+1)
% 2(a-1)^2 + 2a^2 - 1 = 4(a^2 -a +1/4)
%\item
%Find the smallest integer $n$ such that $\sqrt{31n+2015}$ is an integer.
%\item
%Given that $a = \sqrt{19 - 8\sqrt{3}}$, show that $\frac{a^4 - 6a^3 - 2a^2 + 18a + 23}{a^2-8a+15} = 5$.
\item
Solve for $x$: $\sqrt{x^2 + 9} + \sqrt{x^2-9} = 5 + \sqrt{7}$. (To avoid squaring twice, use difference of squares to obtain a second equation, then use the two equations together to eliminate one of the square roots.)
\item
Given $a = 4^{1/3} + 2^{1/3} + 1$, evaluate $\frac{3}{a} + \frac{3}{a^2} + \frac{1}{a^3}$. (Write out the expressions for  
$\frac{1 - x^3}{1-x}$ and $(1+y)^3$, and see if you can relate them to the given expressions)
\item
Solve for $x$: $x = \sqrt{x - \frac{1}{x}} + \sqrt{1 - \frac{1}{x}}$.  (To avoid squaring twice, use difference of squares to obtain a second equation, then use the two equations together to eliminate one of the square roots.)
\item
Suppose that $a+b+c=0$ and $a^3 + b^3 + c^3=0$.  Show that $a^n + b^n + c^n = 0$ for all odd values of $n$. (Look at two cases:  (a) at least one of $a,b,c$ is equal to 0; (b) exactly two of the numbers have the same sign (without loss of generality, you may assume that $a,b>0$ and $c<0$)).
\item
Given $a^2-9a+1=0$, find $\displaystyle a^2 - 7a + \frac{18}{a^2+1}$.  (solve the first equation for $a^2$ and for $a^2+1$, and use  substitutions.)
\item
Given that $x_1$ and $x_2$ are both solutions to the equation $x^2+1 = 1/x$, find $2021^{x_1 - x_2}$  (graph the functions $y = x^2+1$ and $y = 1/x$).
\item
Given that $x+y=3$ and $xy=1$, evaluate $x^5 + y^5$  (use the first two expressions to find quadratic equations for $x$ and $y$, then substitute repeatedly for $x^2$ and $y^2$ in $x^5 + y^5$).
\item
Given that $\displaystyle \frac{a+b}{c} = \frac{a+c}{b} = \frac{b+c}{a}$, find the value of $\displaystyle \frac{abc}{(a+b)(b+c)(c+a)}$  (Be careful! There may be more than one answer. Take two of the equations and clear the denominators. Both sides will have a common factor, which may or may not be zero.)
\item
Given $x= 2 + \sqrt{2}$, find $x^4 - 4x^3+7x^2 - 20x + 16$.(Find a quadratic equation satisfied by $2 + \sqrt{2}$.
\item
Given $4x^{-4} - 2x^{-2}=3$ and $x^4 + y^2 = 3$, find $4x^{-4} + y^4$.
\item
Given $30^x = 2010$ and $67^y = 2010$, find $x^{-1} + y^{-1}$.
\item
Given $a+b=6$ and $ab + (c-a)^2 + 9=0$, find $a+b+c$  (Try to find a particular solution for $a,b,c$.  Look at the signs of the terms.)
\item
Simplify $\displaystyle \frac{1234^2}{2469^2 + 2467^2 - 2}$ (no calculator required!)
\item
Given $2^a = 10, 2^ b = 5, 2^c = 200$, compute $a -4041b + 2020c - 6060$. (Exponent rules!)
\item
Given$\displaystyle \frac{xy}{x+y}=1, \frac{yz}{y+z} = 2, \frac{xz}{x+z} = 3$, find $x$. (Take reciprocals and break the fractions apart.  Then add together the equations.)
\item
Given that $a_1, a_2,\ldots a_{1000}$ are the first 1000 terms of a geometric series with $a_1= 1/5$ and $a_{1000} = 20$. The product $a_1 \cdot a_2 \cdot \ldots \cdot a_{1000}$ can be expressed as $2^x$.  Find $x$. (Recall that the $n$th term of a geometric series has the form $a r^n$.  Group terms in the geometric series in pairs.)
\item
Without using a calculator, determine which is larger:  $9^{12}$ or $15^9$. 
%\item
%Given:  $\sqrt{8 - \sqrt{40 + 8 \sqrt{5}}} + \sqrt{8 + \sqrt{40 + 8 \sqrt{5}}}=x$, express $x$ in the form $\sqrt{m} + \sqrt{n}$ where $m,n$ are positive integers. 
\end{enumerate}
\end{exercise} 