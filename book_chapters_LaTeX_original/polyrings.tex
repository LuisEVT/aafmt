%%%% Search for '%%%%' and address issues
%%%% make sure the following is taken care of

% Important results -- Gauss lemma, solution by radicals
% make sure definition of multiplication props is correct
% multiplication -- usual definition (term by term). Then show rearrangement. Make sure the same result is not duplicated elsewhere.
% Matrices -- similarity of multiplication with matrix multiplication.  Shows associativity immediately. Matrices also form a ring (not commutative)
% Preparation for coding theory.


\chap{Polynomials}{poly}
 

 In this chapter we'll be looking at  polynomials from an algebraic point of view. First we'll review basic  polynomial arithmetic that you've seen in high school; then we'll jump off from there and see how far we can generalize. In doing so, we will be naturally led to the algebraic notion of a \term{ring}, which somewhat resembles a group but has two operations instead of just one.
\bigskip

This chapter contains contributions from David Weathers, Johnny Watts, Jennifer Lazarus, and Semi Harrison (edited by C.T.). Thanks to Tom Judson for material used in this chapter. 

\section{General remarks}
Undoubtedly you've seen polynomials quite a bit in high school math. You've added and multiplied them; you've graphed them; you've factored them; you've found roots. Let's take a moment to remind ourselves why polynomials and their operations are important.	

Polynomials are used to express relationships between variables. For example, we may consider the situation  of a vehicle that is moving on a straight road. We'll use $x$ to denote the position, and $t$ to denote time.   If (for example) the vehicle has an initial displacement of 100 meters, initial velocity equal to 40 m/sec, and constant acceleration $-5$ $\text{m/sec}^2$, then we may write the following relationship between position ($x$) and time ($t$):
$$ x =  -\frac{5}{2} t^2+ 40t + 100$$
(the factor of $1/2$ in the $t^2$ term comes from calculus). In this equation, we have expressed $x$ as a function of $t$: in other words, $t$ is the \term{independent variable}, and $x$ is the \term{dependent variable}.

Now suppose a second vehicle is moving along the same road  in the opposite direction. We'll represent this vehicle's position as $y$, and suppose that $y$ depends on $t$ as follows:
$$ y =   t^2- 30t + 600.$$

If we're interested in the position of the second vehicle relative to the first, then we should take $y - x$, which we can also write as $y + (-1)x$:

$$ y + (-1)x =  (t^2- 30t + 600) + (-1)(-\frac{5}{2} t^2+ 40t + 100).$$

This equation illustrates two operations with polynomials, namely \term*{scalar multiplication}\index{polynomial!scalar multiplication of}\index{scalar multiplication!of polynomials} and \term*{polynomial addition}\index{polynomial!addition}.  Naturally we may perform the operations and obtain:
$$ y -x =  \frac{7}{2} t^2- 70t + 500.$$
If we are interested in the time(s) at which the two vehicles meet (hopefully without colliding!), then we need to find the solution(s) (also known as the \term*{roots}\index{roots!of polynomial}\index{polynomial!roots} of  $y-x = 0$, or
$$   \frac{7}{2} t^2 - 70t + 500 = 0.$$
It is interesting to note that even though the coefficients of this polynomial are rational numbers, in general the solution(s) will not be rational numbers. (In fact, we know from the quadratic formula that in some cases the solutions are not even real numbers!)

Now suppose instead that the two vehicles are moving on two perpendicular roads which cross at $(0,0)$.  In this case, the square of the distance between the two vehicles is given by (using the Pythagorean theorem)
\begin{align*}
(\text{Distance between vehicles})^2 &= x^2 + y^2 \\
&= (-\frac{5}{2} t^2+ 40t + 100)^2 + (t^2- 30t + 600)^2.
\end{align*}
Here we see both polynomial addition and \term*{polynomial multiplication}. Using polynomial arithmetic (see next section), we find:
\begin{align*}
(\text{Distance between vehicles})^2 = \frac{29}{4}t^4 - 260t^3+3200t^2-28,000t+370000.
\end{align*}
If we would like to find the time(s) at which the relative distance is equal to 500, we should solve

$$500^2 = \frac{29}{4}t^4 - 260t^3+3200t^2-28,000t+370000,$$
which can be rearranged to give
$$ \frac{29}{4}t^4 - 260t^3+3200t^2-28,000t+120000=0.$$
This equation has two real solutions and two complex solutions. (In Section~\ref{sec:FTOA} we saw that a  polynomial of degree four always has at least 1 and at most 4 distinct complex roots.) The real solutions correspond to the two times that the cars are 500 meters apart. 

\begin{exercise}{poly1}
Suppose vehicle 1 has an initial position of $x_0=150$ m, an initial velocity of $v_0=60$  m/sec, and a constant acceleration of $a=-8$ $\text{m/sec}^2$. Additionally, suppose vehicle 2 has an initial position of $y_0=80$ $\text{m}$, an initial velocity of $ v_0=-50$  m/s, and a constant acceleration of $a=2$ $\text{m/sec}^2$.
\begin{enumerate}[(a)]
\item
Express the position of the second vehicle relative to the first, assuming they are moving on the same road in opposite directions. Determine the time(s) at which the vehicles meet.
\item
Determine the time, $t>0$, at which the distance between the vehicles is equal to $400$ $\text{m}$, if the vehicles are moving on two perpendicular roads which cross at $(0, 0)$. Give an answer that is correct to three decimal places.
\end{enumerate}
\end{exercise}
% a) y-x=y+(-1)x=(t^2-50t+80)+(-1)(-4t^2+60t+150)=5t^2-110t-70. 5t^2-110t-70=0
% t_1=11-3\sqrt{15}$\approx$-0.61895 sec. t_2=11+3\sqrt{15}$\approx$22.619 sec
% b) (Distance between vehicles)^2=x^2+y^2=(-4t^2+60t+150)^2+(t^2-50t+80)^2=17t^4-580t^3+5060t^2+10000t+28900
%17t^4-580t^3+5060t^2+10000t+28900=400^2
% t_1$\approx$-4.6735 sec t_2$\approx$5.6650 sec

We hope that this discussion will convince you that the study of polynomials and their operations is eminently practical, and not just a mathematical pastime.

\section{Review of polynomial arithmetic\quad
\sectionvideohref{2FuGOJ3kHTU&list=PL2uooHqQ6T7PW5na4EX8rQX2WvBBdM8Qo&index=47}}\label{sec:polyStripe}

%	A tennis player hits a ball with an initial velocity of 95 ft/s, at a height of 3.5 feet and an angle of $30^{\circ}$. The general equation for the height of a projectile, $y(t)$  as a function of time is as follows:
%$$y(t)=\frac{1}{2}gt^2+v_0 \sin(\theta)t+y_0,$$ 
% where $ g≈-32.2 ft/s^2 $, $v_0$ is the initial velocity, $y_0$ is the starting height, and $\theta$ is the launch angle.
%After substitution, we have the following quadratic polynomial:
%$$y(t)=-16.1 t^2+47.5t+3.5$$
%Now suppose, we want to know  the time, $t$, that the ball willl be at a height of 8 feet.
%After substituting y(t) with the height of 8 feet and rearranging, we have,
%$$ -16.1t^2+47.5t-4.5=0$$
%Solving the above quadratic polynomial for the time, $t$, of interest, is routine and shows just one of a plethora of applications of polynomial algebra.

Let's briefly review what you've previously learned about polynomial arithmetic.  We'll cover polynomial addition, subtraction, and multiplication: polynomial division is a bit more complicated, so we'll talk about that later.

We'll begin with an example. Let
$$ p(x)  = x^3 -3x +2  \quad \text{and} \quad q(x)  = 5x^3+3x^2 -6x +5.$$
Then we can add $p(x)$ and $q(x)$ as follows: 
\begin{align*}
p(x) + q(x) 
& =  ( x^3 - 3 x + 2 ) + ( 5x^3+3 x^2 - 6 x + 5 ) \\
& =   (1+5)x^3 +3x^2+(-3-6)x + (2+5) \\
& = 6x^3 + 3 x^2 - 9 x + 7
\end{align*}
Notice, we first grouped together terms with the same power of  $x$, and then we added the coefficients.

Multiplication of polynomials is a bit more involved, so we'll start with polynomials of single terms (monomials) and work our way up from there.  Suppose we have:
\begin{align*} 
p(x)  = 5x^3\text{ and }  
q(x)  = 3x^2.
\end{align*}
Then their product is
\begin{align*}
p(x)  q(x) 
& =  5x^3 3 x^2 \\
& = (5 \cdot 3)x^{(3 + 2)},\\
& = 15x^5,
\end{align*}
where we combined the coefficients and the exponents (remember your exponent rules!).  

Let's extend ourselves a bit and multiply a polynomial of two terms by a monomial:
\begin{align*} 
p(x)  = 5x^3 + 2x \text{ and }
q(x)  = 3x^2.
\end{align*}
According to the distributive law, we multiply each term in the first polynomial with the second polynomial:
\begin{align*}
p(x)  q(x) 
& = ( 5x^3 +2x)  3x^2 \\
&= 5x^3 3x^2 + 2x 3x^2\\
& = (5 \cdot 3)x^{(3 + 2)} + (2 \cdot 3)x^{(1+2)}\\ 
& = 15x^5 + 6x^3.
\end{align*}
In order to  multiply a two term polynomial by another two term polynomial, e.g.
\begin{align*} 
p(x)  = 5x^3 + 2x \text{ and } q(x)  = 3x^2 - 6x,
\end{align*}
we extend the distributive law even further.  Like before, each term in the first polynomial is being multiplied by the second polynomial.
Then the product is
\begin{align*}
p(x)  q(x) 
& = ( 5x^3 +2x)  (3x^2 - 6x) \\
& = 5x^3 (3x^2 - 6x) +2x(3x^2 - 6x) 
\end{align*}
At this point we just have the sum of two terms, each involving a monomials times a two-term polynomial, which we now know can be calculated using the distributive property,
\begin{align*}
& = 5x^3 (3x^2 - 6x) +2x(3x^2 - 6x) \\
& = (15x^5 -30x^4) + (6x^3-12x^2)\\
&= 15x^5 - 30x^4 + 6x^3 - 12x^2
\end{align*}
This is just the same result as the FOIL method\index{FOIL (FLOI) method} you learned in high school, but thinking in terms of the distributive property has the advantage of being applicable to polynomials that have more than just two terms each.  
For instance, with
\begin{align*}
p(x)  = 5x^3 + 4x^2 - 2x \text{ and } q(x)  = 3x^2 - 6x,
\end{align*}
we obtain
\begin{align*}
p(x) q(x)
& = 5x^3 (3x^2 - 6x) +4x^2(3x^2 - 6x)-2x(3x^2 - 6x) \\
& = (15x^5 -30x^4) + (12x^4 - 24x^3)  + (-6x^3 + 12x^2)\\
&= 15x^5 - 30x^4 + 12x^4-24x^3 - 6x^3 + 12x^2\\
&= 15x^5 + (-30+12)x^4 + (-24-6)x^3 +12x^2 \\
&= 15x^5 - 18x^4 - 30x^3 +12x^2.
\end{align*}

Again notice that we are grouping like terms by exponent.  Later, when we give a more general way of multiplying polynomials, this method of distribution is what you need to have in mind.


\begin{exercise}{poly2}
\begin{enumerate}[(a)]
\item
Let $p(x)=4x^2+7x$ and $q(x)=-2x^2-3x+2$. Using polynomial arithmetic, compute both the sum and the product of $p(x)$ and $q(x)$.
\item
Let $p(x)=3x^2+8x-2$ and $q(x)=2x^2-5x+9$. Using polynomial arithmetic, compute both the sum and the product of $p(x)$ and $q(x)$. 
\end{enumerate}
\end{exercise}



\section{Polynomial operations in summation notation}

In the preceding section, we discussed familiar polynomial operations. In this section we give general formulas for these operations in terms of summation notation. These formulas are important both theoretically and practically: theoretically, because they give us a way to express general polynomial operations in proofs; and practically, because they provide instructions for programming polynomial operations on computers.

 First we give the summation representation for an arbitrary polynomial: 

\begin{defn}\label{def:sumPoly}  A polynomial may be written as

\[f(x) = a_0 + a_1 x +a_2 x^2 + \cdots + a_N x^N = \sum^{N}_{n=0} a_i x^n, \]

Where $a_n$  is the \term{coefficient} \index{Polynomial!coefficients} of $x^n$,  $n=1,2,\ldots N$. It is possible for $a_n = 0$, in which case we usually omit the corresponding $x^n$ term (for instance, we write $-7 + x^2 $ rather than $-7 + 0x +  x^2  $). When we write a polynomial as a sum in this way we will \emph{assume} that $a_N \neq 0$ (here $a_N$ is called the \term{leading coefficient}\index{Polynomial! leading coefficient}.  Thus the largest power of $x$ that appears in the polynomial is $x^N$: this largest power is called the \term{degree}\index{Degree!of a polynomial} of the polynomial.
\end{defn}

\begin{rem}
According to  Definition~\ref{def:sumPoly}, we write polynomials in ascending order This differs from Section~\ref{sec:polyStripe}, where we wrote polynomials in descending order as is customary in secondary school. Since the operation `+'  is commutative, the two ways are equivalent: but we will increasingly use this new way, which turns out to be useful for a number of reasons. 
\end{rem}

 \begin{example}{poly3.5} Express the following polynomials in summation notation:
\begin{enumerate}[(a)]
\item
$p_1(x)=1 + x+x^2+x^3$
\item
$p_2(x)=0 + x+2x^2+3x^3+4x^4+5x^5+6x^6+7x^7$
\item
$p_3(x)=5 x+4x^2+3x^3+2x^4 $
\item
$p_4(x)=x+4x^2+9x^3+16x^4+25x^5$
\item
$p_5(x)=-3ix^3+4x^4+5i^5x^5$ (note that here $i$ denotes $\sqrt{-1}$)
\item
$p_6(x)=\sqrt{2}\cis(\pi/3)x^3+\sqrt{3}\cis(2\pi/3)x^6-x^9 + 2\cis(4\pi/3)x^{12} + \sqrt{5}\cis(5\pi/3)x^{15} + \sqrt{6}x^{18}$ 
\item
$p_7(x)=0+\frac{1}{2}x+\frac{2}{3}x^2+\frac{3}{4}x^3+\frac{4}{5}x^4$
\item
$p_8(x)=i+(1+2i)x+(2+3i)x^2+(3+4i)x^3+(4+5i)x^4+(5+6i)x^5+(6+7i)x^6$
\end{enumerate}
 
\noindent{Answers:}

\begin{enumerate}[(a)]
\item
$p_1(x)  = \sum_{j=0}^{3} x^j$
\item
$p_2(x)=\sum_{m=0}^{7} mx^m$
\item
$p_3(x)=\sum_{k=1}^{4} (6-k)x^k $
\item
$p_4(x)=\sum_{\ell=0}^{5} \ell^2x^\ell$
\item
$p_5(x)  = \sum_{n=3}^{5}ni^nx^n$
\item
$p_6(x) = \sum_{q=1}^{6}\sqrt{q+1}\cis(q\pi/3)x^{3q}$
\item
$p_7(x) = \sum_{i=0}^{5}\frac{ix^i}{i+1}$
\item
$p_8(x) = \sum_{a=0}^{6}(a+(a+1)i)x^a$
\end{enumerate}

Note that we don't always begin the sum at 0, depending on the polynomial. Also, the power of $x$ may be a function of the index, as in $p_6$.

\end{example}

\begin{exercise}{poly3l}
Write down the polynomial that each summation represents.
\begin{enumerate}[(a)]
\item
$p(x)  = \sum_{j=0}^{6}( j^2+2)x^{3j}$
\item
$p(x)  = \sum_{r=2}^{5} (r+1)i^rx^r$
\item
$p(x)  = \sum_{s=0}^{8} (-1)^sx^{2s}$
\end{enumerate}
\end{exercise}

\begin{exercise}{poly4}
Re-express the following polynomials in summation notation,  and give the degree of each polynomial.
\begin{enumerate}[(a)]
\item
$2 + 0x + 6x^2 + 0x^3 + 10x^4 + 0x^5 + 14x^6 +0x^7 +  18x^8$
\item
$ \cis(\pi/2)x^2 + 0x^3 + \cis(\pi)x^4 + 0x^5  + \cis(3\pi/2)x^6 + 0x^7$
\item
$1+ 2x + 4x^2 + 8x^3 + 16x^4 + 32x^5$
\item
$1+4x+11x^2+30x^3+85x^4$
\hyperref[sec:polyrings:hints]{(*Hint*)} 
\item
$1-\frac{1}{3}x + \frac{1}{5}x^2 - \frac{1}{7}x^3 + \frac{1}{9}x^4 - \frac{1}{11}^5$
\end{enumerate}
\end{exercise}

\begin{rem}
Summation notation is convenient for representing polynomials whose coefficients exhibit a simple pattern. If there is no apparent pattern, then summation notation is not beneficial. For example, suppose:


$$p(x) = 7+ 22x^2 + x^3-6x^6+4x^8-x^9.$$


Since there's no clear pattern in the coefficients, we wouldn't use summation notation.
\end{rem}
Although the following definition may seem rather obvious, nonetheless we should state it to be precise.

\begin{defn} Two polynomials are said to be \term{equal}\index{Equality!of polynomials}  if and only if their corresponding coefficients are equal. That is, if we let    
$$
p(x)  = \sum^{M}_{m=0} a_m x^m; \qquad
q(x)  = \sum^{N}_{n=0} b_n x^n,
$$
then $p(x) = q(x)$ if and only if $M=N$ and $a_m = b_m$ for all $0 \leq m \leq M$.
\end {defn}

Now we're ready to express our arithmetical rules in summation notation. 

\begin{defn}
We define the \term*{sum of two polynomials}\index{Sum! of polynomials}\index{Polynomials!sum} as follows.  Let
\begin{align*}
p(x)  = \sum^{M}_{m=0} a_m x^m ; \qquad
q(x)  = \sum^{N}_{n=0} b_n x^n,
\end{align*}
Then the sum of $p(x)$ and $q(x)$ is
\[
p(x) + q(x) =  \sum_{k=0}^{\max(M,N)} (a_k + b_k) x^k.
\]

 \end{defn}

Notice that we have taken the upper limit of the sum to $\max(M,N)$ in order to make sure to include all nonzero terms from both polynomials. 

Now that we have a formula for adding polynomials, the next step is to  formula for multiplying polynomials, using summation notation.
To do this, let's repeat the polynomial multiplication procedure we used in Section~\ref{sec:polyStripe}, only this time we'll use two general polynomials instead of specific examples. As with addition, we use
$$
p(x)  = \sum^{M}_{m=0} a_m x^m; \qquad
q(x)  = \sum^{N}_{n=0} b_n x^n.
$$
In the final example in Section~\ref{sec:polyStripe}, what we did was split up the first polynomial, and multiply  each term of the first polynomial by the second polynomial. When applied to $p(x)$ and $q(x)$, this becomes:
\begin{align*}
p(x)q(x)&= a_0x^0 \cdot q(x) + a_1x^1 \cdot q(x) + \ldots + a_Nx^N \cdot q(x) \\
&= \sum^M_{m=0} a_mx^m \cdot q(x)\\
&= \sum^M_{m=0} a_mx^m \cdot \left(\sum^{N}_{n=0} b_n x^n\right),
\end{align*}
where in the last equation we have replaced $q(x)$ with its expression in summation notation. Now since $a_mx^m$ is constant with respect to $n$, we may pull $a_mx^m$ inside the sum over $n$, which gives:

\[
p(x) q(x) =\sum_{m=0}^{M}\sum_{n=0}^{N}(a_mx^m) b_n x^{n}=\sum_{m=0}^{M}\sum_{n=0}^{N}a_m b_n x^{m+n},
\]
where we have used our multiplication rule for monomials:  $(a_mx^m) b_n x^{n}=a_m b_n x^{m+n}$.

Although this expression is correct, it's kind of a hodgepodge. The reason is that not all the terms with the same power of $x$ are grouped together. So let's try to collect terms according to like power of $x$. 

We'll start with $x^0$. Since terms have the form $a_m b_n x^{m+n}$, this means we need to find all values of $m$ and $n$ such that $m+n=0$.  Since both $m$ and $n$ are nonnegative, the only possibility is $m=0,n=0$, which gives the term $a_0 b_0 x^{0}$.

Next let's look at $x^1$. In this case we want terms $a_m b_n x^{m+n}$ which have $m+n=1$.  There are two: $a_1 b_0 x^{1}$ and $a_0 b_1 x^{1}$.

If we treat $x^2$ similarly, we have three terms: $a_2 b_0 x^{2}$, $a_1 b_1 x^{2},$ and $a_0 b_2 x^{2}$.  Then $x^3$ has four terms: $a_3 b_0 x^{3}$, $a_2 b_1 x^{3}$, $a_1b_2 x^{3}$, and  $a_0b_3x^3$.  Do you see the pattern? For $x^k$ we will get $k+1$ terms:  $a_k b_0 x^{k}$,$a_{k-1} b_1 x^{k}$, $\ldots$, $a_1 b_{k-1} x^{k}$, and $a_0 b_{k} x^{k}$.  Since $+$ is commutative, we may sum these terms together to obtain the coefficient of $x^k$, which we will denote as $c_k$:

\[
c_k  = a_0  b_k + a_1 b_{k -1} + \cdots + a_{k -1} b _1 + a_k b_0 = \sum_{j = 0}^k a_j b_{k - j},
\]
This is the coefficient of $x^k$ in the summation notation expression for the product. At last we have our general formula:

\begin{defn}\label{Product of polynomials} The \term*{product of two polynomials}\index{Product!of polynomials}\index{Polynomials!product} $p(x)= \sum^{M}_{m=0} a_m x^m$ and $q(x)=\sum^{N}_{n=0} b_n x^n$ is given by: 
\[
p(x) q(x) = \sum_{k=0}^{M+N} c_k x^k,
\]
where
\[
c_k=  \sum_{j = 0}^k a_j b_{k - j}
\]
for each $k$.  
\end {defn}

Let's verify the formula by computing $f(x)$, where:
\[ f(x)=(1+x^2-2x^3)(x+4x^3). \]
$f(x)$ is the product of $p(x)$ and $q(x)$, where:
\[p(x)= 1x^0 + 0x^1 + 1x^2 + (-2)x^3  \quad \text{and} \quad q(x)= 0x^0 + 1x^1 + 0x^2 + 4x^3 \]
Both polynomials have degree 3, so the degree of the product is 3+3=6:
\[
p(x) q(x) = \sum_{k=0}^{m+n} c_k  x^k =  \sum_{k=0}^{6} c_k x^k. 
\]
Now all we have to do is find the values of the seven coefficients $c_0,\ldots,c_6$, some of which may be zero.  Let us start with $c_0$:
\[ c_0 = \sum_{i = 0}^0 a_i b_{0 - i} = a_0b_0= 0 \cdot 1 = 0. \]
Already we've found a term that is zero.  We still need to find six more coefficients--how about we look at the fifth coefficient:
\[ c_4 =  \sum_{i = 0}^4 a_i b_{4 - i} =   a_0b_4 + a_1b_3 +a_2b_2 + a_3b_1 + a_4b_0.\]
Notice that $a_4=b_4=0$ since $p(x)$ and $q(x)$ both have degree 3, so the first and last terms are both 0. Altogether we have 
\[ c_4=0+0\cdot 4+1\cdot 0+(-2)\cdot 1+0=-2.\]
Doing the same for the other coefficients gives us:
\[f(x)= 0x^0+ 1x^1 + 0x^2 + 5x^3 + (-2)x^4 + 4x^5 + (-8)x^6 \]
Getting rid of the zero terms and dealing with the negatives gives us the simplified version:
\[f(x)=x+5x^3-2x^4+4x^5-8x^6. \]
\begin{exercise}{mult2way}

Perform the following polynomial multiplications in two ways: first, by distributing and collecting terms; and second, by using the coefficient formula in Definition \ref{Product of polynomials} directly.  Verify that the two methods agree.
\begin{enumerate}[(a)]
\item
$(-5+x)(3x+ x^2)$
\item
$(-\sqrt{3}+x)(2\sqrt{3}+5x^3)$
\item
$(7/2- 3x+4x^2)(2+x^3)$
\item
$(- 7x^2 + 4x^3+8x^5 )(3-5x+10x^2 )$
\end{enumerate}
\end {exercise}

In some cases the coefficient formula may greatly simplify calculations.

\begin{exercise}{multform}
\begin{enumerate}[(a)]
\item
Give the coefficient of $x^{100}$ in the polynomial $p(x)^2$, where $p(x) = \sum_{n=0}^{100} x^n$
\item
Give the coefficient of $x^{25}$ in the polynomial $p(x) \cdot q(x)$, where $p(x) = \sum_{n=0}^{25} nx^n$  and $q(x) = \sum_{m=0}^{25} x^m$.
\item
give the coefficient of $x^{33}$ in the polynomial 
$p(x) \cdot q(x)$, where  $p(x) = \sum_{n=1}^{33} \frac{x^n}{n}$ and  $q(x) = \sum_{n=0}^{32} (33-n)x^n$.
\end{enumerate}
\end {exercise}


Another useful fact to keep in mind is that polynomial multiplication is associative.  First, let's see this in an example.


\begin{exercise}{poly5}
Show that the multiplication of two linear polynomials and one quadratic polynomial is associative. (use $a_0+a_1x, b_0+b_1x$, and $c_0+c_1x+c_2x^2$ as your polynomials.)
\end{exercise}

Now we'd like to prove associativity in general. Here's where summation notation comes in really handy. To show associativity, we need to show that $(p(x)q(x))r(x)=(p(x)q(x))r(x)$.  As we stated before, the product of two polynomials $p(x)$ and $q(x)$ written in summation notation is:
\[
p(x) q(x) =\sum_{i=0}^{m}\sum_{j=0}^{n}a_i b_j x^{i+j}
\]
So let's introduce a third polynomial, $r(x)$, with degree $l$ and coefficients $\{ c_i, i=1 \ldots l$, and calculate its product with $(p(x)q(x))$:
\begin{align*}
(p(x) q(x))r(x) =&  \Big( \sum_{i=0}^{m}\sum_{j=0}^{n}a_i b_j x^{i+j} \Big)\Big(\sum_{k=0}^{l} c_k x^k \Big)  \\
=&   \sum_{i=0}^{m}\sum_{j=0}^{n} \Big(a_i b_j x^{i+j}\Big(\sum_{k=0}^{l} c_k x^k \Big) \Big)  \\
=& \sum_{i=0}^{m}\sum_{j=0}^{n}\sum_{k=0}^{l} a_i b_j x^{i+j} \cdot c_k x^k \\
=&  \sum_{i=0}^{m}\sum_{j=0}^{n}\sum_{k=0}^{l} a_i b_j  c_k x^{i+j+k}. 
\end{align*}
Here we used our summation and exponent rules that you know and love so well.  Now let's see if we get the same result for $p(x)  (q(x)r(x))$. Following similar steps, we get:
\begin{align*}
p(x)  (q(x)r(x)) =& \Big(\sum_{i=0}^{m} a_ix^i \Big) \Big( \sum_{j=0}^{n} \sum_{k=0}^{l}  b_j x^j c_k  x^k \Big)    \\
=& \sum_{i=0}^{m} \Bigg( a_i x^i \Big(\sum_{j=0}^{n} \sum_{k=0}^{l}   b_j c_k x^{j+k} \Big) \Bigg)\\
=&   \sum_{i=0}^{m}\sum_{j=0}^{n}\sum_{k=0}^{l}   a_i x^i \cdot b_j c_k x^{j+k}\\
=&  \sum_{i=0}^{m}\sum_{j=0}^{n} \sum_{k=0}^{l} a_i b_j  c_k x^{i+j+k}. 
\end{align*}
This  is exactly the same expression as before, so it is true that $(p(x)q(x))r(x)=p(x)(q(x)r(x))$; therefore, multiplication of polynomials is associative. 
%With summation notation, we can prove that polynomials are associative by using definition 13; however, such a proof can get quite involved.  There is an easier proof that we will show in the next section, but in the mean time let us look at a specific example of the associativity of polynomials.  Suppose we have three linear terms:
%\[A(x)=a_0x^0 + a_1x^1 \qquad B(x)=b_0x^0 +b_1x^1  \qquad  C(x)=c_0x^0 + c_1x^1  \]
%To show associativity, we need to show that $A*(B*C)=(A*B)*C$.  Lets do this term by term, $A*(B*C)$ is:
%\[a_0x^0 + a_1x^1 [(b_0x^0 +b_1x^1)*( c_0x^0 + c_1x^1)] \]
%\[=a_0x^0 + a_1x^1 [b_0x^0c_0x^0 + b_0x^0c_1x^1 + b_1x^1c_0x^0 + b_1x^1c_1x^1]  \]
%\begin{align*}
%&=a_0x^0(b_0x^0c_0x^0 + b_0x^0c_1x^1 + b_1x^1c_0x^0 + b_1x^1c_1x^1 )\\
%& + a_1x^1(b_0x^0c_0x^0 + b_0x^0c_1x^1 + b_1x^1c_0x^0 + b_1x^1c_1x^1 ) 
%\end{align*}
%\begin{align*}
%=& a_0x^0b_0x^0c_0x^0+a_0x^0b_0x^0c_1x^1 + a_0x^0b_1x^1c_0x^0 + a_0x^0b_1x^1c_1x^1  \\
%&+ a_1x^1b_0x^0c_0x^0 + a_1x^1b_0x^0c_1x^1 +a_1x^1b_1x^1c_0x^0 +  a_1x^1b_1x^1c_1x^1 
%\end{align*}
%So without attempting to simplify these terms, try as you might, we have 8 seperate terms.  Now let us see the terms we get from $(A*B)*C$:
%\[ [(a_0x^0 + a_1x^1)*(b_0x^0 +b_1x^1)]*(c_0x^0 + c_1x^1) \]
%\[ =(a_0x^0b_0x^0 + a_0x^0b_1x^1 +a_1x^1b_0x^0 + a_1x^1b_1x^1)*(c_0x^0 + c_1x^1 ) \]
%\begin{align*}
%=& a_0x^0b_0x^0c_0x^0 + a_0x^0b_1x^1c_0x^0 +a_1x^1b_0x^0c_0x^0 + a_1x^1b_1x^1c_0x^0 \\
%&+ a_0x^0b_0x^0c_1x^1 + a_0x^0b_1x^1c_1x^1 +a_1x^1b_0x^0c_1x^1 + a_1x^1b_1x^1c_1x^1 
%\end{align*}
%Again we end up with 8 terms, which turn out to be the same terms as $A*(B*C)$, thus $A*(B*C)=(A*B)*C$ for this \emph{specific} example.  That was a lot of work just to prove that two linear terms with one quadratic term is associative.  What if we had three cubic terms, or three 5th order polynomials?  Clearly this method is not feasible if we want to prove the general case of associativity, which should hold true for all orders of polynomials, not just the linear or quadratic versions .  That is the motivation behind the next section where we can prove associativity for any order of polynomials we can think of using matricies, so when the next section gets tough, just imagine sitting there and multiplying out an infinite combination of polynomials!


%To show that polynomial multiplication is associative, let
%\begin{align*}
%p(x)  = \sum_{i=0}^{m} a_i x^i; \qquad
%q(x)  = \sum_{i=0}^{n} b_i x^i; \qquad
%r(x) = \sum_{i=0}^{p} c_i x^i. 
%\end{align*}
%Then
%\begin{align*}
%[p(x) q(x)] r(x) 
%& =
%\left[
%\left(
%\sum_{i=0}^{m} a_i x^i 
%\right)
%\left( 
%\sum_{k=0}^{n} b_k x^k
%\right)
%\right]
%\left(
%\sum_{i=0}^{p} c_i x^i
%\right) 
%\end{align*}
%For the first two terms we can use Definition 13 to combine them:
%%need to put in a ref statement there
%\begin{align*}
%& =
%\left[
%\sum_{j=0}^{m+n}
%\left( 
%\sum_{k=0}^{i} a_k b_{i-k}
%\right) x^i
%\right]
%\left(
%\sum_{i=0}^{p} c_i x^i
%\right) \\
%& =
%\sum_{i=0}^{m+n+p} 
%\left[
%\sum_{k=0}^{i}
%\left(
%\sum_{j=0}^k a_j b_{k-j} 
%\right) c_k
%\right]
% x^i \\
%& =
%\sum_{i=0}^{m+n+p} 
%\left(
%\sum_{j+k+l=i} a_j b_k c_r
%\right) x^i 
%\end{align*}
%That was the crucial line in the proof, because at this point we can freely choose which pair we want to multiply first.  We already started by multiplying the $a$ and $b$ terms first, so in order to show associativity we need to multiply the $b$ and $c$ terms first.  So at this point we are essentially working in reverse:
%\begin{align*}
%& =
%\sum_{i=0}^{m+n+p}
%\left[
%\sum_{j=0}^{i} a_j 
%\left(
%\sum_{k=0}^{i-j} b_k c_{i-j-k}
%\right)
%\right]  x^i \\
%& =
%\left(
%\sum_{i=0}^{m} a_i x^i 
%\right)
%\left[
%\sum_{i=0}^{n+p} 
%\left(
%\sum_{j=0}^{i} b_j c_{i-j}
%\right) x^i
%\right] \\
%& =
%\left(
%\sum_{i=0}^{m} a_i x^i 
%\right)
%\left[
%\left( 
%\sum_{i=0}^{n} b_i x^i
%\right)
%\left(
%\sum_{i=0}^{p} c_i x^i
%\right)
%\right] \\
%& = p(x) [ q(x) r(x) ]
%\end{align*}
%This proof seems very involved: we will show a much easier proof shortly. The commutativity and distribution properties of polynomial multiplication are proven similarly (but much more easily!).  We shall leave the proofs of these properties as exercises:
%%%% Give answer for answer key
\begin{exercise}{polymult}
Using summation notation, prove the following properties of polynomial multiplication:
\begin{enumerate}[(a)]
\item
Commutativity:~~ 
$p(x) q(x) = q(x) p(x)$;
\item
Distributivity across addition:~~
$p(x) (q(x) + r(x)) =p(x)q(x) + p(x)r(x)$.
\end {enumerate}
\end {exercise}

\section{More exotic polynomials}

So far we've performed algebraic operations on polynomials with integer, rational, real, or complex coefficients. We may identify different sets of polynomials according to the type of coefficient used. For instance we may define:
\begin{itemize}
\item
$\mathbb{Z}[x]$ is the set of polynomials in the variable $x$ with integer coefficients;
\item
$\mathbb{Q}[x]$ is the set of polynomials in the variable $x$ with rational coefficients;
\item
$\mathbb{R}[x]$ is the set of polynomials in the variable $x$ with real coefficients;
\item
$\mathbb{C}[x]$ is the set of polynomials in the variable $x$ with complex coefficients.
\end{itemize}

However, we can generalize polynomials far beyond these cases. In this section, we present several additional types of polynomials. 


Consider first $\mathbb{Z}_n[x]$, for the integers mod $n$. For example, two polynomials $p(x)$ and $q(x)$ in $\mathbb{Z}_4[x]$ are
\begin{align*} 
p(x) & = 1+3x+x^3\\
q(x) & = 2+2x+3x^2+3x^3.
\end{align*}
In this case, we should consider the variable $x$ as representing an \emph {unknown element in }$\mathbb{Z}_4$, so the '+' operation in these expressions should be interpreted as \emph{addition in }$\mathbb{Z}_4$. All operations on coefficients will also make use of addition mod $\mathbb{Z}_4$.
We may then evaluate $p(x)+q(x)$ as follows:
\begin{align*}
p(x) + q(x) &=(1+ 2)+(3+ 2)x+(0+ 3)x^2+ (1+ 3)x^3\\
&= 3+x+3x^2.
\end{align*}
To multiply, we may use the distributive law as before, with the understanding that we are distributing \emph{modular multiplication} over \emph{modular addition}. As before, we group together all terms with like powers of $x$ and use modular arithmetic to combine  these terms into a single term.  The result is:
\begin{align*}
p(x)  q(x)  =  &1\cdot 2+(3 \cdot 2 + 1 \cdot 2)x+(3 \cdot 2 +  1 \cdot 3)x^2+\\
&(1\cdot 2 + 3\cdot 3 +  1 \cdot 3)x^3+ (1\cdot 2 +  3\cdot 3)x^4 +\\
&(1 \cdot 3)x^5+ (1 \cdot 3)x^6\\
= &2+1x^2 + 2x^3+ 3x^4+ 3x^5+3x^6.
\end{align*}

It turns out that $\mathbb{Z}_2[x]$ in particular is of great practical use (in polynomial codes), so we include an exercise to get you warmed up.

\begin{exercise}{poly6}
Compute the sum and product of $p(x)$ and $q(x)$, where both polynomials are in $\mathbb{Z}_2[x]$.
\begin{enumerate}[(a)]
\item
$p(x)=  1+ x +x^2$, $q(x)= 1+x+x^2+x^3$
\item
$p(x)= 1 + x^2+x^4$, $q(x)=x^2+x^3+x^4 $.
\item
$p(x)= 1 +x+x^2+x^3+x^4 $, $q(x)=p(x)$.
\end{enumerate}
\end{exercise}
\begin{exercise}{poly7}
Compute the sum and product of $p(x)$ and $q(x)$.
\begin{enumerate}[(a)]
\item
$p(x)= 2x^2 + x + 1$, $q(x)=x^3 +3x^2$,where both polynomials are in $\mathbb{Z}_5[x]$.
\item
$p(x)= 2x^4 + 3x^3 + 4x^2+1$, $q(x)=x^3 +2x^2+5$, where both polynomials are in $\mathbb{Z}_6[x]$.
\end{enumerate}
\end{exercise}

Our second example is $\mathbb{P}[x][y]$, which represents polynomials in the variable $y$ whose coefficients are polynomials in a different variable $x$.  For example, two polynomials $p(x,y)$ and $q(x,y)$ are
\begin{align*} 
p(x,y) & = (1+3x)+(1+x^2)y+(5x)y^2\\
q(x,y) & = (3x)+(2+2x)y+(4x^2)y^2.
\end{align*}
We may add them as follows:
\begin{align*}
p(x,y) + q(x,y) &=((1+3x)+3x)+((1+x^2)+(2+2x))y+(5x+4x^2)y^2\\
&= (1+6x)+(3+2x+x^2)y+(5x+4x^2 )y^2.
\end{align*}
We will perform multiplication of $p(x,y)$ and $q(x,y)$ using the summation formula for coefficients found in Definition~\ref{Product of polynomials}.
%\[
%p(x) q(x) = \sum_{k=0}^{M+N} c_k x^k,
%\]
%where
%\[
%c_k=  \sum_{j = 0}^k a_j b_{k - j}
%\]
%for each $k$.  

\begin{align*}
 c_0 = \sum_{i = 0}^0 a_i b_{0 - i} = a_0b_0= (1+3x) \cdot (3x)= 3x+9x^2. 
\end{align*}
\begin{align*}
 c_1 = \sum_{i = 0}^1 a_i b_{0 - i} = a_0b_1+a_1b_0&= ((1+3x) \cdot (2+2x))+((1+x^2)\cdot (3x))\\
&=(1+8x+6x^2)+(2+2x+2x^2+2x^3)\\
&=3+10x+8x^2+2x^3. 
\end{align*}
\begin{align*}
 c_2 = \sum_{i = 0}^2 a_i b_{0 - i} = a_0b_2+a_1b_1+a_2b_0&=((1+3x) \cdot (4x^2))+(1+x^2)\cdot(2+2x)+((5x)\cdot(3x)\\
&= (4x^2+12x^3)+(2+2x+2x^2+2x^3)+15x^2\\
&=2+2x+21x^2+14x^3. 
\end{align*}
\begin{align*}
c_3 = \sum_{i = 0}^3 a_i b_{0 - i} = a_1b_2+a_2b_1&= ((1+x^2) \cdot (4x^2))+((5x)\cdot(2+2x)\\
&=  (4x^2+4x^4)+(10x+10x^2)\\
&=10x+14x^2+4x^4. 
\end{align*}
\begin{align*}
 c_4 = \sum_{i = 0}^4 a_i b_{0 - i} = a_2b_2= 5x \cdot 4x^2= 20x^3. 
\end{align*}
Therefore, we have the following:

\begin{align*}
p(x,y)q(x,y)=& c_0+c_1y+c_2y^2+c_3y^3+c_4y^4\\
=&(3x+9x^2)+(3+10x+8x^2+2x^3)y+(2+2x+21x^2+14x^3)y^2\\
&+(10x+14x^2+4x^4)y^3+(20x^3)y^4.
\end{align*}

\begin{exercise}{poly8}
Compute the sum and product of $p(x)$ and $q(x)$ where:

$p(x,y)=  (1+8x)+(3-2x)y^2$ and

$q(x,y)= (5x+6x^2)y-(2+6x)y^2$

\end{exercise}

Next we consider $\mathbb{M}_n[x]$, the set of polynomials in the variable $x$ with $n{\times}n$ matrix coefficients. Suppose that two polynomials $p(x)$ and $q(x)$ are:

$$p(x)=  \begin{bmatrix} 
-1 & 5  \\
-2 & 4 
\end{bmatrix}
+\begin{bmatrix} 
4 & 3 \\
2 & -4 
\end{bmatrix}
 x+\begin{bmatrix} 
-1 & 0 \\
2 & 1 
\end{bmatrix}
 x^2$$

$$q(x)=  \begin{bmatrix} 
4 & 2 \\
-3 & 6
\end{bmatrix}
+\begin{bmatrix} 
2 & -1 \\
0 & 9 
\end{bmatrix}
 x+\begin{bmatrix} 
9 & -1 \\
8 & 1 
\end{bmatrix}
 x^2$$

We add $p(x)$ and $q(x)$ as follows:
\begin{align*}
p(x) + q(x) =&\bigg( \begin{bmatrix} 
-1 & 5  \\
-2 & 4 
\end{bmatrix}
+\begin{bmatrix} 
4 & 3 \\
2 & -4 
\end{bmatrix}
 x+\begin{bmatrix} 
-1 & 0 \\
2 & 1 
\end{bmatrix}
 x^2\bigg)\\
&+\bigg(\begin{bmatrix} 
4 & 2 \\
-3 & 6
\end{bmatrix}
+\begin{bmatrix} 
2 & -1 \\
0 & 9 
\end{bmatrix}
 x+\begin{bmatrix} 
9 & -1 \\
8 & 1 
\end{bmatrix}
 x^2\bigg)
\end{align*}
\begin{align*}
= &\bigg(\begin{bmatrix} 
-1 & 5 \\
-2 & 4
\end{bmatrix}
+\begin{bmatrix} 
4 & 2 \\
-3 & 6 
\end{bmatrix}\bigg)
 +\bigg(\begin{bmatrix} 
4 & 3 \\
2 & -4 
\end{bmatrix}
+\begin{bmatrix}
2 & -1\\
0 & 9
\end{bmatrix}\bigg)x\\
&+\bigg(\begin{bmatrix}
-1 & 0 \\
2 & 1
\end{bmatrix}
+\begin{bmatrix} 
9 & -1 \\
8 & 1 
\end{bmatrix}\bigg)
 x^2\\
%&=\begin{bmatrix} 
%-1+4 & 5+2 \\
%-2+(-3) & 4+6
%\end{bmatrix}
%+\begin{bmatrix} 
%4+2 & 3+(-1) \\
%2+0 & -4+9
%\end{bmatrix}x
%+\begin{bmatrix}
%-1+9 & 0+(-1)\\
%2+8 & 1+1\end{bmatrix}x^2\\
=&\begin{bmatrix} 
3 & 7 \\
-5 & 10
\end{bmatrix}
+\begin{bmatrix} 
6 & 2 \\
2 & 5
\end{bmatrix}x
+\begin{bmatrix}
8 & -1\\
10 & 2\end{bmatrix}x^2
\end{align*}

Again, we will use the summation formula for the coefficients to compute the product of $p(x)$ and $q(x)$.
\[ c_0 = \sum_{i = 0}^0 a_i b_{0 - i} = a_0b_0= \begin{bmatrix} 
-1 & 5 \\
-2 & 4
\end{bmatrix} \cdot \begin{bmatrix} 
4 & 2 \\
-3 & 6
\end{bmatrix}= \begin{bmatrix} 
-19& 28\\
-20 & 20
\end{bmatrix}. \]

\[ c_1 = \sum_{i = 0}^1 a_i b_{0 - i} = a_0b_1+a_1b_0= \begin{bmatrix} 
-1 & 5 \\
-2 & 4
\end{bmatrix} \cdot \begin{bmatrix} 
2 & -1\\
0 & 9
\end{bmatrix}+ \begin{bmatrix} 
4 & 3\\
2 & -4
\end{bmatrix} \cdot\begin{bmatrix}
4 & 2\\
-3 & 6
\end{bmatrix}=\begin{bmatrix}
-2 & 46\\
-4 & 38
\end{bmatrix}+\begin{bmatrix}
7 & 26\\
20 & -20
\end{bmatrix}=\begin{bmatrix}
5 & 72\\
16 & 18
\end{bmatrix}.\]

\[ c_2 = \sum_{i = 0}^2 a_i b_{0 - i} = a_0b_2+a_1b_1+a_2b_0= \begin{bmatrix} 
-1 & 5 \\
-2 & 4
\end{bmatrix} \cdot \begin{bmatrix} 
9 & -1\\
8 & 1
\end{bmatrix}+ \begin{bmatrix} 
4 & 3\\
2 & -4
\end{bmatrix} \cdot\begin{bmatrix}
2 & -1\\
0 & 9
\end{bmatrix}+\begin{bmatrix}
-1 & 0\\
2 & 1
\end{bmatrix}+\begin{bmatrix}
4 & 2\\
-3 & 6
\end{bmatrix}=\begin{bmatrix}
31 & 6\\
14 & 6
\end{bmatrix}+\begin{bmatrix}
8 & 23\\
4 & -38
\end{bmatrix}+\begin{bmatrix}
-4 & -2\\
5 & 10
\end{bmatrix}=\begin{bmatrix}
35 & 27\\
23 & -22
\end{bmatrix}.\]

\[ c_3 = \sum_{i = 0}^3 a_i b_{0 - i} = a_1b_2+a_2b_1= \begin{bmatrix} 
4 & 3 \\
2 & -4
\end{bmatrix} \cdot \begin{bmatrix} 
9 & -1\\
8 & 1
\end{bmatrix}+ \begin{bmatrix} 
-1 & 0\\
2 & 1
\end{bmatrix} \cdot\begin{bmatrix}
2 & -1\\
0 & 9
\end{bmatrix}=\begin{bmatrix}
60 & -1\\
-14 & -6
\end{bmatrix}+\begin{bmatrix}
-2 & 1\\
4 & 7
\end{bmatrix}=\begin{bmatrix}
58 & 0\\
-10 & 1
\end{bmatrix} .\]

%We compute the product of $p(x)$ and $q(x)$ as follows:
%
%$$p(x)q(x) =\bigg( \begin{bmatrix} 
%-1 & 5  \\
%-2 & 4 
%\end{bmatrix}
%+\begin{bmatrix} 
%4 & 3 \\
%2 & -4 
%\end{bmatrix}
% x+\begin{bmatrix} 
%-1 & 0 \\
%2 & 1 
%\end{bmatrix}
% x^2\bigg)\cdot$$
%$$\bigg(\begin{bmatrix} 
%4 & 2 \\
%-3 & 6
%\end{bmatrix}
%+\begin{bmatrix} 
%2 & -1 \\
%0 & 9 
%\end{bmatrix}
% x+\begin{bmatrix} 
%9 & -1 \\
%8 & 1 
%\end{bmatrix}
% x^2\bigg)$$
%$$=\begin{bmatrix}
%-1 & 5\\
%-2 & 4
%\end{bmatrix}\cdot
%\bigg(\begin{bmatrix} 
%4 & 2 \\
%-3 & 6
%\end{bmatrix}
%+\begin{bmatrix} 
%2 & -1 \\
%0 & 9 
%\end{bmatrix}
% x+\begin{bmatrix} 
%9 & -1 \\
%8 & 1 
%\end{bmatrix}
% x^2\bigg)+$$
%$$\begin{bmatrix}
%4 & 3\\
%2 & -4
%\end{bmatrix}x\cdot
%\bigg(\begin{bmatrix} 
%4 & 2 \\
%-3 & 6
%\end{bmatrix}
%+\begin{bmatrix} 
%2 & -1 \\
%0 & 9 
%\end{bmatrix}
% x+\begin{bmatrix} 
%9 & -1 \\
%8 & 1 
%\end{bmatrix}
% x^2\bigg)+$$
%$$\begin{bmatrix}
%-1 & 0\\
%2 & 1
%\end{bmatrix}x^2\cdot
%\bigg(\begin{bmatrix} 
%4 & 2 \\
%-3 & 6
%\end{bmatrix}+
%\begin{bmatrix} 
%2 & -1 \\
%0 & 9 
%\end{bmatrix}x+
%\begin{bmatrix} 
%9 & -1 \\
%8 & 1 
%\end{bmatrix}x^2\bigg)$$
%$$=\begin{bmatrix}
%-1 & 5  \\
%-2 & 4 
%\end{bmatrix}\cdot
%\begin{bmatrix} 
%4 & 2 \\
%-3 & 6 
%\end{bmatrix}+
%\begin{bmatrix}
%-1 & 5  \\
%-2 & 4 
%\end{bmatrix}\cdot
%\begin{bmatrix} 
%2 & -1 \\
%0 & 9 
%\end{bmatrix}x+
%\begin{bmatrix}
%-1 & 5  \\
%-2 & 4 
%\end{bmatrix}\cdot
%\begin{bmatrix} 
%9 & -1 \\
%8 & 1 
%\end{bmatrix}x^2+$$
%$$\begin{bmatrix}
%4 & 3  \\
%2 & -4 
%\end{bmatrix}x\cdot
%\begin{bmatrix} 
%4 & 2 \\
%-3 & 6 
%\end{bmatrix}+
%\begin{bmatrix}
%4 & 3  \\
%2 &-4 
%\end{bmatrix}x\cdot
%\begin{bmatrix} 
%2 & -1 \\
%0 & 9 
%\end{bmatrix}x+
%\begin{bmatrix}
%4 & 3  \\
%2 & -4 
%\end{bmatrix}x\cdot
%\begin{bmatrix} 
%9 & -1 \\
%8 & 1 
%\end{bmatrix}x^2+$$
%$$\begin{bmatrix}
%-1 & 0  \\
%2 & 1
%\end{bmatrix}x^2\cdot
%\begin{bmatrix} 
%4 & 2 \\
%-3 & 6 
%\end{bmatrix}+
%\begin{bmatrix}
%-1 & 0  \\
%2 & 1 
%\end{bmatrix}x^2\cdot
%\begin{bmatrix} 
%2 & -1 \\
%0 & 9 
%\end{bmatrix}x+
%\begin{bmatrix}
%-1 & 0  \\
%2 & 1 
%\end{bmatrix}x^2\cdot
%\begin{bmatrix} 
%9 & -1 \\
%8 & 1 
%\end{bmatrix}x^2$$
%$$=\begin{bmatrix}
%-1 & 5  \\
%-2 & 4 
%\end{bmatrix}\cdot
%\begin{bmatrix} 
%4 & 2 \\
%-3 & 6 
%\end{bmatrix}+
%\begin{bmatrix}
%-1 & 5  \\
%-2 & 4 
%\end{bmatrix}\cdot
%\begin{bmatrix} 
%2 & -1 \\
%0 & 9 
%\end{bmatrix}x+
%\begin{bmatrix}
%-1 & 5  \\
%-2 & 4 
%\end{bmatrix}\cdot
%\begin{bmatrix} 
%9 & -1 \\
%8 & 1 
%\end{bmatrix}x^2+
%\begin{bmatrix}
%4 & 3  \\
%2 & -4 
%\end{bmatrix}\cdot
%\begin{bmatrix} 
%4 & 2 \\
%-3 & 6 
%\end{bmatrix}x+$$
%$$\begin{bmatrix}
%4 & 3  \\
%2 &-4 
%\end{bmatrix}\cdot
%\begin{bmatrix} 
%2 & -1 \\
%0 & 9 
%\end{bmatrix}x^2+
%\begin{bmatrix}
%4 & 3  \\
%2 & -4 
%\end{bmatrix}\cdot
%\begin{bmatrix} 
%9 & -1 \\
%8 & 1 
%\end{bmatrix}x^3+
%\begin{bmatrix}
%-1 & 0  \\
%2 & 1
%\end{bmatrix}\cdot
%\begin{bmatrix} 
%4 & 2 \\
%-3 & 6 
%\end{bmatrix}x^2+
%\begin{bmatrix}
%-1 & 0  \\
%2 & 1 
%\end{bmatrix}\cdot
%\begin{bmatrix} 
%2 & -1 \\
%0 & 9 
%\end{bmatrix}x^3+$$
%$$\begin{bmatrix}
%-1 & 0  \\
%2 & 1 
%\end{bmatrix}\cdot
%\begin{bmatrix} 
%9 & -1 \\
%8 & 1 
%\end{bmatrix}x^4$$
%$$=\begin{bmatrix} 
%-19 & 28  \\
%-20 & 20 
%\end{bmatrix}+
%\begin{bmatrix} 
%-2 & 46 \\
%-4 & 38
%\end{bmatrix}x+
%\begin{bmatrix} 
%31 & 6 \\
%14 & 6 
%\end{bmatrix}x^2+
%\begin{bmatrix} 
%7 & 26 \\
%20 & -20
%\end{bmatrix}x+
%\begin{bmatrix} 
%8 & 23 \\
%4 & -38 
%\end{bmatrix}x^2+
%\begin{bmatrix} 
%60 & -1 \\
%-14 & -6
%\end{bmatrix}x^3+
%\begin{bmatrix} 
%-4 & -2  \\
%5 & 10
%\end{bmatrix}x^2+$$
%$$\begin{bmatrix} 
%-2 & 1 \\
%4 & 7
%\end{bmatrix}x^3+
%\begin{bmatrix} 
%-9 & 1 \\
%26 & -1 
%\end{bmatrix}x^4$$
%$$=\begin{bmatrix} 
%-19 & 28  \\
%-20 & 20 
%\end{bmatrix}+
%\bigg(\begin{bmatrix} 
%-2 & 46 \\
%-4 & 38
%\end{bmatrix}+
%\begin{bmatrix} 
%7 & 26 \\
%20 & -20
%\end{bmatrix}\bigg)x+
%\bigg(\begin{bmatrix} 
%31 & 6 \\
%14 & 6 
%\end{bmatrix}+
%\begin{bmatrix} 
%8 & 23 \\
%4 & -38 
%\end{bmatrix}+
%\begin{bmatrix} 
%-4 & -2  \\
%5 & 10
%\end{bmatrix}\bigg)x^2+$$
%$$\bigg(\begin{bmatrix} 
%60 & -1 \\
%-14 & -6
%\end{bmatrix}+
%\begin{bmatrix} 
%-2 & 1 \\
%4 & 7
%\end{bmatrix}\bigg)x^3+
%\begin{bmatrix} 
%-9 & 1 \\
%26 & -1 
%\end{bmatrix}x^4$$
%$$=\begin{bmatrix} 
%-19 & 28  \\
%-20 & 20 
%\end{bmatrix}+
%\begin{bmatrix} 
%-2 & 46 \\
%-4 & 38
%\end{bmatrix}x+
%\begin{bmatrix} 
%31 & 6 \\
%14 & 6 
%\end{bmatrix}x^2+
%\begin{bmatrix} 
%7 & 26 \\
%20 & -20
%\end{bmatrix}x+
%\begin{bmatrix} 
%8 & 23 \\
%4 & -38 
%\end{bmatrix}x^2+
%\begin{bmatrix} 
%60 & -1 \\
%-14 & -6
%\end{bmatrix}x^3+
%\begin{bmatrix} 
%-4 & -2  \\
%5 & 10
%\end{bmatrix}x^2+$$\begin{align*}
%$$\begin{bmatrix} 
%-2 & 1 \\
%4 & 7
%\end{bmatrix}x^3+
%\begin{bmatrix} 
%-9 & 1 \\
%26 & -1 
%\end{bmatrix}x^4$$
%$$=\begin{bmatrix} 
%-19 & 28  \\
%-20 & 20 
%\end{bmatrix}+
%\begin{bmatrix} 
%5 & 72 \\
%16 & 18
%\end{bmatrix}x+
%\begin{bmatrix} 
%35 &27 \\
%23 & -22
%\end{bmatrix}x^2+
%\begin{bmatrix} 
%58 & 0 \\
%-10 & 1
%\end{bmatrix}x^3+
%\begin{bmatrix} 
%-9 & 1 \\
%26 & -1 
%\end{bmatrix}x^4$$
%

\begin{exercise}{poly9}
Compute the sum and product of $p(x)$ and $q(x)$, where:

$$p(x)=  \begin{bmatrix} 
1 & 2 \\
3 & 4 
\end{bmatrix}
+\begin{bmatrix} 
1 & 0 \\
2 & 4 
\end{bmatrix}
 x+\begin{bmatrix} 
1 & 0 \\
0 & 1 
\end{bmatrix}
 x^2$$

$$q(x)=  \begin{bmatrix} 
1 & 0 \\
0 & 1 
\end{bmatrix}
+\begin{bmatrix} 
1 & 2 \\
3 & 4 
\end{bmatrix}
 x^2$$
\end{exercise}

%Lastly, we consider the set of polynomials in two variables. Consider the following polynomials:
%
%$p(x)=2+3xy+x^2y$ and
%
%$q(x)=5xy+2xy^2$
%
%The sum of $p(x)$ and $q(x)$ is computed as follows:
%
%\begin{align*}
%p(x)+q(x)&=(2+3xy+x^2y)+(5xy+2xy^2)\\
%&=2+(3+5)xy+x^2y+2xy^2\\
%&=2+8xy+x^2y+2xy^2\\
%\end{align*}
%
%We calculate the product as follows:
%
%\begin{align*}
%p(x)q(x)&=(2+3xy+x^2y)(5xy+2xy^2)\\
%&=2(5xy)+2(2xy^2)+3xy(5xy)+3xy(2xy^2)+x^2y(5xy)+x^2y(2xy^2)\\
%&=10xy+4xy^2+15x^2y^2+6x^2y^3+5x^3y^2+2x^3y^3\\
%\end{align*}
Are these sets of polynomials ($\ZZ[x], \RR[x]$ and so on) also groups? I'm glad you asked! See if you can figure it out:

\begin{exercise}{poly10}
\begin{enumerate}[(a)]
\item
Which of the following are groups under addition: $\mathbb{N}[x], \mathbb{Z}[x], \mathbb{Q}[x], \mathbb{R}[x]$, $\mathbb{C}[x]$, $\mathbb{Z}_5[x]$,$\mathbb{Z}_6[x]$, $\mathbb{P}[x][y]$, and/or $\mathbb{M}_n[x]$.
\item
Which of the following are groups under multiplication: $\mathbb{N}[x]$, $\mathbb{Z}[x]$,$\mathbb{Q}[x]$,$\mathbb{R}[x]$, $\mathbb{C}[x]$, $\mathbb{Z}_5[x]$,$\mathbb{Z}_6[x]$, $\mathbb{P}[x][y]$, and/or $\mathbb{M}_n[x]$? 
\emph{Explain} your answers.
\end{enumerate}
\end{exercise}

Note what's common to all examples.  In each case, the coefficients form an Abelian group under The coefficients also have a multiplication operation that is associative and distributes over the + operation. Any set with two operations that satisfy these properties is called a ring. The study of rings is a very important subject in Abstract Algebra. In this book, however, we will not go further into the general theory of rings.

In many important polynomial rings, the multiplication defined on coefficients satisfies some key additional properties. These additional properties are:
\begin{itemize}
\item
commutivity
\item
 multiplicative identity
\item
 multiplicative inverse for nonzero coefficients 
\end{itemize}

If the set of coefficients satisfies these additional properties, it is called a field. In the following discussion, we will be considering the case where the polynomial coefficients form a field.We will see that this allows us to perform polynomial long division and finding gcd's, very much like we did with integers in Chapter~\ref{modular}.

\begin{exercise}{poly47}
%Exercise:  Which of the following are polynomial rings over fields?  Z[x], R[x],C[x],Z_4[x], Z_5[x], P{x][y], M[x].
\end{exercise}

In this chapter we won't go beyond polynomials, but you should be aware that there's more to the story, and the study of rings forms a large part of modern abstract algebra.

\section{Polynomial rings over fields\quad
\sectionvideohref{Q5q-LL4QW4o&list=PL2uooHqQ6T7PW5na4EX8rQX2WvBBdM8Qo&index=48}} 
 We have noted above that polynomials can have different types of coefficients. 
In this section we will  impose some properties on the coefficients that, although quite general, will enable us to prove several interesting properties. But first, let's relate polynomials to the summation notation that we discussed in the previous chapter:




\begin{defn} \label{def:polyring}(\emph{Polynomial rings over fields})~~
A set of polynomials is called a \term{polynomial ring over a field}\index{Polynomial!ring over a field}\index{Ring!polynomial, over a field} if two operations (which we denote as $+$ and $\cdot$)  are defined on the set of coefficients and
\begin{itemize}
\item
The coefficients form an abelian group under +, with identity element 0;
\item
The nonzero coefficients form an abelian group under $\cdot$, with identity element 1.
\item The operations $+$ and $\cdot$ on the set of coefficients satisfy the distributive property: for any three coefficients $a,b,c$, we have $a \cdot (b + c) = a \cdot b + a \cdot c$ and 
$(b + c) \cdot a = b \cdot a + c \cdot a$.
\end{itemize}
\end{defn}

\begin{rem}
the clause ``over a field'' is added because of the requirement that the  nonzero coefficients form a group under multiplication. In the most general definition of a polynomial ring, the set of nonzero coefficients is not required to have a multiplicative inverse.
\end{rem}

%%%% Add 
\begin{exercise}{poly11}
\begin{enumerate}[(a)]
\item
According to Definition~\ref{def:polyring},is $\mathbb{Z}_6[x]$ a polynomial ring over a field? \emph{Explain} your answer.
\item
According to Definition~\ref{def:polyring},is $\mathbb{Z}_{11}[x]$ a polynomial ring over a field? \emph{Explain} your answer.
\item
What are the conditions on $n$ such that $\mathbb{Z}_n[x]$ is a polynomial ring over a field?
\end{enumerate}
\end{exercise}
We see from the previous exercise that $F[x]$ fails to be a polynomial ring over a field when $F^*$ is not an abelian group. Why do we require this? It turns out that if we don't, then the polynomials have some nasty properties that we don't want.

\begin{exercise}{poly12}
\begin{enumerate}[(a)]
\item
Find two nonzero polynomials in $\mathbb{Z}_4[x]$ of degree 1 and 3 respectively  whose product is 0.
\item
Is it possible to find two nonzero polynomials $\mathbb{Z}_5[x]$ whose product is 0? \emph{Explain} your answer.
\item
Suppose $n=p\cdot q$, where $p$ and $q$ are positive integers. Show that there exist two nonzero polynomials in $\mathbb{Z}_n[x]$ whose product is 0.
\end{enumerate}
\end{exercise}

\begin{rem}
In the following, whenever we refer to `polynomials'' without specifically mentioning what type of coefficients it has, we are presuming that the polynomials are elements of a polynomial ring over a field.
\end{rem}


In the previous section, we showed properties of polynomials in the familiar case where the coefficients are real numbers. Now let's 
do the same thing, but this time for arbitrary polynomial rings over fields.
 



Throughout this book, we have encountered various mathematical structures and shown that in many cases they possess group properties. Let's now give polynomials that same treatment. Since we have two operations, addition and multiplication, it is possible that polynomials are groups under either or both of these operations. Of course, it's possible that polynomials may have different properties depending on what type of coefficients they have, so all of the following proofs will depend \emph{only} on the coefficient properties listed in Definition 5.  

First let us take a look at the commutativity and associativity of addition.  While commutativity is not a necessary property for groups, we include the following proof because it uses techniques that are useful in many polynomial proofs.

\begin{prop}{polysumcommuteassociate} Polynomial addition is both commutative: 
	\[p(x)+q(x) = q(x) + p(x),\]	
and associative:	
	\[(p(x) + q(x)) + r(x)  = p(x) + (q(x) + r(x)).\]	
\end {prop}
\begin{proof}{}
First, we show commutativity:	
Given polynomials,
\begin{align*}
p(x)  = \sum^{n}_{i=0} a_i x^i; \qquad
q(x)  = \sum^{m}_{i=0} b_i x^i,
\end{align*}	
then 
\[
p(x) + q(x) =  \sum_{i=0}^{\max(m,n)} (a_i + b_i) x^i,
\]
and
\[
p(x) + q(x) =  \sum_{i=0}^{\max(m,n)} (b_i + a_i) x^i.
\]
Since the coefficients are abelian under +, we have $a_i + b_i = b_i + a_i$ for all $i$. It follows that all coefficients of $p(x) + q(x)$ are equal to the corresponding coefficients of $q(x) + p(x)$. By the definition of polynomial equality, this means that 
$p(x) + q(x) = q(x) + p(x)$.  Note that the condition that the coefficients be commutative is satisfied for all of the polynomials we have been considering so far, including $\mathbb{Z}[x]$, $\mathbb{R}[x]$, $\mathbb{C}[x]$, and so on.

\end {proof}
%%%% Provide solution

\begin{exercise}{}
Using the definition of polynomial equality and polynomial addition, complete Proposition \ref{proposition:poly:polysumcommuteassociate} by proving the associativity of polynomial addition.
\end {exercise}

It is also true that polynomials under addition have an identity and an inverse.

\begin{prop}{polyadditiveidentity} Polynomials have an additive identity.
\end{prop}
\begin{exercise}{}
Show that polynomials have an additive identity by specifying the identity polynomial, and showing that it satisfies the identity property.
\end{exercise}

\begin{prop}{polyadditiveinverse} Every polynomial has an additive inverse.  (In the following, we will write the additive inverse of $p(x)$ as $-p(x)$).
\end{prop}
\begin{exercise}{}
Given any polynomial $f(x) = \sum_{i=0}^{n} a_i x^i$, show that an additive inverse to that polynomial exists by (a) specifying the inverse and (b) showing that it satisfies the inverse property.
\end{exercise}

\begin{prop}{polyadditivegroup} Polynomials under addition form a group. The group is abelian if the additive group of coefficients is abelian.
\end{prop}

\begin{exercise}{}
Prove Proposition \ref{proposition:poly:polyadditivegroup}.
\end{exercise}


Polynomials also have \emph{some} group properties under multiplication.

\begin{exercise}{}
Show that the polynomial $p(x) = 1x^0$ is a multiplicative identity for the set of polynomials $\mathbb{C}[x]$.
\end{exercise}

Polynomials in general do not have multiplicative inverses \emph{that are polynomials}.  Of course, in high-school algebra you defined $1/p(x)$ as the multiplicative inverse of the polynomial $p(x)$, but $1/p(x)$ is not a polynomial so it doesn't count!

\begin{exercise}{}
\begin{enumerate}[(a)]
\item
Consider the  polynomial $p(x)= 1x$ as an element of $\mathbb{R}[x]$. Show there is no polynomial in $\mathbb{R}[x]$ that is a multiplicative inverse of $p(x)$.
\item
Prove or disprove: polynomial rings over fields are also commutative groups over multiplication.
\end{enumerate}
\end{exercise}

%%% rewrite and use the summation notation form of polynomial product
Another useful fact to keep in mind is that polynomial multiplication is associative.  First, let's see this in an example.



%\section{Isomorphism with Matrices}
%Another way of proving associativity and commutativity of polynomial multiplication is by using an isomorphism to matrices.  We can take polynomials, rewrite them in a very specific way as a matrix, and then add or multiply these matricies to get the same result.  This is the same idea as the isomorphism between complex numbers and ordered pairs that you studied in the isomorphism chapter, except here it is polynomials and matrices that are isomorphic:
%
%%%%\begin{figure}[htb]
%%%%	   \center{\includegraphics[width=4.in]
%%%%	         {images/matrixpolyiso.png}}
%%%%	  \caption{\label{fig:groups:matrixpolyiso} Multiplication is the ``same'' for polynomials and matrices. }
%%%%\end{figure}
%
%To show that polynomials and matrices are isomorphic, we first need to define how to represent a polynomial using a matrix:
%\[ M_{ij}=
%\begin{cases}
%a_{i-j} ~ i \ge j \\
%0 ~ i \le j
%\end{cases} \]
%If you write this out, it takes the form:
%\[
%\left( \begin{array}{cccccc}
%a_0 & 0 & 0 & 0 & 0 & \cdots\\
%a_1 & a_0 & 0 & 0 & 0 & \cdots\\
%a_2 & a_1 & a_0 & 0 & 0 & \cdots\\
%a_3 & a_2 & a_1 & a_0 & 0 & \cdots\\
%a_4 & a_3 & a_2 & a_1 & a_0 & \cdots\\
%\vdots & \vdots & \vdots & \vdots & \vdots & \ddots\\
%\end{array} \right)\\
%\] 
%Take the following polynomial product:
%\[(x^3+2x+3)\cdot (x^2+3)\]
%Through simple multiplication we know the product is:
%\[x^5 + 2x^3 + 3x^2 + 3x^3 + 6x + 9 = x^5 + 5x^3 + 3x^2 + 6x + 9\]
%Now let's take the two polynomials and convert them to $6\times 6$ matrices.  (The reason to use a $6 \times 6$ matrix will be apparent after the conversion.)
%\[ \left( \begin{array}{cccccc}
%3 & 0 & 0 & 0 & 0 & 0\\
%2 & 3 & 0 & 0 & 0 & 0\\
%0 & 2 & 3 & 0 & 0 & 0\\
%1 & 0 & 2 & 3 & 0 & 0\\
%0 & 1 & 0 & 2 & 3 & 0\\
%0 & 0 & 1 & 0 & 2 & 3\\
%\end{array} \right)\] 
%Here the constant coefficient is assigned to every entry on the main diagonal of the matrix, the $x$ coefficient is assigned to the diagonal one space below, and so on.  Now to do the same with the second polynomial.
%\[ \left( \begin{array}{cccccc}
%3 & 0 & 0 & 0 & 0 & 0\\
%0 & 3 & 0 & 0 & 0 & 0\\
%1 & 0 & 3 & 0 & 0 & 0\\
%0 & 1 & 0 & 3 & 0 & 0\\
%0 & 0 & 1 & 0 & 3 & 0\\
%0 & 0 & 0 & 1 & 0 & 3\\
%\end{array} \right)\] 
%Now multiply the two matrices.
%\[ \left( \begin{array}{cccccc}
%3 & 0 & 0 & 0 & 0 & 0\\
%2 & 3 & 0 & 0 & 0 & 0\\
%0 & 2 & 3 & 0 & 0 & 0\\
%1 & 0 & 2 & 3 & 0 & 0\\
%0 & 1 & 0 & 2 & 3 & 0\\
%0 & 0 & 1 & 0 & 2 & 3\\
%\end{array} \right)\cdot \\
%\left( \begin{array}{cccccc}
%3 & 0 & 0 & 0 & 0 & 0\\
%0 & 3 & 0 & 0 & 0 & 0\\
%1 & 0 & 3 & 0 & 0 & 0\\
%0 & 1 & 0 & 3 & 0 & 0\\
%0 & 0 & 1 & 0 & 3 & 0\\
%0 & 0 & 0 & 1 & 0 & 3\\
%\end{array}\right)=\\
%\left( \begin{array}{cccccc}
%9 & 0 & 0 & 0 & 0 & 0\\
%6 & 9 & 0 & 0 & 0 & 0\\
%3 & 6 & 9 & 0 & 0 & 0\\
%5 & 3 & 6 & 9 & 0 & 0\\
%0 & 5 & 3 & 6 & 9 & 0\\
%1 & 0 & 5 & 3 & 6 & 9\\
%\end{array} \right)\] 
%
%Now reversing the process, we use the main diagonal to get the constant coefficient, the second diagonal as the $x$ coefficient and so on to yield:
%\[x^5 + 5x^3 + 3x^2 + 6x + 9\]
%Which happens to be the the same answer as our polynomial multiplication above.  Now the reason we picked a $6\times6$ matrix was to account for each term in the answer.  If we had used a smaller matrix, the $x^5$ term would not have been accounted for in the lower left hand corner of the answer matrix.  
%
%\begin{exercise}
%Given two quadratic polynomials:
%\[ 2x^2-8x+3 \quad \text{and} \quad -9x^2+3x+3 \]
%Find the sum of the two by writing them as $4\times4$ matrices and using matrix addition.
%\end {exercise}
%
%\begin{exercise}{matrixcommute}
%\begin{enumerate}[(a)]
%\item
%Matrices are known for not being commutative over matrix multiplication.  Multiply the above matrices in opposite order to show the result is the same.
%\end {enumerate}
%\end {exercise}
%
%Now while the above example is nice, it takes a bit more rigor to prove that there is an isomorphism.  Given polynomials:
%\[p(x) = \sum_{i=0}^{m} a_i x^i\]
%\[q(x) = \sum_{j=0}^{n} b_j x^i\]
%Where $a_i$ and $b_i$ are arbitrary coefficients of index $i$ and $j$, and $m$ and $n$ are the orders of the polynomials, define a relation $\Phi$ from a polynomial to an $m+n+1 \times m+n+1$ matrix as:
%\[\Phi(p(x)) = \\
%\left( \begin{array}{cccccc}
%a_0 & 0 & 0 & 0 & 0 & \cdots\\
%a_1 & a_0 & 0 & 0 & 0 & \cdots\\
%a_2 & a_1 & a_0 & 0 & 0 & \cdots\\
%a_3 & a_2 & a_1 & a_0 & 0 & \cdots\\
%a_4 & a_3 & a_2 & a_1 & a_0 & \cdots\\
%\vdots & \vdots & \vdots & \vdots & \vdots & \ddots\\
%\end{array} \right)\\
%\] 
%To show this is an isomorphism, we must first show that $\Phi$ is 1 to 1 and onto.
%
%\noindent
%\emph{Proof of 1 to 1}:~~
%If $\Phi(p) = \Phi(q)$
%then
%\[\Phi(p(x)) = \\
%\left( \begin{array}{cccccc}
%a_0 & 0 & 0 & 0 & 0 & \cdots\\
%a_1 & a_0 & 0 & 0 & 0 & \cdots\\
%a_2 & a_1 & a_0 & 0 & 0 & \cdots\\
%a_3 & a_2 & a_1 & a_0 & 0 & \cdots\\
%a_4 & a_3 & a_2 & a_1 & a_0 & \cdots\\
%\vdots & \vdots & \vdots & \vdots & \vdots & \ddots\\
%\end{array} \right)\\
%=\\
%\left( \begin{array}{cccccc}
%b_0 & 0 & 0 & 0 & 0 & \cdots\\
%b_1 & b_0 & 0 & 0 & 0 & \cdots\\
%b_2 & b_1 & b_0 & 0 & 0 & \cdots\\
%b_3 & b_2 & b_1 & b_0 & 0 & \cdots\\
%b_4 & b_3 & b_2 & b_1 & b_0 & \cdots\\
%\vdots & \vdots & \vdots & \vdots & \vdots & \ddots\\
%\end{array} \right)=\\
%\Phi(q(x)) \\
%\] 
%For two matrices to be equal, each of the terms in a given row/column in one matrix must be equal to the corresponding term in the other matrix.  Therefore, $a_0 = b_0$ ; $a_1 = b_1 \cdots$ for all indeces of $a$ and $b$.  Which means all the coefficients $a_i$ and $b_i$ are equal.  Which by definition of polynomials, means $p=q$.  Therefore $\Phi(p) = \Phi(q) \Rightarrow p = q$.
%
%Proof of onto.
%\noindent For any given matrix:
%
%\[M=\left( \begin{array}{cccccc}
%a_0 & 0 & 0 & 0 & 0 & \cdots\\
%a_1 & a_0 & 0 & 0 & 0 & \cdots\\
%a_2 & a_1 & a_0 & 0 & 0 & \cdots\\
%a_3 & a_2 & a_1 & a_0 & 0 & \cdots\\
%a_4 & a_3 & a_2 & a_1 & a_0 & \cdots\\
%\vdots & \vdots & \vdots & \vdots & \vdots & \ddots\\
%\end{array} \right)\\
%\] 
%
%Select the polynomial \[p=\sum_{i=0}^{m} a_i x^i.\]
%It follows from the definition of $\Phi$ that $\Phi(p) = M$.
%
%
%\noindent \emph{Proof of operation preservation:}~~ 
%Rather than giving a formal proof, we observe the pattern.  On the one hand, we have
%\[\Phi(pq)= \Phi(\sum_{i=0}^{m+n}(\sum_{j=0}^{i}a_jb_{i-j})x^i)=\Phi(a_0b_0 + (a_1b_0 + a_0b_1)x + (a_2b_0 + a_1b_1 + a_0b_2)x^2\cdots)\]
%\[\left( \begin{array}{cccccc}
%{a_0b_0} & 0 & 0 & 0 & 0 & \cdots\\
%{a_1b_0+a_0b_1} & {a_0b_0} & 0 & 0 & 0 & \cdots\\
%{a_2b_0 + a_1b_1 + a_0b_2} & {a_1b_0+a_0b_1} & {a_0b_0} & 0 & 0 & \cdots\\
%\vdots & {a_2b_0 + a_1b_1 + a_0b_2} & {a_1b_0+a_0b_1} & {a_0b_0} & 0 & \cdots\\
%\vdots & \vdots & {a_2b_0 + a_1b_1 + a_0b_2} & {a_1b_0+a_0b_1} & {a_0b_0} & \cdots\\
%\vdots & \vdots & \vdots & \vdots & \vdots & \ddots\\
%\end{array} \right).
%\]
%On the other hand, we have
%\[
%\Phi(p)\cdot\Phi(q) =
%\left( \begin{array}{cccccc}
%a_0 & 0 & 0 & 0 & 0 & \cdots\\
%a_1 & a_0 & 0 & 0 & 0 & \cdots\\
%a_2 & a_1 & a_0 & 0 & 0 & \cdots\\
%a_3 & a_2 & a_1 & a_0 & 0 & \cdots\\
%a_4 & a_3 & a_2 & a_1 & a_0 & \cdots\\
%\vdots & \vdots & \vdots & \vdots & \vdots & \ddots\\
%\end{array} \right)\\
%\cdot\\
%\left( \begin{array}{cccccc}
%b_0 & 0 & 0 & 0 & 0 & \cdots\\
%b_1 & b_0 & 0 & 0 & 0 & \cdots\\
%b_2 & b_1 & b_0 & 0 & 0 & \cdots\\
%b_3 & b_2 & b_1 & b_0 & 0 & \cdots\\
%b_4 & b_3 & b_2 & b_1 & b_0 & \cdots\\
%\vdots & \vdots & \vdots & \vdots & \vdots & \ddots\\
%\end{array} \right).\\
%\] 
%If we multiply these two matrices, we will find the result is the same as $\Phi(pq)$. This completes the proof of isomorphism.
%
%This was a lot of work, but it has an immediate benefit. Since matrix multiplication is associative, it follows immediately that polynomial multiplication is also associative. This replaces the long proof of associativity that we used earlier.
%
%%Split into different section.%also define fields = rings+multiplicative inverse.
%We have seen that polynomials have two closed operations: addition and multiplication. Under addition, the polynomials are an abelian group. Under multiplication, the polynomials possess all the properties of a commutative group \emph{except} identity. Also, multiplication distributes over addition (just like with real numbers). When a set of objects possesses two operations satisfying these conditions, it is known as a \emph{commutative ring} \index{Commutative ring}.   
%
%%In other words, a set $R$ is a commutative ring if:
%%
%%\begin{enumerate}[1.]
%%\item
%%$a+b = b + a$ for all $a,b \in R$
%%\item
%%(a+b) + c = a + (b + c) for all $a,b,c \in R$
%%\item
%%There is a $z \in R$ such that $z+a = a$ for all $a\in R$
%%\item
%%For all $a\in R$ there exists a $-a$ such that $a + (-a) = z$
%%\item
%%For all $a,b,c \in R$ it is true that $(ab)c = a(bc)$
%%\item
%%For $a,b,c\in R$; $a(b+c) = ab + bc$ and $(a+b)c = ac + bc$
%
%%Use different coefficients for polynomials as exercises, Q[x], Z[x], 
%%add exercise for proof of field, Q[x], Z[x], Z5[x]
%%\end {enumerate}
%\begin{exercise}{proofofring}
%Prove or disprove that the following sets are rings.
%\begin{enumerate}[(a)]
%\item
%The set of complex numbers with integer coefficients with complex addition and multiplication.
%\item
%$M \times N$ matrices using matrix addition of the form $a_{mn} + b_{mn} = c_{mn}$  and matrix multiplication.
%\item
%The set of golden rectangles composed of pairs $l,w$ such that $l$ is the length, $w$ is the width and it is always true that $w*(1+\sqrt{5}) / 2 = l$ where addition is defined as $l_3 = l_1 + l_2$ and $w_3 = w_2 + w_1$; and multiplication is defined as $w_3 = w_1 * w_2$ and $l_3 = (l_2 * l_1) / [ (1 + \sqrt{5}) / 2 ]$.
%\end{enumerate}
%\end {exercise}

\section{The Division Algorithm for Polynomials\quad
\sectionvideohref{qc8GBcukwr4&list=PL2uooHqQ6T7PW5na4EX8rQX2WvBBdM8Qo&index=49}}
%only works if polynomial coefficients are in a field.
In the chapter on modular arithmetic, we used the following fact about integers: for any two integers $a$ and $b$  with $b > 0$, then there exist unique
integers $q$ and $r$ such that $a = bq+r$, where $0 \leq r < b$. This fact was known to the ancient Greeks, who proved it using what's known as the \emph{division algorithm}.\footnote{As we said before, you may find a proof in any book on number theory. Or, take a look at:  \url{http://2000clicks.com/mathhelp/NumberTh09EuclidsAlgorithm.aspx}.} It turns out that a similar
division algorithm\index{Division algorithm!for polynomials} exists for
polynomials. In this section we'll  give the proof. But first, as usual, some examples.
%add exercise for long division of polynomials with coefficients of Z[p]; p=2,3,5,7
%work an example in Z2 before giving exercise.

\begin{example}{poly_division} 
Dividing polynomials is very similar to long division of real numbers.  
 For example,
suppose that we divide $x^3 - x^2 + 2 x - 3$ by $x - 2$.  
\begin{center}
\begin{tabular}{rrcrcrcr}
        &  $x^2$  &  $+$  &      $x$  &  $+$  &    $4$  &       &       \\ \cline{2-8}
 \multicolumn{1}{r|}{$x - 2$}
        &  $x^3$  &  $-$  &    $x^2$  &  $+$  &  $2 x$  &  $-$  &  $3$  \\
        &  $x^3$  &  $-$  &  $2 x^2$  &       &         &       &       \\ \cline{2-8}
        &         &       &    $x^2$  &  $+$  &  $2 x$  &  $-$  &  $3$  \\
        &         &       &    $x^2$  &  $-$  &  $2 x$  &       &       \\ \cline{4-8}
        &         &       &           &       &  $4 x$  &  $-$  &  $3$  \\
        &         &       &           &       &  $4 x$  &  $-$  &  $8$  \\ \cline{6-8}
        &         &       &           &       &         &       &  $5$
\end{tabular}
\end{center}
In the example, we need to take the leading power term of $x$ in the divisor and multiply by something that will make it equal to the the leading power term in the dividend.  In this case it's $x^2$.  This gives $x^2\cdot(x-2) = x^3 - 2x^2$  Subtract from the dividend to yield a remainder of $x^2 + 2x - 3$ and repeat until the remainder is of a degree less than the divisor.
 
Hence, $x^3 - x^2 + 2 x - 3 = (x - 2) (x^2 + x + 4 ) + 5$.  And simply multiplying out the right side will show that these are indeed equal.
\end{example}

\begin{example}{poly_division} 
Divide $(2x^3+3x^2+x+4)$ by $(x+2)$ where  both polynomials are in $\mathbb{Z}_5$.
\begin{center}
\begin{tabular}{rrcrcrcr}
        &  $2x^2$  &  $+$  &      $4x$  &  $+$  &    $3$  &       &       \\ \cline{2-8}
 \multicolumn{1}{r|}{$x + 2$}
        &  $2x^3$  &  $+$  &    $3x^2$  &  $+$  & $ x$  &  $+$  &  $4$  \\
        &  $2x^3$  &  $+$  &    $4 x^2$  &       &         &       &       \\ \cline{2-8}
        &         &       &                $4x^2$  & $+$  &  $ x$  &  $+$  &  $4$  \\
        &         &       &                $4x^2$  &  $+$  & $ 3x$  &       &       \\ \cline{4-8}
        &         &       &           &       &                         $3 x$  & $+$  & $4$  \\
        &         &       &           &       &                          $3x$  & $+$  & $1$  \\ \cline{6-8}
        &         &       &           &       &         &       &                               $3$
\end{tabular}
\end{center}
\end{example}

%Explain that Cayley table for multiplication would help in determining what terms go in the quotient. (refer to remainder equation). Also explain that the equivalent mod form a negative number can be with number line(refer back to chapter 3) division alorithm to work. This only works for Z_p. More generally, you can only use the algorithm in in a field.
%Also test out long division for matrices. Answer no, bc they don't all have inverses(give example with det=1/0)
%same for polynomials with polynomial coefficients
%same for poly with coefficient of another variable
% Show example in Z_4 (x^2+x+!)/(2x+1). 2*?=1, BUT there is no multiplicative inverse to multiply by 2 in Z_4. Every coefficient must have a multiplicative inverse for the 
\begin{exercise}{div1}
Find $q(x)$ and $r(x)$ in the following equations.
\begin{enumerate} [(a)]
\item $x^2+3x+27=(x-2)q(x) + r(x)$
\item $15x^3+13x-27=(x-5)q(x) + r(x)$
\item $10x^3 - x^2+3x+27=(2x^2-4)q(x) + r(x)$
\end {enumerate}
\end {exercise}

\begin{exercise}{div2}
\begin{enumerate} [(a)]
\item Divide  $ ( 3x^6 + x^5 +4x^4 +2)$  by $ ( x+3) $ where both polynomials are in  $\mathbb{Z}_5$.
\item Divide $ (x^7 + x^5 + x^3 + x)$  by $ ( x + 1 ) $ where both polynomials are in $ \mathbb{Z}_2$.
\end {enumerate}
\end {exercise}


And now, for the proof:

\begin{prop} {Division for Polynomials} \emph{(Division algorithm for polynomials)}
Let $f(x)$ and $g(x)$ be nonzero polynomials where the degree of $g(x)$ is greater than 0.  Then there exists unique polynomials $q(x)$ and $r(x)$ such that 
\[
f(x) = g(x)q(x) + r(x)
\]
where the degree of $r(x)$ is less than the degree of $g(x)$.
\end {prop}

\begin{rem}
This proposition is true for \emph{any} polynomial ring over a field. You may check the following proof works for any polynomial ring over a field, and not just real polynomials.
\end{rem}
%%%move long division and show how to find gcd of polynomials-assign an exercise or two (euclid to find gcd. then find diophantine)

\begin{proof}
We will first prove the existence of $q(x)$ and $r(x)$. We define a set $S$ as follows: 
\[S = \{f(x) - g(x) h(x), \text{ for all } h(x) \in P(x) \}.\] 
 This set is nonempty since $f(x) \in
S$. 
Let
$r(x)$ be  a polynomial of smallest degree in $S$.\footnote{At this point we can't assume that there's only one such polynomial, so we have to say ``a polynomial'' rather than ``the polynomial''.} This means that there must exist a $q(x)$ such that  
\[
r(x) = f(x) - g(x) q(x).
\]
We need to show that the degree of $r(x)$ is less than the degree of
$g(x)$. Let's prove this by contradiction. So we assume the contrary, namely  that $\deg g(x) \leq \deg r(x)$. 
Let $n,m$ be the degree of $g(x),r(x)$ respectively, where $n \leq m$. Then we may write
\[
g(x) = a_0 + a_1 x + \cdots + a_n x^n
\]
and
\[
r(x) = b_0 + b_1 
x + \cdots + b_m x^m, \]
where $a_n \neq 0$ and $b_m \neq 0$. 
Taking a cue from the process of long division, we define a new polynomial $r'(x)$ by
\begin{align*}
r'(x) := r(x) - b_m a_n^{-1} x^{m-n}g(x)
\end{align*}
It's tedious to write out all the terms of $r'(x)$. Fortunately, it's not really necessary. We only need to remark that the degree of $r'(x)$ is less than the degree of $r(x)$, since the leading-order terms of $r(x)$ and $b_m(a_n^{-1}) x^{m-n}g(x)$ are both $b_m x^m$, so they cancel. We may plug in $r(x) = f(x) - g(x) q(x)$ to obtain
\begin{align*}
r'(x) &:=  f(x) - g(x) q(x) - b_m a_n^{-1}  x^{m-n}g(x)\\
&= f(x) - g(x) \left( q(x) - b_m a_n^{-1}  x^{m-n}\right).
\end{align*}
This shows that $r'(x)$ is also in $S$  (look back at the definition and see!).  But 
$\deg r'(x) < \deg r(x)$, which contradicts our condition that $r(x)$ is an element of $S$ with smallest degree. The rules of proof by contradiction allow us to conclude that our assumption is false: namely, it must be true that $\deg g(x) > \deg r(x)$.
This finishes the proof of existence.

To show that  $q(x)$ and $r(x)$ are unique, suppose that polynomials $q'(x)$ and $r'(x)$ satisfy $f(x) = g(x) q'(x)
+ r'(x)$, so that
\[
f(x) = g(x) q(x) + r(x) = g(x) q'(x) + r'(x).
\]
This implies
\[
g(x) [q(x) - q'(x) ] = r'(x) - r(x).
\]
If $q(x) - q'(x)$ is not the zero polynomial, then $ \deg( g(x) \leq \deg( g(x) [q(x) - q'(x) ])$ , which implies
\[
\deg g(x) \leq \deg( g(x) [q(x) - q'(x) ] )= \deg( r'(x) - r(x) ).
\]
However, the degrees of both $r(x)$ and $r'(x)$ are strictly less than
the degree of $g(x)$, so their difference can't have such a large degree.  It follows that $q(x) - q'(x) = 0$, which implies that $q(x)=q'(x)$ and $r(x)=r'(x)$.
\end{proof}
%% find gcd of polynomials 
\section{Polynomial roots\quad
\sectionvideohref{Uh-czcHjyq4&list=PL2uooHqQ6T7PW5na4EX8rQX2WvBBdM8Qo&index=50}}
The Fundamental Theorem of Algebra (discussed in Section~\ref{sec:FTOA}) asserts that a real polynomial of degree $n$ has at most $n$ roots. (Actually, this is just the easy part of the theorem, as we mentioned before.) You may have learned this before from previous math classes, but you've probably never seen it proved. Seek no more: the proof is at hand. Not only that--our results will apply to  \emph{any} polynomial ring over a field, including $\RR[x]$, $\CC[x]$, $\QQ[x]$, and $\ZZ_p[x]$ (when $p$ is prime).  

The following proposition gives us a way to relate polynomial values to polynomial remainders.

\begin{prop}{polyremainder}
Let $f(x)$ be any polynomial in any polynomial ring over a field. When $f(x)$ is divided by $x-a$, the remainder is $f(a)$.
\end {prop}

\begin{proof}
According to Proposition~\ref{proposition:poly:Division for Polynomials}, if we divide $f(x)$ by $x-a$, it will produce two unique polynomials $q(x)$ and $r(x)$ such that $f(x) = (x-a)q(x) + r(x)$.  Since the degree of $x-a$ is 1, then according to the division algorithm, the degree of $r(x)$ must be less than 1.  Therefore $r(x)$ has to be a constant.  We will show this replacing $r(x)$ with $r$ where $r$ is an real number.  This yields:
\[f(x) = (x-a)q(x) + r.\]
If we set $x=a$ then we get:
\[f(a) = (a-a)q(x) + r \implies f(a) = 0 \cdot q(x) + r \implies f(a) = r.\]
\end {proof}

This proposition can save lots of time when finding remainders under division by monomials.

\begin{exercise}{}
\begin{enumerate}[(a)]
\item
Find the remainders when $\sum_{k=1}^{100} k x^{k-1}$ is divided by $x-1$ and  $x+1$, respectively.
\item
Find the remainders when $\sum_{k=0}^{100} \left( \frac{1}{2} \right)^k x^k$ is divided by $x-1/2$ and $x+1/2$, respectively.
\item
Find the remainders when $\sum_{k=0}^{100} 3^k x^{k}$ is divided by $x+1/9$ and $x - 1/9$, respectively.
\end{enumerate}
\end{exercise}

The following proposition in an important special case of Proposition~\ref{proposition:poly:polyremainder}.

\begin{prop}{divide}
$a$ is a root of $f(x)$ if and only if $x-a$ divides $f(x)$.
\end {prop}

\begin{proof}
From Proposition~\ref{proposition:poly:polyremainder} $f(x) = (x-a) \cdot q(x) + f(a)$, so $f(a) = 0$ if and only if $f(x)=(x-a) \cdot q(x)$ which is true if and only if $x-a$ divides $f(x)$.
\end{proof}

%irreduceable polynomials
%It is possible for a polynomial to always have a remainder regardless of which polynomial is used to divide it.  A polynomial that cannot be evenly divided cannot be further factored.  A polynomial that cannot be further factored we call an \term{irreducible polynomial}.  \index{Irreducible Polynomial} These polynomials serve a similar function for polynomials as prime numbers do for integers.  Since these polynomials cannot be reduced, we can make deductions as to the behavior of a product of irreducible polynomials.  For example the polynomial $(x+a)$ where $a$ is a real number cannot be factored any further, and the polynomial $mx+b$ where $m$ and $b$ are reals can only be factored into $(x+b/m)(m)$.  A like with real numbers, if we know that $f(x) \cdot g(x) = 0$ then either $f(x) = 0$ or $g(x) = 0$.  Now it is possible to get a higher degree polynomial to be irreducible, but the number set that is used can affect the irreducibility of a polynomial.  For example $x^2 + 1$ is irreducible over the reals, but could be factored over the complex numbers into $(x-i)(x+i)$.  

We will need one more fact in order to prove (the easy part of) the Fundamental Theorem of Algebra. From basic algebra, we know that if $ab=0$ then either $a=0$ or $b=0$ (see also 
Proposition~\ref{proposition:complex:nonzero_complex_product} of Chapter~\ref{complex}).  It turns out that the same is true for polynomials, as we now show.

\begin{prop}{polyremainder}
Suppose $F[x]$ is a polynomial ring over a field, and suppose $p(x),  q(x) \in F[x]$. Then $p(x)  q(x)=0$ iff either $p(x)=0$ or $q(x)=0$.
\end {prop}
\begin{proof}
Since this is a ``iff'' proof, we must actually prove two things:  

First we prove the ``if'' part.  It is clear from the formula for multiplication of polynomials that if either $p(x)=0$ or $q(x)=0$, then the product $p(x)q(x)$ must also be 0. That was easy!

The ``only if'' part is harder. We will prove the contrapositive, namely that  $p(x)\neq\ 0$ and  $ q(x)\neq\ 0$ implies that  $p(x)q(x) \neq 0$.
Let 
\[p(x) =  \sum_{i=0}^{m} a_i x^i \mathrm{~ and ~} q(x) =  \sum_{j=0}^{n} b_j x^j,\] 
where $a_m \neq\ 0$ and $b_n\neq\ 0$.
We can then write
 \[p(x) q(x) = \sum_{k=0}^{m+n} c_k, \mathrm{~~where~~} c_{k} =  \sum_{i=0}^{k}a_i b_{k-i}.\]
Consider the coefficient $c_{m+n}$, which may be expanded out as
\[
c_{m+n} =  a_0 b_{m+n} + a_1 b_{m+n-1} + \ldots  + a_{m}b_{n} + \dots +  a_{n+m-1}b_{1} + a_{n+m}b_{0}.
\]
Take a look at these terms for a moment. Which of them are nonzero?  Notice how we've separated out the term $a_{m}b_{n}$ in the middle of the expansion. Since $a_{m} \neq 0$ and $b_{n} \neq 0$, this term is nonzero. Now, are there
any other nonzero terms?  All terms have the  form $a_i b_j$, and for every other term in the series (besides $a_{m}b_{n}$) we have either $i>m$ or $j>n$.  If $i>m$ then $a_i=0$, since the degree of $p(x)$ is $m$ and all coefficients of terms of higher degree are 0.  For the same reason, if $j>n$ then $b_j=0$. It follows that except for the term $a_{m}b_{n}$, all other terms $a_ib_j$ are 0, which implies that $c_{m+n} = a_mb_n \neq 0$. But this means that $p(x)q(x)$ has a nonzero term, namely
$c_{m+n}x^{m+n}$, so $p(x)q(x) \neq 0$.    The proof is completed.
\end{proof}

And here's the result we've been waiting for. Now that we've prepared the ground, it's not so difficult to prove.

\begin{prop}{FTOA}
In any polynomial ring over a field, If $f(x)$ is a polynomial of degree $n$ then the equation $f(x)=0$ has at most $m$ solutions.
\end {prop}


\begin{proof}
Suppose $a_1$ is a solution to $f(x)=0$. Then by Proposition~\ref{proposition:poly:divide} it follows that $x-a_1$ divides $f(x)$. Therefore $f(x) = (x-a_1) g_{n-1}(x)$ where the degree of $g_{n-1}(x)=n-1$. 

Now if $a_2 \neq a_1$ is another solution then using our above result we have
\[ f(a_2) = (a_2 - a_1)g_{n-1}(a_2) = 0. \]
Since $a_2 - a_1 \neq 0$, it follows that $g_{n-1}(a_2) = 0$. So we can write $g_{n - 1}(x) = (x-a_2)g_{n-2}(x)$ where the degree of $g_2(x) = n-2$. 

Continuing in the same way, if there are distinct roots $a_1,a_2,...,a_n$ then 
\[
f(x) = (x - a_1)(x - a_2)...(x - a_n)g_0,\]
 where the degree of $g_0$ is 0 (in other words, $g_0$ is a constant.). So there can't be any more solutions, $a_{n+1}$, because $(x-a_{n+1})$ doesn't divide $g_0$.
\end {proof}

A special case of this theorem tells us something very, very important about the coefficients of a polynomial ring over a field. First, a definition:

\begin{defn}
A set $F$ is called a \term{field} if it satisfies the following properties:
\begin{itemize}
\item
Two binary operations (denoted by `$+$' and `$\cdot$') are defined on $F$.  
\item
$F$ is an Abelian group under $+$. We denote the identity element by 0. 
\item
$F \setminus 0$ is an Abelian group under $\cdot$.
\end{itemize}
\end{defn}

The previous theorem immediately gives us an extremely important general property of fields:

\begin{prop}{}
Let $F$ be a field, and let $c$ be any element $F$.  Then $c$ has at most $n$ $n^{\text{th}}$ roots.
\end {prop}


\begin{proof}
Given the field $F$ let $F[x]$ be the associated polynomial ring over the field $F$. The polynomial $x^n-c$ is an element of $F[x]$. By Proposition~\ref{proposition:poly:FTOA}, the equation $x^n-c=0$ has at most $n$ solutions.  This is exactly the same thing as saying that $c$ has at most $n$ $n^{\text{th}}$ roots.
\end {proof}

The theory of fields is an extremely important part of abstract algebra, with many very beautiful and important results. 


\section{Proof that $U(p)$ is cyclic}

Mathematics has many mysterious and wonderful connections. Surprisingly, we can use Proposition~\ref{proposition:poly:FTOA} to prove something about group theory.  
Recall that $U(n)$ is the group of units in $\ZZ_n$, where a ``unit'' is an element with a multiplicative inverse. If $p$ is a prime, then $U(p)$ is the set of all nonzero elements of $\ZZ_p$. In some coding theory applications, it's important to find elements of $U(p)$ which have a very larger order (recall that the \term*{order} of a group element $g \in G$ (denoted by $|g|$ is the smallest positive integer $n$ such that $g^n = {\var id}$). We know from Lagrange's theorem that $|g|$ divides $|G|$.  The largest possible value for $|g|$ is $|G|$: but this is \emph{only} possible when $G$ is cyclic. (Why?)  We are thus interested in primes $p$ such that $U(p)$ is cyclic. 

One might say to oneself, ``Oh, if $p$ is prime then $U(p)$ must be cyclic since it's of prime order.'' Alas, this is not true:

\begin{exercise}{}
Show that if $p$ is a prime greater than 3, then $|U(p)|$ is \emph{not} a prime.
\end{exercise}

But all is not lost! Even though  $|U(p)|$ is not prime, we can still prove that $U(p)$ is cyclic using some of our results from the previous section. 

\begin{prop}{Up_cyclic}
$U(p)$ is cyclic for every prime $p$.
\end {prop}
\begin{proof}
First, notice that Proposition~\ref{proposition:poly:FTOA} says that there are at most $m$ solutions to the equation $x^m = 1$ in $\ZZ_p$. Since 0 is not a solution, it follow that all of these solutions are also in $U(p)$.

Also, according to the factorization of Abelian groups (Proposition~\ref{FactorabelianGroup}), there exists an isomorphism $\phi$:
\[\phi: U(p)  \rightarrow {\mathbb{Z}_{{p_1}^{e_1}}}  \times   {\mathbb{Z}_{{p_2}^{e_2}}}  \times  ... \times   {\mathbb{Z}_{{p_k}^{e_k}}},\]
where $p_1, p_2, \ldots, p_k$ are all primes.  It's not necessarily true a priori  that all of the $p_j$'s are distinct: but if they are, then Proposition~\ref{RelativelyPrime} tells us that $U(p)$ must be cyclic. 

So it all comes down to proving that all of the $p_j$'s are distinct. We will prove this by contradiction. Thus, we may suppose that $p_i = p_j$ for some $i \neq j$. 
Now consider the following two elements of the direct product:
\[
g_i = ( 0, \ldots ,\underbrace{p_i^{e_i-1}}_{i'\text{th place}},\ldots,0) \mathrm{~ and ~} g_j = ( 0, \ldots ,\underbrace{p_j^{e_j-1}}_{j'\text{th place}},\ldots,0).
\]
It is then possible to prove that (recall that ``$|g|$ '' is the order of the group element $g$ )
\[ |g_i| = p_i \mathrm{~ and ~}  |g_j| = p_j. \]

\begin{exercise}{orders}
Given the above definitions of $g_i$ and $g_j$, show that   $|g_i| = p_i$ and  $|g_j| = p_j$.
\hyperref[sec:polyrings:hints]{(*Hint*)} 
\end{exercise}
As a result of the above exercise, Proposition~\ref{cosets_theorem_7} of the Cosets chapter  enables us to conclude that 
\[
|g_i^n| = p_i \mathrm{~ and ~} |g_j^n| = p_j \mathrm{~ for~~} (n = 1,...,p_j-1).
\]
Since $p_i = p_j$, we have at least $2(p_i-1)$ elements in ${\mathbb{Z}_{{p_1}^{e_1}}}  \times   {\mathbb{Z}_{{p_2}^{e_2}}}  \times  ... \times   {\mathbb{Z}_{{p_k}^{e_k}}}$ of order $p_i$. By 
Proposition~\ref{isomorph_theorem_1} of the Isomorphisms chapter, this means there are  $2(p_i-1)$ elements of $U(p)$ which have order $p_i$, and all of these elements are solutions of the equation $x^{p_i}-1=0$ (Why?). 
But at the beginning of this proof, we demonstrated that there can only be at most $p_i$ solutions. This contradiction shows us our supposition is false, so all of the $p_i$'s in the direct product must be unequal.
\end{proof}


\begin{exercise}{}
\begin{enumerate}[(a)]
\item
Show that $U(6),U(8),$ and $U(9)$ are cyclic.
\item
Give an example of a positive integer $n$ for which $U(n)$ is \emph{not} cyclic.
\item
Is it possible to specify exactly the positive integers $n$ for which $U(n)$ is cyclic?  (You may do some internet research to answer this one.  Hint: the answer has something to do with the name ``Gauss''.)
\end{enumerate}
\end{exercise}

%
%\begin{defn}
%A vector spaae over $\mathbb{R}$ consists of a set \emph{V} along with two operations '+' and  ' $\cdot$ ' subject to the conditions that for all vectors $ \vec {v},\vec{w},\vec{u} \in$ \emph{V} and all scalars r, s $\in \mathbb{R}$:
%\begin{enumerate} [(1)]
%\item
%the set \emph{V} is closed under vector addition, that is, $ \vec{v} + \vec{w} \in \emph{V}$
%\item
%vector addition is commutative, $\vec{v} + \vec{w} = \vec{w} + \vec{v}$
%\item
%vector addition is associative, ($\vec{v} + \vec{w}) + \vec{u} = \vec{v} + (\vec{w} + \vec{u}$)
%\item
%there is a zero vector $\vec{0} \in$ \emph{V} such that $\vec{v} + \vec{0} = \vec{v}$ for all $\vec{v} \in \emph{V}$
%\item
%each $\vec{v} \in \emph{V}$ has an additive inverse $\vec{w} \in \emph{V}$ such that $\vec{w} + \vec{v} = \vec{0}$
%\item
%the set \emph{V} is closed under scalar multiplication, that is,$ r\cdot \vec{v} \in \mathbb{R}$
%\item
%addition of scalars distributes over scalar multiplication, $(r+ s) \cdot \vec{v} = r \cdot \vec{v} + s \cdot \vec{v}$
%\item
%scalar multiplication distributes over vector addition,$ r \cdot( \vec{v} + \vec{w}) = r \cdot \vec{v} + r \cdot \vec{w}$
%\item
%ordinary multiplication of scalars associates with scalar multiplication, $(rs) \cdot \vec{v} = r \cdot (s \cdot \vec{v})$
%\item
%multiplication by the scalar 1 is the identity operation, $1 \cdot \vec{v} = \vec{v}$.
%\end{enumerate}
%\end{defn}
%
%\begin{example}{}
%  The set $\mathbb{R}^3$ is a vector space if the operations '$ + $' and  '$ \cdot$' have their usual meaning.
%
%$\left(\begin{array}{c}x_1\\x_2\\x_3\end{array}\right)$ $\mbox{~+~}$ $\left(\begin{array}{c}y_1\\y_2\\y_3\end{array}\right)$ = $\left(\begin{array}{c}x_1+y_1\\x_2+y_2\\x_3+y_3\end{array}\right)$ ~~~ r$\cdot$$\left(\begin{array}{c}x_1\\x_2\\x_3\end{array}\right)$ = $\left(\begin{array}{c}rx_1\\rx_2\\rx_3\end{array}\right)$
%
%We shall check all of the conditions.
%For (1), closure of addition,for any $ v_1,v_2,v_3,w_1,w_2,w_3$ $\in$ $\mathbb{R}$ the result of the vector sum
%
%$\left(\begin{array}{c}v_1\\v_2\\v_3\end{array}\right)$ $\mbox{~+~}$ $\left(\begin{array}{c}w_1\\w_2\\w_3\end{array}\right)$ = $\left(\begin{array}{c}v_1+w_1\\v_2+w_2\\v_3+w_3\end{array}\right)$ $\mbox{$\in$}$ $\mathbb{R}^3$
%
%For (2), that addition of vectors commutes,
%
%$\left(\begin{array}{c}v_1\\v_2\\v_3\end{array}\right)$ $\mbox{~+~}$ $\left(\begin{array}{c}w_1\\w_2\\w_3\end{array}\right)$ = $\left(\begin{array}{c}v_1+w_1\\v_2+w_2\\v_3+w_3\end{array}\right)$ 
%
% =  $\left(\begin{array}{c}w_1+v_1\\w_2+v_2\\w_3+v_3\end{array}\right)$ = $\left(\begin{array}{c}w_1\\w_2\\w_3\end{array}\right)$ $\mbox{~+~}$ $\left(\begin{array}{c}v_1\\v_2\\v_3\end{array}\right)$
%
%Condition(3), vector addition is associative.
%
%( $\left(\begin{array}{c}v_1\\v_2\\v_3\end{array}\right)$ $\mbox{~+~}$ $\left(\begin{array}{c}w_1\\w_2\\w_3\end{array}\right)$  ) + $\left(\begin{array}{c}u_1\\u_2\\u_3\end{array}\right)$ = $\left(\begin{array}{c}(v_1+w_1)+u_1\\(v_2+w_2)+u_2\\(v_3+w_3)+u_3\end{array}\right)$ 
%
%= $\left(\begin{array}{c}v_1+(w_1+u_1)\\v_2+(w_2+u_2)\\v_3+(w_3+u_3)\end{array}\right)$ =  $\left(\begin{array}{c}v_1\\v_2\\v_3\end{array}\right)$ $\mbox{~+~}$( $\left(\begin{array}{c}w_1\\w_2\\w_3\end{array}\right)$   + $\left(\begin{array}{c}u_1\\u_2\\u_3\end{array}\right)$)
%
%For (4),
% $\left(\begin{array}{c}v_1\\v_2\\v_3\end{array}\right)$ $\mbox{~+~}$ $\left(\begin{array}{c}0\\0\\0\end{array}\right)$ = $\left(\begin{array}{c}v_1\\v_2\\v_3\end{array}\right)$
%
%For (5), For any $v_1,v_2,v_3$ $\in$ $\mathbb{R}$ we have
%
%$\left(\begin{array}{c}-v_1\\-v_2\\-v_3\end{array}\right)$ $\mbox{~+~}$ $\left(\begin{array}{c}v_1\\v_2\\v_3\end{array}\right)$ = $\left(\begin{array}{c}0\\0\\0\end{array}\right)$
%
%The checks for the five conditions having to do with scalar multiplication are similar.
%
%For (6), closure under scalar multiplication, where r,$v_1,v_2,v_3$ $\in$ $\mathbb{R}$,
%
%r $\cdot$ $\left(\begin{array}{c}v_1\\v_2\\v_3\end{array}\right)$ = $\left(\begin{array}{c}rv_1\\rv_2\\rv_3\end{array}\right)$ $\in$ $\mathbb{R}^3$
%
%For (7),
%$( r + s )$ $\cdot$ $\left(\begin{array}{c}v_1\\v_2\\v_3\end{array}\right)$ = $\left(\begin{array}{c}( r + s )v_1\\( r + s )v_2\\( r + s )v_3\end{array}\right)$
% = $\left(\begin{array}{c}rv_1+sv_1\\rv_2+sv_2\\rv_3+sv_3\end{array}\right)$ 
%
%= r $\cdot$ $\left(\begin{array}{c}v_1\\v_2\\v_3\end{array}\right)$ + s $\cdot$ $\left(\begin{array}{c}v_1\\v_2\\v_3\end{array}\right)$
%
%For (8), that scalar multiplication distributes from the left over vector addition, we have this.
%
%r $\cdot$ ($\left(\begin{array}{c}v_1\\v_2\\v_3\end{array}\right)$ $\mbox{~+~}$ $\left(\begin{array}{c}w_1\\w_2\\w_3\end{array}\right)$ ) = $\left(\begin{array}{c}r(v_1+w_1)\\r(v_2+w_2)\\r(v_3+w_3)\end{array}\right)$
%
% = $\left(\begin{array}{c}rv_1+rw_1\\rv_2+rw_2\\rv_3+rw_3\end{array}\right)$ = r $\cdot$ ($\left(\begin{array}{c}v_1\\v_2\\v_3\end{array}\right)$) + r $\cdot$ ($\left(\begin{array}{c}w_1\\w_2\\w_3\end{array}\right)$)
%
%For the ninth condition,
%
%( rs ) $\cdot$ $\left(\begin{array}{c}v_1\\v_2\\v_3\end{array}\right)$ = $\left(\begin{array}{c}(rs)v_1\\(rs)v_2\\(rs)v_3\end{array}\right)$ = $\left(\begin{array}{c}r(sv_1)\\r(sv_2)\\r(sv_3)\end{array}\right)$ =  r$\cdot$(s  $\cdot$ $\left(\begin{array}{c}v_1\\v_2\\v_3\end{array}\right)$)
%
%And the tenth condition,
%1 $\cdot$ $\left(\begin{array}{c}v_1\\v_2\\v_3\end{array}\right)$ =  $\left(\begin{array}{c}1v_1\\1v_2\\1v_3\end{array}\right)$ =  $\left(\begin{array}{c}v_1\\v_2\\v_3\end{array}\right)$
%\end {example}
%
%The set $\mathbb{R}[x]$ of polynomials with real coefficients is a vector space over $\mathbb{R}$, using the standard operations on polynomials. In fact, a polynomial ring is an infinite-dimensional vector space. Each individual polynomial is of a finite degree, but the set has no single bound on the degree of all of its members. For instance, We can think of $1 + 4x + 7x^2$ as corresponding to $( 1, 4, 7, 0, 0,...)$.
%
%\begin{exercise}{}
%Show that a polynomial ring also satisfies the properties of a vector space when given the standard operations on polynomials.( Use the set of polynomials with real coefficients $\{ a_0 + a_1x +  ... + a_nx^n | n \in \mathbb{N}$ and $a_0,...,a_n \in \mathbb{R}\}$)
%\end{exercise}
%
%In a general vector space, you don't multiply vectors with each other (only scalar multiplication and addition). But we've seen that you can multiply polynomials. So it seems polynomials have more restrictions than a vector space.
%There is an example of vector space that does have multiplication. Consider, for instance, vector space of $n \times n$ matrices. In this case we have a vector space with both ` + '  and ` $\cdot$ '.
%
%We can directly relate polynomials to matrices.
%
%\begin{example}{}
%Represent $p(x) = 3x^2 - 7x + 2$ as a vector:$\left[\begin{array}{c}2\\-7\\3\\0\\0\end{array}\right]$
%(notice listing the coefficients of $x^n$ from lowest to highest)
%
%Notice that $x \cdot p(x)$ = $\left[\begin{array}{c}0\\2\\-7\\3\\0\end{array}\right]$
%or $x \cdot p(x)$ = $\left[\begin{array}{ccccc}0 & 0 & 0 & 0 & 0\\1 & 0 & 0 & 0 & 0\\0 & 1 & 0 & 0 & 0\\0 & 0 & 1 & 0 & 0\\0 & 0 & 0 & 1 & 0\end{array}\right]$$\left[\begin{array}{c}0\\2\\-7\\3\\0\end{array}\right]$
%
%So it seems that $x$ can be identified with the matrix $\left[\begin{array}{ccccc}0 & 0 & 0 & 0 & 0\\1 & 0 & 0 & 0 & 0\\0 & 1 & 0 & 0 & 0\\0 & 0 & 1 & 0 & 0\\0 & 0 & 0 & 1 & 0\end{array}\right]$
%
%and $x^2$ can be identified with the matrix$\left[\begin{array}{ccccc}0 & 0 & 0 & 0 & 0\\0 & 0 & 0 & 0 & 0\\1 & 0 & 0 & 0 & 0\\0 & 1 & 0 & 0 & 0\\0 & 0 & 1 & 0 & 0\end{array}\right]$.
%\end{example}
%
%\begin{exercise}{}
%Find the matrix related to each polynomial.
%\begin{enumerate}[(a)]
%\item
%$7x^2 - 2x + 3$
%\item
%$3x^4 + 5x^2 - 2$
%\end{enumerate}
%\end{exercise}
%
%\begin{example}{}
%Use $6 \times 6$ matrices to show that the matrix related to $-4x^2 + 3x - 2$ multiplied by the matrix related to $7x^2 - 2x - 5$ is equal to the matrix related to $(-4x^2 + 3x - 2 )(7x^2 - 2x - 5)$.
%
%$\left[\begin{array}{cccccc}-2 & 0 & 0 & 0 & 0 & 0\\3 & -2 & 0 & 0 & 0 & 0\\-4 & 3 & -2 & 0 & 0 & 0\\0 & -4 & 3 & -2 & 0 & 0\\0 & 0 & -4 & 3 & -2 & 0\\0 & 0 & 0 & -4 & 3 & -2\end{array}\right]$$\left[\begin{array}{cccccc}-5 & 0 & 0 & 0 & 0 & 0\\-2 & -5 & 0 & 0 & 0 & 0\\7 & -2 & -5 & 0 & 0 & 0\\0 & 7 & -2 & -5 & 0 & 0\\0 & 0 & 7 & -2 & -5 & 0\\0 & 0 & 0 & 7 & -2 & -5\end{array}\right]$ 
%
%= $\left[\begin{array}{cccccc}10 & 0 & 0 & 0 & 0 & 0\\-11 & 10 & 0 & 0 & 0 & 0\\0 & -11 & 10 & 0 & 0 & 0\\29 & 0 & -11 & 10 & 0 & 0\\-28 & 29 & 0 & -11 & 10 & 0\\0 & -28 & 29 & 0 & -11 & 10\end{array}\right]$
%
%$(-4x^2 + 3x - 2 )(7x^2 - 2x - 5) =  -28x^4 + 29 x^3 - 11x + 10$
%
%: $\left[\begin{array}{cccccc}10 & 0 & 0 & 0 & 0 & 0\\-11 & 10 & 0 & 0 & 0 & 0\\0 & -11 & 10 & 0 & 0 & 0\\29 & 0 & -11 & 10 & 0 & 0\\-28 & 29 & 0 & -11 & 10 & 0\\0 & -28 & 29 & 0 & -11 & 10\end{array}\right]$
%
%\end{example}
%To show \emph{isomorphism},l
%et $\varphi : P[x] \mapsto M_{\infty \times \infty}$ for a polynomial \[p(x) = \sum^ {N}_{m=0} a_{m}x^{m}\] then we have $\varphi(p(x))$ is a matrix $P$ with entries.
%
%$[\varphi(p)]_{i,j} = \left\{\begin{array}{rcl}\ a_m  &\mbox{if}&   i - j = m,   m=0,... N \\  0 &\mbox{otherwise}& \end{array}\right.$
%
% Prove 1-1. 
%
%Suppose $p(x)$ and $q(x)$ are polynomials and $p(x) \neq q(x)$. We can write \[p(x) = \sum^ {N}_{m=0} a_{m}x^{m}\] and \[q(x) = \sum^ {N'}_{m'=0} b_{m'}x^{m'}\]
%Since $p(x) \neq q(x)$, there must be some k such that $ a_k \neq b_k $.
%But then according to the definition it follows
%
%$[\varphi(p)]_{k+1, 1} = a_k$
%
%$[\varphi(q)]_{k+1, 1} = b_k$
%
%Therefore, $\varphi(p) \neq \varphi(q)$ since $a_k \neq b_k$.
%The mapping $\varphi$ is \emph{not} onto all infinite matrices.
%
%\begin{exercise}{}
%Find an infinite matrix M such that $M \neq \varphi(p(x))$ for any polynomial $p(x)$.
%\end{exercise}
%
%\begin{exercise}{}
%Show that if $M = \varphi(p(x))$ for some polynomial $p(x)$ then M is \emph{subdiagonal} that is, all entries above the diagonal are 0.
%\end{exercise}
%
%\begin{exercise}{}
%Show that if $M = \varphi(p(x))$ then M is \emph{banded} that is, $M_{i+k, j+k} = M_{i, j}$ for any positive integers i, j, k.
%\end{exercise}
%
%Let us define B as the set of all subdiagonal banded matrices. The previous 2 exercises have shown that $\varphi$ maps $P[x]$ into B. In fact,$ \varphi$ maps $P[x]$ onto B.
%
%\begin{exercise}{}
%Let M be an arbitrary matrix in B. Suppose the first column of M is the column vector $\left(\begin{array}{c}v_1\\v_2\\v_3\\.\\.\end{array}\right)$
%
%Find a polynomial $p(x)$ such that $\varphi(p(x)) = M$.
%\end{exercise}
%
%Until now we've been talking about how $\varphi$ preserves the operation under multiplying polynomials. It turns out that $\varphi$ is also an isomorphism under adding polynomials.
%
%\begin{exercise}{}
%Let \[p(x) = \sum^ {N}_{m=0} a_{m}x^{m}\] and \[q(x) = \sum^ {N'}_{m'=0} b_{m'}x^{m'}\]. Show that $\varphi(p + q) = \varphi(p) + \varphi(q)$.
%\end{exercise}
%
%\begin{example}{}
%Show $\varphi(x)\varphi(x) = \varphi(x^2)$.
%
%$\left[\begin{array}{ccc}0 & 0 & 0 \\1 & 0 & 0 \\0 & 1 & 0 \end{array}\right] \left[\begin{array}{ccc}0 & 0 & 0 \\1 & 0 & 0 \\0 & 1 & 0 \end{array}\right]
%=\left[\begin{array}{ccc}0 & 0 & 0 \\0 & 0 & 0 \\1 & 0 & 0 \end{array}\right]
%= \varphi(x^2)$
%\end{example}
%
%\begin{example}{}
%Show $\varphi(x^2)\varphi(x) = \varphi(x^3)$.
%
%$\left[\begin{array}{cccc}0 & 0 & 0 & 0 \\0 & 0 & 0 & 0 \\1 & 0 & 0 & 0\\ 0 & 1 & 0 & 0 \end{array}\right] \left[\begin{array}{cccc}0 & 0 & 0 & 0 \\1 & 0 & 0 & 0 \\0 & 1 & 0 & 0\\0 & 0 & 1 & 0 \end{array}\right]
%= \left[\begin{array}{cccc}0 & 0 & 0 & 0 \\0 & 0 & 0 & 0 \\0 & 0 & 0 & 0\\1 & 0 & 0 & 0 \end{array}\right]
%= \varphi(x^3)$
%\end{example}
%
%The previous 2 examples show that $\varphi(x^m) \varphi(x) = \varphi(x^{m+1})$.
%
%\begin{example}{}
%$\varphi(x^m)\varphi(x^n) = \varphi(x^m)\varphi(x)\varphi(x^{n-1})$
%
%$~~~~~~~~~~~~~~~~= \varphi(x^{m+1})\varphi(x^{n-1})$
%
%$~~~~~~~~~~~~~~~~= \varphi(x^{m+1})\varphi(x)\varphi(x^{n-2})$
%
%$~~~~~~~~~~~~~~~~= \varphi(x^{m+2})\varphi(x^{n-2})$
% 
%After repeating these steps, we have $\varphi(x^m)\varphi(x^n) = \varphi(x^{m+n})$
%\end{example}
%
%This example suggests $\varphi$ preserves the operation of multiplication that is,
%$ \varphi(pq) = \varphi(p)\varphi(q)$.
%
%Be careful here! On the left- hand side we're taking the product of polynomials and taking $\varphi$ of the result. On the right- hand side, we're converting 2 polynomials into matrices, and multiplying the matrices.
%So there are 2 different multiplication operations on the 2 sides of the equation.
%So far we haven't proven that $\varphi$ preserves the operation of multiplication. We'll do that later - but first let's show some other properties of $\varphi$.
%
%We have proven $\varphi$ is 1-1 and onto earlier.
%Remember that we are wanting to prove that $\varphi$ preserves the operation of multiplication that is, $ \varphi(pq) = \varphi(p)\varphi(q)$.
%But, notice that both polynomials and matrices also have another operation in common, namely addition. We can therefore ask : Does $\varphi$ preserve the operation of addition? In other words, is it true that $\varphi(p + q) = \varphi(p) + \varphi(q)$?
%
%\begin{exercise}{}
%Show that if $p(x) = 3x^2 - 7x + 4$ and $q(x) = 2x^3 + 2x - 2$, then $\varphi(p + q) = \varphi(p) + \varphi(q)$.
%\end{exercise}
%
%The preceding exercise shows one particular example. We'd like to prove this in general.
%
%\begin{prop}{}
%Let  \[p(x) = \sum^ {N}_{m=0} a_{m}x^{m}\] and  \[q(x) = \sum^ {N'}_{m'=0} b_{m'}x^{m'}\].
%Then $\varphi(p + q) = \varphi(p) + \varphi(q)$.
%
%Proof : We can suppose that $ m \geq m' $ ( if $m'> m$, we just exchange p and q in the proof). In this case we have
% \[p(x) + q(x) = \sum^ {N}_{m=0} a_{m}x^{m} + \sum^{N}_{m=0} b_{m}x^{m}\] 
% \[=\sum^{N}_{m=0}(a_{m} + b_{m})x^{m}\]
% The reason we can replace the m' and N' in the second sum is that $N\geq N'$ ; the coefficients $b_j$ for $j > N'$ are just equal to 0.
%
% Now, using our formula for $\varphi$ we have
%
%$[\varphi(p + q)]_{i, j} = \left\{\begin{array}{rcl}\ a_m + b_m   &\mbox{if}&   i - j = m,  m = 0,...N \\ 0   &\mbox{otherwise}& \end{array}\right.$
%
%Comparing this with the 2 formulas
%
%$[\varphi(p)]_{i, j} = \left\{\begin{array}{rcl}\ a_m   &\mbox{if}&   i - j = m,  m = 0,...N \\ 0   &\mbox{otherwise}& \end{array}\right.$
%
%and 
%$[\varphi(q)]_{i, j} = \left\{\begin{array}{rcl}\ b_m   &\mbox{if}&   i - j = m,  m = 0,...N \\ 0   &\mbox{otherwise}& \end{array}\right.$
%
%It's clear that $[\varphi(p + q)]_{i, j} = [\varphi(p)]_{i, j} + [\varphi(q)]_{i, j}$ for every i and j. In other words, all of the matrix entries  of $\varphi(p + q)$ are equal to the sum of corresponding entries of $\varphi(p)$ and $\varphi(q)$.
%
%Therefore $\varphi(p + q) = \varphi(p) + \varphi(q)$.
%\end{prop}
%
%We have actually shown something is significant. Notice that polynomials and infinite matrices are both groups under addition.
%So we have shown that $\varphi$ is an one-to-one map of one group into another group which preserves the group operation `+ '.
%We can do even better than that.
%
%\begin{exercise}{}
%Show that B ( that is, the subdiagonal banded matrices) is a group under addition.
%\end{exercise}
%
%\begin{exercise}{}
%Show that $\varphi : P[x] \mapsto B$ is an isomorphism between the addition groups $( P[x] , +)$ and $(B, +)$.
%\end{exercise}
%
%We just need one more piece in the puzzle to show that $\varphi$ does preserve the operation of multiplication.
%
%\begin{exercise}{}
%\begin{enumerate}[(a)]
%\item
%Show $\varphi(ax) \varphi(bx) = \varphi(abx^2)$
%\item
%Show $\varphi(ax) \varphi(bx^2) = \varphi(abx^3)$
%\item
%Show $\varphi(ax) \varphi(bx^3) = \varphi(abx^4)$
%\item
%Fill in the blank (You don't need to prove this, just follow the pattern of part (a), (b),(c)).
%$\varphi(ax) \varphi(bx^k) = \varphi(  ~~~~~~~~~~~         )$
%\item
%Using part (d) repeatedly with $a = b = 1$, show that $\varphi(x^k) = (\varphi(x))^k$.
%\item
%Using part (e), show that $\varphi(x^k) \varphi(x^l) = \varphi(x^{k + l})$.
%\item
%Using the fact that $\varphi(ax^k) = a\varphi(x^k)$, show that $\varphi(ax^k) \varphi(bx^l) = \varphi(abx^{k + l})$.
%\end{enumerate}
%\end{exercise}
%
%Now we're finally ready to prove that $\varphi$ preserves the operation of multiplication.
%
%\begin{prop}{}
%Let  \[p(x) = \sum^ {N}_{m=0} a_{m}x^{m}\] and  \[q(x) = \sum^ {N'}_{m'=0} b_{m'}x^{m'}\].
%Then $\varphi(pq) = \varphi(p) \varphi(q)$.
%
%Proof : we'll start from the left-hand side of the equation $\varphi(pq) = \varphi(p) \varphi(q)$ and show it's equal to the right-hand side.
%\end{prop}
%
%\begin{exercise}{}
%Fill in the blanks the following proof.
% \[\varphi(pq) = \varphi((\sum^ {N}_{m=0} a_{m}x^{m})( \sum^{N'}_{m'=0} b_{m'}x^{m'}))\] 
% \[=\varphi(\sum^ {N}_{m=0}\sum^{N'}_{m'=0} a_{m}b_{m'} x^{(~~~~~)} )\] 
% \[=\sum^ {N}_{m=0}\sum^{N'}_{m'=0}\varphi( a_{m}b_{m'} x^{(~~~~~)} )\] 
% \[=\sum^ {N}_{m=0}\sum^{N'}_{m'=0}\varphi( ~~~~~~~ )\varphi(~~~~~~~)\]
%\[= (\sum^ {N}_{m=0}\varphi(~~~~~~~))( \sum^{N'}_{m'=0}\varphi(~~~~~~~))\] 
%\[=\varphi (\sum^ {N}_{m=0}(~~~~~~~))\varphi( \sum^{N'}_{m'=0}(~~~~~~~))\] 
%\[= \varphi(~~) \varphi(~~)\]
%\end{exercise}
%
%
%
%
%
%\section{Irreducible Polynomials}
% 
% 
%A nonconstant polynomial $f(x) \in F[x]$ is \term{
%irreducible}\index{Polynomial!irreducible}\index{Irreducible polynomial}
%over a field $F$ if $f(x)$ cannot be expressed as a product of two
%polynomials $g(x)$ and $h(x)$ in $F[x]$, where the degrees of $g(x)$
%and $h(x)$ are both smaller than the degree of $f(x)$.  Irreducible
%polynomials function as the ``prime numbers'' of polynomial rings.
% 
% 
%\begin{example}{poly_irred}
%The polynomial $x^2 - 2 \in {\mathbb Q}[x]$ is irreducible since it
%cannot be factored any further over the rational numbers. Similarly,
%$x^2 + 1$ is  irreducible over the real numbers. 
%\end{example}
% 
% 
%\begin{example}{finite_poly}
%The polynomial $p(x) = x^3 + x^2 + 2$ is irreducible over ${\mathbb
%Z}_3[x]$. Suppose that this polynomial was reducible over ${\mathbb
%Z}_3[x]$.  By the division algorithm there would have to be a factor
%of the form $x - a$, where $a$ is some element in ${\mathbb Z}_3[x]$.
%Hence, it would have to be true that $p(a) = 0$.  However,
%\begin{align*}
%p(0) & = 2 \\
%p(1) & = 1 \\
%p(2) & = 2.
%\end{align*}
%Therefore, $p(x)$ has no zeros in ${\mathbb Z}_3$ and must be
%irreducible. 
%\end{example}
%
%
%\begin{lemma}\label{poly:integer_coef_lemma}
%Let $p(x) \in {\mathbb Q}[x]$.  Then
%\[
%p(x) = \frac{r}{s}(a_0 + a_1 x + \cdots + a_n x^n),
%\]
%where $r, s, a_0, \ldots, a_n$ are integers, the $a_i$'s are
%relatively prime, and $r$ and $s$ are relatively prime. 
%\end{lemma}
% 
% 
%\begin{proof}
%Suppose that
%\[
%p(x) = \frac{b_0}{c_0} + \frac{b_1}{c_1} x + \cdots + \frac{b_n}{c_n}
%x^n,
%\]
%where the $b_i$'s and the $c_i$'s are integers. We can rewrite $p(x)$
%as 
%\[
%p(x) = \frac{1}{c_0 \cdots c_n} (d_0 + d_1 x + \cdots + d_n x^n),
%\]
%where $d_0, \ldots, d_n$ are integers. Let $d$ be the greatest common
%divisor of $d_0, \ldots, d_n$.  Then
%\[
%p(x) = \frac{d}{c_0 \cdots c_n} (a_0 + a_1 x + \cdots + a_n x^n),
%\]
%where $d_i = d a_i$ and the $a_i$'s are relatively prime. Reducing $d
%/(c_0 \cdots c_n)$ to its lowest terms, we can write
%\[
%p(x) = \frac{r}{s}(a_0 + a_1 x + \cdots + a_n x^n), 
%\]
%where $\gcd(r,s) = 1$.
%\end{proof}
