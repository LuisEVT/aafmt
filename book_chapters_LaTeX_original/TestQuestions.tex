\chap{Example Test Questions}{Practice}

The following exercises were taken from various exams. The list will be added to in subsequent editions of this text.  Contributions are welcome.


\section{In the Beginning}
\begin{enumerate}[(1)]
\item
Simplify:
$\displaystyle{ \left(\frac{(x+y)^{x+y}(x-y)^{x-y}}{(x^2 - y^2)^x}\right)}$
\item
Simplify:
$ \displaystyle{\left( \frac{6^6}{2^2 3^3} +  \frac{2^8 3^6}{6^3}\right)^{1/2}} $
\item
Simplify:   $ \displaystyle{\frac{(a+b)(b+c) + (a-b)(b-c)}{a+c} }$
\item
Given the expression:  $( a(bc - cb) + (ac - ca)b) + c(ab - ba)$:
\begin{enumerate}[(a)]
\item
Simplify, using the associative property ONLY.
\item
Simplify, using the associative and distributive properties ONLY.
\item
Simplify, using associative, distributive, and commutative properties.
\end{enumerate}

\item
Given the expression:~
 $(((a-b)+b)+b)(a-b) + b^2$
\begin{enumerate}[(a)]
\item
Simplify the expression using the associative law ONLY.
\item
Simplify the expression using the associative and distributive laws ONLY.
\item
Simplify the expression using the associative, distributive, and commutative laws.
\end{enumerate}

\item
Give an example (using actual numbers) to show that division is not associative.
\item
Suppose $ab>cb, b < 0,$ and $c<0$.  For each of the following statements, either prove that it is always true, or give an example to show that
it is not always true:
\begin{enumerate}[(a)]
\item
$a > b$ \qquad 
\item
$a < 0$.
\item
$b < c$ \qquad 
\item
$a < c$ \qquad 
\end{enumerate}

\end{enumerate}

\section{Complex Numbers}

\begin{enumerate}

\item
Evaluate: $(\sqrt{6}+3\sqrt{2}i)^6/36$.

\item
Prove that  $(z + \bar{z})(z - \bar{z})$ is real for any complex number $z$

\item
Suppose that $z$ is a complex number such that $z^{-1} = \bar{z}$.
\begin{enumerate}[(a)]
\item
 Find the modulus of $z$.
\item
How many solutions does this equation have?
\end{enumerate}
	
\item
Find all solutions to:  $\displaystyle{\frac{\bar{z}^4}{z} = 8i.}$
(\emph{Hint}:: Use polar form.)

\item
Find all solutions to:  $\bar{z}^3(z^2) = -32i.$
\item
Evaluate:  $\displaystyle{\frac{ \overline{3 + 8i} }{7 + 6i}}$.
\item
Evaluate:  $\displaystyle{( \overline{4 -7i} ) \cdot (\overline{3 + 3i})^{-1}}$.
\item
$z$ and $w$ are complex numbers. $z$ has modulus 7 and complex argument $\pi/9$, while $w$ has modulus $\sqrt{7}$ and argument $\pi/6$.  What are the modulus and argument of $z^3 w^{-4}$?

\item
Evaluate $\left(\frac{1-i}{2}\right)^{10}$.  Show your work. Give your answer in the form $a + bi$, with no decimals.

\item
Evaluate $ i^{2x+3}$, where $x$ is the smallest prime number that is greater than $1000^{1000}$.

\item
It is possible to raise numbers to imaginary powers.  A famous formula of Euler states that   
$e^{i\pi}= -1$, Where $e$ is the mathematical constant  $2.71828 \ldots$ .  Use this expression to show that $i^i$  is a positive real number between 0 and 1.  (\emph{Hint}: take the square root of both sides of Euler's formula.)

\item
Simplify:  $(z + \bar{z})(z - \bar{z}) + \overline{(z + \bar{z})(z - \bar{z})}$. \emph{Show your work.}

\item
A cubic polynomial of the form $x^3 + ax^2 + bx + c$  ($a,b,c$ are real)  has roots $11$ and $3-i$.  Find $a,b,c$.

\item
A polynomial of the form $x^4 + a_3x^3 + a_2x^2 + a_1x + a_0$  ($a_0,a_1,a_2,a_3$ are real)  has roots $3+2i$ and $1-i$.  Find $a_0,a_1,a_2,a_3$.

\item
Find all $6^{\text{th}}$ roots of  $8i$.


\item
Find all fifth roots of $-1-i$.

\item
Suppose $\bar{z} = iz$ and $z \neq 0$. Show that $z^4$ is a negative real number.

\item
Draw a picture of the following set in the complex plane:  $|\textrm{Re}[z] | = 2$.  (Recall that $\textrm{Re}[z]$ means the real part of $z$.)

\item
Show that there exists a complex number $z$ such that $\sin(t) = \text{Re}[z \cdot \cis(t)]$, and find $z$.

\item
\begin{enumerate}[(a)]
\item 
Find a complex number $w$ such that $\sin(t) - \cos(t) = \textrm{Re}[ w \cdot \cis(t) ]$. Express $w$ in complex polar form.  
\item
Using part (a), show that  $\sin(t) - \cos(t)$ can be written as $ A \cdot \cos(t + \theta)$. Find $A$ and $\theta$.  
\end{enumerate}

\end{enumerate}



\section{Modular Arithmetic}

\begin{enumerate}

\item
December 25, 2015 is on a Friday.  What day of the week is December 1, 2018?  (Note: 2016 has 366 days).

\item
A certain computer program takes 4,923 hours to run to completion. The program is started on Monday at 9:00 a.m.
\begin{enumerate}[(a)]
\item
What is the time on the clock when the program completes?
\item
On what day of the week does the program complete?
\end{enumerate}



\item
Make tables that show the results of:
\begin{enumerate}[(a)]
\item \label{Mod3TablesEx-multiplication}
multiplication modulo~$4$.
\item \label{Mod3TablesEx-subtraction}
addition modulo~$4$
\end{enumerate}
For both (a) and (b), all table entries should be  $\class{0} \ldots \class{3}$.



\item
Find integers $m$ and $n$ that solve the following equation: $4801m + 500n = 1337$.

\item
Solve these simultaneous congruences: $x \equiv 1 \bmod{2}; x \equiv 2 \bmod{3}; x \equiv 3 \bmod{5}; x\equiv4 \bmod{7}$.


\item
Show that  modular multiplication distributes over modular addition: That is, given $x,y,z \in \integer_n$ we have
\[
x \odot (y \oplus z) = (x \odot y) \oplus  (x \odot z).
\]
(\emph{Hint}:  First apply Exercise~\ref{exercise:modular:ops} part (c)  with $e=f=x, a=b=y, c=d=z$. Then, use the ordinary distributive law. on the left side of the modular equivalence.  Then, use  Proposition~\ref{proposition:modular:number_remainder} parts (a) and (b) to return to an expression with modular operations.  Finally, use Proposition~\ref{proposition:modular:equiv_mod_n} to obtain an equality instead of modular equivalence.)

\item
 Find a solution to:  $411m + 312n = 41 $.

\item
Find all solutions to: $277x \equiv 149 \pmod{113}$.

\item
Find all solutions to:  $228 x - 104 \equiv 777 \pmod{56}$


\item
Find all solutions to:  $ 470x - 120 \equiv 852 \pmod{93}$

\item
Compute:  mod($30!,19$).  (Note: $30!$ means $1 \cdot 2 \cdot 3 \cdot \ldots \cdot 30$.)

\item
Compute mod( $3^{100}$,5). 


\item
Evaluate: 
\begin{enumerate}[(a)]
\item
 gcd(111,507) 
\item
gcd(182,367) 
\item
gcd(39,409)
\end{enumerate}

\item
Find values of $m$ and $n$ that solve the following equations:
\begin{enumerate}[(a)]
\item
$88m + 97n = 19$
\item
 411m + 312n = 41 
\item
105m + 75n = 225
\end{enumerate}


\item
\begin{enumerate}[(a)]
\item
Show that if $\bmod(x,4)=1$, then $\bmod(x^2,8)=1$.
\item
Show that if $\bmod(x,4)=2$, then $\bmod(x^2,8) = 4$.
\item
Show that if $\bmod(x,4)=3$ then $\bmod(x^2,8) = 1$.
\item	
Show that if $x,y$  are integers such that $\bmod(x^2+y^2,8)=2$, then both $x$ and $y$ must be odd.
\item	
Show that if $x,y,z$ satisfy $x^2 + y^2 + 6 = 8z$, then $x$ and $y$ must both be odd.
\end{enumerate}

\item
\begin{enumerate}[(a)]	
\item
Show that if $\bmod(x,4)=1$, then $\bmod(x^4,16)=1$.
\item	
Show that if $\bmod(x,4)=2$ then $\bmod(x^4,16) = 0$.
\item	
Show that if $\bmod(x,4)=3$ then $\bmod(x^4,16) = 1$.
\item	
Show that if $\bmod(x,4)=0$ then $\bmod(x^4,16) = 0$.
\item
Show that if $\bmod(x^4+y^4,16) = 0$ then $x$ and $y$ must both be even.
\item	
Show that if $x,y, z$ are integers such that $x^4+y^4 = 16z$, then $x$ and $y$ must both be even.
\end{enumerate}

\item
Find $x,y \in \integer_{51}$ such that $x \neq \class{0}$ and $y \neq \class{0}$, but $x \cdot y = \class{0}$.


\end{enumerate}

\section{Introduction to Cryptography}

\begin{enumerate}

\item
Perform the following matrix multiplication mod 44. Simplify before multiplying

$$\left(
\begin{array}{cc}
444 & 486 \\
890 & 606
\end{array}
\right)
\left(
\begin{array}{cc}
1103 & 2200 \\
990 & 133
\end{array}
\right)$$

\item
Perform the following matrix multiplication mod 37. Simplify before multiplying

$$\left(
\begin{array}{cc}
409 & 372 \\
743 & 189
\end{array}
\right)
\left(
\begin{array}{cc}
-105 & 410 \\
 -300& -225
\end{array}
\right)$$

\item
\begin{enumerate}[(a)]
\item
Show that if $m$ is odd, then $\mod(m^2,8) = 1$.

\item
If$\mod(n,8)=1$ and $n = (x+y)(x-y)$, show that  $x$ must be odd and $y$ is divisible by 4.

\item
If$\mod(n,8)=7$ and $n = (x+y)(x-y)$, show that  $y$ must be odd.

\item
If $\mod(n,8)=5$ and $n = (x+y)(x-y)$, show that  $\mod(y,4) = 2$.

\item
If$\mod(n,8)=3$ and $n = (x+y)(x-y)$, show that  $y$ is odd.
\end{enumerate}

\item 
Consider an affine cryptosystem working on an alphabet with 27 letters. For each of the following functions, (i) determine whether the  function is a valid encoding function; (ii) if the function is valid, find the decoding function. Express the decoding function in the form: $f^{-1}(c) = (a\odot c) \oplus b$ where $a$ and $b$ are numbers between 0 and 26.
\begin{enumerate}[(a)]
\item
$f(p) = (4 \odot p) \oplus 7$
\item	
$f(p) = (5 \odot p) \oplus 12$ 
\item
$f(p) = (6 \odot p) \oplus  11$
\end{enumerate}

\item
The general form for an affine cryptosystem encoding function is $f(p) =(a\odot p) \oplus b$.
\begin{enumerate}[(a)]
\item
 How many different possible values of $a$ are there, for an affine cryptosystem that works on  an alphabet of 16 letters?
\item	
For the same situation as (a), how many different possible values are there for $b$?
\item	
What is the total number of affine cryptosystems that work on an alphabet of 16 letters?
\end{enumerate}

\item
A polyalphabetic cryptosystem on an alphabet with 7 letters uses the encoding function:
$f(p)= Ap + b$,where $A=[3 , 5; 5, 6]$ and b =$ [4 ; 2]$. (\emph{Here we are using Matlab format for matrices.})  
What is the decoding function? Express your answer as:  $ f^{-1}(c) = Bc + d$, where $B$  is a $2\times 2$ matrix and $d$ is a $2 \times 1$ vector.  Both $B$ and $d$ should have entries between 0 and 6.

\item	
Make a spreadsheet to compute  $\bmod(47^x, 116)$, where $x = 64$.

\item
Make a spreadsheet to compute $\bmod(91^x, 97)$, where $x = 35$.

\item	
Make a spreadsheet that uses the brute-force method to factor 1010011.

\item	
Make a spreadsheet that uses the brute-force method to factor 6666661.

\item 
Make a spreadsheet that uses the Fermat method to factor the following number: 65072743

\item 	
Make a spreadsheet that uses the Fermat method to factor the following number: 70081027

\item	
Given that 1234567 is the number $n$ for a RSA cryptosystem.  The encoding key is 113. The decoding key is one of the following three numbers: 140797, 140897, 140997. Which of these numbers is the correct decoding key? Prove your answer.

\item
Given that 145279  is the number n for a RSA cryptosystem.  The encoding key is 113. The decoding key is one of the following three numbers:
133579, 134579, 135679.
Which of these numbers is the correct decoding key? Prove your answer.
	
\end{enumerate}


\section{Set Theory}

\begin{enumerate}

\item
Let ${\mathbb N}$ be the universal set and suppose that
\begin{align*}
A &= \{ x \in {\mathbb N} : x \text{ is a perfect square (that is, } x=y^2 \text{ where } y \text{ is a natural number)}\} \\ 
B &= \{ x \in {\mathbb N} : x \text{ is divisible by 6}\} \\ 
C &= \{ x \in {\mathbb N} : x  \equiv 2 \text{ (mod 3)} \\
D &= \{ x \in {\mathbb N} : x  \equiv 0 \text{ (mod 3)} \\
\end{align*} 
Specify each of the following sets. You may specify a set either by describing a property, by enumerating the elements, or as one of the four sets $A, B, C, D$:
\begin{enumerate}[(a)]
\item
$(A \cap B)$
\item
$B \cap C$
\item
$C \cup (B \setminus D)$.
\end{enumerate}



\item
Prove using set identities ONLY:  
\[A = (A\setminus (B\cup C)) \cup (A\cap (B\setminus C)) \cup  (A \cap C).\]    
Set diagram or element-by-element proofs are verboten (although you may use them to help you arrive at a proof).

\item
Given a set $S$, let $G$ be the set of all subsets of $S$. We will define an operation `\textcircled{s}' on $G$ as follows.  If $X$ and $Y$ are two elements of $G$  then we define $X \textcircled{s} Y$ as follows:
\[ X \textcircled{s} Y = (X \cup Y) \setminus (X \cap Y) .\]
\begin{enumerate}[(a)]
\item
Prove that the set $G$ is closed under the operation $\textcircled{s}$.
\item
It turns out that the operation $\textcircled{s}$ has an identity: that is, there is a set $Z \in G$ such that
\[  Z \textcircled{s} X = X \textcircled{s} Z = X  \qquad \text{for any $X \in G$.} \]
Which element of $G$ has this property?  \emph{Prove} your answer.  (\emph{Hint}:  The answer is either the empty set $\emptyset$ or the entire set $S$. Tell which one, and prove your answer.)  
\item
Given a set $X \in G$, what is the inverse of $X$ under the operation $\textcircled{s}$?  \emph{Prove} your answer. 
\item
Is the operation $\textcircled{s}$ defined on the set $G$ commutative? \emph{Prove} your answer. 
\item
It also turns out that $\textcircled{s}$ is associative (you don't need to prove this).  Is $G$ a group under the operation $\textcircled{s}$?  \emph{Prove} your answer.
\end{enumerate}


\end{enumerate}

\section{Functions}

\begin{enumerate}

\item
Here is a function~$f$ given by a table of values.

\begin{center}
\begin{tabular}{c|c}
$x$ & $f(x)$ \\ \hline

0 & 3 \\
1 & 4 \\
2 & 0 \\
3 & 1 \\
4 & 2 \\
\end{tabular}
\end{center}

\begin{enumerate}[(a)]
\item  \label{FunctionByTableEx-domain}
What are the domain and range  of~$f$?
\item  \label{FunctionByTableEx-pairs}
Represent $f$ by an arrow diagram.
\item
Is $f$ a bijection? If so, give a table for $f^{-1}$.
\item  \label{FunctionByTableEx-formula}
Find a formula to represent~$f$. (\emph{Hint}:  Consider arithmetic mod 5).
\end{enumerate}

\item
Let $k$ be a positive integer, and Define $g : \integer \times \integer  \rightarrow \integer \times \integer$ by: $g(m, n) = (m + n,  m - kn)$.
\begin{enumerate}[(a)]
\item
Prove or disprove:  $g$ is one-to-one.
\item
Prove or disprove:  $g$ is onto.
\end{enumerate}

\item
Let $ f:Y \rightarrow X$ and $g:X \rightarrow Y$ be functions   such that  $f \compose g(x) = x$ for all $x \in X$.
\begin{enumerate}[(a)]
\item
Give an example to show that $g$ may not be the inverse of $f$.
\item
Suppose that $g$ is onto. Then prove that $g = f^{-1}$.
\end{enumerate}

\item
Given $f : A \rightarrow B$ and $g : B \rightarrow C$, such that $f$  is not onto but  $g \compose f$  is onto. Prove that $g$ is not one-to-one.

\item
Given $f : A \rightarrow B$ and $g : B \rightarrow C$, such that $g$  is one-to-one  but  $g \compose f$  is not one-to-one. Prove that $f$ is not one-to-one.

\item
Let $a \in \mathbb{Z}_4$, and define $f_a \colon \mathbb{Z}_4 \to \mathbb{Z}_4$ by $f_a(x) = a \odot x$.  (Here the multiplication is mod 4.) 
\begin{enumerate}[(a)]
\item \label{LinearWhenBijectionExer-not0}
For which values of $a$  is $f_a$ a bijection? (Hint: $f_0$ is not a bijection (explain why), and $f_1$ is a bijection (explain why).  You need to check $f_2$ and $f_3$.
\item \label{LinearWhenBijectionExer-0}
For all values of $a$ for which $f_a$ is a bijection, find the inverse of $f_a$.
\end{enumerate}

\item
 For the given sets $A$ and~$B$:
\[ A = \{\var{a},\var{b},\var{c}\}, B = \{\var{d},\var{e}\} \]
\begin{enumerate}[(a)] 
\item How many different functions are there from $A$ to $B$?
\item Write each function from~$A$ to~$B$ as a set of ordered pairs. 
\item  Indicate which of the functions are onto.
\item Indicate which of the functions are one-to-one.
\end{enumerate}

\item
For each function, either prove that it is a bijection, or prove that it is not.
\begin{enumerate}[(a)]
\item \label{modular9}
 $g \colon {\mathbb Z}_6 \to {\mathbb Z}_6$ defined by $g(x)= (x \odot 3) \oplus  (x \odot 2)$ .
\item \label{modular_m6}
 $g \colon {\mathbb Z}_6 \to {\mathbb Z}_6$ defined by $g(x) = (x \odot 2) \oplus (x \odot 2) $ .
 \end{enumerate}

\item
Find formulas for $(f \compose g)(x)$ and $(g \compose f)(x)$, where 
 $f(x) = ax^2 $ and $g(x) = \sqrt{b x^2 + c}$ (where $a,b,c \in \real$)

 In each case, determine whether $g$ is an inverse of~$f$.
 \begin{enumerate}[(a)]
 \item \label{VerifyInverseExers-(x^2)}
$f \colon \real^+ \to \real^+$ is defined by $f(x) =2x^2$ and 
 \\ $g \colon \real^+ \to \real^+$ is defined by $g(y) = \sqrt{y}/2$.
 \item \label{VerifyInverseExers-(sqrt(x+1)-1)}
$f \colon \real^+ \to \real^+$ is defined by $f(x) = \sqrt{x+1} - 1$ and 
 \\ $g \colon \real^+ \to \real^+$ is defined by $g(y) = y^2 + 2y$.
 \end{enumerate}



\end{enumerate}

\section{Equivalence Relations}

\begin{enumerate}

\item
\begin{enumerate}[(a)]
\item
Let $X =\{a, b\}$. List all the equivalence relations on $X$.
\item
Let $Y =\{a, b,c\}$. List all the equivalence relations on $Y$.
\end{enumerate}

\item
Let $A$ be a set, and let $\sim$ be a relation on $A$. Given an element $a$ in  $A$, we shall call $a$ an ``orphan'' if there does not exist \emph{any} $b$ in $A$ such that $a \sim b$. In other words, $a$ is an orphan if it is not related to any other element of $A$.  

Suppose that the relation $\sim$ is symmetric, transitive, and has no orphans.  
\begin{enumerate}[(a)]
\item
Show that $\sim$ is also reflexive.
\item
Prove or disprove:  $\sim$ is an equivalence relation.
\end{enumerate}

\item
Define the relation $\sim_t$ by:  $(a,b) \sim_t  (c,d)$  iff $a \equiv c \pmod{3}$ and $b \equiv d \pmod{2}$, where $a,b,c,d  \in \integer_4$.  Define the relation $\sim_s$  by:  $(a,b) \sim_s (c,d)$ iff $a \equiv d \pmod{3}$ and 
$b \equiv c \pmod{2}$, where $a,b,c,d \in \integer_4$.
\begin{enumerate}[(a)]
\item
Prove or disprove: $\sim_t$ is an equivalence relation. If it is, give the equivalence classes for $\sim_t$.
\item
Prove or disprove: $\sim_s$ is an equivalence relation. If it is, give the equivalence classes for $\sim_s$.
\end{enumerate}

\item
Let $A =  \ZZ_{12}$  (that is, the integers from 0 to 11). Draw a digraph for each of the following binary relations on~$A$: 
 \begin{enumerate}[(a)]
 \item \label{DrawBinRelExer-married}
 $ R_a = \{\, (x,y) \mid  x \equiv y \text{(mod }4 \,\} .$
  \item \label{DrawBinRelExer-lived}
 $ R_b = \{\, (x,y) \mid  x \equiv 2y \text{(mod }3 \,\} .$
 \item
Explain why $R_a$ is an equivalence relation, and give the equivalence classes for $R_a$
\item
Explain why $R_b$ is \underline{not} an equivalence relation.
\end{enumerate}

\item
Let $f\colon \ZZ_8 \to \ZZ_8$ be defined by $f(x) =  x^2 $. Define a relation $\sim$ on $\ZZ_8$ by: $n \sim m$ iff $f(n) = f(m)$.
\begin{enumerate}[(a)]
\item
Show that $\sim$ is an equivalence relation: that is, show that $\sim$ is reflexive, symmetric, and transitive.
\item
According to Proposition~\ref{EquivRel->Part}, this equivalence relation produces a partition on  $\ZZ_8$. List the sets in the partition.
\end{enumerate}

\item
 Show that$f(x)=x^2$ provides a well-defined function from~$\integer_5$ to~$\integer_5$. That is, show that if $a,b \in \integer$, such that 
\[ [a]_5 = [b]_5, \text{ then } [ a^2]_7 = [ b^2]_7.\]


\item
Show that there is a well-defined function 
$f \colon \integer_4 \to \integer_{12}$, given by \\
$ f \bigl( [a]_4 \bigr) = [a]_{12}$. 
That is, show that if $[a]_{12} = [b]_{12}$, then $f \bigl( [a]_4 \bigr) = f \bigl( [b]_4 \bigr)$.



\end{enumerate}

\section{Symmetries of Plane Figures}

\begin{enumerate}

\item
Consider a regular hexagon, with vertices labeled 1,2,3,4,5,6 (in counterclockwise order). Let $s$ be the reflection that  leaves 1 invariant, that is, $s(1) = 1$.
\begin{enumerate}[(a)]
\item
Write the tableau representations of $s \compose r^2$ and $ r^4 \compose s$.
\item
Express $r^3 \compose  s \compose  r^2 \compose  s \compose  r\compose s$ in the form $s \compose r^p$ , where $p$ is an integer between 0 and 5.
\end{enumerate}

\item
Given a regular $n$-gon, with vertices labeled 1,2,…,$n$ (in counterclockwise order).  You may assume that $n>5$. Let $r$ be the rotation that satisfies $r(1)=2$, and let $s$ be the reflection that satisfies $s(1)=1$.
\begin{enumerate}[(a)]
\item
What is $r^{-1}(n)$?
\item
What is $r^2(3)$?  
\item
Suppose there exists a reflection $\mu_1$ such that $\mu_1(1)=5$. 
\item
What is  $\mu_1(2)$?
\item  
Express $\mu_1$ in terms of $r$ and $s$.
\item
If $n$ is odd, how many vertices are left unchanged by $\mu_1$?
\item 
Suppose there is exactly one vertex $k$ such that  $\mu_1(k)=k$.  Find $k$.
\item 
Consider the symmetry given by $s \compose r^m \compose s^3 \compose r^p$, where $m$ and $p$ are positive integers. Is this symmetry a reflection or a rotation?  Prove your answer. 
\end{enumerate}

\item
With reference to an equilateral triangle with vertices labeled $A,B,C$ counterclockwise, let $r$ be the 120-degree counterclockwise rotation and let $s$ be the reflection that leaves vertex $A$ fixed. 
% We have the relations:
%\[ r^3 = {\var id}, \qquad s^2 = {\var id}, \qquad r \compose s = s \compose r^2. \]
\begin{enumerate}[(a)]
\item
{\var id} is one symmetry of the triangle. Express the 5 other symmetries of the triangle in terms of $r$ and $s$.  
\item
Fill in the blanks $<1>,  <2>$, and $<3>$: 
\[r^{<1>} = {\var id}, \qquad s^{<2>} = {\var id},\qquad  r \compose s = s \compose <3>. \]
\item
Use the relations that you wrote down in part (b) to complete the Cayley table for rotations of the triangle.  All rotations (besides ${\var id}$) should be expressed in terms of $r$ and $s$.
\end{enumerate}


\end{enumerate}




