\section{Hints for ``Permutations'' exercises}
\label{sec:Permutations:Hints} 

\noindent Exercise \ref{exercise:Permutations:64}: The first blank should be replaced by $k$

\noindent Exercise \ref{exercise:Permutations:switchyard1}(c): Take advantage of the previous part.

\noindent Exercise \ref{exercise:Permutations:switchyard2}(b): Note for instance that $(1 \;2 \; 3) = (1 \; 2) (2 \; 3)$.

\noindent Exercise \ref{exercise:Permutations:switchyard2}(c): Note for instance that $(1 \;4 ) = (1 \; 2 \; 3 \; 4) \compose (1 \; 2 \; 3 )^{-1}$.

\noindent Exercise \ref{exercise:Permutations:Sn_even_odd}: Use the cycle structures you found in Exercise~\ref{exercise:Permutations:cycle_types}

% \noindent Exercise \ref{exercise:Permutations:S_nequiv}: Think about the relation between partitions and equivalence relations.

\noindent Exercise \ref{exercise:Permutations:A_nGroupProps}(b): If you write $\sigma$ as the product of transpositions $\tau_1 \cdots \tau_{n}$, then what is $\sigma^{-1}$?

\noindent Exercise \ref{exercise:Permutations:A_nGroupProps}(c): If $\sigma = \tau_1 \cdots \tau_{n}$ and $\mu = \lambda_1 \cdots \lambda_{m}$, then what about $\sigma \mu$?

\noindent Exercise \ref{exercise:Permutations:prove}(a): If $\sigma$ is even, then what about $(1 \; 2) \compose \sigma$?

\noindent Exercise \ref{exercise:Permutations:prove}(b): If $\mu$ is odd, then what about $(1 \; 2) \compose \mu$? Also, what is $f( (1 \; 2) \compose \mu )$?

\noindent Exercise \ref{exercise:Permutations:prove}(c): If $(1 \; 2) \sigma_1 = (1 \; 2) \sigma_2$, then what can you conclude about $\sigma_1$ and $\sigma_2$? Why are you able to conclude this?


\noindent Exercise \ref{exercise:Permutations:Ad4}: Let $\ell$ be the length of $\sigma$: then what is the order of $\sigma$? On the other hand, let $k$ be the order of $\sigma^2$: then what do you know about $\sigma^{2k}$?


\bigskip

\subsection{Hints for additional exercises (Section~\ref{sec:Permutations:AdditionalExercises})}

\noindent Exercise \ref{ex:Permutations:Ad1}: Consider the cycle structure.

\noindent Exercise \ref{ex:Permutations:Ad2}: We know that $\sigma$ can be written as the product of disjoint cycles. So let $\sigma_1, \sigma_2, \ldots \sigma_m$ be disjoint cycles such that  $\sigma = \sigma_1 \sigma_2 \ldots \sigma_m$, and let $\ell_j$ be the length of the cycle $\sigma_j$. How many transpositions does it take to construct each of these disjoint cycles? And what is the largest possible value of the sum of $\ell_j$?

\noindent Exercise \ref{ex:Permutations:Ad3}: Use the notation of the previous problem, and write a formula (in terms of $\ell_1 \ldots \ell_m$ and $m$) for the number of transpositions it takes to construct $\sigma$.
