\section{Study guide  for ``Set Theory''  chapter}
\label{sec:Sets:StudyGuide} 

\subsection*{Section \ref{sec:Sets:Basics}, Set Theory}
\subsubsection*{Concepts:}
\begin{enumerate}
\item 
Definition of a set
\item 
Sets of sets
\item 
Universal set
\item 
Subsets and proper subsets
\item 
Empty set
\item 
Union and intersection of sets
\item 
Disjoint sets
\item 
Complement set
\item 
Difference of sets
\end{enumerate}

\subsubsection*{Competencies}
\begin{enumerate}
\item
Given a description of the elements of a set, list the elements (and vice versa). (\ref{exercise:Sets:set1}) 
\item
Be able to describe sets of sets. (\ref{exercise:Sets:SetSet}) 
\item
Be able to specify sets given operations on the sets.  (\ref{exercise:Sets:14}, \ref{exercise:Sets:20})
\end{enumerate}


\subsection*{Section \ref{sec:Sets:Properties}, Properties of set operations}
\subsubsection*{Concepts:}
\begin{enumerate}
\item 
Properties of set operations
\item
De Morgan's Laws
\end{enumerate}

\subsubsection*{Key Formulas}
\begin{enumerate}
\item
Given any sets $A, B$, it is always true that $A \cap B \subset A \mathrm{~~~and~~~} \\ A \subset A \cup B.$
\item    
Properties of set operations: Let $A$, $B$, and $C$ be subsets of a universal set $U$. Then
\begin{enumerate}
\item
$A \cup A' = U$ and $A \cap A' = \emptyset$
\item
$A \cup A = A$, $A \cap A = A$, and $A \setminus A = \emptyset$;
\item
$A \cup \emptyset = A$ and $A \cap \emptyset = \emptyset$;
\item
$A \cup U = U$ and $A \cap U = A$; 
\item
$A \cup (B \cup C) = (A \cup B) \cup C$ and  $A \cap (B \cap C) = (A \cap B) \cap C$;
\item
$A \cup B = B \cup A$ and $A \cap B = B \cap A$;
\item
$A \cup (B \cap C) = (A \cup B) \cap (A \cup C)$;
\item
$A \cap (B \cup C) = (A \cap B) \cup (A \cap C)$. 
\end{enumerate}
\item
De Morgan's Laws: Let $A$ and $B$ be sets. Then 
\begin{enumerate}
\item
$(A \cup B)' = A' \cap B'$; 
\item
$(A \cap B)' = A' \cup B'$.
\end{enumerate}
\end{enumerate}

\subsubsection*{Competencies}
\begin{enumerate}
\item
Prove set identities algebraically, making use of the above properties of set operations. (\ref{exercise:Sets:30}) 
\end{enumerate}


\subsection*{Section \ref{sec:Sets:SetGroup}, Do the subsets of a set form a group?}
\subsubsection*{Concepts:}
\begin{enumerate}
\item 
Group properties   (Definition \ref{definition:ModularArithmetic:group})
\end{enumerate}

\subsubsection*{Competencies}
\begin{enumerate}
\item
Be able to prove or disprove group properties of set operations. (\ref{exercise:Sets:cup_group}) 
\end{enumerate}
