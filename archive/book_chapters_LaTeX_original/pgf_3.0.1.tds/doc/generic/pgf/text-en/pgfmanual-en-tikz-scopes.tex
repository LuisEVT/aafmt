% Copyright 2006 by Till Tantau
%
% This file may be distributed and/or modified
%
% 1. under the LaTeX Project Public License and/or
% 2. under the GNU Free Documentation License.
%
% See the file doc/generic/pgf/licenses/LICENSE for more details.

\section[Hierarchical Structures: Package, Environments, Scopes, and Styles]
{Hierarchical Structures:\\
  Package, Environments, Scopes, and Styles}

The present section explains how your files should be structured when
you use \tikzname. On the top level, you need to include the |tikz|
package. In the main text, each graphic needs to be put in a
|{tikzpicture}| environment. Inside these environments, you can use
|{scope}| environments to create internal groups. Inside the scopes
you use |\path| commands to actually draw something. On all levels
(except for the package level), graphic options can be given that
apply to everything within the environment.



\subsection{Loading the Package and the Libraries}

\begin{package}{tikz}
  This package does not have any options.

  This will automatically load the \pgfname\ and the |pgffor| package.

  \pgfname\ needs to know what \TeX\ driver you are intending to use. In
  most cases \pgfname\ is clever enough to determine the correct driver
  for you; this is true in particular if you use \LaTeX. One
  situation where \pgfname\ cannot know the driver ``by itself'' is when
  you use plain \TeX\ or Con\TeX t together with |dvipdfm|. In this case,
  you have to write |\def\pgfsysdriver{pgfsys-dvipdfm.def}|
  \emph{before} you input |tikz.tex|.
\end{package}


\begin{command}{\usetikzlibrary\marg{list of libraries}}
  Once \tikzname\ has been loaded, you can use this command to load
  further libraries. The list of libraries should contain the names of
  libraries separated by commas. Instead of curly braces, you can also
  use square brackets, which is something Con\TeX t users will
  like. If you try to load a library a second time, nothing will
  happen.

  \example |\usetikzlibrary{arrows.meta}|

  The above command will load a whole bunch of extra arrow tip
  definitions.

  What this command does is to load the file
  |tikzlibrary|\meta{library}|.code.tex| for each \meta{library} in
  the \meta{list of libraries}. If this file does not exist, the file
  |pgflibrary|\meta{library}|.code.tex| is loaded instead. If this
  file also does not exist, an error message is printed. Thus, to
  write your own library file, all you need to do is to place a file
  of the appropriate name somewhere where \TeX\ can find it. \LaTeX,
  plain \TeX, and Con\TeX t users can then use your library.
\end{command}



\subsection{Creating a Picture}

\subsubsection{Creating a Picture Using an Environment}

The ``outermost'' scope of \tikzname\ is the |{tikzpicture}|
environment. You may give drawing commands only inside this
environment, giving them outside (as is possible in many other
packages) will result in chaos.

In \tikzname, the way graphics are rendered is strongly influenced by
graphic options. For example, there is an option for setting the color used
for drawing, another for setting the color used for filling, and also
more obscure ones like the option  for setting the prefix used in the
filenames of temporary files written while plotting functions using an
external program. The graphic options are specified in \emph{key
  lists}, see Section~\ref{section-graphic-options} below for
details. All graphic options are local to the |{tikzpicture}| to which
they apply.

\begin{environment}{{tikzpicture}\opt{\oarg{options}}}
  All \tikzname\ commands should be given inside this
  environment, except for the |\tikzset| command. Unlike other
  packages, it is not possible to use, say, |\pgfpathmoveto| outside
  this environment and doing so will result in chaos. For \tikzname,
  commands like |\path| are only defined inside this environment, so
  there is little chance that you will do something wrong here.

  When this environment is encountered, the \meta{options} are
  parsed, see Section~\ref{section-graphic-options}. All options given
  here will apply to the whole picture.

  Next, the contents of the environment is processed and the graphic
  commands therein are put into a box. Non-graphic text is suppressed
  as well as possible, but non-\pgfname\ commands inside a
  |{tikzpicture}| environment should not produce any ``output'' since
  this may totally scramble the positioning system of the backend
  drivers. The suppressing of normal text, by the way, is done by
  temporarily switching the font to |\nullfont|. You can, however,
  ``escape back'' to normal \TeX\ typesetting. This happens, for
  example, when you specify a node.

  At the end of the environment, \pgfname\ tries to make a good guess
  at the size of a bounding box of the graphic and
  then resizes the picture box such that the box has this size. To ``make its
  guess,'' everytime \pgfname\ encounters a coordinate, it updates the
  bounding box's size such that it encompasses all these
  coordinates. This will usually give a good
  approximation of the bounding box, but will not always be
  accurate. First, the line thickness of diagonal lines is not taken
  into account correctly. Second, control points of a curve often lie far
  ``outside'' the curve and make the bounding box too large. In this
  case, you should use the |[use as bounding box]| option.

  The following key influences the baseline of the resulting
  picture:
  \begin{key}{/tikz/baseline=\meta{dimension or coordinate or \texttt{default}} (default 0pt)}
    Normally, the lower end of the picture is put on the baseline of
    the surrounding text. For example, when you give the code
    |\tikz\draw(0,0)circle(.5ex);|, \pgfname\ will find out that the
    lower end of the picture is at $-.5\mathrm{ex} - 0.2\mathrm{pt}$
    (the 0.2pt are half the line width, which is 0.4pt) and that the
    upper end is at $.5\mathrm{ex}+.5\mathrm{pt}$. Then, the lower end
    will be put on the baseline, resulting in the following:
    \tikz\draw(0,0)circle(.5ex);. 

    Using this option, you can specify that the picture should be
    raised or lowered such that the height \meta{dimension} is on the
    baseline. For example, |\tikz[baseline=0pt]\draw(0,0)circle(.5ex);|
    yields \tikz[baseline=0pt]\draw(0,0)circle(.5ex); since, now, the
    baseline is on the height of the $x$-axis.

    This options is often useful for ``inlined'' graphics as in
\begin{codeexample}[]
$A \mathbin{\tikz[baseline] \draw[->>] (0pt,.5ex) -- (3ex,.5ex);} B$
\end{codeexample}

    Instead of a \meta{dimension} you can also provide a coordinate in
    parentheses. Then the effect is to put the baseline on the
    $y$-coordinate that the given \meta{coordinate} has \emph{at the
      end of the picture}. This means that, at the end of the picture,
    the \meta{coordinate} is evaluated and then the baseline is set
    to the $y$-coordinate of the resulting point. This makes it easy
    to reference the $y$-coordinate of, say, the baseline of nodes.
\begin{codeexample}[]
Hello
\tikz[baseline=(X.base)]
  \node [cross out,draw] (X) {world.};
\end{codeexample}

\begin{codeexample}[]
Top align:
\tikz[baseline=(current bounding box.north)]
  \draw (0,0) rectangle (1cm,1ex);
\end{codeexample}
	
	Use |baseline=default| to reset the |baseline| option to its initial configuration.
  \end{key}

  \begin{key}{/tikz/execute at begin picture=\meta{code}}
    This option causes \meta{code} to be executed
    at the beginning of the picture. This option must be
    given in the argument of the |{tikzpicture}| environment itself
    since this option will not have an effect otherwise. After all,
    the picture has already ``started'' later on. The effect of
    multiply setting this option accumulates.

    This option is mainly used in styles like the |every picture|
    style to execute certain code at the start  of a picture.
  \end{key}

  \begin{key}{/tikz/execute at end picture=\meta{code}}
    This option installs \meta{code} that will be executed
    at the end of the picture. Using this option multiple times will
    cause the code to accumulate. This option must also be given in
    the optional argument of the |{tikzpicture}| environment.

\begin{codeexample}[]
\begin{tikzpicture}[execute at end picture=%
  {
    \begin{pgfonlayer}{background}
      \path[fill=yellow,rounded corners]
        (current bounding box.south west) rectangle
        (current bounding box.north east);
    \end{pgfonlayer}
  }]
  \node at (0,0) {X};
  \node at (2,1) {Y};
\end{tikzpicture}
\end{codeexample}
  \end{key}

  All options ``end'' at the end of the picture. To set an option
  ``globally'' change the following style:
  \begin{stylekey}{/tikz/every picture (initially \normalfont empty)}
    This style is installed at the beginning of each picture.
\begin{codeexample}[code only]
\tikzset{every picture/.style=semithick}
\end{codeexample}
  \end{stylekey}

  Note that you should not use |\tikzset| to set options directly. For
  instance, if you want to use a line width of |1pt| by default, do
  not try to say |\tikzset{line width=1pt}| at the beginning of your
  document. This will not work since the line width is changed in many
  places. Instead, say
\begin{codeexample}[code only]
\tikzset{every picture/.style={line width=1pt}}
\end{codeexample}
  This will have the desired effect.
\end{environment}

In other \TeX\ formats, you should use the following commands instead:

\begin{plainenvironment}{{tikzpicture}\opt{\oarg{options}}}
  This is the plain \TeX\ version of the environment.
\end{plainenvironment}

\begin{contextenvironment}{{tikzpicture}\opt{\oarg{options}}}
  This is the Con\TeX t version of the environment.
\end{contextenvironment}


\subsubsection{Creating a Picture Using a Command}

The following command is an alternative to |{tikzpicture}| that is
particular useful for graphics consisting of a single or few
commands.

\begin{command}{\tikz\opt{\oarg{options}}\marg{path commands}}
  This command places the \meta{path commands} inside a
  |{tikzpicture}| environment. The \meta{path commands} may
  contain paragraphs and fragile material (like verbatim text).

  If there is only one path command, it need not be surrounded by
  curly braces, if there are several, you need to add them (this is
  similar to the |\foreach| statement and also to the rules in
  programming languages like Java or C concerning the placement of
  curly braces).

  \example |\tikz{\draw (0,0) rectangle (2ex,1ex);}| yields
  \tikz{\draw (0,0) rectangle (2ex,1ex);}

  \example |\tikz \draw (0,0) rectangle (2ex,1ex);| yields
  \tikz \draw (0,0) rectangle (2ex,1ex);
\end{command}


\subsubsection{Handling Catcodes and the Babel Package}

Inside a \tikzname\ picture, most symbols need to have the category 
code 12 (normal text) in order to ensure that the parser works
properly. This is typically not the case when packages like |babel|
are used, which change catcodes aggressively.

To solve this problem, \tikzname\ provides a small library also called
|babel| (which can, however, also be used together with any other
package that globally changes category codes). What it does is to
reset the category codes at the beginning of every |{tikzpicture}| and
to restore them at the beginning of every node. In almost all cases,
this is exactly would you would expect and need, so I recommend to
always load this library by saying |\usetikzlibrary{babel}|. For
details on what, exactly, happens with the category codes, see
Section~\ref{section-library-babel}.


\subsubsection{Adding a Background}

By default, pictures do not have any background, that is, they are
``transparent'' on all parts on which you do not draw
anything. You may instead wish to have a colored background behind
your picture or a black frame around it or lines above and below it or
some other kind of decoration.

Since backgrounds are often not needed at all, the definition of
styles for adding backgrounds has been put in the library package
|backgrounds|. This package is documented in
Section~\ref{section-tikz-backgrounds}.


\subsection{Using Scopes to Structure a Picture}

Inside a |{tikzpicture}| environment you can create scopes
using the |{scope}| environment. This environment is available only
inside the |{tikzpicture}| environment, so once more, there is little
chance of doing anything wrong.

\subsubsection{The Scope Environment}

\begin{environment}{{scope}\opt{\oarg{options}}}
  All \meta{options} are local to the \meta{environment
  contents}. Furthermore, the clipping path is also local to the
  environment, that is, any clipping done inside the environment
  ``ends'' at its end.

\begin{codeexample}[]
\begin{tikzpicture}[ultra thick]
  \begin{scope}[red]
    \draw (0mm,10mm) -- (10mm,10mm);
    \draw (0mm,8mm) -- (10mm,8mm);
  \end{scope}
  \draw (0mm,6mm) -- (10mm,6mm);
  \begin{scope}[green]
    \draw (0mm,4mm) -- (10mm,4mm);
    \draw (0mm,2mm) -- (10mm,2mm);
    \draw[blue] (0mm,0mm) -- (10mm,0mm);
  \end{scope}
\end{tikzpicture}
\end{codeexample}

  The following style influences scopes:
  \begin{stylekey}{/tikz/every scope (initially \normalfont empty)}
    This style is installed at the beginning of every scope.
  \end{stylekey}

  The following options are useful for scopes:
  \begin{key}{/tikz/execute at begin scope=\meta{code}}
    This option install some code that will be executed
    at the beginning of the scope. This option must be
    given in the argument of the |{scope}| environment.

    The effect applies only to the current scope, not to subscopes.
  \end{key}
  \begin{key}{/tikz/execute at end scope=\meta{code}}
    This option installs some code that will be executed
    at the end of the  current scope. Using this option multiple times
    will  cause the code to accumulate. This option must also be given
    in the optional argument of the |{scope}| environment.

    Again, the effect applies only to the current scope, not to subscopes.
  \end{key}
\end{environment}

\begin{plainenvironment}{{scope}\opt{\oarg{options}}}
  Plain \TeX\ version of the environment.
\end{plainenvironment}

\begin{contextenvironment}{{scope}\opt{\oarg{options}}}
  Con\TeX t version of the environment.
\end{contextenvironment}


\subsubsection{Shorthand for Scope Environments}

There is a small library that makes using scopes a bit easier:
\begin{tikzlibrary}{scopes}
  This library defines a shorthand for starting and ending |{scope}|
  environments.
\end{tikzlibrary}
When this library is loaded, the following happens: At certain places
inside a \tikzname\ picture, it is allowed to start a scope just using
a single brace, provided the single brace is followed by options in
square brackets:

\begin{codeexample}[]
\begin{tikzpicture}
  { [ultra thick]
    { [red]
      \draw (0mm,10mm) -- (10mm,10mm);
      \draw (0mm,8mm)  -- (10mm,8mm);
    }
    \draw (0mm,6mm) -- (10mm,6mm);
  }
  { [green]
    \draw (0mm,4mm) -- (10mm,4mm);
    \draw (0mm,2mm) -- (10mm,2mm);
    \draw[blue] (0mm,0mm) -- (10mm,0mm);
  }
\end{tikzpicture}
\end{codeexample}

In the above example, |{ [thick]| actually causes a
  |\begin{scope}[thick]| to be inserted, and the corresponding closing
    |}| causes an |\end{scope}| to be inserted.

The ``certain places'' where an opening brace has this special meaning
are the following: First, right after the semicolon that ends a path. Second,
right after the end of a scope. Third, right at the beginning of a
scope, which includes the beginning of a picture. Also note that some
square bracket must follow, otherwise the brace is treated as a normal
\TeX\ scope.


\subsubsection{Single Command Scopes}

In some situations it is useful to create a scope for a single
command. For instance, when you wish to use algorithm graph drawing
in order to layout a tree, the path of the tree needs to be surrounded
by a scope whose only purpose is to take a key that selects a layout
for the scope. Similarly, in order to put something on a background
layer, a scope needs to be created. In such cases, where it will 
cumbersome to create a |\begin{scope}| and |\end{scope}| pair just for
a single command, the |\scoped| command may be useful:

\begin{command}{\scoped\opt{\oarg{options}}\meta{path command}}
  This command works like |\tikz|, only you can use it inside a
  |{tikzpicture}|. It will take the following \meta{path command} and
  put it inside a |{scope}| with the \meta{options} set. The
  \meta{path command} may either be a single command ended by a
  semicolon or it may contain multiple commands, but then they must be
  surrounded by curly braces.
  \begin{codeexample}[]
\begin{tikzpicture}
  \node [fill=white] at (1,1) {Hello world};
  \scoped [on background layer]
    \draw (0,0) grid (3,2);
\end{tikzpicture}
  \end{codeexample}
\end{command}


\subsubsection{Using Scopes Inside Paths}

The |\path| command, which is described in much more detail in later
sections, also takes graphic options. These options are local to the
path. Furthermore, it is possible to create local scopes within a
path simply by using curly braces as in
\begin{codeexample}[]
\tikz \draw (0,0) -- (1,1)
           {[rounded corners] -- (2,0) -- (3,1)}
           -- (3,0) -- (2,1);
\end{codeexample}

Note that many options apply only to the path as a whole and cannot be
scoped in this way. For example, it is not possible to scope the
|color| of the path. See the explanations in the section on paths for
more details.

Finally, certain elements that you specify in the argument to the
|\path| command also take local options. For example, a node
specification takes options. In this case, the options apply only to
the node, not to the surrounding path.



\subsection{Using Graphic Options}
\label{section-graphic-options}

\subsubsection{How Graphic Options Are Processed}

Many commands and environments of \tikzname\ accept
\emph{options}. These options are so-called \emph{key lists}. To
process the options, the following command is used, which you can also
call yourself. Note that it is usually better not to call this command
directly, since this will ensure that the effect of options are local
to a well-defined scope.

\begin{command}{\tikzset\marg{options}}
  This command will process the \meta{options} using the |\pgfkeys|
  command, documented in detail in Section~\ref{section-keys}, with
  the default path set to |/tikz|. Under normal circumstances, the
  \meta{options} will be lists of comma-separated pairs of the form
  \meta{key}|=|\meta{value}, but more fancy things can happen when you
  use the power of the |pgfkeys| mechanism, see
  Section~\ref{section-keys} once more.

  When a pair \meta{key}|=|\meta{value} is processed, the following
  happens:
  \begin{enumerate}
  \item If the \meta{key} is a full key (starts with a slash) it is
    handled directly as described in Section~\ref{section-keys}.
  \item Otherwise (which is usually the case), it is checked whether
    |/tikz/|\meta{key} is a key and, if so, it is executed.
  \item Otherwise, it is checked whether |/pgf/|\meta{key} is a key
    and, if so, it is executed.
  \item Otherwise, it is checked whether \meta{key} is a color and, if
    so, |color=|\meta{key} is executed.
  \item Otherwise, it is checked whether \meta{key} contains a dash
    and, if so, |arrows=|\meta{key} is executed.
  \item Otherwise, it is checked whether \meta{key} is the name of a
    shape and, if so, |shape=|\meta{key} is executed.
  \item Otherwise, an error message is printed.
  \end{enumerate}

  Note that by the above description, all keys starting with |/tikz|
  and also all keys starting with |/pgf| can be used as \meta{key}s in
  an \meta{options} list.
\end{command}


\subsubsection{Using Styles to Manage How Pictures Look}

There is a way of organizing sets of graphic options ``orthogonally''
to the normal scoping mechanism. For example, you might wish all your
``help lines'' to be drawn in a certain way like, say, gray and thin
(do \emph{not} dash them, that distracts). For this, you can use
\emph{styles}.

A style is a key that, when used, causes a set of graphic options to
be processed. Once a style has been defined, it can be used like any
other key. For example, the predefined |help lines| style, which you
should use for lines in the background like grid lines or construction
lines.
\begin{codeexample}[]
\begin{tikzpicture}
  \draw             (0,0) grid +(2,2);
  \draw[help lines] (2,0) grid +(2,2);
\end{tikzpicture}
\end{codeexample}

Defining styles is also done using options. Suppose we wish to define
a style called |my style| and when this style is used, we want the
draw color to be set to |red| and the fill color be set to
|red!20|. To achieve this, we use the following option:
\begin{codeexample}[code only]
my style/.style={draw=red,fill=red!20}
\end{codeexample}

The meaning of the curious |/.style| is the following: ``The key
|my style| should not be used here but, rather, be defined. So, set up
things such that using the key |my style| will, in the following, have
the same effect as if we had written |draw=red,fill=red!20| instead.''

Returning to the help lines example, suppose we prefer blue help
lines. This could be achieved as follows:
\begin{codeexample}[]
\begin{tikzpicture}[help lines/.style={blue!50,very thin}]
  \draw             (0,0) grid +(2,2);
  \draw[help lines] (2,0) grid +(2,2);
\end{tikzpicture}
\end{codeexample}

Naturally, one of the main ideas behind styles is that they can be
used in different pictures. In this case, we have to use the
|\tikzset| command somewhere at the beginning.
\begin{codeexample}[]
\tikzset{help lines/.style={blue!50,very thin}}
% ...
\begin{tikzpicture}
  \draw             (0,0) grid +(2,2);
  \draw[help lines] (2,0) grid +(2,2);
\end{tikzpicture}
\end{codeexample}

Since styles are just special cases of |pgfkeys|'s general style
facility, you can actually do quite a bit more. Let us start with
adding options to an already existing style. This is done using
|/.append style| instead of |/.style|:
\begin{codeexample}[]
\begin{tikzpicture}[help lines/.append style=blue!50]
  \draw             (0,0) grid +(2,2);
  \draw[help lines] (2,0) grid +(2,2);
\end{tikzpicture}
\end{codeexample}
In the above example, the option |blue!50| is appended to the style
|help lines|, which now has the same effect as
|black!50,very thin,blue!50|. Note that two colors are set, so the
last one will ``win.'' There also exists a handler called
|/.prefix style| that adds something at the beginning of the style.

Just as normal keys, styles can be parameterized. This means that you
write \meta{style}|=|\meta{value} when you use the style instead of
just \meta{style}. In this case, all occurrences of |#1| in
\meta{style} are replaced by \meta{value}. Here is an example that
shows how this can be used.

\begin{codeexample}[]
\begin{tikzpicture}[outline/.style={draw=#1,thick,fill=#1!50}]
  \node [outline=red]  at (0,1) {red};
  \node [outline=blue] at (0,0) {blue};
\end{tikzpicture}
\end{codeexample}

For parameterized styles you can also set a \emph{default} value using
the |/.default| handler:

\begin{codeexample}[]
\begin{tikzpicture}[outline/.style={draw=#1,thick,fill=#1!50},
                    outline/.default=black]
  \node [outline]      at (0,1) {default};
  \node [outline=blue] at (0,0) {blue};
\end{tikzpicture}
\end{codeexample}

For more details on using and setting styles, see also
Section~\ref{section-keys}.
