\section{Hints for ``Applications (I): Introduction to Cryptography'' exercises}
\label{sec:Cryptography:Hints}

\noindent Exercise \ref{exercise:Cryptography:mat1}: Prove by contradiction.  If $A$ has an inverse, then there exists a matrix $B$ such that $AB = I$.  Take the determinant of this equation, and show that it produces a contradiction to the fact that $(a \odot d)  \ominus (b \odot c)$ has no inverse.

\noindent Exercise \ref{exercise:Cryptography:primes}: It is possible to list all of the numbers between $1$ and $pq$ which are \emph{not} relatively prime to $pq$.

\noindent Exercise \ref{exercise:Cryptography:power}(c): Remember your exponent rules!

\noindent Exercise \ref{exercise:Cryptography:brute}: Consider the case where $n$ is the product of two \emph{equal} factors:  $n=a \cdot a$. Then how large must $a$ be?  Compare this with the general case where $n$ is the product of two unequal factors:  $n = xy$. Show that the \emph{smaller} of these two factors must be smaller than $a$.

\noindent Exercise \ref{exercise:Cryptography:Fermat}: Suppose $n=ab$. Choose $a$ to be the smaller factor.  Write $a = x-y$ and $b = x+y$, and solve for $x$ and $y$. To finish the proof, you need to prove that $x$ and $y$ must both be integers.

\noindent Exercise \ref{exercise:Cryptography:smallest_value}: Solve for $x$. What value of $y$ makes $x$ as small as possible?

\noindent Exercise \ref{exercise:Cryptography:FermatEfficient}(a): Prove by contradiction.~~(b): Write $m = 2k+1$.~~(d): Use part (c), part (b), and the distributive law.~~(e): This is similar to part(b).