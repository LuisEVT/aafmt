\section{Solutions for ``Induction''}
\subsection{Proofs for Section~\ref{sec:Induction:BasicInduction}}
\subsubsection{Induction proofs, type I: sum/product formulas}


\textbf{Proof of (a):}
We seek to show that, for all $n\in\NN$,  
\[
\sum_{i=1}^n i(i+1)=\frac{n(n+1)(n+2)}{3}. 
\tag{$*$}
\]

\textbf{Base case:} When $n=1$, the left side of ($*$) is $1\cdot(1+1) =2$,
and the
right side is $1\cdot(1+1)(1+2)/3=2$, so both sides are equal and ($*$) is
true for $n=1$.

\textbf{Induction step:} Let $k\in\NN$ be given and suppose 
($*$) is true for $n=k$. Then
\begin{align*}
\sum_{i=1}^{k+1}i(i+1)
&=
\sum_{i=1}^{k}i(i+1)
+(k+1)(k+2)
\\
&=\frac{k(k+1)(k+2)}{3} 
+(k+1)(k+2)
\quad \text{(by induction hypothesis)}
\\
&=\frac{(k+1)(k+2)(k+3)}{3}.
\end{align*}
Thus, ($*$) holds for $n=k+1$, and the proof of the induction step is complete. 

\textbf{Conclusion:} By the principle of induction,  it follows that
($*$) is true for all $n\in\NN$.  
\medskip
We furnish no proofs for (b) and (c).  Note that (b) is a special case of (c).

\medskip


\textbf{Proof of (d):}
We seek to show that, for all $n\in\NN$,
\[
\sum_{i=0}^n i! i
= (n+1)!-1. 
\tag{$*$}
\]

\textbf{Base case:} When $n=1$, the left side of ($*$) is $0+1 \cdot 1! =1$,
and the
right side is $(1+1)!-1=1$, so both sides are equal and ($*$) is
true for $n=1$.

\textbf{Induction step:} Let $k\in\NN$ be given and suppose 
($*$) is true for $n=k$. Then
\begin{align*}
\sum_{i=1}^{k+1}i\cdot i!
&=
\sum_{i=1}^{k}i\cdot i! + (k+1)(k+1)!
\\
&= (k+1)!-1 + (k+1)(k+1)!
\quad \text{(by induction hypothesis)}
\\
&=(k+1)!(k+2)-1
\\
&=(k+2)!-1.
\end{align*}
Thus, (2) holds for $n=k+1$, and the proof of the induction step is complete. 

\textbf{Conclusion:} By the principle of induction, 
($*$) is true for all $n\in\NN$.  

\subsubsection{Induction proofs, type II: Inequalities}

We will give detailed proofs for (c), (d), (e). The other inequalities can
be proved similarly.

\medskip

\textbf{Proof of (c):}
A direct check of the inequality for the first few values of $n$ shows
that the left-right pairs in the stated inequality 
are $(1,2),(2,4),(6,8),(24,16),(120,32)$. Thus, the inequality
fails for $n=1,2,3$, but holds for $n=4,5$. This suggests that it indeed
holds for all $n$ from $4$ onwards. We will prove this by induction,
i.e., we will show that 
\[
n!>2^n
\tag{$*$}
\]
holds for all $n\ge4$.

\textbf{Base case:} For $n=4$, the left and right sides of ($*$) are 
$24$ and $16$, respectively, so ($*$) is true in this case. 

\textbf{Induction step:} Let $k\ge4$ be given and suppose 
($*$) is true for $n=k$. Then
\begin{align*}
(k+1)!&=k!(k+1)
\\
&>2^k(k+1)
\quad \text{(by induction hypothesis)}
\\
&\ge 2^k\cdot 2
\quad \text{(since $k\ge4$ and so $k+1\ge2$))}
\\
&=2^{k+1}.
\end{align*}
Thus, ($*$) holds for $n=k+1$, and the proof of the induction step is complete. 

\textbf{Conclusion:} By the principle of induction, 
it follows that ($*$) is true for all $n\ge4$.  

\bigskip

\textbf{Proof of (d) and (e):}
We will prove that for any real number $x>-1$  
\[
(1+x)^n\ge 1+nx.
\tag{$*$}
\]
holds for any $n\in\NN$.
This simultaneously proves both statements (d) and (e): (e) corresponds
to the case $x>0$, while (d) corresponds to the case $-1<x<0$ (with
$x'=-x$ in place of $x$).

\textbf{Base case:} For $n=1$, the left and right sides of ($*$) are 
both $1+x$, so ($*$) holds.

\textbf{Induction step:} Let $k\in\NN$ be given and suppose ($*$) is true
for $n=k$ and any real number $x>-1$.   We seek to show that ($*$) holds
for $n=k+1$ and any real number $x>-1$.

Let $x>-1$ be given. Then
\begin{align*}
(1+x)^{k+1}
&=(1+x)^k(1+x)
\\
&\ge (1+kx)(1+x) 
\quad\text{(by ind. hyp. and since $x>-1$ and thus $(1+x)>0$)}
\\
&=1+(k+1)x +kx^2
\quad\text{(by algebra)}
\\
&\ge 1+(k+1)x 
\quad\text{(since $kx^2\ge0$)}.
\end{align*}
Hence ($*$) holds for $n=k+1$, and the proof of the induction step is
complete.

\textbf{Conclusion:} By the principle of induction, it follows that ($*$)
holds for all $n\in\NN$.

\subsubsection{Induction proofs, type III:
Extension of theorems from $2$ variables to $n$ variables}

We will prove by induction on $n$ the following statement: 

\[
\tag*{$P(n)$:}
\parbox{5in}{\slshape For all real numbers $a_i$ and $b_i$ ($i=1,\dots,n$)
such that $a_i\le b_i$ for all $i$ we
have}
\]
\[
\sum_{i=1}^na_i\le \underline{~~~~~~~}. 
\tag{$*$}
\]
(Note that the condition ``for all real numbers $a_i$ and $b_i$''
must be part of the induction statement we seek to prove.)

\textbf{Base case:} For $n=1$, the 
left and right sides are \underline{~~~} and \underline{~~~~}, respectively, and the
inequality ($*$) therefore follows from our hypothesis that
$a_i\le b_i$ for all $i=1,\dots,n$.  Hence \underline{~~~~} is true.

\textbf{Induction step:}
\begin{itemize}
\item Let $k\ge 1$, and suppose $P(k)$ is true, i.e.,
suppose that 
for $n=$\underline{~~~} and any choice of real numbers 
$a_1,\dots,a_k$ and $b_1,\dots,b_k$ 
satisfying \underline{~~~~} for each $i$, the inequality ($*$) holds.
\item
We seek to show that \underline{~~~} is true, i.e., that 
for $n=$\underline{~~~~}
any choice of real numbers $a_1,\dots,a_{k+1}$ and $b_1,\dots,b_{k+1}$
satisfying \underline{~~~~~} for each $i$, the inequality ($*$) holds.
\item
Let $a_1,\dots,a_{k+1}$ and $b_1,\dots,b_{k+1}$ be given real numbers
such that \underline{~~~~~} for each $i$.
\item Then
\begin{align*}
\sum_{i=1}^{k+1}a_i
&=\underline{~~~~~} + a_{k+1}
\\
&\le \underline{~~~~~~} + a_{k+1}
\quad \text{(by induction hypothesis applied to $a_1,\dots a_k$)}
\\
&\le \sum_{i=1}^{k}b_i + \underline{~~~~~}
\quad \text{(by assumption $a_{k+1}\le b_{k+1}$)}
\\
&=\sum_{i=1}^{k+1}b_i.
\end{align*}
\item 
Thus, ($*$) holds for $n=$\underline{~~~~} and the given numbers $a_1,\dots,a_{k+1}$ 
and $b_1,\dots,b_{k+1}$.
\item Since the $a_1,\dots,a_{k+1}$ 
and $b_1,\dots,b_{k+1}$ were arbitrary
real numbers satisfying $a_i\le b_i$ for each $i$,
we have obtained statement $P(k+1)$, 
and the proof of the induction step is complete.
\end{itemize}


\textbf{Conclusion:} By the principle of induction, 
it follows that $P(n)$ is true for all $n\in\NN$.  
  {\hspace\fill$\square$\par\medskip}

We seek to prove by induction on $n$ the following statement: 
\[
\tag*{$P(n)$:}
\parbox{5in}{\slshape For all real numbers $x_1,\dots,x_n$ 
we have}
\]
\[
\left|\sin\left(\sum_{i=1}^n x_i\right)\right|
\le \sum_{i=1}^n\left|\sin x_i\right|.
\tag{$*$}
\]

The key to the argument is the trig identity
\[
\sin(\alpha+\beta)=\sin\alpha \cos \beta+\sin\beta\cos\alpha,
\]
which is valid for any real $\alpha$ and $\beta$. Since $|\cos x|\le 1$,
this identity implies, via the triangle inequality,
\begin{align*}
\tag{$**$}
|\sin(\alpha+\beta)|
&\le|\sin\alpha \cos \beta|+|\sin\beta\cos\alpha|
\\
&\le|\sin\alpha|+|\sin\beta|.
\end{align*}
The inequality $(**)$ is the case $n=2$ of the statement $(*)$ 
we seek to prove, and will be needed in the induction proof. (One could
also use it as the base case of an induction proof that starts with $n=2$, 
but it is easier to start the induction with $n=1$, where the base case
is trivial.)

\bigskip


\textbf{Base case:} For $n=1$, the 
left and right sides of ($*$) are both equal to $|\sin x_1|$, so  
($*$) holds trivially in this case. Hence $P(1)$ is true.

\textbf{Induction step:}
\begin{itemize}
\item Let $k\ge 1$, and suppose $P(k)$ is true, i.e.,
suppose that ($*$) holds for $n=k$ and any choice of real numbers 
$x_1,\dots,x_k$.
\item
We seek to show that $P(k+1)$ is true, i.e., that for
any choice of real numbers $x_1,\dots,x_{k+1}$
the inequality ($*$) holds. 
\item
Let $x_1,\dots,x_{k+1}$  be given real numbers. 
\item Then
\begin{align*}
\left|\sin\left(\sum_{i=1}^{k+1} x_i\right)\right|
&=\left|\sin\left(\left(\sum_{i=1}^{k} x_i\right) + x_{k+1}\right)\right|
\\
&\le \left|\sin\left(\sum_{i=1}^{k} x_i\right)\right|
+ \left|\sin x_{k+1}\right|
\quad \text{(by ($**$)
with $\alpha=\sum_{i=1}^k x_i$ and $\beta=x_{k+1}$)}
\\
&\le \sum_{i=1}^k\left|\sin x_i\right|
+ \left|\sin x_{k+1}\right|
\quad \text{(by induction hypothesis applied to $x_1,\dots, x_k$)}
\\
&=\sum_{i=1}^{k+1}\left|\sin x_i\right|.
\end{align*}
\item 
Thus, ($*$) holds for $n=k+1$ and the given numbers $x_1,\dots,x_{k+1}$.
\item
Since the $x_1,\dots,x_{k+1}$ were
arbitrary real numbers, we have obtained statement $P(k+1)$, and 
proof of the induction step is complete.
\end{itemize}

\textbf{Conclusion:} By the principle of induction, 
it follows that $P(n)$ is true for all $n\in\NN$.  

We seek to prove by induction on $n$ the following statement: 
\[
\tag*{$P(n)$:}
\parbox{5in}{\slshape For all sets $A_1,\dots,A_n$  
we have}
\]
\[
\left(A_1\cup \dots \cup A_n\right)^c
=A_1^c\cap\dots \cap A_n^c.
\tag{$*$}
\]

The key to the argument is two set version of De Morgan's Law:
\[
(A\cup B)^c = A^c\cap B^c,
\tag{$**$}
\]
which holds for any sets $A$ and $B$.


\textbf{Base case:} For $n=1$, the 
left and right sides of ($*$) are both equal to $A_1^c$,
so ($*$) holds trivially in this case. Hence $P(1)$ is true.
\iffalse
Though not necessary, we can also easily verify the next case,
$n=2$: In this case, the left and right sides of ($*$) are $(A_1\cup A_2)^c$
and $A_1^c\cap A_2^c$, respectively, so the identity is just the two set
version of De Morgan's Law, i.e., ($**$) with $A=A_1$ and $B=A_2$.
\fi

\textbf{Induction step:} 
\begin{itemize}
\item Let $k\ge 1$, and suppose $P(k)$ is true, i.e.,
suppose that ($*$) holds for $n=k$ and any sets $A_1,\dots,A_k$.
\item
We seek to show that $P(k+1)$ is true, i.e., that for
any sets $A_1,\dots,A_{k+1}$, ($*$) holds. 
\item
Let $A_1,\dots,A_{k+1}$  be given sets.
\item
Then
\begin{align*}
\left(A_1\cup \dots \cup A_{k+1}\right)^c
&=\left(\left(A_1\cup \dots \cup A_{k}\right)\cup A_{k+1}\right)^c
\\
&=\left(A_1\cup \dots \cup A_{k}\right)^c\cap A_{k+1}^c
\quad\\
& \qquad \qquad\qquad \text{(by ($**$) with $A=(A_1\cup \dots \cup A_k)$ and 
$B=A_{k+1}$)}
\\
&=\left(A_1^c\cap\dots \cap A_k^c\right)\cap A_{k+1}^c\\
& \qquad \qquad\qquad \text{(by induction hypothesis applied to $A_1,\dots, A_k$)}
\\
&=A_1^c\cap\dots \cap A_k^c\cap A_{k+1}^c.
\end{align*}
\item
Thus, ($*$) holds for $n=k+1$
and the given sets $A_1,\dots,A_{k+1}$.
\item 
Since the $A_1,\dots,A_{k+1}$ were
arbitrary sets, we have obtained statement $P(k+1)$, and the 
proof of the induction step is complete.
\end{itemize}

\textbf{Conclusion:} By the principle of induction, 
it follows that $P(n)$ is true for all $n\in\NN$.  
 

\subsection{Proofs for Section~\ref{sec:Induction:StrongInductionPractice}}
\subsubsection{``Strong induction and recurrences''}

\noindent
\textbf{Proof of (a):}

We seek to show that, for all $n\in\NN$, 
\[
\tag{$*$}
\sum_{i=1}^n F_i=F_{n+2}-1.
\]

\textbf{Base case:} When $n=1$, the left side of ($*$) is $F_1 =1$,
and the
right side is $F_3-1=2-1=1$, so both sides are equal and ($*$) is
true for $n=1$.

\textbf{Induction step:} Let $k\in\NN$ be given and suppose 
($*$) is true for $n=k$. Then
\begin{align*}
\sum_{i=1}^{k+1}F_i
&=
\sum_{i=1}^{k}F_i
+F_{k+1}
\\
&=F_{k+2}-1 + F_{k+1}
\quad \text{(by ind. hyp. $(*)$ with $n=k$)}
\\
&=F_{k+3}-1 
\quad \text{(by recurrence for $F_n$)}
\end{align*}
Thus, ($*$) holds for $n=k+1$, and the proof of the induction step is complete. 

\textbf{Conclusion:} By the principle of induction,  it follows that
($*$) is true for all $n\in\NN$.  

\bigskip

\textbf{Remark:} Here standard induction was sufficient, since we were
able to relate the $n=k+1$ case directly to the $n=k$ case, in the
same way as in the induction proofs for summation formulas like 
$\sum_{i=1}^n i=n(n+1)/2$.  Hence, a single base case was sufficient.

(For convenience, we define $F_0=0$; with this definition, the recurrence
relation $F_n=F_{n-1}+F_{n-2}$ holds for all $n\ge2$ and the above matrix
is well defined for all $n\ge1$.)

\textbf{Base case:} When $n=1$, 
the four entries of the matrix on the right are $F_2=1$, $F_1=1$, $F_1=1$, 
and $F_0=0$, so ($*$) holds in this case. 

\newcommand{\mat}[4]{\begin{pmatrix}#1 & #2 \\ #3 & #4\end{pmatrix}}
\textbf{Induction step:} Let $k\in\NN$ be given and suppose 
($*$) holds for $n=k$. Then
\begin{align*}
\mat 1110^{k+1}
&=\mat 1110^k\mat1110
\\
&=\mat{F_{k+1}}{F_k}
{F_k}{F_{k-1}}\mat1110
\quad\text{(by $(*)$ with $n=k$)}
\\
&=\mat{F_{k+1}+F_k}{F_{k+1}}{F_{k}+F_{k-1}}{F_k}
\quad \text{(by matrix multiplication)}
\\
&=\mat{F_{k+2}}{F_{k+1}}{F_{k+1}}{F_k}
\quad \text{(by recurrence for $F_n$)}.
\end{align*}
Thus, ($*$) holds for $n=k+1$, and the proof of the induction step is complete. 

\textbf{Conclusion:} By the principle of induction,  it follows that
($*$) is true for all $n\in\NN$.  

\textbf{Proof of (b):}

We will use induction to show that the following statement 
holds for all $m\in\NN$:
\[
\tag*{$P(m):$}
\text{
$F_{3m-2}$ and $F_{3m-1}$ are odd,
and $F_{3m}$ is even.}
\]

\textbf{Base case:} When $m=1$, the three Fibonacci numbers appearing in 
$P(m)$ are $F_1=1$, $F_2=1$, and $F_3=2$, and thus are of the required
parity. Hence $P(1)$ is true.

\textbf{Induction step:} Let $k\in\NN$ be given and suppose 
$P(m)$ is true for $m=k$. Thus, $F_{3k-2}$ and $F_{3k-1}$ are odd and 
$F_{3k}$ is even.

By the recurrence for $F_n$, we have $F_{3k+1}=F_{3k}+F_{3k-1}$. Hence
$F_{3k+1}$ is the sum of an even number $F_{3k}$, and an odd number,
$F_{3k-1}$, and therefore odd. 

Similarly, $F_{3k+2}=F_{3k+1}+F_{3k}$, so $F_{3k+2}$ is the sum of an odd
number, $F_{3k+1}$, and an even number, $F_{3k}$, and hence odd.

Finally, $F_{3k+3}=F_{3k+2}+F_{3k+1}$, so $F_{3k+3}$ is the sum of two odd
numbers and hence even. 

Altogether, we have shown that, of the three numbers
$F_{3k+1},F_{3k+2},F_{3k+3}$, the first two are odd and the last one is
even. Thus $P(m)$ holds for $m=k+1$, and the proof of the 
induction step is complete. 

\textbf{Conclusion:} By the principle of induction,  it follows that
$P(m)$ is true for all $m\in\NN$.  

\bigskip

\textbf{Alternative argument:} The above proof lumps together groups of
three consecutive Fibonacci numbers and establishes the desired parity
properties simultaneously for all three numbers.  
Alternatively, one can treat the sequences $F_{3m-2}$, $F_{3m-1}$, and
$F_{3m}$ separately using  the following identity, valid for 
all $n\ge4$:
\[
F_{n}=F_{n-1}+F_{n-2}=(F_{n-2}+F_{n-3})+F_{n-2}=2F_{n-2}+F_{n-3}.
\]
It follows from this identity that if $F_{n-3}$ is odd, then so is $F_{n}$,
and if $F_{n-3}$ is even, then so is $F_n$.  Using 
induction with $n=1$ as the base case then shows that the numbers
$F_1=1,F_4,F_7,\dots$ are all odd.  With $n=2$ as base case one gets that 
$F_2=1,F_5,F_8,\dots$ are all odd, and taking $n=3$ as base case 
shows that $F_3=2,F_6,F_9,\dots$ are all even.


\textbf{Proof of (c):}
We will prove $(*)$ by strong induction.

\textbf{Base step:} 
For $n=1,2,3$, $T_n$ is equal to $1$, whereas the
right-hand side of ($*$) is equal to $2^1=2$, $2^2=4$, and $2^3=8$,
respectively. Thus, ($*$) holds for $n=1,2,3$.

\textbf{Induction step:} Let $k\ge3$ be given and suppose 
($*$) is true for all $n=1,2,\dots,k$. Then
\begin{align*}
T_{k+1}&=T_{k}+T_{k-1}+T_{k-2}
\quad \text{(by recurrence for $T_n$)}
\\
&<2^{k}+2^{k-1}+2^{k-2} \quad \text{(by strong ind. hyp. ($*$)
with $n=k$, $k-1$, and $k-2$)}
\\
&=2^{k+1}\left(\frac12+\frac14+\frac18\right)
\\
&=2^{k+1}\frac{7}{8}
<2^{k+1}.
\end{align*}
Thus, ($*$) holds for $n=k+1$, and the proof of the induction step is complete. 

\textbf{Conclusion:} By the strong induction principle,  it follows that
($*$) is true for all $n\in\NN$.  

\subsubsection{Strong induction and representation problems}

\textbf{Proof of (a):}

A quick direct check shows that the 
positive numbers $n<15$ that have a
representation $n=3x+7y$ with $x,y\in\NN\cup \{0\}$) are exactly
$3,6,7,9,10,12,13,14$.  We now use strong induction to show that from
$12$ onwards every integer has a representation in the above form.
In other words, we will prove that the following 
statement holds for all $n\ge12$:
\[
\text{$n$ has a representation $(*)$ $n=3x+7y$ with $x,y\in\NN\cup \{0\}$}
\tag{$P(n)$}
\]
\textbf{Base case:} For $n=12,13,14$, the representations $12=3\cdot 4$,
$13=3\cdot 2+7$ and $14=7\cdot 2$ show that $P(n)$ is true.

\textbf{Induction step:} 
Let $k\ge14$ be given and suppose $P(k')$ is true for all $k'$ with 
$k'=12,13,\dots,k$, i.e., suppose that all such $k'$ have a representation in
the form $(*)$. We seek to show that $k+1$ also has a representation of
this form. 

Write  $k+1=3+k'$, so that $k'=k-2$. Note that $k'\le k$ and  also
$k'\ge 12$ since we assumed $k\ge14$. Thus, we can apply the strong
induction hypothesis to $k'$ and obtain a representation 
\[
k'=3x+7y,
\]
where $x,y\in\NN\cup\{0\}$. Adding $3$ to both sides of this
representation, we get
\[
k+1=k'+3= 3x+7y+3=3(x+1)+7y,
\]
which is a representation of the desired form for $k+1$.
Hence $P(k+1)$ is true, and the proof of the induction step is complete.

\textbf{Conclusion:} By the strong induction principle, 
it follows that $P(n)$  is true for all $n\ge12$.

\bigskip

\textbf{Remark:} Note that, in the induction step, in order to be able
to apply the induction hypothesis with $k'=k-2$ we need to ensure that 
$k'$ is at least $12$. This in turn requires $k$ 
to be at least $14$ in the induction step,
and the cases $k=12,13,14$ to be treated as base
cases.


\textbf{Proof of (b):}
We will prove by strong induction that the following statement  
holds for all $n\in\NN$:

\[\text{$n$ has a representation}(*) ~~n=2^{i_1}+\dots + 2^{i_h} \]
\[ \text{with distinct integers $i_1,\dots,i_h\in\NN\cup\{0\}$.}
\tag{$P(n)$}
\]
\textbf{Base case:} The integer $n=1$ has the representation $1=2^0$,
which is of the desired form. Hence $P(n)$ holds for $n=1$.


\textbf{Induction step:} 
Let $k\ge1$ be given and suppose $P(n)$ is true for all positive
integers $ n\le k$, i.e., suppose that all such $n$ have a
representation in the form $(*)$. We seek to show that $n=k+1$ also has a
representation of this form. 

Let $2^m$ be the largest (integer) power of $2$ that satisfies $2^m\le k+1$.

If $2^m=k+1$, then $k+1$ has a representation of the desired form
(namely as a sum of a single power of $2$, $2^m$), and we are done.

If $2^m<k+1$, we let $k'=k+1-2^m$.  Since $2^m\ge1$
and $2^m<k+1$, $k'$ is an integer with $1\le k'\le k$. 
Hence we can apply the strong induction hypothesis to $k'$ and obtain a
representation of $k'$ as a sum of distinct powers of $2$.

Adding $2^m$ to this representation gives a representation of $k+1=k'+2^m$
as a sum of powers of $2$. To complete the proof of the induction step,
we still need to show that the powers of $2$ involved here are distinct.

Since the powers of $2$ representing $k'$ were already distinct, it
suffices to show that the added power $2^m$ cannot occur among the
powers in the representation for $k'$. 
To do this, we exploit the fact that $2^m$ was chosen as the largest
power of $2$ below $k+1$. Thus, we have
\[
2^m< k+1 < 2^{m+1}.
\]
Subtracting $2^m$ from both sides, we get 
\[
0< k+1-2^m<2^{m+1}-2^m=2^m,
\]
and since $k'=k+1-2^m$, it follows that $k'$ is strictly less than
$2^m$, and so $2^m$ cannot occur in the representation for $k'$.
This is what we wanted to show.

Thus, the representation for $k+1$ that we obtained is indeed a
representation of the desired form and the proof of the 
induction step is complete.

\textbf{Conclusion:} By the strong induction principle, 
it follows that $P(n)$  is true for all $n\ge1$.

\bigskip

\textbf{Remarks:} The argument used above in the induction step is called a
``greedy'' algorithm: In constructing a binary representation for $k+1$, one
starts  out by using the largest possible ``building block'' (namely, the
largest power of $2$ that is $\le k+1$), and then uses the strong induction
hypothesis to make up for the left over part. 


An alternative, but less flexible, approach to the induction step is as
follows: If $k+1$ is even, then $k+1=2k'$, where $k'$ is an integer in
the range $1\le k'\le k$. By the strong induction hypothesis $k'$
has a representation as sum of distinct powers of $2$, and multiplying this
representation by $2$ gives a representation of the desired form for $k+1$. 
If $k+1$ is odd, a similar argument based on the representation $k+1=2k'+1$
yields the same conclusion. 

The latter approach, however, relies heavily on specific arithmetic properties
of the powers of $2$ and does not generalize to other sequences like Fibonacci
numbers or factorials. By contrast, the ``greedy'' approach is one that can be
used for many representation problems and, in fact, is the standard way to
handle such problems.

\bigskip

\textbf{Uniquess of representation:} The above argument proves only the
\emph{existence} of a representation, not its uniqueness.  One way to prove the
uniqueness is by contradiction: Assume there are positive integers with
multiple representations, let $n$  be the smallest of these exceptional
integers, and derive a contradiction from this assumption.

Another way to prove uniquess is to incorporate the uniqueness claim into the
statement $P(n)$ to be proved. The strengthened statement requires an additional
argument in the induction step showing that uniqueness holds for $k+1$,
provided it holds for all $k'\le k$. This is not difficult; the key observation
is that any representation of $k+1$ must necessarily involve the power $2^m$ 
defined above.

\textbf{Proof of (c):}
We will prove by strong induction that the following statement  
holds for all $n\in\NN$.
\[
\text{$n$ has a representation 
$(*)$ $\sum_{i=1}^rd_ii!$
with $d_i\in\{0,1,\dots,i\}$.}
\tag{$P(n)$}
\]
\textbf{Base case:} The integer $n=1$ has the representation $1=1\cdot 1!$,
which is of the desired form. Hence $P(n)$ holds for $n=1$.


\textbf{Induction step:} 
Let $k\ge1$ be given and suppose $P(n)$ is true for all positive
integers $ n\le k$, i.e., suppose that all such $n$ have a
representation in the form $(*)$. We seek to show that $n=k+1$ also has a
representation of this form. 

Let $r$ be the \emph{largest} integer such that $r!\le k+1$; i.e., $r$ is the
unique integer for which  
\[
r!\le k+1 < (r+1)!.
\tag{1}
\]

If $r!=k+1$, then $k+1$ has a representation of the desired form,
and we are done.

If $r!<k+1$, we let $k'=k+1-r!$.  Since $1\le r!<k+1$, 
$k'$ is an integer in the range $1\le k'\le k$. 
Hence we can apply the strong induction hypothesis to $k'$ and obtain a
representation of $k'$ as a finite sum of terms $d_ii!$, with ``digits'' 
$d_i$ in the range $0\le d_i\le i$.  

Adding $r!$ to this representation gives a representation of $k+1=k'+r!$
as a sum of factorials.  To complete the induction step, we need to make sure
that this new representation still satisfies the constraints $0\le d_i\le i$ 
on the digits. 

If $r!$ does not occur in the representation of $k'$, this is clearly the case.

If $r!$ does occur in the representation of $k'$ with an associated
``digit'' $d_r$ satisfying $d_r\le r-1$, 
then adding $r!$ to this representation gives a
representation with $d_r$ replaced by $d_r+1$ and all other digits unchanged, 
and since $d_r\le r-1$, the new digit $d_r+1$ satisfies the required
constraint, $d_r+1\le r$.

It remains to consider the case when $k'$ has a representation involving $r!$
in which the associated digit is maximal, i.e., $d_r=r$. But then 
\[
k+1=k'+r!\ge d_rr!+r! =r\cdot r! + r!=(r+1)!,
\]
so $(r+1)!\le k+1$, contradicting (1). Therefore this case is impossible. 

Hence, in each case we have obtained a representation of $k+1$ of the 
desired form and the proof of the 
induction step is complete.

\textbf{Conclusion:} By the strong induction principle, 
it follows that $P(n)$  is true for all $n\ge1$.

\subsection{Proofs for Section~\ref{sec:Induction:NonFormulaProofs}}

\textbf{Proof of (a):}
We use a variation of the above argument (showing that an
$n$-element set has $2^n$ subsets).  For brevity, we call a subset with an odd
number of elements an ``odd subset'', and a subset with an even number of
elements an ``even subset.'' 

Let $P(n)$ denote the statement that \textbf{any set with $n$ elements has
$2^{n-1}$ odd subsets and $2^{n-1}$ even subsets.}
We use induction to show that $P(n)$ holds for all $n\in\NN$.

\textbf{Base case:} A $1$-element set $A=\{a_1\}$ has exactly 
one even subset, the empty set $\emptyset$ (since the empty set has $0$
elements, and $0$ is an even number), and one odd subset, $\{a_1\}$, so $P(1)$
is true.

\textbf{Induction step:} 
Let $k\in\NN$ be given and suppose 
$P(k)$ is true, i.e., that any $k$-element set has $2^{k-1}$ even subsets and
$2^{k-1}$ odd subsets. 
We seek to show that $P(k+1)$  is true as well, i.e., that
any $(k+1)$-element set has $2^{k}$ even subsets and $2^k$ odd subsets.

Let $A$ be a set with $(k+1)$ elements.  
Choose an element $a$ in $A$, and set $A'=A-\{a\}$. 

We again classify the subsets of $A$ into two types: (I) subsets that do
\emph{not} contain $a$, and (II) subsets that do contain $a$.
The subsets of type (I) are exactly the subsets of the set
$A'$. Since $A'$ has $k$ elements, the induction
hypothesis can be applied to this set and we get that there are 
$2^{k-1}$ even subsets and $2^{k-1}$ odd subsets of type (I).

The subsets of type (II) are exactly the sets of the form $B=B'\cup
\{a\}$, where $B'$ is a subset of $A'$, and hence are in one-to-one
correspondence with subsets $B'$ of $A'$.  Moreover, $B$ is
an odd subset of
$A$ if and only if the associated set $B'$ is an even subset of $A'$,
and an even subset of
$A$ if and only if the associated set $B'$ is an odd subset of $A'$.
By the
induction hypothesis there are $2^{k-1}$ even subsets of $A'$, and $2^{k-1}$
odd subsets of $A'$.   Hence there are $2^{k-1}$ odd subsets of type (II), and
$2^{k-1}$ even subsets of type (II).

Since there are $2^{k-1}$ even subsets of each of the types (I) and
(II), the total number of even subsets of  $A$ is $2^{k-1}+2^{k-1}=2^{k}$. 
Similarly, the total number of odd subsets of $A$ is $2^{k-1}+2^{k-1}=2^k$.

Since $A$ was an arbitrary $(k+1)$-element set, we have proved that any
$(k+1)$-element set has $2^{k}$ even subsets and $2^k$ odd 
subsets. Thus $P(k+1)$ is true,
completing the induction step. 

\textbf{Conclusion:}
By the principle of induction, it follows that
$P(n)$  is true for all $n\in\NN$.

\textbf{Proof of (b):}

For brevity, we call a set of lines \emph{generic} if it satisfies the
conditions in the statement, namely that no two lines are parallel, and no
three lines intersect at the same point.

Let $P(n)$ denote the statement that 
\textbf{the number of regions created by $n$ generic lines in the plane is
$1+\frac{n(n+1)}{2}$}. 
We will use induction to show that $P(n)$ holds for all $n\in\NN$.


\textbf{Base case:} A single line divides the plane into $2$ regions. Since 
$1+1(1+1)/2=2$, this proves $P(1)$.
for $n=1$. 

\textbf{Induction step:} Let $k\in\NN$ be given, and suppose $P(n)$
holds for $n=k$, i.e., suppose that any $k$ generic lines in the plane create
$1+k(k+1)/2$ regions.  

Let $k+1$ lines $L_1,L_2,\dots,L_{k+1}$ be given that are generic in the above
sense.  Then the first $k$ lines, $L_1,\dots,L_k$ are also generic and, by the
induction hypothesis, these $k$ lines divide the plane into $1+k(k+1)/2$
regions.

Now consider the line $L_{k+1}$.  By the ``generic'' property, this line
 intersects each of the lines $L_1,\dots,L_k$ at exactly one point,
and the $k$ intersection points are all distinct and hence divide $L_{k+1}$
into $k+1$ segments.  Each of these segments divides one of the regions created
by the first $k$ lines into two parts, and hence increases the region count by
$1$.  Since there are $k+1$ such segments, the added line $L_{k+1}$ increases
the region count by $k+1$. Thus the total number of regions created by
the lines $L_1,\dots,L_{k+1}$ is  
\[
1+\frac{k(k+1)}{2}+k+1=\frac{k(k+1)+2k+4}{2}=1+\frac{(k+1)(k+2)}{2},
\]
which is the desired formula for the number of regions created by $k+1$
lines.  Hence $P(k+1)$ holds, and the proof of the induction step is
complete.

\textbf{Conclusion:} By the principle of induction, $P(n)$ holds for all
$n\in\NN$.

\textbf{Proof of (c):}

Let $P(n)$ denote the statement that \textbf{the sum of the interior
angles in an $n$-sided polygon is $(n-2)\pi$.}
We will use induction to show that $P(n)$ holds for 
all integers $n\ge3$.


\textbf{Base case:} The sum of the angles in a triangle is $\pi$, which
agrees with the formula 
$(n-2)\pi$ when $n=3$. Thus the statement $P(n)$
holds for $n=3$. 

\textbf{Induction step:} Let $k\in\NN$ with $k\ge 3$ be given, and
assume $P(n)$ holds for $n=k$, i.e., suppose that, for any $k$-sided polygon,
the sum of the interior angles is $(k-2)\pi$. 


Let $P$ be a $(k+1)$-sided polygon. Pick a vertext $P_1$ of $P$ at which the
interior angle is $<\pi$. (It is clear that such a vertex must exist.)
Let $P_0$ and $P_1$ denote the vertices of $P$ adjacent to $P_1$,
let $T$ be the triangle $P_0P_1P_2$, and 
$P'$ the polygon obtained from $P$ by removing the triangle $T$,
i.e., with the two sides $P_0P_1$ and $P_1P_2$ replaced by $P_0P_2$

Then $P'$ has $k$ sides, so by the induction hypothesis the sum of the
interior angles in $P'$ is $(k-2)\pi$. Also, since $T$ is a triangle,
the sum of the interior angles in $T$ is $\pi$.  
The sum of the interior angles in the original polygon $P$ is equal
to the sum of the interior angles of $P'$ plus the sum of the interior
angles of $T$, i.e.,  $(k-2)\pi + \pi = ((k+1)-2)\pi$. 
This is the desired formula for the sum of the interior angles of a
$(k+1)$-sided polygon, so we have proved $P(k+1)$. 

\textbf{Conclusion:} By the principle of induction, $P(n)$ holds for
every integer $n\ge3$.

\textbf{Proof of (d):}

Let $P(m)$ denote the statement that,  \textbf{given any group of $2m+1$ people
with pairwise distinct mutual distances, there is at least one survivor in the
pie fight (in the sense of not getting hit by a pie)}.
We will show by induction on $m$ that $P(m)$ holds for all $m\in\NN$.

\textbf{Base case:} When $m=1$, there are $2m+1=3$ fraternity members in the
group, By the ``distinct distance'' assumption, the triangle created by these
three members has a unique minimal side. Therefore the two members standing at
the endpoints of this side throw pies at each other, while the third person
hits one of these two, but does not get hit. Thus, this third person
``survives'' the fight, and hence the statement $P(m)$ holds for $m=1$.

\textbf{Induction step:} Let $k\in\NN$ and suppose $P(m)$ holds for $m=k$.
Consider a group of $2(k+1)+1=2k+3$ fraternity members, say
$F_1,\dots,F_{2k+2},F_{2k+3}$,
with distinct mutual distances.  Since the distances are
distinct, there exists a unique minimal distance among the distances between
pairs of fraternity members. Without loss of generality, we may assume that
$F_{2k+2}$ and $F_{2k+3}$ are the two members whose mutual distance is minimal. 
Then these two members throw pies at each other, while the remaining members, 
i.e., $F_1,\dots, F_{2k+1}$, throw pies at each other or at $F_{2k+2}$
or $F_{2k+3}$. 

If all remaining members $F_1,\dots, F_{2k+1}$ only throw  pies at themselves
(and  not at $F_{2k+2}$ or $F_{2k+3}$), then the induction hypothesis
immediately guarantees that there is a survivor among these members. 

Now consider the case when some of the members
$F_1,\dots,F_{2k+1}$
have  $F_{2k+2}$ or
$F_{2k+3}$ as their nearest target, and thus, by the rules of the game, throw a
pie at $F_{2k+2}$ or $F_{2k+3}$.  In this case, removing   $F_{2k+2}$ and
$F_{2k+3}$ as possible targets forces these members to target the nearest
neighbor among 
$F_1,\dots,F_{2k+1}$.
We can then again apply   the
induction hypothesis to obtain a survivor among $F_1,\dots,F_{2k+1}$.
Since $F_{2k+2}$ and $F_{2k+3}$ only target themselves, that person remains a
survivor after adding these two members back in. 

Thus, in either case, we have shown that there is a survivor in the given group of 
$2(k+1)+1$ fraternity members. Hence $P(k+1)$ holds, and the induction step is
complete.   

\textbf{Conclusion:} By the principle of induction, $P(m)$ holds for all
$m\in\NN$.

\subsection{Solutions to Section~\ref{sec:Induction:FallaciesAndPitfalls} } 

\textbf{Example~\ref{example:Induction:fall1}}

Here there is no problem with the induction step, but the base case
is not valid despite the claim that it is true in this case.
Moral: Make sure to \emph{really} check the base case. Simply 
stating that it is true doesn't make it true!

\textbf{Example~\ref{example:Induction:fall2}}
The base step is valid, as is the induction step \emph{provided $k$
is at least $2$.} However, when $k=1$, the induction step breaks down since in
this case there is no overlap in the variables in (1) and (2), so one cannot
``chain together'' these equalities. Formally, this means that in the
implication chain $P(1)\implies P(2)\implies P(3)\implies P(4)\implies\cdots$, the
first link (from $P(1)$ to $P(2)$) is broken, while all other links are valid.
This single broken link is enough to render the induction argument invalid.


\textbf{Example~\ref{example:Induction:fall3}}

The base step is valid, and the induction step is valid, too,
\emph{provided $k$ is at least $1$.} However, the first $k$-value for which
we need the induction step is $k=0$ (since $n=0$ is our base case). When
$k=0$, the equation $k+1=i+j$ reduces to $1=i+j$, 
and the constraints on $i,j$ become $0\le i,j\le 0$. Since there is no  
choice of $i,j$ satisfying both these constraints  and the equation $i+j=1$,
the argument breaks down in this case. (When $k\ge1$, there is no problem
since choosing $i=1$ and $j=k$ we can satisfy both $i+j=k+1$ and 
the constraints $0\le i,j\le k$.)

\textbf{Example~\ref{example:Induction:fall4}}
The base step is valid, and the induction step is valid, too,
\emph{provided $k$ is at least $1$.} However, as in the previous example, 
when $k=0$ (the first $k$-value we need in the induction step), something goes wrong:
Namely, in this case, $k-1=0-1=-1$ is negative and hence out of range of 
the induction hypothesis.

\textbf{Example~\ref{example:Induction:fall5}}

The base step is valid, 
but there is a problem with the induction step: In this step
the induction hypothesis is applied with $x-1$ and $y-1$ in place
of $x$ and $y$. However, this requires that $x-1$ and $y-1$ be positive
integers, something that we are not assured.
For example, if $x=1$, then $x-1=0$, so $x-1$ is out of range, and therefore we
cannot apply the induction hypothesis with $x-1$ as the $x$-value. 
This renders the induction step invalid \emph{for all values of $k$}.








