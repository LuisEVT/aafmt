\section{Study guide  for ``Abstract Groups: Definitions and Basic Properties''  chapter}\label{sec:Groups:study} 


\subsection*{Section \ref{groups_section_define}, Formal definition of a group}
\subsubsection*{Concepts}
\begin{enumerate}
\item %%%%%%%%%Had trouble wording this one....KW%%%%%%
A \term{binary operation} or \term{law of composition} is a function from $G \times G \rightarrow G$. The binary operation takes a pair $(a,b)$ and produces an element in $G$ which is denoted as $ab$ or $a \circ b$, and is  called the \term{composition} of $a$ and $b$ (Definition~\ref{group_op}). This definition generalizes the binary operations we have seen before, such as +, $\cdot$, $\oplus$, $\odot$, and function composition.
\item
A \term{group} is a set $G$ with a binary operation $\circ$ that meets the requirements of closure, identity element, inverse element, and associativity. The group may be denoted by $(G,\circ)$, or just $G$ if the operation $\circ$ is clear from the context  (Definition~\ref{definition:Groups:group_definition}, and remark following)
\item
If $ab=ba$ for all possible pairs of elements of $(G, \circ)$ then $G$ is \term{abelian} or \term*{commutative}\index{Group!commutative}. (Definition~\ref{abelian_group})
\item
 A group is \term*{finite}\index{Groups!finite or infinite}  if it contains a finite number of elements.  The \term*{order}\index{Order!of a group} of a group $G$ is equal to the number of elements in the group, and is denoted by $|G|$. If a group is not finite then it is infinite and we may denote this by $|G| = \infty$. (Definition~\ref{definition:Groups:group_order})
\item
The \term{trivial group}\index{Group!trivial} is the group that consists of only the identity element, $e$. (Definition~\ref{trivial_group})
\end{enumerate}

\subsubsection*{Competencies}
\begin{enumerate}
\item
Know and be able to verify the four defining properties of a group (Ex. \ref{exercise:Groups:group_prop},\ref{exercise:Groups:trivial})
\end{enumerate}


\subsection*{Section \ref{Examples}, Examples}
\subsubsection*{Concepts:}
\begin{enumerate}
\item
A set of numbers  with superscript * indicates that 0 is excluded  (e.g. $\mathbb{R}^{\ast},\mathbb{C}^{\ast})$.
\item
Groups of numbers that include 0 (such as ${\mathbb R}, {\mathbb C}, {\mathbb Q}$) are assumed to have group operation + (unless otherwise stated),  groups that exclude 0 (such as ${\mathbb R}^{\ast}, {\mathbb C}^{\ast}, {\mathbb Q}^{\ast}$) are assumed to be multiplicative (unless otherwise stated).
\item
Given groups $G$ and $H$, the \term{product}\index{Groups!product of} of $G$ and $H$, $G \times H$, is defined as the set of pairs $\left\{(g,h), g \in G, h \in H \right\}$.  If $(g_1, h_1)$ and $(g_2, h_2) \in G \times H$, then the group operation $(g_1, h_1) \circ (g_2, h_2) := (g_1g_2, h_1h_2)$. (Definition~\ref{definition:Groups:defProductOfGroups})
\item
The set of positive integers less than $n$ and relatively prime to $n$ is a group under multiplication mod $n$. This group is called the \term*{group of units} of ${\mathbb Z}_n$\index{Group!of units $U(n)$}  and is denoted by $U(n)$.  (Definition~\ref{DefnOfUnits}, Remark~\ref{UnMul})
\item
The group of $2 \times 2$ matrices $GL_2({\mathbb C})$ with nonzero determinant is called the 2-dimensional general linear groups over the complex numbers. (Exercise \ref{exercise:Groups:GL2_group})
\end{enumerate}

\subsubsection*{Competencies}
\begin{enumerate}
\item
Prove whether or not excluding zero changes whether a specified set is still a group. (\ref{exercise:Groups:Rstar_group}, \ref{exercise:Groups:C_star}, \ref{exercise:Groups:15})
\item
Determine if a given operation on a set produces a group, and if so is it an abelian group. (\ref{exercise:Groups:integer_coordinate_plane}, \ref{exercise:Groups:19})
\item
Compute the solutions for the given elements as products of groups. (\ref{exercise:Groups:prod_of_groups})
\item
Show that $g$ is the identity element of $G \iff g \circ h = h$. (\ref{exercise:Groups:ident})
\item
Create Cayley tables and answer various questions about their componenets. (\ref{exercise:Groups:cayley}, \ref{exercise:Groups:Cayley_groups}, \ref{exercise:Groups:Cayley_groups2}, \ref{exercise:Groups:Cayley_groups3})
\item
Prove that $U(8)$ is a group using a Cayley table and determine if it is abelian. (\ref{exercise:Groups:27}, \ref{exercise:Groups:28})
\item
Prove that $U(n)$ is a group under multiplication mod $n$ for any $n$ and is abelian. (\ref{exercise:Groups:U(n)_abgroup}, \ref{exercise:Groups:UnAbel})
\item
Prove that a matrix, or a subset of a matrix,  is a group under different operations and determine the order and if it is abelian. (\ref{exercise:Groups:M2x2_group}, \ref{exercise:Groups:Mn_group})
\item
Understand $GL_n({\mathbb C})$ as the subset of ${\mathbb M}_n({\mathbb C})$ and determining if it is a group under matrix multiplication. (\ref{exercise:Groups:GL2_group}, \ref{exercise:Groups:GL_n_group})
\end{enumerate}


\subsection*{Section \ref{Basic properties of groups}, Basic properties of groups}
\subsubsection*{Concepts:}
\begin{enumerate}
\item 
Identity element of groups: 
	\begin{itemize}
	\item
	The identity element, $e$, for a group, $G$, is unique. So, $\exists$ only one $e \in G$ such that $eg = ge = g\  \forall\  g \in G$. (Proposition~\ref{proposition:Groups:id_unique})
	\end{itemize}
\item
Inverses element of Groups:
	\begin{itemize}
	\item
	Any element $g \in G$, has a unique inverse, $g'$. (Proposition~\ref{proposition:Groups:inv_unique}, Exercise~\ref{exercise:Groups:inverse_unique})
	
	\item
	If $a, b \in G$, then $(ab)^{-1} = b^{-1}a^{-1}$. Notice that you switch the order of the elements when you distribute the power to each element. (Proposition~\ref{proposition:Groups:group_inv_reverse}, Exercise~\ref{exercise:Groups:42})

	\item
	Remember that no matter what elements or binary operations you are given finding the inverse is just like high school algebra! i.e. You used inverses to solve $5x = 6$ for $x$.  It's the same concept. You are taking the reciprocal (inverse) and multiplying (composing) it to the other side.
	\end{itemize}

\item
The inverse of an inverse is just the original element.  $\forall a \in G, (a^{-1})^{-1} = a$. (Proposition~\ref{proposition:Groups:inv_inv}, Exercise~\ref{exercise:Groups:44})

\item
Let $a, b \in G$, then $ax = b$ and $xa = b$ have unique solutions in $G$. (Proposition~\ref{proposition:Groups:group_equations}, Exercise~\ref{exercise:Groups:48})

\item
The law of cancellation, which is what we use in basic algebra, is when we multiply by the inverse to cancel out the element.  This works for any group $G$.  So, if $a, b, c \in G$, then $ba = ca \implies b = c$ and $ab = ac \implies b = c$. Notice this is showing both left and right cancellation, it holds true in all groups. (Proposition~\ref{proposition:Groups:cancel}, \ref{exercise:Groups:56})

\item
Exponential notation:
	\begin{itemize}
	\item
	If $g \in G$, then the identity element can be written as $g^0 = e$.\\
	For $n \in {\mathbb N}$:\\
	\[  	g^n := \underbrace{g \cdot g \cdots g}_{n \; {\rm times}} \]\\
	and\\
	\[g^{-n} := (g^{-1})^n =  \underbrace{g^{-1} \cdot g^{-1} \cdots g^{-1}}_{n \; {\rm times}}. \]\\
	(Definition~\ref{definition:Groups:DefGroupExponents})
	
	\item
	Law of exponents still applies to groups:\\
	$\forall g, h \in G$\\
		\begin{enumerate}
 		\item
		$g^mg^n = g^{m+n}$ for all $m, n \in {\mathbb Z}$; 
 
		\item
		$(g^m)^n = g^{mn}$ for all $m, n \in {\mathbb Z}$; 
 
		\item
		$(gh)^n = (h^{-1}g^{-1})^{-n}$ for all $n \in {\mathbb Z}$. Furthermore, if $G$ is abelian, then $(gh)^n = g^nh^n$. 
 		\end{enumerate}
	(Proposition~\ref{proposition:Groups:exponent_laws}, Exercise~\ref{exercise:Groups:59})
	\end{itemize}
\end{enumerate}

\subsubsection*{Competencies}
\begin{enumerate}
\item
Determine the inverse of an element given it's group and operation. (\ref{exercise:Groups:39})

\item
Calculate the inverse of products in given groups. (\ref{exercise:Groups:inv_prod})

\item
Determine if a group is abelian or not. (\ref{exercise:Groups:group_abelian}, \ref{exercise:Groups:54})

\item
Solve for x using the inverse property of diferent operations. (\ref{exercise:Groups:49}, \ref{exercise:Groups:50}, \ref{exercise:Groups:51}, \ref{exercise:Groups:52}, \ref{exercise:Groups:53})

\item
Determine the equality of equations using exponential notation. (\ref{exercise:Groups:57})
\end{enumerate}


\subsection*{Section \ref{groups_section_subgroups}, Subgroups}
\subsubsection*{Concepts:}
\begin{enumerate}
\item 
A subgroup, a subset of a group, is also a group. All subgroups are subsets.
	
\item
Subgroup $H$ of $(G, \circ)$ is a subset $H$ of $G$ that has $G$ restricted to $H$ and $H$ is a set in it's own right. (Definition~\ref{subgroup_defn})

\item
An example: $2{\mathbb Z}$ is a sbgroup of ${\mathbb Z}$ under the operation of addition.  (Example~\ref{example:Groups:Z2subgroup})

\item
The associative property holds for any subset of a group $G$ under that groups operation.

\item
A subset $H$ of a group $G$ is a subgroup if and only if:
	\begin{enumerate}[(a)]
	\item
	The identity of $G$ is in $H$.

	\item
	If $h_1, h_2 \in H$, then $h_1h_2 \in H$ (that is, $H$ is closed under the group operation)

	\item
	If $h \in H$, then $h^{-1} \in H$
	\end{enumerate}
(Proposition~\ref{proposition:Groups:subgroup_prove})

\item
If $H$ is a subset of $G$, then $H$ is a subgroup of $G \iff h \neq \emptyset$, and whenver $g, h \in H$ then $gh^{-1}$ is in $H$. (Proposition~\ref{proposition:Groups:subgroup_prove_2}, Example~\ref{example:Groups:prov})
\end{enumerate}

\subsubsection*{Competencies}
\begin{enumerate}
\item
Prove that a given operation is a subgroup. (\ref{exercise:Groups:63}, \ref{exercise:Groups:64})

\item
Prove the identity element for a group and a subgroup are equal.  (\ref{exercise:Groups:65})

\item
Show that different sets are subgroups of other groups, be able to find their order, and show whether or not they are abelian. (\ref{exercise:Groups:T_subgroup}, \ref{exercise:Groups:67}, \ref{exercise:Groups:68}, \ref{exercise:Groups:69}, \ref{exercise:Groups:70})

\item
Using Proposition~\ref{proposition:Groups:subgroup_prove_2} prove if a given is a subgroup of a group. (\ref{exercise:Groups:73})
\end{enumerate}


\subsection*{Section \ref{cyclic_groups}, Cyclic groups}
\subsubsection*{Concepts:}
\begin{enumerate}
\item 
Following Example~\ref{example:Groups:construct_Z}, understand how to construct subgroups by starting with the smallest subset possible and adding in the elements of identity and inverse.  Think of it as building a Cayley table.

\item
The set generated by an element $a \in G$ is denoted by $\langle a \rangle$ and is defined as the set obtained by repeated multiplication of the identity element, $e$, by the group elements $a$ and $a^{-1}$.  This can be written as  \[ \langle a \rangle = \{ \ldots, a^{-3}, a^{-2}, a^{-1}, e, a, a^2, a^3, \ldots \} \] or \[ \langle a \rangle = \{ a^{k} : k \in \mathbb Z \} \] $\langle a \rangle$ is sometimes called the orbit of $a$. (Definition~\ref{defn_set_gen})

\item
When using the ``+'' operation write $\langle a \rangle  = \{ na : n \in {\mathbb Z} \}$. (Remark~\ref{rem_orbits})

\item
When an entire group is generated by some element $a$, such that $G = \langle a \rangle$, then $G$ is called a cyclic group and $a$ is a generator of $G$. i.e. ${\mathbb Z}$ is a cyclic group and 1 is the generator, a group can have more than one generator though. (Definition~\ref{cyclic_generator}, Example~\ref{example:Groups:construct_Z6})
\\
\\
This also applies to other types of groups such as $U(9)$. (Example~\ref{example:Groups:Cyclic_U9})

\item
$\langle a \rangle$ is always a group. If we let $G$ be any group and $a \in G$, then the set \[\langle a \rangle  = \{ a^k : k \in {\mathbb Z}\}\] is a subgroup of $G$, any subgroup of $G$ that contains $a$ must also contain $\langle a \rangle$. (Example~\ref{example:Groups:Cyclic_Z3}, Theorem~\ref{OrbitIsSubgroup})

\item
$\langle a \rangle$ is called the cyclic subgroup generated by $a$ for each $a \in G$. (Definition~\ref{defn_cyclic_subgroup})

\item
The order of $a$ is the smallest positive integer $n$ such that $a^n = e$. The order of $a$ is written as $|a| = n$. If there is no $n$ that gives $e$ then the order of $a$ is infinite and we write $|a| = \infty$. (Definition~\ref{definition:Groups:DefOrder}, Example~\ref{example:Groups:Z_1_order})

\item
If $G$ is a finite group and $a \in G$, then $|a| = |\langle a \rangle|$. (Exercise~\ref{exercise:Groups:OrderEltCyclic}, Proposition~\ref{proposition:Groups:OrderEltCyclic})

\item
Not every group is a cyclic group. (Example~\ref{example:Groups:Not_Cyclic_S3})

\item
Every cyclic group is abelian. (Theorem~\ref{cyclic_abelian})

\end{enumerate}

\subsubsection*{Competencies}
\begin{enumerate}
\item
List the set given an orbit and a group.  (\ref{exercise:Groups:gen_3_Rstar})

\item
Show the generator of a group.  (\ref{exercise:Groups:Z_construct_-1}, \ref{exercise:Groups:Z6_construct_5}, \ref{exercise:Groups:U9_construct})

\item
Prove that if $\langle a \rangle$ is a generator of $G$, then so is $\langle a^{-1} \rangle$. (\ref{exercise:Groups:82})

\item
Prove that a subgroup of a group is indeed still a group.  (\ref{exercise:Groups:subgroup_3Z}, \ref{exercise:Groups:87})

\item
Show for a finite group that there has to exist a ${\mathbb N} m > 0$ such that $a^m = e$, where $a \in G$, and $G$ is a finite group. (\ref{exercise:Groups:90})

\item
Determine the order of a given order for elements of a group. (\ref{exercise:Groups:Z6_orders}, \ref{exercise:Groups:U9_orders})

\item
Find the order of elements of a group and the cyclic subgroup created by each element.  Determine the relationship, if any, between $|a|$ and $|\langle a \rangle|$ (\ref{exercise:Groups:Z12_orders})

\item
Show different properties associated with a finite group $G$, where $a \in G$ and $|a| = n$. (\ref{exercise:Groups:finite_inv}, \ref{exercise:Groups:OrderEltCyclic})

\item
Show that given a finite group $G$ and letting $a \in G$ such that $|a| = n$, for $n > 0$, that there exists a ${\mathbb N} m$ such that $a^{-1} = a^m$, in terms of $m$ and $n$. (\ref{exercise:Groups:97})

\end{enumerate}


%\subsection*{Section \ref{}, }
%\subsubsection*{Concepts:}
%\begin{enumerate}
%\item 
%
%\item
%
%\item
%
%\item
%
%\end{enumerate}
%
%\subsubsection*{Competencies}
%\begin{enumerate}
%\item
%
%\item
%
%\item
%
%\end{enumerate}





