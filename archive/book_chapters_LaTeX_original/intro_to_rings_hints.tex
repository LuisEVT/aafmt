\section{Hints for ``Introduction to Rings'' exercises}
\label{sec:Rings:Hints} 

\noindent Exercise \ref{exercise:Rings:Iso_Z10_to_Z5}:  Look at the Cayley tables and imagine that the tables contain only the first element in each pair.

\noindent Exercise \ref{exercise:Rings:Z3_homo}: 
This is similar to Exercise \ref{exercise:Rings:Iso_Z10_to_Z5}.

\noindent Exercise \ref{exercise:Rings:abc_iso}:
Part (b):  Check if $f$ has an inverse.

\noindent Exercise \ref{exercise:Rings:ringIsPrId}:
Can you think of an element in $R$ that generates all of $R$?

\noindent Exercise \ref{exercise:Rings:zerodivisor}:  Take the expression $ab=0$ and multiply both sides on the left by $a^{-1}$.

\noindent Exercise \ref{exercise:Rings:zdzp}:  Use Proposition~\ref{proposition:Rings:zerodivisor} part b and remember that $\mathbb{Z}_p$ is a field.
