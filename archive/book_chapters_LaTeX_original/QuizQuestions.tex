\chap{Quiz Questions}{QuizPractice_chap}

\begin{enumerate}[(1)]
\item
Simplify:
$ (ab)^x(bc)^y(ca)^{x+y}$
\item
Simplify:
$ \displaystyle{\left(\frac{3}{2}\right)^3 \left(\frac{4}{27}\right)} $
\item
Simplify:   $ \displaystyle{(x+y)(x-z) - (x-y)(x+z) }$
\item
Given the expression:  $(x \cdot y + ((y+z)+w) \cdot x) + w \cdot x$:
\begin{enumerate}[(a)]
\item
Simplify, using the associative property ONLY.
\item
Simplify, using the associative and distributive properties ONLY.
\item
Simplify, using associative, distributive, and commutative properties.
\end{enumerate}
\end{enumerate}

\begin{enumerate}[(1)]
\item
Give an example to show that subtraction is not associative.
\item
Suppose $a>b$ and $ab < 0$.  What can you conclude about $a$ and $b$?
\item
Simplify:   $\displaystyle{ \frac{5}{x+1} + \frac{6}{x-1} + \frac{7}{1-x}}$
\item
Simplify: $\displaystyle{\frac{y^{-1}x^2 - y x^4}{1-y} - x^3}$.
\item
Given an example of an irrational number.
\item
Given an example of a nonzero complex number that has real part equal to 0.
\item
Evaluate: $(1+i)^2$.
\item
Evaluate the modulus (absolute value) of $1+i$.
\end{enumerate}

\begin{enumerate}[(1)]
\item
Evaluate:  $\displaystyle{\frac{4 - 2i}{3 + 3i}}$.
\item
Evaluate:  $(1 + \sqrt{3}i)^2$.
\item
Evaluate:  $(1 + \sqrt{3}i)^3$.
\item
$( \sqrt{5} + i \sqrt{7}) - i (\overline{\sqrt{7} + \sqrt{5}i})$
\item
$z$ and $w$ are complex numbers. The modulus of $z$ is 4 and the modulus of $w$ is 2. What is the modulus of $zw$?
\item
$z$ and $w$ are complex numbers. The argument of $z$ is $\pi/4$ and the argument of $w$ is $\pi/3$. What is the argument of $zw$?
\end{enumerate}

\begin{enumerate}[(1)]
\item
A car begins at mile marker 3 on an 8-mile track.  The car goes forward 18 miles, backwards 7 miles, then forward 5 miles. The car then repeats the same pattern 5 more times.  At what mile marker does the car end up?
\item
The first day of March is a Tuesday.  What day of the week is the first day of April?  What day of the week is the first day of May?  (March has 31 days, April has 30).
\item
Let $z = 3 \cis (\pi/6)$.  Find $z^3$ in complex polar form, and re-express in rectangular form.
\item
Find all cube roots of -1.
\end{enumerate} 

Given the expression
$$ (yx + (x(y+x) + (x+y)x) ) - yx$$
\begin{enumerate}[(a)]
\item
Simplify the expression using the associative law.
\item
Simplify the expression using the associative and distributive laws.
\item
Simplify the expression using the associative, distributive, and commutative laws.
\end{enumerate}

\begin{enumerate}[(1)]
\item
A cubic polynomial of the form $x^3 + ax^2 + bx + c$  ($a,b,c$ are real)  has roots $-2$ and $1 - 2i$.  Find $a,b,c$.
\item
Find all $4^{\text{th}}$ roots of  $3 \sqrt{3} + 3i$.
\item
Find all solutions to:$423 x - 351 \equiv 564 \text{(mod 7)}$
\end{enumerate}

\begin{enumerate}[(1)]
\item
A polynomial of the form $x^4 + a_3x^3 + a_2x^2 + a_1x + a_0$  ($a_0,a_1,a_2,a_3$ are real)  has roots $1-i$ and $-2-i$.  Find $a_0,a_1,a_2,a_3$.
\item
Find all $5^{\text{th}}$ roots of  $-i$.
\item
Find all solutions to:$ 80x - 1000 \equiv 200 \text{(mod 9)}$
\item
Compute:  mod($((111 \cdot 444) + 777) \cdot 48273,11$).
\item
Compute in $\mathbb{Z}_7$:  $6 \odot 6 \odot 6 \odot 6 \odot 6 \odot 6$. 
\end{enumerate}


\begin{enumerate}[(1)]
\item
Show that if $m$ is odd, then $\mod(m^2,8) = 1$.
\item
If$\mod(n,8)=1$ and $n = (x+y)(x-y)$, show that  $x$ must be odd and $y$ is divisible by 4.
\item
If$\mod(n,8)=7$ and $n = (x+y)(x-y)$, show that  $y$ must be odd.
\item
If$\mod(n,8)=5$ and $n = (x+y)(x-y)$, show that  $\mod(y,4) = 2$.
\item
If$\mod(n,8)=3$ and $n = (x+y)(x-y)$, show that  $y$ is odd.
\end{enumerate}

Instructions:  You may use a basic calculator that does addition, multiplication, division, and subtraction. No other helps
\bigskip

\begin{enumerate}[(1)]
\item
Evaluate:  (a)~gcd(951,351) \qquad (b)~gcd(391,663) \qquad (c)~gcd(501,473)
\item
Find values of $m$ and $n$ that solve the following equations:

(a) $88m + 97n = 19$ \qquad (b) 411m + 312n = 41 \qquad 105m + 75n = 225
\item
Perform the following matrix multiplications:

(a)~
$\left(
\begin{array}{cc}
1 & 4 \\
2 & 5
\end{array}
\right)
\left(
\begin{array}{cc}
-2 & 1 \\
3 & -2
\end{array}
\right)$ \qquad
(b)~
$\left(
\begin{array}{cc}
-5 & 7 \\
6 & -5
\end{array}
\right)
\left(
\begin{array}{cc}
9 & 7 \\
8 & 10
\end{array}
\right)$ \\

(c)~
$\left(
\begin{array}{cc}
4 & 9 \\
8 & 2
\end{array}
\right)
\left(
\begin{array}{cc}
-5 & 1 \\
2 & 7
\end{array}
\right)$ 
\item
Perform the following matrix multiplications mod 26:

(a)~
$\left(
\begin{array}{cc}
11 & 17 \\
14 & 19
\end{array}
\right)
\left(
\begin{array}{cc}
7 & 22 \\
3 & 10
\end{array}
\right)$ \quad
(b)~
$\left(
\begin{array}{cc}
23 & 24 \\
21 & 19
\end{array}
\right)
\left(
\begin{array}{cc}
14 & 9 \\
7 & 5
\end{array}
\right)$
\end{enumerate}
\bigskip

{\bf Next class} (October 8): In-class practice and quiz on exercises 78, 88 in Modular Arithmetic chapter, and Exercises 2-4 of the .  {\bf updated version}  of Chapter 7  on Blackboard


\begin{enumerate}
\item
Find all solutions:
\begin{enumerate}[(a)]
\item
$221 x \equiv 542 \pmod{629}$
\item
$107x \equiv 319 \pmod{444}$
\item
$459x \equiv 639 \pmod{1125}$
\item
$418 x \equiv 421 \pmod{589}$
\item
$105x \equiv 195 \pmod{5320}$
\end{enumerate}
\item
Find all solutions
\begin{enumerate}[(a)]
\item
$x \equiv 3 \pmod{7},~~x\equiv 2 \pmod{4},~~ x\equiv 5 \pmod{3}$
\item
$x\equiv 4 \pmod{2},~~x \equiv 7 \pmod{13}, ~~ x \equiv 5 \pmod{26}$
\item
$x\equiv 1 \pmod{5},~~ x\equiv 1 \pmod{11},~~x\equiv 1 \pmod{22}$
\end{enumerate}
\item
Do problems 3b, 4a,b in the updated version of Chapter 7.
\end{enumerate}



