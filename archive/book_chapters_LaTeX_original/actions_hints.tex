\section{Hints for ``Group Actions, with Applications'' exercises}
\label{sec:GroupActions:Hints} 


\noindent Exercise \ref{exercise:GroupActions:Action7}(b): Notice $0 \in 2\mathbb{Z}$, but $1+0$ is not in $2\mathbb{Z}$. So the action of $\mathbb{Z}$ on $2\mathbb{Z}$ is not well-defined. (d) Note that the group operation of $\mathbb{C}$ is addition.

\noindent Exercise \ref{exercise:GroupActions:Stabilizers2}: Note there are two rows for stabilizers of faces, because some stabilizers of faces have order 2 and some have order 4.

\noindent Exercise \ref{exercise:GroupActions:CountingFormula2}(a):   There are two elements.~~(d):   There are three elements.

\noindent Exercise \ref{exercise:GroupActions:CubeCounts}(a):   Express $|G|$ two different ways by applying the Counting Formula to edges, and then to faces.

\noindent Exercise \ref{exercise:GroupActions:CubeCounts}(c):  You may take the ratio (faces/edges) / (vertices/edges).

\noindent Exercise \ref{exercise:GroupActions:Stabilizers2}:   There are two rows for faces, because there are two kinds of stabilizers for faces.

\noindent Exercise \ref{exercise:GroupActions:Tetra5}(b): It may be helpful to calculate the rotation using cycle notation.

\noindent Exercise \ref{exercise:GroupActions:Tetra5a} For example, $R_{Bb}$ and  $R_{Bb}^2$ are the 120- and 240- degree rotations around the axis $\overset{\leftrightarrow}{Bb}$, Both stabilize face $b$. So 
$G_b = \{ {\var id}, R_{Bb}, R_{Bb}^2 \}$. The same group stabilizes another set as well--can you figure out which one?  


\noindent Exercise \ref{exercise:GroupActions:Octa5a}: How many group elements (rotations) are in $G_{x_+}$?  What else do they stabilize?

\noindent Exercise \ref{exercise:GroupActions:Dodeca5}(b): What is $|G|$ according to the counting formula?  How many stabilizers have we found so far?

\noindent Exercise \ref{exercise:GroupActions:Soccer1}(b): Use the Counting Formula.

\noindent Exercise \ref{exercise:GroupActions:Soccer2}(b): See the previous hint.

\noindent Exercise \ref{exercise:GroupActions:IntLatNotGset}: Does $a+h$ have to be in $H$?

\noindent Exercise \ref{exercise:GroupActions:RtCosetAction}: $H$ itself is a coset, and take $g_1=(123)$ and $g_2=(23)$.  Is it true that acting on $H$ by $g_1$ followed by $g_2$ is the same as acting on $H$ by $g_2 g_1$?

\noindent Exercise \ref{exercise:GroupActions:Conj13}: (b) Take the answer to part (a), and apply the rotation $r_x^2$ (why does this work?)  (c) Find a rotation that map the fixed point set of $r_y\compose r_z$ to the fixed point set of  $r_z\compose r_y$.
