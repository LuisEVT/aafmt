\chapter{Test 1, Part 1, Number Theory Math 301 (Thron) }\label{Actual}

\begin{enumerate}[(1)]
%\item
%Simplify:   $ \displaystyle{\frac{(a+b)(b-c) + (a-b)(b+c)}{c-a} }$

\item
Given the expression:~
 $(r+p)(s+q) - (p+s)(q+r)$
\begin{enumerate}[(a)]
\item
Simplify the expression using the associative law ONLY.
\item
Simplify the expression using the associative and distributive laws ONLY.
\item
Simplify the expression using the associative, distributive, and commutative laws.
\end{enumerate}

\item
Evaluate:  $\displaystyle{ \left(\frac{i}{3 + 4i} \right) +  \left(\frac{2}{4 + 3i} \right)} $. Express your answer in the form $a + bi$.
\emph{Show your procedure.}  You may use a calculator to do arithmetic, but you must do the procedure by hand.
\item
A certain manufacturing process takes 500 hours to run to completion. The program is started on Wednesday at noon.
\begin{enumerate}[(a)]
\item
What is the time of day when the program completes?
\item
On what day of the week does the program complete?
\end{enumerate}

\item
$z$ and $w$ are complex numbers. $z$ has modulus $\sqrt{20}$ and complex argument $\pi/4$, while $w$ has modulus $\sqrt{5}$ and argument $3\pi/4$.  What are the modulus and argument of $\displaystyle{z \cdot \overline{w}}$? \emph{Show your procedure.}  You may use a calculator to do arithmetic, but you must do the procedure by hand. Simplify your answer as much as possible

\item 
For the equation:  $157m + 50n = 3 $ ($m,n$ are integers):
\begin{enumerate}[(a)]
\item
Find a solution to the equation.
\item
Find \underline{all} solutions to the equation.
\end{enumerate}


%\item
%Compute mod( $2^{x}$,3), where $x = 5280^{5280}+1$. \emph{Explain} your answer. 
%
%%%% Actual
%\item
%Find all solutions to: $z^4 = -16$.
%
%
%\item
%Recall that $\textrm{Re}[z]$ means the real part of $z$, and  $\textrm{Im}[z]$ means the imaginary part of $z$.
%\begin{enumerate}[(a)]
%\item
%Define the set $A$ in the complex plane as:    $A$ = \{ all complex numbers  $z$ such that Im[$z$]  = -1 \}.  Draw a picture of $A$ in the complex plane.
%\item
%Define the set $B$ in the complex plane as:    $B$ = \{ all complex numbers  $z$ such that Re[$z$]  = 2 \}.  Draw a picture of $B$ in the complex plane.
%\item Describe the set $A \cap B$.
%\end{enumerate}
%
%




\end{enumerate}