\section{Solutions for ``Functions''}
\noindent\textbf{\textit{ (Chapter \ref{functions}})}\bigskip

%\noindent\textbf{Exercise \ref{exercise:functions:}:}\\  %KW unnamed exercise

\noindent\textbf{Exercise \ref{exercise:functions:6}:}\\

\noindent\textbf{Exercise \ref{exercise:functions:7}:}\\

\noindent\textbf{Exercise \ref{exercise:functions:9}:}\\

\noindent\textbf{Exercise \ref{exercise:functions:funtable}:}\\

\noindent\textbf{Exercise \ref{exercise:functions:ExOrderPairs}:}\\
$A=\{a,b,c,d\}$, $B=\{1,3,5,7,9\}$\\
b. function from $A \implies B$\\
d. function from $A \implies B$\\
f. function from $A \implies B$\\
h. not a function from $A \implies B$\\

%\noindent\textbf{Exercise \ref{exercise:functions:}:}\\  %KW unnamed exercise

\noindent\textbf{Exercise \ref{exercise:functions:23}:}\\
a. Domain = $\{2,4,6,8,10\}$\\
b. Range = $\{7,9,11,13,15\}$\\
c. $g(6)=11$\\
d. $g(7)$ is undefined\\
e. $g=\{(2,7),(4,9),(6,11),(8,13),(10,15)\}$\\
g. $g(n)=n+5$\\

%\noindent\textbf{Exercise \ref{exercise:functions:}:}\\  %KW unnamed exercise

\noindent\textbf{Exercise \ref{exercise:functions:16}:}\\

%\noindent\textbf{Exercise \ref{exercise:functions:}:}\\  %KW unnamed exercise

\noindent\textbf{Exercise \ref{exercise:functions:27}:}\\ %KW not updated, Adam did part b, which is not part of the current question.
%For the given sets $A$ and~$B$, write each function from~$A$ to~$B$ as a set of ordered pairs. (It turns out that if $|A| = m$ and $|B| = n$, then the number of functions from~$A$ to~$B$ is~$n^m$. Do you see why?)
\begin{multicols}{2}
\begin{enumerate}
\item \label{FunctionsChapExers-FindAll-abc,d}
%$A = \{\var{a},\var{b},\var{c}\}$, $B = \{\var{d}\}$
\item \label{FunctionsChapExers-FindAll-ab,cd}
%$A = \{\var{a},\var{b}\}$, $B = \{\var{c},\var{d}\}$
\item \label{FunctionsChapExers-FindAll-a,bcd}
%$A = \{\var{a}\}$, $B = \{\var{b},\var{c},\var{d}\}$
\item \label{FunctionsChapExers-FindAll-ab,cde}
%$A = \{\var{a},\var{b}\}$, $B = \{\var{c},\var{d},\var{e}\}$
\item 
%$A = \{\var{a},\var{b},\var{c}\}$, $B = \{\var{d},\var{e}\}$
\end{enumerate}
\end{multicols}

%old answer in suppl
%b. $A=\{a,b\}$, $B=\{c,d\}$\\
%$f=\{(a,c),(b,d)\}$ Range: $\{c,d\}$\\
%$g=\{(a,d),(b,c)\}$ Range: $\{d,c\}$\\
%$h=\{(a,c),(b,c)\}$ Range: $\{c,c\}$\\
%$k=\{(a,d),(b,d)\}$ Range: $\{d,d\}$\\

\noindent\textbf{Exercise \ref{exercise:functions:atomic}:}\\

\noindent\textbf{Exercise \ref{exercise:functions:arrow}:}\\

\noindent\textbf{Exercise \ref{exercise:functions:11Exers-pairs}:}\\

\noindent\textbf{Exercise \ref{exercise:functions:11Exers-pairs2}:}\\

%\noindent\textbf{Exercise \ref{exercise:functions:}:}\\  %KW unnamed exercise

\noindent\textbf{Exercise \ref{exercise:functions:horizontal}:}\\

\noindent\textbf{Exercise \ref{exercise:functions:horizontal2}:}\\

%\noindent\textbf{Exercise \ref{exercise:functions:}:}\\  %KW unnamed exercise

%\noindent\textbf{Exercise \ref{exercise:functions:}:}\\  %KW unnamed exercise

\noindent\textbf{Exercise \ref{exercise:functions:11Exers}:}\\
c. $h(x)=x^2$\\
$h(x)$ is not one-to-one because $f(-1)=f(1)=1$ while $-1\neq 1$\\
d. $i(x)=3x+2$\\
Let $x_1,x_2 \in R$. If $f(x_1)=f(x_2)$, then we have:
$3x_1+2=3x_2+2 \implies x_1=x_2$\\
$\implies i(x)$ is a one-to-one function.\\

%\noindent\textbf{Exercise 37:}\\
%a. $f$ is one-to-one because if $x_1\neq x_2$ then we have $f(x_1)\neq f(x_2)$ (contrapositive).\\
%b. $g$ is not one-to-one because $g(2)=g(3)=d$ while $2\neq 3$.\\

\noindent\textbf{Exercise \ref{exercise:functions:40}:} %KW updated e,i, j,m,o
For each function, either prove that it is one-to-one, or prove that it is not.
\begin{enumerate}[(a)]
\item \label{IsIt11linear}
% $f \colon \rational \to \rational$ defined by $f(r)=3r/5 - 2$.

\item \label{IsIt11square0}
% $f \colon {\mathbb R} \to {\mathbb R}$ defined by $f(x)=(x+2)^2$.

\item \label{IsIt11square}
% $f \colon {\mathbb N} \to {\mathbb N}$ defined by $f(n)=(n+2)^2$.

\item
% $f \colon {\mathbb Z} \to {\mathbb Z}$ defined by $f(n)=(n-1)n(n+1)+1$.

\item
% $f \colon {\mathbb N} \to {\mathbb N}$ defined by $f(n)=(n-1)n(n+1)+1$.
\begin{align*}
f(n) &= (n - 1)n(n+1) + 1\\
&= n^3 - n + 1\\
\\
f'(n) &= 3n^2 - 1\\
0 &= 3n^2 -1\\
1 &= 3n^2\\
\frac{1}{3} &= n^2\\
n &= \pm \sqrt{\frac{1}{3}}
\end{align*}
$(\frac{1}{\sqrt{3}}, \infty)$ is where $f(n)$ is increasing and $f'(n) > 0$.  Since, ${\mathbb N} \subset (\frac{1}{\sqrt{3}}, \infty)$, $f(n)$ is strictly increasing as $n$, where $n \in {\mathbb N}$, increases. This shows, that $n_1 > n_2 \implies f(n_1) > f(n_2)$ or $n_1 < n_2 \implies f(n_1) < f(n_2)$.  \\
$\therefore f(n_1) \neq f(n_2)$ and $n_1 \neq n_2$, hence the function is one to one.

\item
% $f \colon A \to A$ defined by $f(x)=(x-1)(x(x+1)$ ,where \\
% $A =\{x \in \mathbb{R} \text{ and }x >1 \}$~~(requires calculus).

\item \label{IsIt11abs}
% $g \colon \real \to \real$ defined by $g(x)= \left|(x+1)/2 \right|$.

\item \label{modular_g}
% $g \colon {\mathbb Z}_6 \to {\mathbb Z}_6$ defined by $g(n)= n \oplus 2$ .

\item \label{modular_m}
% $g \colon {\mathbb Z}_8 \to {\mathbb Z}_8$ defined by $g(n) = n \odot 2 $ .
This function is not one-to-one because $f(0)=f(4)=0$. Since there are more than 1 elements in the domain that map to the same element in the co-domain. 

\item \label{modular_m2}
% $g \colon {\mathbb Z}_{11} \to {\mathbb Z}_{11}$ defined by $g(n) =  n \odot 2$ .
Since,
\begin{multicols}{3}
$g(0) = 0 \cdot 2 = 0$\\
$g(1) = 1 \cdot 2 = 2$\\
$g(2) = 2 \cdot 2 = 4$\\
$g(3) = 3 \cdot 2 = 6$\\
$g(4) = 4 \cdot 2 = 8$\\
$g(5) = 5 \cdot 2 = 10$\\
$g(6) = 6 \cdot 2 = 1$\\
$g(7) = 7 \cdot 2 = 3$\\
$g(8) = 8 \cdot 2 = 5$\\
$g(9) = 9 \cdot 2 = 7 $\\
$g(10) = 10 \cdot 2 = 9$
\end{multicols}
Since all the elements in the domain map to different elements in the co-domain the function $g(n)=n\cdot 2$ when $g\colon {\mathbb Z}_{11}\to {\mathbb Z}_{11} $ is one-to-one.

\item 
% $g_x \colon {\mathbb Z}_7 \to {\mathbb Z}_7$ defined by $g(n)= n \odot x$ , where $x$ can be any fixed element of ${\mathbb Z}_7$.

\item 
% $g \colon {\mathbb Z}_8 \to {\mathbb Z}_8$ defined by $g(x)= n \odot n \odot n$ .

\item 
% $g \colon {\mathbb Z}_7 \to {\mathbb Z}_7$ defined by $g(x)= n \odot n \odot n$ .
The function $g(x)=n\cdot n\cdot n$ is not one-to-one because\\ 
$g(1)=1\cdot 1\cdot 1=1\cdot 1=1=4\cdot 2=2\cdot 2\cdot 2=g(2)$. Since there exist more than one element in the domain mapping to the same element in the co-domain the function is not one-to-one.

\item
% $g \colon {\mathbb C}\setminus \{0\}  \to {\mathbb C}\setminus \{0\} $ defined by $g(z) =  z^{-1}$ .

\item
% $r \colon A  \to {\mathbb R} $ defined by  
%$r(z) = \text{Re}[z] + \text{Im}[z]$, where \\
% $A = \{ z \in  \mathbb{C}: \text{Im}[z] > 0 \}$. 
If the domain contained $z=-1+i$ and $z=-2+2i$ then it will map to the same co-domain of $-1+1=0=-2+2$. Since two different elements in the domain map to the same element in the co-domain. This function is not one-to-one.
\end{enumerate}

%no longer exists
%f. $g: Z_7 \implies Z_7$ with $g(x)=x\odot 2$\\
%$g(x)$ is one-to-one function because if $x_1\neq x_2$, then we will have $g(x_1)\neq g(x_2)$.\\

\noindent\textbf{Exercise \ref{exercise:functions:atomic_onto}:}\\

\noindent\textbf{Exercise \ref{exercise:functions:45}:}\\
If the function $f$ is onto then the range of $f$ equals the codomain of $f$. There is no element in the codomain that is not in the range.\\

\noindent\textbf{Exercise \ref{exercise:functions:OntoExers-pairs}:}\\
a. It's onto. For any element in the codomain, there is an element in the domain thap maps to it.\\
b. It's not onto. Element $\diamondsuit$ doesn't map to any element in the domain.\\

%\noindent\textbf{Exercise \ref{exercise:functions:}:}\\  %KW unnamed exercise

\noindent\textbf{Exercise \ref{exercise:functions:horizontal_onto}:}\\

\noindent\textbf{Exercise \ref{exercise:functions:OntoExers}:}\\
c. $c(x)=x^2$\\
$c(x)$ is not onto because there is a number $-2 \in R$ in the codomain that doesn't map to any element in the domain.\\
d. $d(x)=3x+2$\\
$d(x)$ is onto because for any element $m$ in the codomain, we can find\\ $x=\displaystyle\frac{m-2}{3}$ in the domain that maps to $m$.\\

\noindent\textbf{Exercise \ref{exercise:functions:OntoExers2}:} %kw updated
%For each of the following  functions, either prove that it is onto, or prove that it is not.
\begin{enumerate}[(a)]
\item
% $g \colon {\mathbb C}  \to {\mathbb C} $ defined by $g(z) =  z^2+1$.

\item
% $g \colon {\mathbb C}\setminus \{0\}  \to {\mathbb C} $ defined by $g(z) =  z^{-1}$ .

\item
% $g \colon {\mathbb C}\setminus \{1\}  \to {\mathbb C}\setminus \{0\} $ defined by $g(z) =  (z-1)^{-1}$ .
We need to prove that given any $z,w\in C\setminus \{0\}$ we need to find a value of $w$ that makes $g(w)=z$. Therefore, $(w-1)^{-1}=z \implies w=(z)^{-1}+1$. Then we get:
\begin{align*}
g(w) &=(w-1)^{-1}\\
&=((z)^{-1}+1-1)^{-1}\\
&=(z^{-1})^{-1}\\
&=z
\end{align*}
$\therefore g(z)=(z-1)^{-1}$ is onto when $g\colon {\mathbb C}\setminus \{1\}\to {\mathbb C}\setminus \{0\}$

\item
% $g \colon {\mathbb R} \times [0,1]  \to {\mathbb C} $ defined by $g(\,(x,y)\,) =  |x| \cis ( 2\pi y)$.
For $z\in \mathbb(C), z$ can be written as $z=a+ib$. Now $z=r(cos(\theta)+isin(\theta))\in [0,\infty]$ and $0\leq \theta \leq 2\pi, \theta=tan^{-1}(\frac{b}{a})$\\
Let $(r,\frac{\theta}{2\pi})=(x,y)$, then:
\begin{align*}
g((x,y))&=g((r,\frac{\theta}{2\pi})) &\\
&=|r|cos(2\pi(\frac{\theta}{2\pi}))+isin(2\pi(\frac{\theta}{2\pi})) & \text{(substituion)} \\
&=|r|(cos(\theta)+isin(\theta)) & \text{(basic arithmetic)}\\
&=z 
\end{align*}
$\therefore g\colon \mathbb{R}\times [0,1]\to \mathbb{C}$ defined by $g(x,y)=|x|cis(2\pi y)$ is onto.
\end{enumerate}

\noindent\textbf{Exercise \ref{exercise:functions:OntoExers3}:} %kw updated
%For each of the following  functions, either prove that it is onto, or prove that it is not.
\begin{enumerate}[(a)]
\item
Not onto
\item \label{modular_m3}
% $g \colon {\mathbb Z}_5 \to {\mathbb Z}_{5}$ defined by $g(x) =  (x \odot 2) \oplus 3$ .
Onto
\item 
% $g \colon {\mathbb Z}_7 \to {\mathbb Z}_7$ defined by $g(x)= (x \odot x) \oplus 1 $ .
The elements in the domain and co-domain are $\{0,1,2,3,4,5,6,7,8\}$\\
Since $g(x)=x\odot x\odot x$, we get\\
$g(0)=g(3)=g(6)=0$ and\\ 
$g(1)=g(4)=g(7)=1$ and\\ 
$g(2)=g(5)=g(8)=8$\\        
Since every element in the co-domain does not get mapped to by an element in the domain, $\mathbb{Z}_{9}\to \mathbb{Z}_{9}$ is defined by $g(x)=x\odot x\odot x$ is not onto.

\item 
% $g \colon {\mathbb Z}_9 \to {\mathbb Z}_9$ defined by $g(x)= x \odot x \odot x$ .
The elements in the domain and co-domain are $\{0,1,2,3,4,5,6\}$\\
Since $g(x)=x\odot x\odot x$, we get\\
$g(0)=g(0)$ and\\ 
$g(6)=0$ and\\ 
$g(1)=g(2)=g(4)=1$ and\\ 
$g(3)=g(5)=g(6)=6$\\
Since every element in the co-domain does not get mapped to by an element in the domain, $\mathbb{Z}_{7}\to \mathbb{Z}_{7}$ is defined by $g(x)=x\odot x\odot x$ is not onto.

\item 
% $g \colon {\mathbb Z}_7 \to {\mathbb Z}_7$ defined by $g(x)= x \odot x \odot x$ .

\end{enumerate}

%\noindent\textbf{Exercise \ref{exercise:functions:}:}\\  %KW unnamed exercise

%\noindent\textbf{Exercise \ref{exercise:functions:}:}\\  %KW unnamed exercise

\noindent\textbf{Exercise \ref{exercise:functions:BijectRtoRExer}:}\\
d. $d(x)=-15x-12$\\
$d(x)$ is one-to-one because if $d(x_1)=d(x_2)$ then $x_1=x_2$.\\
$d(x)$ is onto because for any element y in the codomain, we have $x=\displaystyle\frac{y+12}{-15}$ in the domain that maps to $y$.\\
$\implies d(x)$ is a bijection.\\
e. $e(x)=x^3$\\
Similarly, we have $e(x)$ is both one-to-one and onto $\implies e(x)$ is a bijection.\\

\noindent\textbf{Exercise \ref{exercise:functions:LinearWhenBijectionExer}:} %KW updated
%Let $a,b \in \mathbb{R}$, and define $f \colon \mathbb{R} \to \mathbb{R}$ by $f(x) = a x + b$. 
\begin{enumerate}[(a)]
\item \label{LinearWhenBijectionExer-not0}
%Show that if $a \neq 0$, then $f$ is a bijection.
For this function to be a bijection, there needs to exist an $a,b\in mathbb{R}$ where the function can be one-to-one and onto. First I will prove that the function can be one-to-one, then I will prove that it can be onto.
	\begin{itemize}
	\item 
	(one-to-one): Given $a,b\in \mathbb{R}$, such that $f(x_{1}) = f(x_{2})$, we have
	\begin{align*}
	ax_{1} + b &= ax_{2} + b\\
	\frac{(ax_{1} + b) - b}{a} &= \frac{(ax_{2} + b) - b}{a}
	\end{align*}
	Which implies $x_{1} = x_{2}$, so $f$ is one-to-one
	
	\item
	(onto): Given $y\in \mathbb{R}$ let $x = \frac{y-b}{a}$. Then,
	\begin{center}
	\begin{align*}
	f(x) &= ax + b\\
	&= a(\frac{y-b}{a}) + b\\
	&= y - b + b\\
	&= y
	\end{align*}
	\end{center}
	We need to verify that $x$ is in the domain of $f$ for every $y$ in the co-domain.
	\begin{align*}
	y_{1}\le y\le y_{2}  &\implies y_{1} - b\le y - b\le y_{2} - b \qquad \text{[basic algebra]}\\
	&\implies  \frac{y_{1} - b}{a}\le \frac{y - b}{a}\le \frac{y_{2} - b}{a} \qquad \text{[basic algebra]}
	\end{align*}
	If $a\neq 0$, then $\frac{y_{1} - b}{a} = x_{1}$ and $\frac{y_{2} - b}{a} = x_{2}$ have real solutions.\\
	$\therefore f(x) = ax + b$ for $a,b\in \mathbb{R}$ and when $f\colon \mathbb{R}\to \mathbb{R}$.\\
	Since this function is both one-to-one and onto, it is a bijection.\\
	\end{itemize}

\item \label{LinearWhenBijectionExer-0}
%Show that if $a = 0$, then $f$ is \emph{not} a bijection.
If $a = 0$, then $f(x) = 0x + b = b$. Therefore, all elements in the domain map to the same element in the range. This means that $f(x) = ax + b$ is not one-to-one when $a = 0$ and cannot be a bijection, since a bijection is both one-to-one and onto.  
\end{enumerate}

\noindent\textbf{Exercise \ref{exercise:functions:LinearWhenBijectionExer2}:}\\ %KW updated
%Let $a,b \in \mathbb{R}$, and define $f \colon [1,2] \to [4,7]$ by $f(x) = a x + b$.  Find all values of $a$ and $b$ such that $f$ is a bijection.
We saw in Exercise \ref{exercise:functions:LinearWhenBijectionExer} that the function $f(x) = ax + b$ is a bijection as long as $a\neq 0$. We proved that it is one-to-one and that it is onto if: 
\begin{align*}
\frac{y_{1} - b}{a} &= x_{1} &\text{\ and\ }& &\frac{y_{2} - b}{a} = x_{2}
\end{align*}
Now, let's substitute our co-domain values in for $y_{1}$ and $y_{2}$, substitute our domain values in for $x_{1}$ and $x_{2}$, and solve for $a$ and $b$.
\begin{align*}
\frac{4 - b}{a} &= 1 &\text{\ and\ }& &\frac{7 - b}{a} = 2\\
&&\text{OR}\\
\frac{7 - b}{a} &= 1 &\text{\ and\ }& &\frac{4 - b}{a} = 2
\end{align*}
By basic arithmatic,
\begin{align*}
2 &= \frac{7 - b}{a}\\
&= \frac{3}{a} + \frac{4 - b}{a} \\
&= \frac{3}{a} + 1 &\text{(substitution)}\\
\implies a &= 3 &\text{(basic algebra)}\\
\implies 1 &= \frac{4 - b}{3} &\text{(substitution)}\\
\implies b &= 1 &\text{(basic algebra)}\\
&\text{OR}\\
1 &= \frac{7 - b}{a}\\
&= \frac{3}{a} + \frac{4 - b}{a} \\
&= \frac{3}{a} + 2 &\text{(substitution)}\\
\implies a &= -3 &\text{(basic algebra)}\\
\implies 2 &= \frac{4 - b}{-3} &\text{(substitution)}\\
\implies b &= 10 &\text{(basic algebra)}
\end{align*}

%\noindent\textbf{Exercise \ref{exercise:functions:}:}\\  %KW unnamed exercise

\noindent\textbf{Exercise \ref{exercise:functions:OntoCasesExer}:}\\ 

\noindent\textbf{Exercise \ref{exercise:functions:OntoCasesExer2}:}\\   

\noindent\textbf{Exercise \ref{exercise:functions:OntoCasesExer3}:} %KW still working on the cases
%Define function~$h$ from~$\mathbb{R}$ to~$\mathbb{R}$ by:
%$$ h(x) = 
%\begin{cases}
%x^{3} & \mbox{if $|x| > 1$} \\
%x^{1/3} & \mbox{if $|x| \le 1$}
% . \end{cases} $$
%Prove or disprove:
\begin{enumerate}[(a)]
\item
%$h$ is onto
There are two cases: (i) $|y|>1$; (ii) $|y|\le 1$.  

In case (i), then for arbitrary $|y|>1$ we may define $x = y^{1/3}$.  Since $x \in \mathbb{R}$ and $|x|>1$, it follows that $h(x) = x^3 $. By substitution, $h(x) = (y^{1/3})^3 = y$ by exponent rules. 

In case (ii), then for arbitrary $|y|\le1$ we may define $x = y^3$.  Since $x \in \mathbb{R}$ and $|x|\le 1$, it follows that $h(x) = x^{1/3} $. By substitution, $h(x) = (y^{3})^{1/3} = y$ by exponent rules.
\item
We want to show that $h(x_1)=h(x_2)$ implies $x_1=x_2$ for any $x_1,x_2$ in the domain of $h$.

Suppose $h(x_1)=h(x_2)=y$. There are two cases: (i) $|y|>1$; (ii) $|y|\le 1$. In case (i), Note that $x_2,x_1 >1$ since $|x_1| \le1$ implies $h(x_1) = (x_1)^{1/3}$ which has absolute value less than 1. By the definition of $h$, $|x_2| > 1$ implies that $h(x_2) = x_2^3$ and $|x_1| > 1$ implies that $h(x_1) = x_1^3$. By substitution we have $x_2^3 = x_1^3$, and by algebra it follows that $x_2=x_1$. Case (ii) is similar. 
\end{enumerate}

%\noindent\textbf{Exercise \ref{exercise:functions:}:}\\  %KW unnamed exercise

\noindent\textbf{Exercise \ref{exercise:functions:bijection1}:}\\

\noindent\textbf{Exercise \ref{exercise:functions:NxNBijection}:}\\

\noindent\textbf{Exercise \ref{exercise:functions:NxN}:}\\

\noindent\textbf{Exercise \ref{exercise:functions:ZxZ}:}
%Define $g \colon \mathbb{Z} \times \mathbb{Z} \to \mathbb{Z} \times \mathbb{Z}$ by $g(m,n) = (m + n, m - n)$. 
\begin{enumerate}[(a)]
\item  \label{NxNBijection-mpmn-notonto}  
%*Prove or disprove: $g$ is  onto.
Let's suppose $g$ is onto. Then $\exists$ a $(m,n)\in {\mathbb Z}\times {\mathbb Z}$ with $f((m,n)) = (1,0)$ Therefore,\\
\begin{center}
$g(x) =
\begin{cases}
$m + n = 1$\\
$m - n = 0$\\
\end{cases}$
\end{center}
Add these two equations to get $2m = 1 \implies m = \frac{1}{2}$, which is a contradiction to the statement that $m = 1$. Therefore $g$ is not onto.

\item  \label{NxNBijection-mpmn-11}  
%Prove or disprove: $g$ is one-to-one.
Let $(m_{1},n_1{)},(m_{2},n_{2})\in \mathbb{Z}\times \mathbb{Z}$ with $f((m_{1},n_{1})) = f((m_{2},n_{2}))$. Then
\begin{align*}
(m_{1} + n_{1},m_{1} - n_{1}) &= (m_{2} + n_{2},m_{2} - n_{2})\\
\therefore\ \ 
\begin{cases}
m_{1} + n_{1} &= m_{2} + n_{2} \hspace{1cm}  (1)\\
m_{1} - n_{1} &= m_{2} - n_{2}  \hspace{1cm} (2)\\
\end{cases}
\end{align*}
By adding equations $(1)$ and $(2)$ together we get $2m_{1} = 2m_{2} \implies m_{1} = m_{2}$\\
By subtracting equations $(1)$ and $(2)$ we get $2n_{1} = 2n_{2} \implies n_{1} = n_{2}$\\
Which means $(m_{1},n_{1}) = (m_{2},n_{2})$ and is one-to-one.
        
\item
%Prove or disprove: $g$ is a bijection.
Since $g$ is not onto, it is not a bijection.
\end{enumerate}

\noindent\textbf{Exercise \ref{exercise:functions:AxBxCBijectionEx}:}\\ 

%\noindent\textbf{Exercise 62:}\\
%$f: R\implies R$ by $f(x)=ax+b$ with $a,b \in R$\\
%a. When $a\neq 0$: $f(x)$ is a bijection.\\
%b. When $a=0$: $f(x)$ is not a bijection.\\

\noindent\textbf{Exercise \ref{exercise:functions:RealWorldCompositionExer}:}\\
b. husband $\circ$ mother : father\\
e. mother $\circ$ sister : mother\\
f. daughter $\circ$ sister : niece\\

\noindent\textbf{Exercise \ref{exercise:functions:func_comp_assoc}:}\\
$(h\circ(g\circ f))(x)=h(g(y))=h(w)=z$\\
$((h\circ g)\circ f)(x)=...=z$\\
$\implies (h\circ(g\circ f))(x)=((h\circ g)\circ f)(x)$\\

\noindent\textbf{Exercise \ref{exercise:functions:ComposeDomainMatch}:}\\

\noindent\textbf{Exercise \ref{exercise:functions:ComposeExers-form}:}\\
a. $f(x)=3x+1$ and $g(x)=x^2+2$\\
$(f\circ g)(x)=3x^2+7$\\
$(g\circ f)(x)=9x^2+6x+3$\\
b. $f(x)=3x+1$ and $g(x)=\displaystyle\frac{x-1}{3}$\\
$(f\circ g)(x)=x$\\
$(g\circ f)(x)=x$\\
c. $f(x)=ax+b$ and $g(x)=cx+d$\\
$(f\circ g)(x)=acx+ad+b$\\
$(g\circ f)(x)=cax+cb+d$\\

\noindent\textbf{Exercise \ref{exercise:functions:ComposeExers-pairs}:}\\
a. $g\circ f=\{(1,\clubsuit),(2,\diamondsuit),(3,\heartsuit),(4,\spadesuit)\}$\\
b. $g\circ f=\{(1,\clubsuit),(2,\clubsuit),(3,\clubsuit),(4,\clubsuit)\}$\\


%\begin{exercise}\label{exercise::ComposeExers-form2} 
% The folllowing formulas define functions $f$ and~$g$ from~$\mathbb{C}$ to~$\mathbb{C}$. Find formulas for $f \compose g(x)$ and $g \compose f(x)$.
%\begin{enumerate}[(a)]
%\item \label{composeC1}
% $f(r \cis \theta) = (r+3)\cis(\theta - \pi/6)$ and $g(r \cis \theta) = (r \cis \theta)^2$ 
%\item \label{composeC2}
% $f(a + bi) = 3a + 4bi$ and $g(a + bi) = (a+bi)^2$ 
%\item \label{composeC3}
% $f(r \cis \theta) = \log r + i \theta$ and $g(a+bi) = e^a \cis b$ 
%\item \label{composeC4}
% $f(r \cis \theta) = r^3 \cis (\theta + 2)$ and $g(r \cis \theta) = 2r \cis (\theta + 4)$ 
%\item \label{composeC5}
% $f(z) = |z|$ and $g(z) = -z$ 
%\end{enumerate}
%\end{exercise}
\noindent\textbf{Exercise \ref{exercise:functions:ComposeExers-form2}:}\\
\begin{enumerate}[(a)]
\item 
 $f\circ g(r \cis \theta) = (r^2+3)\cis(2\theta - \pi/6)$ and $g \circ f(r \cis \theta) = (r+3)^2 \cis (2\theta - \pi/3)$ 
\item \label{composeC2}
 $f\circ g(a + bi) = 3(a^2-b^2) + 8abi$ and $g\circ f (a + bi) = (3a+4bi)^2 = (9a^2 - 16b^2) + 24abi$ 
\item \label{composeC3}
 $f \circ g (a+bi) = a + bi$ and $g \circ f(r \cis \theta) = r \cis \theta$ 
\item \label{composeC4}
 $f \circ g(r \cis \theta) = 8r^3 \cis (\theta + 6)$  and $g \circ f(r \cis \theta) = 2r^3 \cis (\theta + 6)$ 
\item \label{composeC5}
 $f\circ g(z) = |z|$ and $g \circ f(z) = -|z|$ 
\end{enumerate}

%\begin{exercise}\label{exercise::ComposeExers-form3} 
% The folllowing formulas define functions $f$ and~$g$ from~$\mathbb{Z}_k$ to~$\mathbb{Z}_k$ for different values of $k$. Find formulas for $f \compose g(x)$ and $g \compose f(x)$.
%\begin{enumerate}[(a)]
%\item \label{composeZ1}
%$f,g:\mathbb{Z}_{15} \rightarrow \mathbb{Z}_{15}$, where  $f(n) = 7 \odot n \oplus 6$ and $g(m) = 6\odot m \oplus 2$ 
%\item \label{composeZ2}
%$f,g:\mathbb{Z}_{25} \rightarrow \mathbb{Z}_{15}$, where  $f(n) = n \odot n$ and $g(m) = m \oplus 3$ 
%\item \label{composeZ3}
%$f,g:\mathbb{Z}_{7} \rightarrow \mathbb{Z}_{7}$, where  $f(n) = 3\odot n \oplus 5$ and $g(m) =4 \odot m \oplus 6$ 
%\item \label{composeZ4}
%$f,g:\mathbb{Z}_{20} \rightarrow \mathbb{Z}_{20}$, where  $f(n) = 4\odot n \oplus 19$ and $g(m) =5 \odot m \oplus 9$ 
%\end{enumerate}
%\end{exercise}
\noindent\textbf{Exercise \ref{exercise:functions:ComposeExers-form3}:}\\
\begin{enumerate}[(a)]
\item 
$f\compose g(m) = 12 \odot m \oplus 5$ and $g \compose f(n) = 12\odot n \oplus 8$
\item 
$f\compose g(m) = m \odot m \oplus 6 \odot m \oplus 9$ and $g \compose f(n) = n\odot n \oplus 3$
\item 
$f\compose g(m) = 5 \odot m \oplus 2$ and $g \compose f(n) = 5\odot n \oplus 5$
\item 
$f\compose g(m) = 15$ and $g \compose f(n) = 4$
\end{enumerate}


\noindent\textbf{Exercise \ref{exercise:functions:CompositionTheoryExers}:} %KW updated
\begin{enumerate}[(a)]
\item \label{CompositionTheoryExers-gof11} 
%Suppose $f \colon A \to B$ and $g \colon B \to C$. Show that if $f$ and~$g$ are one-to-one, then $g \compose f$ is one-to-one. 
  
\item \label{CompositionTheoryExers-f11} 
%Suppose $f \colon A \to B$ and $g \colon B \to C$. Show that if $g \compose f$ is one-to-one, then $f$ is one-to-one.
Assume $g\circ f$ is one to one. If $a_{1},a_{2}\in A$ and $f(a_{1}) = f(a_{2})$, then $g(f(a_{1})) = g(f(a_{2}))$; that is $(g\circ f)(a_{1}) = (g\circ f)(a_{2})$. This implies $a_{1} = a_{2}$, because $g\circ f$ is one-to-one. Therefore, $f$ is one-to-one.

\item \label{CompositionTheoryExers-gonto} 
%Suppose $f \colon A \to B$ and $g \colon B \to C$. Show that if $g \compose f$ is onto, then $g$ is onto.
 
\item \label{CompositionTheoryExers-egNotOnto} 
%Give an example of functions $f \colon A \to B$ and $g \colon B \to C$, such that $g \compose f$ is onto, but $f$ is not onto. 
% (\emph{Hint}: Let $A = B = \real$, $C = [0,\infty)$ and $f(x) = x^2$. Find a $g$ that works, and verify your example.)
If $f\colon \mathbb{R}\to \mathbb{R}$ defined by $f(x) = e^x$ and $f\colon \mathbb{R}\to \mathbb{R}$ defined by
\begin{center}
$g(x) =
\begin{cases}
ln(x)\hspace{1cm} \text{for\ } x>0\\
0 \hspace{1.7cm} \text{for\ } x\leq 0\\
\end{cases}$
\end{center}
then $(g\circ f)(x):\mathbb{R}\to \mathbb{R}$ is  defined by $(g\circ f)(x) = ln(e^{X}) = x$ which is onto.

\item \label{CompositionTheoryExers-fonto} 
%Suppose $f \colon A \to B$ and $g \colon B \to C$. Show that if $g \compose f$ is onto, and $g$~is one-to-one, then $f$ is onto.
Since $g$ is one-to-one, $\forall\  b_{1},b_{2}\in B$ such that $g(b_{1}) = g(b_{2}),\ b_{1} = b_{2}$ and since $g\circ f$ is onto, $\forall\ c\in C\ \exists\ a\in A$ such that $g(f(a))=c$.\\
Since $g\colon B\to C,\ g(b) = c$ and $g(f(a)) = c$ we must conclude that $\forall\ b\in B\ \exists\ a\in A$ such that $f(a) = b$. Therefore $f$ is onto.

\item  \label{CompositionTheoryExers-gNot11} 
%Define $f \colon [0,\infty) \to \real$ by $f(x) = x$. Find a function $g \colon \mathbb{R} \to \mathbb{R}$ such that $g \compose f$ is one-to-one, but $g$ is \emph{not} one-to-one.
% and $g \colon \mathbb{R} \to \mathbb{R}$ by $g(x) = |x|$. Show that $g \compose f$ is one-to-one, but $g$ is \emph{not} one-to-one.
Since $g\colon \mathbb{R}\to \mathbb{R}$ the function $g(x) = |x|$ is not one-to-one because $g(-2) = 2 = g(2)$. But $(g\circ f)(x)\colon [0,\infty)\to \mathbb{R}$ and $(g\circ f)(x) = |x|$ which means that every element in the co-domain of $g\circ f$ gets mapped to by one and only one element of the domain of $g\circ f$ since the domain is only the right half of $|x|$. 
 
\item  \label{CompositionTheoryExers-what} 
%Suppose $f$ and~$g$ are functions from~$A$ to~$A$. If $f(a) = a$ for every $a \in A$, then what are $f \compose g$ and $g \compose f$?
\end{enumerate}

\noindent\textbf{Exercise \ref{exercise:functions:BijectionComposeExer}:} %KW updated
%Suppose $f \colon A \to B$ and $g \colon B \to C$.
\begin{enumerate}[(a)]
\item \label{BijectionComposeExer-gf}
% Show that if $f$ and~$g$ are bijections, then $g \compose f$ is a bijection.
% \hyperref[sec:functions:hints]{(*Hint*)}
To prove that $g\circ f$ is a bijection, we need to show that it is both onto and one-to-one.
	\begin{itemize}        
	\item
	Onto: Both $f$ and $g$ are onto, since they are bijections. To prove $g\circ f$ is onto, we must establish that if $x\in C$, then $\exists$ an element of $x\in A$ such that $(g\circ f)(x) = z$\\
	So let $z\in C$. Because $g$ is onto, $\exists\ y\in B$ such that $g(y) = z$. Since $f$ is also onto, $\exists\ x\in A$ such that $f(x) = y$. Now, we have:
	\begin{align*}
	(g\circ f)(x) = g(f(x)) = g(y) = z
	\end{align*}
	$\therefore\ g\circ f$ is onto.
	 
	\item
	One-to-one: Both $f$ and $g$ are one-to-one. To prove that $g\circ f$ is one-to-one, we shall prove that if $x_{1},x_{2}\in A$ and $(g\circ f)(x_{1}) = (g\circ f)(x_{2})$, then $x_{1} = x_{2}$\\
	Since $g$ is one-to-one, if $(g\circ f)(x_{1}) = (g\circ f)(x_{2})$, then $f(x{_1}) = f(x_{2})$. Therefore, $x_{1} = x_{2}$ since $f$ is one-to-one. Proving that $g\circ f$ is one-to-one.
	
	\item
	Since $g\circ f$ is both onto and one-to-one, it is a bijection when $f$ and $g$ are bijections.  
	\end{itemize}
	  
\item  \label{BijectionComposeExer-g}
%Show that if $f$ and $g\compose f$ are bijections, then $g$ is a bijection.
Since $f$ is a bijection it is one-to-one. Thus $\forall\ a_{1},a_{2}\in A$ such that $f(a_{1}) = f(a_{2}),\ a_{1} = a_{2}$. Since $g\circ f$ a bijection, it is onto, thus $\forall\ c\in C\ \exists\ a\in A$ such that $g(f(a)) = c$.\\
Since $f\colon A\to B,\ f(a) = b$ and $g(f(a)) = c \implies g(b) = c$. Therefore $g$ is onto.

\item \label{BijectionComposeExer-f}
%Show that if $g$ and~$g \compose f$ are bijections, then $f$ is a bijection.
Since $g$ and $g\circ f$ are bijections, $g$ is one-to-one and $g\circ f$ is onto. By Exercise \ref{exercise:functions:CompositionTheoryExers}(e) $f$ is onto. Also, in Exercise \ref{exercise:functions:CompositionTheoryExers}(b) we proved that if $g\circ f$ is one-to-one, $f$ is one-to-one.\\
Since $f$ has been proven to be both onto and one-to-one it is bijective.
\end{enumerate}

\noindent\textbf{Exercise \ref{exercise:functions:InverseMakesBijectionExer}:}\\

\noindent\textbf{Exercise \ref{exercise:functions:88}:}\\ %KW updated
%Prove part (b) of Proposition~\ref{proposition:functions:xgyInverseExer}.
Suppose that  $y = f(x)\Leftrightarrow x = g(y)\ \forall\ x,y$ in the respective domains of $f$ and $g$. Then for any $y\in X$, we may define $z$ as $z = g(y)$. By the $\Leftrightarrow$ statement it follows that $y = f(x)$. But then we may substitute the first equation into the second and obtain $f(g(y)) = f(x) = y$. Since $y$ was an arbitrary element of $X$, it follows that $f(g(y)) = y\ \forall\ x\in X$.\\

\noindent\textbf{Exercise \ref{exercise:functions:conv88}:}\\ %KW updated
%Prove the converse of Proposition~\ref{proposition:functions:xgyInverseExer}. That is, given that \[g \bigl( f(x) \bigr) = x \text{ for all } x \in X \quad \text{and} \quad f \bigl( g(y) \bigr) = y \text{ for all }y \in Y,\] it follows that \[ \forall x \in X, \forall y \in Y, \bigl(y = f(x) \eiff x = g(y) \bigr). \]
$\forall\ x\in X$ let $f(x) = y$, then $x = g(f(x)) = g(y)$.\\
$\therefore\ \forall\ x\in X, y = f(x) \implies x = g(y)$\\
\\  
$\forall\ y\in Y$ let $g(y) = x$, then $y = f(g(y)) = f(x)$.\\
$\therefore\ \forall\ y\in Y, x = g(y) \implies y = f(x)$\\
\\   
$\therefore\ $ given that
\begin{align*}
g\bigl(f(x) \bigr) = x\ \forall\ x \in X \text{\ and\ } f\bigl(g(y)\bigr) = y\ \forall\ y\in Y,
\end{align*}
it follows that
\begin{align*} 
\forall\ x\in X,\ \forall\ y\in Y,\ \bigl(y=f(x) \Leftrightarrow x = g(y)\bigr).
\end{align*} 

\noindent\textbf{Exercise \ref{exercise:functions:VerifyInverseExers}:}\\
a. We have $f(g(x))=x+6-6=x$ and $g(f(x))=...=x$\\
$\implies g$ is an inverse of $f$.\\
c. $f(g(x))=g(f(x))=x \implies g$ is an inverse of $f$.\\

\noindent\textbf{Exercise \ref{exercise:functions:92}:}\\
$g$ has an inverse.\\

\noindent\textbf{Exercise \ref{exercise:functions:InverseBijection2}:}\\

\noindent\textbf{Exercise \ref{exercise:functions:inverMod}:}\\

\noindent\textbf{Exercise \ref{exercise:functions:InverseUniqueExers}:} %KW updated 5.7.13 homework
\begin{enumerate}[(a)]
\item \label{InverseUniqueExers-bij}
%Prove that any inverse of a bijection is a bijection.
Suppose $f\colon A\to B$ is a bijection. Then, the inverse, $g\colon B\to A$ is a bijection if it is both injective and surjective.\\
	\begin{itemize}
	\item 
	Injective: Let $x_{1},x_{2}\in B$ such that $g(x_{1}) = g(x_{2})$. Then:
	\begin{align*}
	x_{1} &= f(g(x_{2})) &\text{(definition of inverse)}\\
	&= f(g(x_{1})) &\text{(applying $f$ to both sides)}\\
	&= x_{2} &\text{(definition of inverse)}
	\end{align*}
	$\therefore\ g:B\to A$ is injective.

	\item 
	Surjective: For some $a\in A$, let $b = f(a)$. Then $g(b) = g(f(a)) = a$.\\
	$\therefore\ g\colon B\to A$ is surjective.
            
	\item
	Since $g\colon B\to A$ is both surjective and injective it is bijective.
	\end{itemize}
	
\item \label{InverseUniqueExers-unique}
%Show that the inverse of a function is \emph{unique}: if $g_1$ and~$g_2$ are inverses of~$f$, then $g_1 = g_2$.
%\hyperref[sec:functions:hints]{(*Hint*)} 
Let $e$ be the identity mapping. Since $g_{1}$ and $g_{2}$ are inverses of $f$:
\begin{align*}
f\circ g_{1} = f\circ g_{2} = g_{1}\circ f = g_{2}\circ f = e
\end{align*}
        
\begin{align*}
g_{1} &= g_{1}\circ e &\text{(definition of identity)}\\
&= g{1}\circ (f\circ g_{2}) &\text{(substitution)}\\
&= (g_{1}\circ f)\circ g_{2} &\text{(associative Law)}\\
&= e\circ g_{2} &\text{(substitution)}\\
&= g_{2} &\text{(definition of identity)}
\end{align*}
\end{enumerate}

\noindent\textbf{Exercise \ref{exercise:functions:IdAInverse}:}
\begin{enumerate}[(a)]
\item
%Show that $\Id_A$ is invertible.\hyperref[sec:functions:hints]{(*Hint*)}
$Id_A$ is one-to-one because if $Id_A(a_1)=Id_A(a_2)$ then $a_1=a_2$.\\
$Id_A$ is onto because for any $a\in$ codomain $A$, we can always have $a\in$ domain $A$ that maps to $a$.\\
$\implies Id_A$ is bijection $\implies Id_A$ is invertible (Prop. 89)

\item
%%Find the inverse of $\Id_A$.\hyperref[sec:functions:hints]{(*Hint*)}
$Id_A$ is its own inverse.
\end{enumerate}

\noindent\textbf{Exercise \ref{exercise:functions:InverseIdentityExers}:} %KW updated, 5.7.17 homework
\begin{enumerate}[(a)]
\item \label{InverseIdentityExers-InvOfComp}
%Suppose $f \colon A \to B$ and $g \colon B \to C$ are bijections. Show that $(g \compose f)^{-1} = f^{-1} \compose g^{-1}$.
%\hyperref[sec:functions:hints]{(*Hint*)}
If a function is invertible, then it has to be a bijective function. Since $f$ and $g$ are bijections, Exercise \ref{exercise:functions:InverseUniqueExers} tells us:
	\begin{itemize}
	\item
	Since $f\colon A\to B\ \exists$ a $f^{-1}\colon B\to A$ which is bijective.\\
	$\therefore\ $ by Definition \ref{def:invfna}, $f(f^{-1}(y)) = y$ and $f^{-1}(f(x)) = x$.
	
	\item
	Since $g\colon B\to C,\ \exists$ a $g^{-1}\colon C\to B$ which is bijective.\\
	$\therefore\ $ by Definition \ref{def:invfna}, $g(g^{-1}(z)) = z$ and $g^{-1}(g(y)) = y$.
	\end{itemize}

By Exercise \ref{exercise:functions:BijectionComposeExer}, since $f$ and $g$ are bijective functions $(f\circ g)\colon A\to C$ is also a bijection.\\
$\therefore\ \exists\ (f\circ g)^{-1}\colon C\to A$.\\
\\
Since the range of $f^{-1}$ is the domain of $g^{-1},\ \exists\ $ a bijective function $(g^{-1}\circ f^{-1})\colon C\to A$\\
\\
To prove $(f\circ g)^{-1} = g^{-1}\circ f^{-1}$ we need to show that $g^{-1}\circ f^{-1}(f\circ g(x)) = x$ and $f\circ g(g^{-1}\circ f^{-1}(z)) = z$\\
\\
Let $g(x) = y$ and $f(y) = z$, then:
	\begin{itemize}
	\item
	\begin{align*}
	g^{-1}\circ f^{-1}(f\circ g(x)) &= g^{-1}\circ (f^{-1}(f(g(x)))) &\text{(Assoc Law)}\\
	&= g^{-1}(f^{-1}(f(y))) &\text{(substitution g(x))}\\
	&= g^{-1}(y) &\text{(substitution)}\\
	&= x
	\end{align*}

	\item
	\begin{align*}
	f\circ g(g^{-1}\circ f^{-1}(z)) &= f(g(g^{-1}(f^{-1}(z)))) &\text{(Assoc. Law)}\\
	&= f(g(g^{-1}(y))) &\text{(substitution $f(y) = z$)}\\
	&= f(y) &\text{(substitution)}\\
	&= z
	\end{align*}
	\end{itemize}
	$\therefore\ (f\circ g)^{-1} = g^{-1}\circ f^{-1}$
	
\item \label{InverseIdentityExers-Comp=Id}
%Suppose $f \colon X \to Y$ and $g \colon Y \to X$. Show that $g$ is the inverse of~$f$ if and only if \[ f \compose g = \Id_Y \text{  and  }g \compose f = \Id_X .\]
%\hyperref[sec:functions:hints]{(*Hint*)}
	\begin{itemize}
	\item 
	First I will prove that given $f\colon X\to Y$ and $g\colon Y\to X$ if $f\circ g = Id_{Y}$ and $g\circ f = Id_{X}$, then $g$ is the inverse of $f$.\\
	\\
	To prove that $g$ is an inverse of $f$ we need to show that $f(g(y)) = y\ \forall\ y\in Y$ and $g(f(x)) = x\ \forall\ x\in X$. By  Proposition~\ref{proposition:functions:xgyInverseExer} and since $Id_{Y}\colon Y\to Y$ by $Id_{Y}(y) = y$ we get:
	\begin{align*}
	Id_{Y} = (f\circ g)(y) = f(g(y)) = y\\
	\end{align*}
	By Proposition~\ref{proposition:functions:xgyInverseExer} and since $Id_{X}\colon X\to X$ by $Id_{X}(x) = x$:
	\begin{align*}
	Id_{X} = (g\circ f)(x) = g(f(x)) = x
	\end{align*}
	Since $f(g(y)) = y\ \forall\ y\in Y$ and $g(f(x)) = x\ \forall\ x\in X$ by Definition \ref{def:invfna} $g$ is the inverse of $f$.
            
	\item
	Next I will prove that given $f\colon X\to Y$ and $g\colon Y\to X$ if $g$ is the inverse of $f$ then, $f\circ g = Id_{Y}$ and $g\circ f = Id_{X}$.\\
	\\           
	Since $g$ is the inverse of $f$, by Definition \ref{def:invfna} and Definition \ref{def:identityMap}, we get for all $y\in Y$:
	\begin{align*}
	(f\circ g)(y) = f(g(y)) = y = Id_{Y}
	\end{align*}
   and for all $x\in X$
	\begin{align*}
	(g\circ f)(x) = g(f(x)) = x = Id_{x}
	\end{align*}
	\end{itemize}
	
\item \label{InverseIdentityExers-InvOfInv}
%Suppose $f \colon X \to Y$ is a bijection. Show that the inverse of~$f^{-1}$ is~$f$. That is, $(f^{-1})^{-1} = f$.
\end{enumerate}

%\noindent\textbf{Exercise \ref{exercise:functions:}:}\\  %KW unnamed exercise

\noindent\textbf{Exercise \ref{exercise:functions:AAclosed}:}\\

\noindent\textbf{Exercise \ref{exercise:functions:GpCompFun}:}\\

\noindent\textbf{Exercise \ref{exercise:functions:abelian_ex}:}\\
