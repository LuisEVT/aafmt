\section{Study guide  for ``Sigma Notation''  chapter}
\label{sec:Sigma:study} 
\subsection*{Section \ref{sec:sigmaExamples}, Lots of examples}
\subsubsection*{Concepts:}
\begin{enumerate}
\item 
Summation notion (sigma notation) -- $\Sigma$ is the symbol used to denote summation it is called sigma
\begin{enumerate}
\item
Index variable -- variable used in the equation that will change and is located beneath the $\Sigma$ symbol
\item
Starting value -- located below the $\Sigma$ and is the value that begins the summation
\item
Final value -- located above the $\Sigma$ and is the last value in the summation
\item
Formula -- located to the right of $\Sigma$, which includes the variable, used to calculate the result
\end{enumerate}
\end{enumerate}

\subsubsection*{Competencies}
\begin{enumerate}
\item
Evaluate expressions given in summation notation. (\ref{exercise:Sigma:Sigma Notation Practice})
\end{enumerate}


\subsection*{Section \ref{sec:sigmaProperties}, Sigma notation properties}
\subsubsection*{Concepts:}
\begin{enumerate}
\item 
Addition and scalar multiplication of sums
\item
Changing the summation index without changing the sum (\ref{exercise:Sigma:sum1})
\end{enumerate}

\subsubsection*{Key formulas}
\begin{enumerate}
\item
Formulas for addition and scalar multiplication of sums: 
\begin{enumerate}
\item
$\displaystyle{\sum_{i=a}^{b} c \cdot d_{i} = c \cdot \sum_{i=a}^{b}  d_{i}}$
\item
$\displaystyle{\sum_{i=0}^{n} \left(x_{i} +y_{i} \right) = \sum_{i=0}^{n} x_{i} + \sum_{i=0}^{n} y_{i}}$
\item
$\displaystyle{\sum_{i=0}^{n} \left(c \cdot x_{i} + d \cdot y_{i} \right) = c \cdot \sum_{i=0}^{n} x_{i} + d \cdot \sum_{i=0}^{n} y_{i}}$
\end{enumerate}
\end{enumerate}

\subsubsection*{Competencies}
\begin{enumerate}
\item
Be able to change the starting value and formula of sigma notations and maintain the same results. (\ref{exercise:Sigma:sum1})
\end{enumerate}


\subsection*{Section \ref{sec:sigmaNested}, Nested sigmas}
\subsubsection*{Concepts:}
\begin{enumerate}
\item 
Nested sigmas -- The entire sum of the inside sigma must be calculated for each value of the index of the outside sigma. \emph{Note} that the index of the outer sum may appear in any or all parts of the inner sum.
\item
Rearranging the order of summation -- exchange the order of the summations and adjust the limits. 
\end{enumerate}

\subsubsection*{Competencies}
\begin{enumerate}
\item
Be able to exchange the order of sums and use other sum manipulation techniques to calculate values of summations. (\ref{exercise:Sigma:nested1}, \ref{exercise:Sigma:nested2})
\end{enumerate}


\subsection*{Section \ref{sec:CommonSums}, Common Sums}
\subsubsection*{Concepts:}
\begin{enumerate}
\item 
Common summation formulas
\item
Geometric series -- sum of non-negative integer powers of a common base
\end{enumerate}

\subsubsection*{Key formulas}
\begin{enumerate}
\item
$\displaystyle{\sum_{i = 1}^{k}i= \frac{k(k + 1)}{2}}$
\item
$\displaystyle{\sum_{i = a}^{k}i} = a + (a + 1) + (a + 2) + \cdots + (k - 1) + k = (k + a) * \frac{k - a + 1}{2}$ where $a$ and $k$ are integers and $a<k$.
\item
\[ \sum_{i=0}^{n-1} ar^i = a \dfrac{1-r^n}{1-r} \]

\end{enumerate}

\subsubsection*{Competencies}
\begin{enumerate}
\item
Be able to write the sum of given integers in sigma notation and give the formula for that sum. (\ref{exercise:Sigma:common1})
\end{enumerate}


\subsection*{Section \ref{sec:sigmaLinAlg}, Sigma notation in linear algebra}
\subsubsection*{Concepts:}
\begin{enumerate}
\item 
Matrix multiplication with sigma notation
\item
Kronecker delta
\item
Abbreviated matrix notations
\item
Matrix transpose
\item
Matrix inverse
\item
Rotation matrices
\item
Matrix traces -- the sum of all the entries on the diagonal
\end{enumerate}

\subsubsection*{Competencies}
\begin{enumerate}
\item
Be able to write the formula for a given entry of a matrix in terms of other matrices. (\ref{exercise:SigmaApp:sigmaAssoc})
\item
Understand the relationship between the Kronecker delta and the identity matrix. Also, how to use it to write matrix equations in summation notation. (\ref{exercise:SigmaApp:KroneckerIdentity})
\item
Be able to write sigma notations in both forms of abbreviated notations. (\ref{exercise:SigmaApp:abbreviated1})
\item
Be able to expand abbreviated notations into unabbreviated expressions. (\ref{exercise:SigmaApp:unabbreviated1})
\item
Be able to express the equations for an identity matrix using summation notation. (\ref{exercise:SigmaApp:inverse1})
\item
Understand the basic properties of traces. (\ref{exercise:SigmaApp:trace1})
\end{enumerate}