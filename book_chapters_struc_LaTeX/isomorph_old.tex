%%%%(c)
%%%%(c)  This file is a portion of the source for the textbook
%%%%(c)
%%%%(c)    Abstract Algebra: Theory and Applications
%%%%(c)    Copyright 1997 by Thomas W. Judson
%%%%(c)
%%%%(c)  See the file COPYING.txt for copying conditions
%%%%(c)
%%%%(c)
\chap{Isomorphisms}{isomorph}

\section{Preliminary examples}\label{isomorph_defn_ex}

Several times in the book so far we have run into the idea of isomorphic groups.  For instance:

\begin{example}\label{example:isomorph:chap1_ex}
In Chapter 1 we pointed out that ${\mathbb C}$ under complex addition and ${\mathbb R} \times {\mathbb R}$ under pointwise addition act exactly the same; that is: 


\noindent
If $a+bi$ and $c+di$ are complex numbers, we can rename them as real ordered pairs

\begin{center}
$a+bi \longrightarrow (a,b)$

$c+di \longrightarrow (c,d)$
\end{center}

\noindent
Then, if we add our two complex numbers and change the result to an ordered pair, 

\begin{center}
$(a+bi) + (c+di) = [(a+c) + (b+d)i] \longrightarrow (a+c, b+d)$
\end{center}

\noindent
we get the same thing as if we had just simply added the two ordered pair versions of the complex numbers

\begin{center}
$(a, b) + (c, d) \longrightarrow (a+c, b+d)$
\end{center}

\noindent
So it doesn't matter whether we add the complex numbers or their corresponding ordered pairs; the result is the same.  The only difference is the label of that result.  We might say that under their respective additions, real ordered pairs and complex numbers are the ``same'' objects, just with  a different face.

Because of this we say that ${\mathbb C}$ and ${\mathbb R} \times {\mathbb R}$ are ``isomorphic groups''. (We'll give a formal definition of ``isomorphic groups'' in the next section.) The method we used of renaming complex numbers as ordered pairs  can be considered as  a function 

\begin{center}
$f : {\mathbb C} \longrightarrow {\mathbb R} \times {\mathbb R}$ such that $f(a + bi) = (a,b)$
\end{center}

\noindent
and this function is called an ``isomorphism'' from ${\mathbb C}$ to  ${\mathbb R} \times {\mathbb R}$.
\end{example}

\begin{exercise}\label{exercise:isomorph:iso_1}
Let $f$ be  the function used in Example 1 to rename complex numbers as ordered pairs. Recall that $r \cis \theta$ is the polar form of a complex number. How would you write $f(r \cis \theta$)?
\end{exercise}

Notice that by saying that ${\mathbb C}$ and ${\mathbb R} \times {\mathbb R}$ are isomorphic, we are expressing that  ${\mathbb C}$ and ${\mathbb R} \times {\mathbb R}$ are ``equivalent" in some sense.  So for example if I redid Example 1 and exchanged the roles of ordered pairs complex numbers, the same result would happen.  This means there has to also be an isomorphism from ${\mathbb R} \times {\mathbb R}$ to ${\mathbb C}$  that shows how we replace ordered pairs with complex numbers, since $f$ in Example 1 only takes complex numbers as inputs and spits out real-ordered pairs.  What would that function be?  If you think about it, the inverse of $f$ would be the natural answer:

\begin{center}
$f^{-1} : {\mathbb R} \times {\mathbb R} \longrightarrow {\mathbb C}$ such that $f(a,b) = (a + bi)$
\end{center}

So our isomorphism $f$ has to have an inverse so that not only ${\mathbb C}$ is isomorphic to ${\mathbb R} \times {\mathbb R}$, but ${\mathbb R} \times {\mathbb R}$ is isomorphic to ${\mathbb C}$.  What does that mean about $f$?  What type of function does it have to be in order to have an inverse?  That's right, a bijection.

\begin{exercise}\label{exercise:isomorph:iso_2}
Prove that the isomorphism in Example 1 is a bijection.
\end{exercise}

\begin{exercise}\label{exercise:isomorph:iso_3} 
Prove that the function $g(a + bi) = (3a, 3b)$ is also an isomorphism from ${\mathbb C}$ to  ${\mathbb R} \times {\mathbb R}$.  That is, prove first that $g$ is a bijection, and then prove that taking $g$ of the sum of two complex numbers agrees with taking $g$ of each complex number separately, then summing the resulting ordered pairs.  You also need to prove that $g$ is a bijection. 
\end{exercise} 

\begin{exercise}\label{exercise:isomorph:iso_4}
Prove that the function $h(a + bi) = (a+2, b+2)$ is {\bf not} an isomorphism from ${\mathbb C}$ to  ${\mathbb R} \times {\mathbb R}$ ({\it Hint:} show a counterexample where the ordered pair corresponding to the sum of two complex numbers is not the same as the sum of the ordered pairs corresponding to the two separate complex numbers )
\end{exercise}


%With that in mind, we have a couple of questions.  Could  ${\mathbb C}$ be isomorphic to ${\mathbb Z} \times {\mathbb Z}$?  No.  Certainly it works right to left:  if we replace any two integer-ordered pairs with their corresponding complex numbers, the sum would be equivalent both ways.  But it does not work left to right.  There are complex numbers, such as $2+1.5i$, that have no integer equivalent; another way of saying this is to say that there are more complex numbers than their are integer-ordered pairs (even though the cardinality of both sets is infinity).  So there would be some complex numbers that we couldn't 

%possible footnote about countably infinite and uncountably infinite here


\begin{example}\label{example:isomorph:sym_ex}
In the Symmetries chapter we also saw some  examples of isomorphic groups.  In particular, we saw that ${\mathbb Z_4}$, the $4^{th}$ roots of unity, and the rotations of a square act exactly the same under modular addition, modular multiplication, and function composition respectively. Let's remind ourselves why. 
The following are the Cayley Tables for ${\mathbb Z_4}$, the $4^{th}$ roots of unity (which we'll denote by $\langle i \rangle$), and the rotations of a square ($R_4$):

\begin{table}[H]
\caption{Cayley table for ${\mathbb Z}_4$}
\label{Z4_add_table}
{\small
\begin{center}
\begin{tabular}{c|cccccccc}
$\oplus$ & 0 & 1 & 2 & 3  \\
\hline
0        & 0 & 1 & 2 & 3  \\
1       & 1 & 2 & 3 & 0  \\
2       & 2 & 3 & 0 & 1 \\
3       & 3 & 0 & 1 & 2 \\

\end{tabular}
\end{center}
}
\end{table}

\begin{table}[H]
\caption{Cayley table for $\langle i \rangle$}
\label{4_roots_table}
{\small
\begin{center}
\begin{tabular}{c|cccccccc}
$\cdot$ & 1 & i & -1 & -i  \\
\hline
1        & 1 &$i$ & -1 &$-i$  \\
i       &$i$ & -1 & $-i$ & 1  \\
-1       & -1 & $-i$ & 1 & $i$ \\
-i       & $-i$ & 1 & $i$ & -1 \\

\end{tabular}
\end{center}
}
\end{table}

\begin{table}[H]
\caption{Cayley table for the $R_4$}
\label{4_rotations_table}
{\small
\begin{center}
\begin{tabular}{c|cccccccc}
$\circ$ & $\var{id}$ & $r_{90}$ & $r_{180}$ & $r_{270}$  \\
\hline
$\var{id}$        & $\var{id}$ & $r_{90}$ & $r_{180}$ & $r_{270}$  \\
$r_{90}$       & $r_{90}$ & $r_{180}$ & $r_{270}$ & $\var{id}$  \\
$r_{180}$       & $r_{180}$ & $r_{270}$ & $\var{id}$ & $r_{90}$ \\
$r_{270}$       & $r_{270}$ & $\var{id}$ & $r_{90}$ & $r_{180}$ \\

\end{tabular}
\end{center}
}
\end{table}

\begin{enumerate}[(1)]
\item
Comparing ${\mathbb Z_4}$ and $\langle i \rangle$, notice that if we identify 

\begin{align*}
  0 \leftrightarrow 1 \qquad
    1 \leftrightarrow i \qquad
    2 \leftrightarrow -1 \qquad
    3 \leftrightarrow -i, 
\end{align*}
    
then the Cayley tables of each are in fact exactly equal to each other.  Which means that if you add any two elements in ${\mathbb Z_4}$ (say 1 and 2), and then multiply their corresponding elements in $\langle i \rangle$ ($i$ and -1), your results from each of these actions are in fact the same (3 and $-i$).

Hence the function $f: {\mathbb Z_4} \longrightarrow \langle i \rangle$  that takes 
\begin{align*}
    0 \longrightarrow 1 ,~~     1 \longrightarrow i,~~    2 \longrightarrow -1,~~   3 \longrightarrow -i  
\end{align*}
 is an isomorphism from ${\mathbb Z_4}$ to the $4^{th}$ roots of unity, and these groups are isomorphic to each other.

\item
Now if we compare $\langle i \rangle$ and $R_4$, using the function  $g: \langle i \rangle \longrightarrow R_4$  defined by
\begin{align*}
1 \longrightarrow \var{id}, ~~\
i \longrightarrow  r_{90},~~
-1 \longrightarrow r_{180},~~
 -i \longrightarrow r_{270}, 
\end{align*}
we see that their Cayley tables are in fact exactly the same.  Hence the $4^{th}$ roots of unity and the rotations of a square are isomorphic to each other, and $g$ is an isomorphism between them.  

\item
Finally, using the  function
$h: {\mathbb Z_4} \longrightarrow R_4$  that takes
\begin{align*}
 0\longrightarrow \var{id},~~
    1 \longrightarrow r_{90},~~
    2 \longrightarrow r_{180},~~
    3 \longrightarrow r_{270}, 
\end{align*}
we see that the Cayley tables for ${\mathbb Z_4}$ and $R_4$ are exactly the same.  Hence ${\mathbb Z_4}$ and the rotations of a square are isomorphic to each other, and $h$ is an isomorphism between them.
\end{enumerate}

\noindent
So ${\mathbb Z_4}$, $R_4$, and $\langle i \rangle$ are all isomorphic to each other. Mathematically we state this as follows:

\[ {\mathbb Z_4} \cong  R_4 \cong \langle i \rangle \]
 
 \end{example}
 
 \begin{exercise}\label{exercise:isomorph:iso_5}
 Prove that each of the functions from the last example are bijections.
 \end{exercise}
 
 \begin{exercise}\label{exercise:isomorph:iso_6}
 Come up with a \emph{different} isomorphism for each pairing of groups in Example \ref{example:isomorph:sym_ex}.
 \end{exercise}
 
 \begin{exercise}\label{exercise:isomorph:iso_7}
 Determine whether each of the following functions are isomorphisms between the groups in Example~\ref{example:isomorph:sym_ex}. Justify your answers. 
 \begin{enumerate}[(a)]
 \item
 $ \phi: {\mathbb Z_4}  \longrightarrow \langle i \rangle$ defined by 
\[  
\phi(0) = 1,~~
 \phi(1) = -1,~~
 \phi(2) = i,~~
 \phi(3) = -i.
\]
 
 \item
$ \theta: {\mathbb Z_4}  \longrightarrow R_4$ defined by 
\[ 
\theta(0) = \var{id},~~
 \theta(1) = r_{270},~~
 \theta(2) = r_{90} ,~~
 \theta(3) = r_{180}. 
 \]

 \item
 $ \omega: \langle i \rangle \longrightarrow R_4$ defined by 
\[ 
 \omega(1) = \var{id},~~ 
\omega(i) = r_{270},~~
\omega(-1) = r_{180},~~
 \omega(-i) = r_{90}.
\]
 \end{enumerate}
 \end{exercise}
 
\section{Formal definitions}
So let's buckle down and get mathematical. We start with a rigorous definition of isomorphic groups:

\begin{defn}\label{isomorph_defn}
Two groups $(G, \cdot)$ and $(H, \circ)$ are \bfii{ isomorphic\/}\index{Group!isomorphic} if there exists a bijection $\phi : G \rightarrow H$ such that the group operation is preserved;  that~is, 
\[
\phi( a \cdot b) = \phi( a) \circ \phi( b)
\]
for all $a$ and $b$ in $G$. If $G$ is isomorphic to $H$, we write $G \cong H$\label{noteisomorph}. The function $\phi$ is called an \bfii{ isomorphism}\index{Group!isomorphism of}\index{Isomorphism!of groups}. 
\end{defn}

Some important properties of isomorphisms follow directly from this definition. First we have:

\begin{thm}\label{IsoId}
Given that  $\phi : G \rightarrow H$ is an  isomorphism, then $\phi$ takes the identity to the identity: that is, if $e$ is the identity of $G$, then  $\phi(e)$ is the identity of $H$.
\end{thm}

\begin{exercise}
Fill in the blanks in the following proof of Proposition~\ref{IsoId}:
\medskip

\noindent
Given that $e$ is the identity of $ \_\_\_\_$ and $h$ is an arbitrary element of $\_\_\_\_$.  Since $\phi$ is a bijection, then there exists $g \in \_\_\_\_$ such that $\phi(\_\_\_\_) = h$.  Then  we have:
\begin{align*}
\phi(e) \circ h &= \phi(e) \circ \phi(\_\_\_\_) & \textrm{(substitution)}\\
&= \phi( e \cdot \_\_\_\_) & \textrm{(definition of \_\_\_\_\_\_\_\_\_\_\_\_\_\_\_\_)}\\
&= \phi( \_\_\_\_) & \textrm{(definition of \_\_\_\_\_\_\_\_\_\_\_\_\_\_\_\_)}\\
&= h & \textrm{(substitution)}
\end{align*}
Following the same steps, we can also show
\begin{align*}
h \circ \phi(e) = \_\_\_\_\_.
\end{align*}
It follows from the definition of identity that $\_\_\_\_\_$ is the identity of the group $\_\_\_\_$.
\end{exercise}


Another important property of isomorphisms is:

\begin{thm}\label{IsoInv}
Given that  $\phi : G \rightarrow H$ is an  isomorphism, then $\phi$ preserves the operation of inverse: that is, for any $g \in G$ we have
\begin{equation*}
\phi(g^{-1}) = (\phi(g))^{-1}.
\end{equation*}
\end{thm}

\begin{exercise}
Fill in the blanks in the following proof of Proposition~\ref{IsoInv}:
\medskip

\noindent
Let $e$ and $f$ be the identities of $G$ and $H$, respectively. Given that $g \in \_\_\_\_\_$, we have:
\begin{align*}
\phi(g) \circ \phi(g^{-1}) &= \phi(g \cdot g^{-1}) & \textrm{(definition of}~\_\_\_\_\_ \_\_\_\_\_ \_\_\_\_\_)\\
&= \phi(e) &\textrm{(definition of}~\_\_\_\_\_ \_\_\_\_\_ \_\_\_\_\_)\\
&= f &\textrm{(Proposition}~\_\_\_\_\_ ).
\end{align*}
Using the same steps, we can also show
\begin{align*}
\phi(g^{-1}) \circ \phi(g) = \_\_\_\_.
\end{align*}
By the definition of inverse, it follows that
\begin{align*}
( \phi(g))^{-1} = \_\_\_\_.
\end{align*}
\end{exercise} 

It's possible to use isomorphisms to create other isomorphisms:

\begin{exercise}\label{exercise:isomorph:InvCompIso}
\begin{enumerate}[(a)]
\item
Given that  $\phi : G \rightarrow H$ is an  isomorphism, show that that  $\phi^{-1} : H \rightarrow G$ is also an  isomorphism.  (\emph{Hint}: According to Definition~\ref{isomorph_defn}, this involves proving two things about $\phi^{-1}$.  What are they?)
\item
Given that  $\phi : G \rightarrow H$ and $\psi : H \rightarrow K$ is an  isomorphism, show that that  $\psi \circ\phi:G \rightarrow K$ is also an  isomorphism.  (\emph{Hint}: You need to prove the same two things as in part (a).  Use results from the Functions chapter.)
\end{enumerate}
\end{exercise}

We said in the previous section that isomorphic groups are ``equivalent'' in some sense. This fact has a formal mathematical statement as well:

\begin{thm}\label{GpEquivRel}
Isomorphism is an equivalence relation on groups. 
\end{thm}

\begin{exercise}
Prove Proposition~\ref{GpEquivRel}.  (\emph{Hint}: Recall that this involves proving the three properties: reflexive, symmetric, and transitive. You may find that Exercise~\ref{exercise:isomorph:InvCompIso}  is useful.)
\end{exercise}

\section{More Examples}\label{iso_more_ex}

Now that we have a formal definition of what it means for two groups to be isomorphic, let's look at some more examples, in order to get a good feel for identifying groups that are isomorphic and those that aren't.
%In the previous examples we only looked at isomorphisms between finite groups.  This simplified our proof process in a couple ways.  One, in showing that the isomorphism was a bijection, all we had to do was verify it visually with the arrow diagram we created, because the arrow diagram contained all the information needed to show that the function was one-to-one and onto.  Two, to show that the function preserved the operations of the respective groups, all we needed was to verify it visually with the Cayley tables, since again the Cayley tables contained all the information needed.

%We were able to take advantage of visual tools like these because the groups we were dealing with were finite (and relatively small in number), and therefore we could represent all the information we needed concisely in a "picture".  However, if we were proving that two \emph{infinite} groups were isomorphic, it would be physically impossible to construct the required arrow diagrams or Cayley tables.  For that matter, if the two groups were finite but large in number, though you could construct the arrow diagrams and Cayley tables, the time required for such an endeavor would make the proof process very long.

%So in such cases, it is necessary to prove our two properties of isomorphisms--that they are bijections and that they preserve the group operations--using the functional and algebraic proofs we introduced in Chapters 4 and 9.  For instance:

From high school and college algebra we are well familiar with the fact that when you multiply exponentials (with the same bases), the result of this operation is the same as if you had just kept the base and added the exponents.  This equivalence of operations is a telltale sign for identifying possible isomorphic groups.  The next two examples illustrate this observation.

For our first example, we   denote the set of integer powers of 2 as $2^{\mathbb Z}$, that is:
\[ 2^{\mathbb Z} \equiv \{\ldots, 2^{-2}, 2^{-1}, 2^0, 2^1, 2^2, \ldots\}. \]
\begin{exercise}
Show that $2^{\mathbb Z}$ with the operation of multiplication is a subgroup of ${\mathbb Q}^{\ast}$.
\end{exercise}

\begin{example}\label{example:isomorph:rational_isomorph}
When elements of $2^{\mathbb Z}$ are multiplied together, their exponents add: we know this from basic algebra. This suggests there should be an isomorphism between ${\mathbb Z}$ and   $2^{\mathbb Z}$. In fact, we may define the function
$\phi: {\mathbb Z} \rightarrow 2^{\mathbb Z}t$ by $\phi( n ) = 2^n$.
To show that this is indeed an isomorphism, by our definition we must show two things: (a)  that the function preserves the operations of the respective groups; and (b) that the function is a bijection:
\begin{enumerate}[(a)] 
\item
We may compute
\[
\phi( m + n ) = 2^{m + n} = 2^m 2^n = \phi( m ) \phi( n ).
\]
\item
By definition the function $\phi$ is onto the subset $\{2^n :n \in {\mathbb Z} \}$ of  ${\mathbb Q}^\ast$.  To show that the map is injective, assume that $m \neq n$.  If we can show that $\phi(m) \neq \phi(n)$, then we are done.  Suppose that $m>n$ and assume that $\phi(m) = \phi(n)$.  Then $2^m = 2^n$ or $2^{m-n} = 1$, which is impossible since $m-n>0$. 
\end{enumerate}

Hence $ {\mathbb Z} \cong 2^{\mathbb Z} \}$.
\end{example}      

 
\begin{example}\label{example:isomorph:RealIsomorph}
As in the previous example, the real powers of $e$ under multiplication acts exactly like addition of those real exponents.  
This suggests that the function $f(x)=e^x$ is an isomorphism between an additive group and a multiplicative group.  The reader will complete this proof of this fact as an exercise.  
\end{example}

\begin{exercise}\label{exercise:isomorph:e_isomorph_proof}
\begin{enumerate}[(a)]
\item
What is the domain and range of $f$?
\item
Using calculus, prove that $f(x)$ is a bijection.
\item
Prove that $f(x)$ preserves the operations of the two groups; that $f(x + y) = f(x)f(y)$.
\item
Now that we know $f(x)$ is an isomorphism, what can we conclude about $({\mathbb R},+)$ and ${\mathbb R}^+,\cdot)$?
\end{enumerate}
\end{exercise}

\begin{exercise}
\begin{enumerate}[(a)]
\item
What is the domain and range of the natural logarithm function $\ln(x)$?
\item
Using the results of the previous exercise and a result from earlier in this chapter, show that the natural logarithm function is an isomorphism. What are the two isomorphic groups?
\item
* Using the fact that $\log_{10}(x) = \ln(x) / \ln(10)$, show that the base 10 logarithm function is also an isomorphism.  What are the two isomorphic groups?
\item 
* Explain how the fact that $\log_{10}(x)$ is an isomorphism enables us to find the product of any two positive real numbers using addition and a base-10 logarithm table. 
\end{enumerate}
\end{exercise}

\begin{exercise}\label{exercise:isomorph:iso_prac1}
Prove that ${\mathbb Z} \cong n{\mathbb Z}$ for $n \neq 0$.
\end{exercise}
 
\begin{exercise}\label{exercise:isomorph:iso_prac2}
Prove that ${\mathbb C}^\ast$ is isomorphic to the subgroup of $GL_2(
{\mathbb R} )$ consisting of matrices of the form 
\[
\begin{pmatrix}
a & b \\
-b & a
\end{pmatrix}
\]
\end{exercise}

We have seen that isomorphic groups bear a strong similarity to one another. They can actually be considered ``equivalent'', in some sense:

\begin{exercise}\label{exercise:isomorph:isomorph_equiv_reln}
Show that isomorphism is an equivalence relation on the set of all groups. 
\end{exercise}

In some cases, it is easy to show that two groups are \emph{not} isomorphic to each other.

\begin{example}\label{example:isomorph:nonisom}
Consider the groups ${\mathbb Z}_8$ and ${\mathbb Z}_{12}$. Can you tell right away that there can't be an isomorphism between them?  Remember, an isomorphism is a one-to-one and onto function: but since 
$|{\mathbb Z}_{12}|>|{\mathbb Z}_{8}|$ there is no onto function from ${\mathbb Z}_8$ to ${\mathbb Z}_{12}$, and so they can not be isomorphic to each other.  Similarly it can be shown that any two finite groups that have differing numbers of elements cannot be isomorphic to each other.
\end{example}
%\[
%\phi( x + y) = e^{x + y} = e^x e^y = \phi( x ) \phi( y).
%\]

\begin{example}\label{example:isomorph:units}
Let us look now at  the group of units of ${\mathbb Z}_8$ and the group of units of ${\mathbb Z}_{12}$; i.e. $U(8)$ and $U(12)$.  We have seen that these  consist of the elements in ${\mathbb Z}_8$ and ${\mathbb Z}_{12}$, that are relatively prime to $8$ and $12$, respectively, so

%cannot be isomorphic since they have different orders; however, it is true that   
\begin{align*}
U(8) & = \{1, 3, 5, 7 \} \\
U(12) & = \{1, 5, 7, 11 \}.
\end{align*}

\begin{exercise}\label{exercise:isomorph:U8_U12_Cayley}
Give the Cayley tables for $U(8)$ and $U(12)$.
\end{exercise}

An isomorphism $\phi : U(8) \rightarrow U(12)$ is then given by
\begin{align*}
1 & \mapsto  1 \\
3 & \mapsto  5 \\
5 & \mapsto  7 \\
7 & \mapsto  11.
\end{align*}

$\phi$ is one-to-one and onto by observation, and we can verify visually that $\phi$ preserves the operations of $U(8)$ and $U(12)$ by the Cayley tables you gave.  Hence $U(8) \cong U(12)$.
\end{example}

\begin{exercise}\label{exercise:isomorph:U8_U12_other}
The function $\phi$ is not the only possible isomorphism between $U(8)$ and $U(12)$.  Define another isomorphism between $U(8)$ and $U(12)$.
\end{exercise}

\begin{exercise}\label{exercise:isomorph:U8_U12_Z2Z2}
Prove that both $U(8)$ and $U(12)$ are isomorphic to ${\mathbb Z}_2 \times {\mathbb Z}_2$ (recall $Z_2 \times Z_2$ is the set of all pairs $(a,b)$ with $a,b \in Z_2$, where 
the group operation is multiplication mod 2 on each element in the pair). 
\end{exercise}

 \begin{exercise}\label{exercise:isomorph:iso_prac3}
Prove that $U(8)$ is isomorphic to the group of matrices
\[
\begin{pmatrix}
1 & 0 \\
0 & 1
\end{pmatrix},
\begin{pmatrix}
1 & 0 \\
0 & -1
\end{pmatrix},
\begin{pmatrix}
-1 & 0 \\
0 & 1
\end{pmatrix},
\begin{pmatrix}
-1 & 0 \\
0 & -1
\end{pmatrix}.
\]
\end{exercise} 

\begin{exercise}\label{exercise:isomorph:iso_prac4}
Show that the matrices
\begin{gather*}
\begin{pmatrix}
1 & 0 & 0 \\
0 & 1 & 0 \\
0 & 0 & 1
\end{pmatrix}
\quad
\begin{pmatrix}
1 & 0 & 0 \\
0 & 0 & 1 \\
0 & 1 & 0
\end{pmatrix}
\quad
\begin{pmatrix}
0 & 1 & 0 \\
1 & 0 & 0 \\
0 & 0 & 1
\end{pmatrix} \\
\begin{pmatrix}
0 & 0 & 1 \\
1 & 0 & 0 \\
0 & 1 & 0
\end{pmatrix}
\quad
\begin{pmatrix}
0 & 0 & 1 \\
0 & 1 & 0 \\
1 & 0 & 0
\end{pmatrix}
\quad
\begin{pmatrix}
0 & 1 & 0 \\
0 & 0 & 1 \\
1 & 0 & 0
\end{pmatrix}
\end{gather*}
form a group. Find an isomorphism of $G$ with a more familiar group of
order~6.
\end{exercise} 

Above we have seen two examples where  Cayley tables make it easy to show that two groups are isomorphic. (Of course, this works best if the groups are not too large, and it certainly doesn't work if the groups are infinite!)  Let us now consider whether it is possible to use Cayley tables to show when groups are \emph{not} isomorphic to each other: 

\begin{example}\label{example:isomorph:Cayley_noniso}
The following are the Cayley tables for ${\mathbb Z}_4$ and $U(5)$.

\begin{table}[H]
\caption{Cayley table for ${\mathbb Z}_4$}
\label{Z4_add_table}
{\small
\begin{center}
\begin{tabular}{c|cccccccc}
$\oplus$ & 0 & 1 & 2 & 3  \\
\hline
0        & 0 & 1 & 2 & 3  \\
1       & 1 & 2 & 3 & 0  \\
2       & 2 & 3 & 0 & 1 \\
3       & 3 & 0 & 1 & 2 \\

\end{tabular}
\end{center}
}
\end{table}

\begin{table}[H]
\caption{Cayley table for $U(5)$\label{U5_table}}
{\small
\begin{center}
\begin{tabular}{c|cccccccc}
$\odot$ & 1 & 2 & 3 & 4  \\
\hline
1        & 1 & 2 & 3 & 4  \\
2       & 2 & 4 & 1 & 3  \\
3       & 3 & 1 & 4 & 2 \\
4       & 4 & 3 & 2 & 1 \\

\end{tabular}
\end{center}
}
\end{table}

Notice that the main diagonals (left to right) of the Cayley tables seem to have a different pattern.  The main diagonal for ${\mathbb Z}_4$ is the alternating sequence, $0, 2, 0, 2$, while the main diagonal of $U(5)$ is the  non-alternating sequence $1, 4, 4, 1$.  It appears at first sight that these two groups must be non-isomorphic.  However, we may rearrange the row and column labels  in Table~\ref{U5_table} to obtain Table~\ref{U5_table2}. From the rearranged table we may read off the isomorphism: $0 \rightarrow 1, 1\rightarrow 2, 2\rightarrow 4, 3 \rightarrow 3$.

\begin{table}[H]
\caption{Rearranged Cayley table for $U(5)$\label{U5_table2}}

{\small
\begin{center}
\begin{tabular}{c|cccccccc}
$\odot$ & 1 & 2 & 4 & 3  \\
\hline
1        & 1 & 2 & 4 & 3  \\
2       & 2 & 4 & 3 & 1  \\
4       & 4 & 3 & 1 & 2 \\
3       & 3 & 1 & 2 & 4 \\

\end{tabular}
\end{center}
}
\end{table}

Note the important point that when we rearranged the table, we used the \emph{same} ordering $(1,2,4,3)$ for both rows and columns.   You don't want to use one ordering for rows, and a different ordering for columns.
\end{example} 
We conclude that it is more difficult to use Cayley tables to prove non-isomorphism, because we have to consider all possible rearrangements of the table. However, in some cases we can still use this method.

\begin{exercise}\label{exercise:isomorph:another_pattern}
\begin{enumerate}[(a)]
\item
Give the Cayley table for $U(12)$. What do you notice about the diagonal entries?
\item
If you rearranged the rows and columns of this Cayley table (always using the same ordering for the rows as for columns) then what happens to the diagonal entries?
\item
Explain why we can use this result to conclude that ${\mathbb Z}_4 \ncong U(12)$.
\end{enumerate}
\end{exercise} 

\begin{exercise}\label{exercise:isomorph:iso_prac5}
Prove or disprove: $U(8) \cong {\mathbb Z}_4$.
\end{exercise}
  
\begin{exercise}\label{exercise:isomorph:iso_prac6}
Let $\sigma$ be the permutation $(12)$, and let $\tau$ be the permutation $(34)$.
Let $G$ be the set $\{ \var{id}, \sigma, \tau, \sigma\tau \}$ together with the operation of composition.
\begin{enumerate}[(a)]
\item
Give the Cayley table for the group $G$.
\item
Prove or disprove: $G \cong {\mathbb Z}_4$.
\item
Prove or disprove: $G \cong U(12)$.
\end{enumerate}
\end{exercise}



%$\psi$ by $\psi(1) = 1$, $\psi(3) = 11$, $\psi(5) = 5$, $\psi(7) = 7$. In fact, both of these groups are isomorphic to 

\begin{example}\label{example:isomorph:not_isomorph_abelian}
Even though $D_3$ and ${\mathbb Z}_6$ possess the same number of elements, we might suspect that they are not isomorphic, because ${\mathbb Z}_6$ is abelian and $D_3$ is non-abelian.  Let's see if the  Cayley tables can help us here:

\begin{table}[H]
{\small
\begin{center}
\begin{tabular}{c|cccccc}
$\circ$  & $\var{id}$     & $\rho_1$ & $\rho_2$ & $\mu_1$ & $\mu_2$ & $\mu_3$ \\
\hline
$\var{id}$     & $\var{id}$     & $\rho_1$ & $\rho_2$ & $\mu_1$ & $\mu_2$ & $\mu_3$ \\
$\rho_1$ & $\rho_1$ & $\rho_2$ & $\var{id}$     & $\mu_3$ & $\mu_1$ & $\mu_2$ \\
$\rho_2$ & $\rho_2$ & $\var{id}$     & $\rho_1$ & $\mu_2$ & $\mu_3$ & $\mu_1$ \\
$\mu_1$  & $\mu_1$  & $\mu_2$  & $\mu_3$  & $\var{id}$    & $\rho_1$& $\rho_2$\\
$\mu_2$  & $\mu_2$  & $\mu_3$  & $\mu_1$  & $\rho_2$& $\var{id}$    & $\rho_1$\\
$\mu_3$  & $\mu_3$  & $\mu_1$  & $\mu_2$  & $\rho_1$& $\rho_2$& $\var{id}$
\end{tabular}
\end{center}
}
\caption{Cayley table for $D_3$}
\label{D3_table}
\end{table}

\begin{table}[H]
\caption{Cayley table for ${\mathbb Z}_6$}
\label{Z6_add_table}
{\small
\begin{center}
\begin{tabular}{c|cccccccc}
$\oplus$ & 0 & 1 & 2 & 3 & 4 & 5  \\
\hline
0        & 0 & 1 & 2 & 3 & 4 & 5  \\
1       & 1 & 2 & 3 & 4 & 5 & 0  \\
2       & 2 & 3 & 4 & 5 & 0 & 1\\
3       & 3 & 4 & 5 & 0 & 1 & 2 \\
4       & 4 & 5 & 0 & 1 & 2 & 3 \\
5       & 5 & 0 & 1 & 2 & 3 & 4 \\

\end{tabular}
\end{center}
}
\end{table}

Note that the Cayley table for ${\mathbb Z}_6$ is symmetric across the main diagonal while the Cayley table for $D_3$ is not.  Furthermore, no matter how we rearrange the row and column headings for the Cayley table for ${\mathbb Z}_6$, the  table will always be symmetric. It follows that there is no way to to match up the two groups' Cayley tables: so $D_3 \ncong {\mathbb Z}_6$.

This  argument via Cayley table works in the case where the two groups being compared are both small, but if the groups are large then it's far too time-consuming (especially if the groups are infinite!). So let us take a different approach, and fall back on our time-tested strategy of proof by contradiction. In the case at hand, this means that we first suppose that $D_3 \cong {\mathbb Z}_6$, and  then find a contradiction based on that supposition. 

So, suppose that the two groups are isomorphic, which means there exists an isomorphism $\phi : {\mathbb Z}_6 \rightarrow  D_3$.  Let $a , b \in D_3$ be two elements such that $a\circ b \neq b \circ a$.  Since $\phi$ is an isomorphism, there exist elements $m$ and $n$ in ${\mathbb Z}_6$ such~that 
\[
\phi( m )  = a \quad \text{and} \quad
\phi( n )  = b.
\]
However,
\[
a\circ b = \phi(m ) \circ  \phi(n) = \phi(m \oplus  n) = \phi(n \oplus m) = \phi(n ) \circ \phi(m) = b \circ a,
\]
which contradicts the fact that $a$ and $b$ do not commute.
\end{example}

Although we have only proven the non-isomorphicity of abelian and non-abelian groups for one particular case, the same  method of proof can be used to prove the following general result.
%Notice that in the general method for proving $D_3 \ncong {\mathbb Z}_6$, the elements $a, b, m$ and $n$ used were completely general and could have been picked from any two groups, one abelian and one not.  Therefore, that proof would work just as well for proving that any abelian group couldn't be isomorphic to a non-abelian group.  Hence 

\begin{thm}\label{abelian_non-abelian}
If $G$ is an abelian group and $H$ is a non-abelian group, then $G \ncong H$.
\end{thm}

\begin{exercise}
Prove Proposition ~\ref{abelian_non-abelian} by imitating the proof in Example ~\ref{example:isomorph:not_isomorph_abelian}.
\end{exercise}

\begin{exercise}\label{exercise:isomorph:iso_prac7}
Prove $D_4 \ncong {\mathbb Z}_8$.
\end{exercise}


Finally, let's look at ${\mathbb Z}$ and ${\mathbb R}$.  We know ${\mathbb Z}$ is a cyclic group with $1$ as the generator, while ${\mathbb R}$ is not cyclic. (Do you remember why?)  We might suspect  that  ${\mathbb Z} \ncong {\mathbb R}$, since one group is cyclic and the other isn't.  This is in fact true, and we'll prove it.   Since ${\mathbb Z}$ and ${\mathbb R}$ are infinite groups though, we can't use Cayley tables, so we have to use another method (three guessess as to what it is):

\begin{thm}\label{not_isomorph_cyclic}
${\mathbb Z}$ is not isomorphic to ${\mathbb R}$.
\end{thm}
\begin{proof}
We will use a proof by contradiction.   Suppose that there exists an isomorphism $\omega: {\mathbb Z} \rightarrow {\mathbb R}$.  Choose any $x \in {\mathbb R}$, and let $m \in {\mathbb Z}$ be the pre-image of $x$, that is $\omega(m) = x$.  It follows that: 

\begin{align*}
x &= \omega(m) = \omega(\underbrace{1 + \ldots + 1}_{m~\textrm{times}})= \underbrace{\phi(1) + \ldots + \phi(1)}_{m~\textrm{times}}.
\end{align*}
Thus $x \in \langle  \phi(1) \rangle$.  But since this is true for \emph{any} $x \in {\mathbb R}$, this means that  $\phi(1)$ is a generator of  ${\mathbb R}$, which means that ${\mathbb R}$ is cyclic. But we've already seen that 
${\mathbb R}$ is \emph{not} cyclic. This contradiction shows that our original supposition must be false: namely, there \emph{cannot} exist an isomorphism $\omega: {\mathbb Z} \rightarrow {\mathbb R}$. This completes the proof.
\end{proof}

Again we can generalize this proof to prove that no cyclic group can be isomorphic to a non-cyclic group.

\begin{thm}\label{cyclic_noncyclic}
Given any group $G$ that is cyclic and any group $H$ that is not cyclic, $G \ncong H$.
\end{thm}

\begin{exercise}
Prove Proposition ~\ref{cyclic_noncyclic}. (\emph{Hint}: Imitate the proof of Proposition~\ref{not_isomorph_cyclic}.)
\end{exercise}

\begin{exercise}\label{exercise:isomorph:noniso_cyclic}
Prove that ${\mathbb Q}$ is not isomorphic to ${\mathbb Z}$.
\end{exercise}

\section{Properties of isomorphisms}\label{iso_properties}
In the last two sections we  proved several properties of isomorphic groups and their corresponding isomorphisms.  We collect these properties (and add a few more) in the following proposition:

\begin{thm}\label{isomorph_theorem_1}
Let $\phi : G \rightarrow H$ be an isomorphism of two groups.  Then the following statements are true. 
\begin{enumerate}[(1)]
 

\rm \item 
$|G| = |H|$. 

\rm \item 
$\phi^{-1} : H \rightarrow G$ is an isomorphism. 

\rm \item 
$G$ is abelian if and only if $H$ is abelian. 

\rm \item 
$G$ is cyclic if and only if $H$ is cyclic. 

\rm \item 
$G$ has a subgroup of order $n$ if and only if $H$ has a subgroup of order $n$.
 
\end{enumerate}
\end{thm}

\begin{proof}
Assertion (1) follows from the fact that $\phi$ is a bijection.  The proofs of (2)--(5) are indicated in the following exercises.

\begin{exercise}
\begin{enumerate}[(a)]
\item 
Show part (2) of Proposition~\ref{isomorph_theorem_1}. (\emph{Hint}: Use Exercise~\ref{exercise:isomorph:InvCompIso}.)
\item
Show part (3) of Proposition~\ref{isomorph_theorem_1}. (\emph{Hint}: Use Proposition~\ref{abelian_non-abelian}. )
\item
Show part (4) of Proposition~\ref{isomorph_theorem_1}. (\emph{Hint}: Use Proposition~\ref{cyclic_noncyclic}. )
\end{enumerate}
\end{exercise} 

\begin{exercise} Fill in the blanks of the following proof of part (5) of Proposition~\ref{isomorph_theorem_1}.
The proof has two parts:

\begin{enumerate}
\item
If $G$ has a subgroup of order $n$, then $H$ also has a subgroup of order $\_\_\_$;
\item
If $H$ has a subgroup of order $\_\_\_$, then $G$ also has a subgroup of order $n$.
\end{enumerate}

To prove (a), let us suppose that $G'$ is a subgroup of $G$ and $|G'| = n$. Then $\phi(G')$ is a subset of $\_\_\_\_$. Note that 
$|\phi(G')| = |G'| = n$ since $\phi$ is a $\_\_\_\_\_\_\_\_\_\_\_\_$. 

We claim that $\phi(G')$ is actually a subgroup of $H$.  To show this, by Proposition~\ref{proposition:groups:subgroup_prove_2} it's enough to show that if $h_1$ and $h_2$ are elements of  $\phi(G')$, then $h_1 h_2^{-1}$ is also an element of $\_\_\_\_\_\_\_\_\_\_\_\_$.

Now given that  $h_1, h_2 \in \phi(G')$, by the definition of $\phi(G')$ it must be true that there exist $g_1, g_2 \in \_\_\_\_\_$ such that $\phi(g_1) = h_1, \phi(g_2) = h_2$. But then we have 
\begin{align*}
h_1 h_2^{-1} &= \phi(g_1) \phi(g_2)^{-1} &\text{(by substitution)}\\
&= \phi(g_1) \phi(g_2^{-1}) &(\text{by Proposition~}\_\_\_\_\_\_\_\_\_\_\_\_)\\
&= \phi(g_1 g_2^{-1}) &(\text{by the definition of }\_\_\_\_\_\_\_\_\_\_\_\_).
\end{align*}
Since $g_1 g_2^{-1}$ is an element of $G'$, it follows that $h_1 h_2^{-1} \in \_\_\_\_\_\_\_\_\_\_\_\_$.
\medskip

To prove (b), we notice that by part (2) of the proposition we have that $\_\_\_\_\_\_\_\_\_\_\_\_$ is an isomorphism from $H$  to $G$. So we can apply part (a) using the  isomorphism $\_\_\_\_\_\_\_\_\_\_\_\_$ instead of $\phi$. This gives us the same conclusion as in (a), except that $G$ and $H$ are switched.
\end{exercise}

\end{proof}

\begin{exercise}
\begin{enumerate}[(a)]
\item
Find groups of order $2, 3, 4, 5,$ and $6$.
Why aren't any of these groups isomorphic to each other?
\item
Find all the subgroups of ${\mathbb Z}_3 \times {\mathbb Z}_3$. Use this
information to show that ${\mathbb Z}_3 \times {\mathbb Z}_3$ is not isomorphic to ${\mathbb Z}_9$. 
\item
Prove $S_4$ is not isomorphic to $D_{12}$.
\item
Find five non-isomorphic groups of order 8, and show that they are not isomophic.
\end{enumerate}
\end{exercise}

\section{Classification up to isomorphism}
We have been emphasizing that two groups that are isomorphic are the ``same'' as far as all group properties are concerned. So if we can characterize a class of groups as isomorphic to a well-understood set of groups, then all of the properties of the well-understood groups carry over to the entire class of groups. We will see two examples of this in the following subsections.

\subsection{Classifying cyclic groups}

Our first classification result concerns cyclic groups.

\begin{thm}\label{isomorph_theorem_2}
All cyclic groups of infinite order are isomorphic to ${\mathbb Z}$.
\end{thm}

\begin{proof}
Let $G$ be a cyclic group with infinite order and suppose that $a$ is a generator of $G$.  Define a map $\phi : {\mathbb Z} \rightarrow  G$ by $\phi : n \mapsto a^n$. Then 
\[
\phi( m+n ) = a^{m+n} = a^m a^n = \phi( m ) \phi( n ).
\]
To show that $\phi$ is injective, suppose that $m$ and $n$ are two elements in ${\mathbb Z}$, where $m \neq n$.  We can assume that $m > n$.  We must show that $a^m \neq a^n$. Let us suppose the contrary; that is, $a^m = a^n$. In this case $a^{m - n} = e$, where $m - n>0$, which contradicts the fact that $a$ has infinite order.  Our map is onto since any element in $G$ can be written as $a^n$ for some integer $n$ and $\phi(n) = a^n$.   
\end{proof}

\begin{exercise}\label{exercise:isomorph:cyclic_inf_isomorph}
Using Proposition ~\ref{isomorph_theorem_2}, prove again that $2^{\mathbb Z} \cong {\mathbb Z}$.
\end{exercise}

\begin{exercise}\label{exercise:isomorph:cyclic_inf_isomorph2}
Prove again that $n{\mathbb Z} \cong {\mathbb Z}$ for $n \neq 0$.
\end{exercise}

\begin{thm}\label{isomorph_theorem_3}
If $G$ is a cyclic group of order $n$, then $G$ is isomorphic to~${\mathbb Z}_n$.  
\end{thm}
 
\begin{proof}
Let $G$ be a cyclic group of order $n$ generated by $a$ and define a map $\phi : {\mathbb Z}_n \rightarrow  G$ by $\phi : k \mapsto a^k$, where $0 \leq k < n$. The proof that $\phi$ is an isomorphism is left as the next exercise. 
\end{proof}

\begin{exercise}\label{exercise:isomorph:phi_cyclic}
Prove that $\phi$ from Proposition ~\ref{isomorph_theorem_3} is an isomorphism.
\end{exercise}

\begin{exercise}\label{exercise:isomorph:cyclic_n_isomorph}
Show that the $n$th roots of unity are isomorphic to ${\mathbb Z}_n$.
\end{exercise} 

\begin{corollary}\label{isomorph_theorem_4}
If $G$ is a  group of order $p$, where $p$ is a prime number, then $G$ is isomorphic to ${\mathbb Z}_p$. 
\end{corollary}

\begin{proof}
This  is a direct result of Corollary~\ref{cosets_theorem_7}.
\end{proof}
 
\medskip

\subsection{Characterization of all finite groups (Cayley's theorem) }
 
In the previous section, we saw that any cyclic group is ``equivalent'' (in the sense of isomorphism) to one of the groups $\mathbb{Z}_n$.  This enables us to easily conceptualize any cyclic group in terms of a standardized group that we're very familiar with. 

Can we do something similar with \emph{all} 
groups? In other words, can we find  a standardized set of groups so that any group can be characterized as identical to one of these standard groups?. 

In a way we already have a standardized characterization of finite groups, because we have seen that every finite group can be represented with a Cayley table.  But this is not really satisfactory, because there are many Cayley tables which do not correspond to any group.

\begin{exercise}\label{exercise:isomorph:CayleyNotGroup}
Give examples of Cayley tables for binary operations that meet each of the following criteria.  (You can make your row and column labels be the set of integers $\{1,2,..n\}$, for an appropriate value of $n$.
\begin{enumerate}
\item
The binary operation has no identity.
\item
The binary operation has an identity, but not inverses for every element
\item
*The binary operation has an identity and inverses, but  the associative law fails.
\end{enumerate}
\end{exercise} 

Although Cayley tables are not adequate for our purpose, it turns out that they provide the key to the characterization we're seeking. Consider first the following simple example.

\begin{example}\label{example:isomorph:cayley_isomorph}
The Cayley table for ${\mathbb Z}_3$ is  
\begin{center}
\begin{tabular}{c|ccc}
$\oplus$   & 0 & 1 & 2 \\
\hline
0     & 0 & 1 & 2 \\
1     & 1 & 2 & 0 \\
2     & 2 & 0 & 1
\end{tabular}
\end{center}
The addition table of ${\mathbb Z}_3$ suggests that it is the isomorphic to the permutation group $S_2 = \{ \var{id}, (0 1 2), (0 2 1) \}$.  One possible isomorphism  is 
\begin{align*}
0 & \mapsto
\begin{pmatrix}
0 & 1 & 2 \\
0 & 1 & 2
\end{pmatrix}
= \var{id} \\
1 & \mapsto
\begin{pmatrix}
0 & 1 & 2 \\
1 & 2 & 0
\end{pmatrix}
= (0 1 2) \\
2 & \mapsto
\begin{pmatrix}
0 & 1 & 2 \\
2 & 0 & 1
\end{pmatrix}
= (0 2 1).
\end{align*}
Notice the interesting ``coincidence'' that  the rows of the Cayley table ( $(0~1~2), (1~2~0)$ and $2~1~0)$ respectively)  ``just happen'' to agree exactly with the second rows of the three tableaus! 

Of course, this ``coincidence'' is no accident.
For example, the second row of the Cayley table is obtained as $(1\oplus 0~1\oplus 1~1\oplus 2)$, and the permutation $\begin{pmatrix}
0 & 1 & 2 \\
1 & 2 & 0
\end{pmatrix}$ 
that is the isomorphic image of 1 is actually the function from ${\mathbb Z}_3 \rightarrow {\mathbb Z}_3$ that takes $n$ to $1 \oplus n$.   The following proposition is basically a generalization of this simple observation.

\end{example}


\begin{thm}[Cayley]\index{Cayley's Theorem}\label{isomorph_theorem_6}
Every group is isomorphic to a group of permutations.
\end{thm}

\begin{proof}
Let $G$ be a group with $|G|$ elements.  We seek a group of permutations $P \subset S_{|G|}$ that is isomorphic to $G$.  For any $g \in G$ we may  define a  function $\sigma_g : G \rightarrow G$ by 
\[
\sigma_g(a) := ga.
\]
  We claim that $\sigma_g$ is a permutation on $G$: you will show this in Exercise~\ref{exercise:isomorph:finish_proof} below.  Let us define the set $P\subset S_{|G|}$ as
\[
P = \{ \sigma_g : g \in G \}.
\]
Let us now define a function $\Phi: G \rightarrow P$ just as we did in Example~\ref{example:isomorph:cayley_isomorph}:
\[ \Phi(g) := \sigma_g. \]
To show that $\Phi$ is an isomorphism, we must show that $\Phi$ is one-to-one, onto, and preserves the group operation. 
You will show that $\Phi$ is one-to-one and onto in Exercise~\ref{exercise:isomorph:finish_proof} below. To show that $\Phi$ preserves the group operation, we need to show that $\Phi(gh) = \Phi(g) \circ \Phi(h)$ for any elements $g, h \in G$. We may show this element-by-element: that is, we show that $\Phi(gh)(a) = (\Phi(g) \circ \Phi(h))(a)$ for an arbitrary $a \in G$ as follows:
\begin{align*}
 \Phi(gh)(a) &= (gh)a & [\text{definition of } \Phi(gh)]\\
&= g(ha) & [\text{associativity of }G]\\
 &= g(\Phi(h)(a)) & [\text{definition of } \Phi(h)]\\
 &=\Phi(g) \circ \Phi(h)(a). & [\text{definition of } \Phi(g)]
 \end{align*}
\end{proof}
\medskip

\begin{exercise}\label{exercise:isomorph:finish_proof}
\begin{enumerate}[(a)]
\item
Show that $\sigma_g : G \rightarrow G$ defined in the above proof is a permutation on $G$. (It is enough to show that $\sigma_g$ is one-to-one and onto. 
\item
Complete the proof of Proposition~\ref{isomorph_theorem_6} by showing that $\Phi:G \rightarrow P$ is one-to-one and onto. 
\end{enumerate}
\end{exercise}

The isomorphism $\Phi: G \rightarrow S_{|G|}$ defined in the proof is known as the \bfii{ left regular representation\/}\index{Left regular representation}\index{Regular representation!left} of~$G$. This is not the only possible isomorphism. There is another isomorphism called the \bfii{ right regular representation\/}\index{Right regular representation}\index{Regular representation!right}, which is presented in the following exercise.

\begin{exercise}\label{exercise:isomorph:r_reg}
The right regular representation  $\Psi: G \rightarrow S_{|G|}$  is defined as follows. For any $g \in G$ define the  function $\tilde{\sigma}_g : G \rightarrow G$ by 
\[
\tilde{\sigma}_g(a) := ag^{-1}.
\]
Define the set $\tilde{P}$ as
\[
\tilde{P} = \{ \tilde{\sigma}_g : g \in G \},
\]
and define the function $\tilde{\Phi}: G \rightarrow P$ as
\[ \tilde{\Phi}(g) := \tilde{\sigma}_g. \]
\begin{enumerate}[(a)]
\item
Show that $\tilde{\sigma}_g : G \rightarrow G$ defined in the above proof is a permutation on $G$. (It follows that the set $\tilde{P}$ is a subset of $S_{|G|}$.)
\item
Show that $\tilde{\Phi}:G \rightarrow \tilde{P}$ is one-to-one and onto. 
 \item
Complete the proof that $G \cong \tilde{P}$ by showing that $\tilde{\Phi}$ preserves the group operation, that is: $\tilde{\Phi}(gh) = \tilde{\Phi}(g) \circ \tilde{\Phi}i(h)$ for any elements $g, h \in G$.
\item
Give the isomorphism  $\tilde{\Phi}$ for the group $\mathbb{Z}_3$.  Is this isomorphism different from $\Phi$ defined on the same group?
\end{enumerate}
\end{exercise}

\histhead

\noindent{\small \histf
Arthur Cayley\index{Cayley, Arthur} was born in England in 1821, though he spent much of the first part of his life in Russia, where his father was a merchant.  Cayley was educated at Cambridge, where he took the first Smith's Prize in mathematics.  A lawyer for much of his adult life, he wrote several papers in his early twenties before entering the legal profession at the age of 25.  While practicing law he continued his mathematical research, writing more than 300 papers during this period of his life.  These included some of his best work.  In 1863 he left law to become a professor at Cambridge.  Cayley wrote more than 900 papers in fields such as group theory, geometry, and linear algebra. His legal knowledge was very valuable to Cambridge; he participated in the writing of many of the university's statutes.  Cayley was also one of the people responsible for the admission of women to Cambridge. 
\histbox
} 
 

\section{Direct Products}\label{isomorph_section_2}

Given two groups $G$ and $H$, it is possible to construct a new group from the Cartesian product of $G$ and $H$, $G \times H$.  Conversely, given a large group, it is sometimes possible to decompose the group; that is, a group is sometimes isomorphic to the direct product of two smaller groups.  Rather than studying a large group $G$, it is often easier to study the component groups of $G$. 
 
 
\subsection{External Direct Products}

If $(G,\cdot)$ and $(H, \circ)$ are groups, then we can make the Cartesian product of $G$ and $H$ into a new group.  As a set, our group is just the ordered pairs $(g, h) \in G \times H$ where $g \in G$ and $h \in H$. We can define a binary operation on $G \times H$ by 
\[
(g_1, h_1)(g_2, h_2) = (g_1 \cdot g_2, h_1 \circ h_2);
\]
that is, we just multiply elements in the first coordinate as we do in $G$ and elements in the second coordinate as we do in $H$.  We have specified the particular operations $\cdot$ and $\circ$ in each group here for the sake of clarity; we usually just write $(g_1, h_1)(g_2, h_2) = (g_1  g_2, h_1 h_2)$.  

\begin{thm}\label{isomorph_theorem_7}
Let $G$ and $H$ be groups. The set $G \times H$ is a group under the operation $(g_1, h_1)(g_2, h_2) = (g_1  g_2, h_1 h_2)$ where $g_1, g_2 \in G$ and $h_1, h_2 \in H$. 
\end{thm}
The proof is outlined in the following exercise. (\emph{Hint}: for each group property to be proved, use the corresponding group property for $G$ and $H$.)

\begin{exercise}\label{exercise:isomorph:proof_thm_7}
\begin{enumerate}
\item
Show that the set $G \times H$ is closed under the binary operation defined in Proposition~\ref{isomorph_theorem_7}. 
\item Show that $(e_G, e_H)$ is the identity of $G \times H$,
where $e_G$ and $e_H$ are the identities of the groups $G$ and $H$ respectively.
\item Show that the inverse of $(g, h) \in G \times H$ is $(g^{-1}, h^{-1})$.  
\item Show that the operation defined in Proposition~\ref{isomorph_theorem_7} is associative.
\end{enumerate}
\end{exercise}

\begin{example}\label{example:isomorph:R2_prodiuct}
Let ${\mathbb R}$ be the group of real numbers under addition.  The Cartesian product of ${\mathbb R}$ with itself, ${\mathbb R} \times {\mathbb R} = {\mathbb R}^2$, is also a group, in which the group operation is just addition in each coordinate; that is, $(a, b) + (c, d) = (a + c, b + d)$.  The identity is $(0,0)$ and the inverse of $(a, b)$ is $(-a, -b)$.
\end{example}

\begin{example}\label{example:isomorph:Z2xZ2}
Consider
\[
{\mathbb Z}_2 \times {\mathbb Z}_2 = \{ (0, 0), (0, 1), (1, 0),(1, 1) \}.
\]
Although ${\mathbb Z}_2 \times {\mathbb Z}_2$ and ${\mathbb Z}_4$ both contain four elements, it is easy to see that they are not isomorphic since for every element $(a,b)$ in ${\mathbb Z}_2 \times {\mathbb Z}_2$, $(a,b) + (a,b) = (0,0)$, but ${\mathbb Z}_4$ is cyclic.
\end{example}

The group $G \times H$ is called the \bfii{ external direct product\/}\index{Direct product of groups!external}\index{External direct product} of  $G$ and $H$. Notice that there is nothing special about the fact that we have used only two groups to build a new group. The direct product
\[
\prod_{i = 1}^n G_i = G_1 \times G_2 \times \cdots \times G_n
\]
of the groups $G_1, G_2, \ldots, G_n$ is defined in exactly the same manner. If $G = G_1 = G_2 = \cdots = G_n$, we often write $G^n$ instead of $G_1 \times G_2 \times \cdots \times G_n$.
 
\begin{example}\label{example:isomorph:Z2^n}
The group ${\mathbb Z}_2^n$, considered as a set, is just the set of all
binary $n$-tuples. The group operation is the ``exclusive or'' of two
binary $n$-tuples. For example, 
\[
(01011101) + (01001011) = (00010110).
\]
This group is important in coding theory, in cryptography, and in many
areas of computer science.  
\end{example}

 
\begin{thm}\label{isomorph:lcm_theorem}
Let $(g, h) \in G \times H$. If $g$ and $h$ have finite orders $r$ and
$s$ respectively, then the order of $(g, h)$ in $G \times H$ is the
least common multiple of $r$ and $s$. \index{Isomorphism!least common multiple theorem}
\end{thm}

 
\begin{proof}
Suppose that $m$ is the least common multiple of $r$ and $s$ and let
$n = |(g,h)|$. Then 
\begin{gather*}
(g,h)^m  = (g^m, h^m) = (e_G,e_H) \\
(g^n, h^n)  = (g, h)^n = (e_G,e_H).
\end{gather*}
Hence, $n$ must divide $m$, and $n \leq m$.  However, by the second
equation, both $r$ and $s$ must divide $n$; therefore, $n$ is a common
multiple of $r$ and $s$. Since $m$ is the {\em least common multiple\/}
of $r$ and $s$, $m \leq n$.  Consequently, $m$ must be equal to~$n$.
\end{proof}
 

\begin{corollary}
Let $(g_1, \ldots, g_n) \in \prod G_i$. If $g_i$ has finite order
$r_i$ in $G_i$, then the order of $(g_1, \ldots, g_n)$ in $\prod G_i$
is the least common multiple of $r_1, \ldots, r_n$.
\end{corollary}
 
 
\begin{example}\label{example:isomorph:Z12xZ60}
Let $(8, 56) \in {\mathbb Z}_{12} \times  {\mathbb Z}_{60}$. Since
$\gcd(8,12) = 4$, the order of 8 is $12/4 = 3$ in ${\mathbb Z}_{12}$.
Similarly, the order of $56$ in ${\mathbb Z}_{60}$ is $15$. The least
common multiple of 3 and 15 is 15; hence, $(8, 56)$ has order 15 in
${\mathbb Z}_{12} \times  {\mathbb Z}_{60}$.
\end{example}

 
\begin{example}\label{example:isomorph:Z2xZ3}
The group ${\mathbb Z}_2 \times {\mathbb Z}_3$ consists of the pairs
\[
\begin{array}{cccccc}
(0,0),& (0, 1),& (0, 2),& (1,0),& (1, 1),& (1, 2).
\end{array}
\]
In this case, unlike that of ${\mathbb Z}_2 \times {\mathbb Z}_2$ and
${\mathbb Z}_4$, it 
is true that ${\mathbb Z}_2  \times {\mathbb Z}_3 \cong {\mathbb Z}_6$. We need
only show that ${\mathbb Z}_2  \times {\mathbb Z}_3$ is cyclic.  It is
easy to see that $(1,1)$ is a generator for ${\mathbb Z}_2  \times {\mathbb
Z}_3$. 
\end{example}

 
The next proposition tells us exactly when the direct product of two
cyclic groups is cyclic. 
 

\begin{thm}\label{Z_pq_theorem}
The group ${\mathbb Z}_m \times {\mathbb Z}_n$ is isomorphic to ${\mathbb
Z}_{mn}$ if and only if $\gcd(m,n)=1$. 
\end{thm}
 

\begin{proof}
Assume first that if ${\mathbb Z}_m \times {\mathbb Z}_n \cong {\mathbb
Z}_{mn}$, then $\gcd(m, n) = 1$. To show this, we will prove the
contrapositive; that is, we will show that if $\gcd(m, n) = d >
1$, then ${\mathbb Z}_m \times {\mathbb Z}_n$ cannot be cyclic. Notice that
$mn/d$ is divisible by both $m$ and $n$; hence, for any element $(a,b)
\in {\mathbb Z}_m \times {\mathbb Z}_n$,  
\[
\underbrace{(a,b) + (a,b)+ \cdots + (a,b)}_{mn/d \; {\rm
times}}
= (0, 0).
\]
Therefore, no $(a, b)$ can generate all of ${\mathbb Z}_m \times {\mathbb
Z}_n$. 

 
The converse follows directly from Proposition~\ref{isomorph:lcm_theorem} since
$\lcm(m,n) = mn$ if and only if $\gcd(m,n)=1$. 
\end{proof}
 

\begin{corollary}\label{RelativelyPrime}
Let $n_1, \ldots, n_k$ be positive integers. Then
\[
\prod_{i=1}^k {\mathbb Z}_{n_i} \cong {\mathbb Z}_{n_1 \cdots n_k}
\]
if and only if $\gcd( n_i, n_j) =1$ for $i \neq j$.
\end{corollary}

 
\begin{corollary}
If
\[
m = p_1^{e_1} \cdots  p_k^{e_k},
\]
where the $p_i$s are distinct primes, then
\[
{\mathbb Z}_m \cong {\mathbb Z}_{p_1^{e_1}} \times \cdots \times {\mathbb
Z}_{p_k^{e_k}}.
\]
\end{corollary}
 
 
\begin{proof}
Since the greatest common divisor of $p_i^{e_i}$ and $p_j^{e_j}$ is
1 for $i \neq j$, the proof follows from Corollary~\ref{RelativelyPrime}.
\end{proof}


\medskip


It turns out that all finite abelian groups are 
isomorphic to direct products of
the form
\[
{\mathbb Z}_{p_1^{e_1}} \times \cdots \times {\mathbb
Z}_{p_k^{e_k}}
\]
where $p_1, \ldots, p_k$ are (not necessarily distinct) primes.

 
 
\subsection{Internal Direct Products}
 

The external direct product of two groups builds a large group out of
two smaller groups.   We would like to be able to reverse this process
and conveniently break down a group into its direct product
components; that is, we would like to be able to say when a group is
isomorphic to the direct product of two of its subgroups.
 

Let $G$ be a group with subgroups $H$ and $K$ satisfying the following
conditions.
\begin{itemize}
 
\item
$G = HK = \{ hk : h \in H, k \in K  \}$;
 
\item
$H \cap K = \{ e \}$;
 
\item
$hk = kh$ for all $k \in K$ and $h \in H$.
 
\end{itemize}
Then $G$ is the \bfii{ internal direct product\/}\index{Direct product of
groups!internal}\index{Internal direct product} of $H$ and $K$.

 
\begin{example}\label{example:isomorph:U8}
The group $U(8)$ is the internal direct product of
\[
H  = \{1, 3 \} \quad \text{and} \quad K  = \{1, 5 \}.
\]
\end{example}

 
\begin{example}\label{example:isomorph:D6_product}
The dihedral group $D_6$ is an internal direct product of its two
subgroups 
\[
H  = \{\var{id}, r^3  \} \quad \text{and} \quad
K  = \{\var{id}, r^2, r^4, s, r^2s, r^4 s   \}.
\]
It can easily be shown that $K \cong S_3$; consequently, $D_6 \cong
{\mathbb Z}_2 \times S_3$. 
\end{example}

 
\begin{example}\label{example:isomorph:S3_not_a_product}
Not every group can be written as the internal direct product of two
of its proper subgroups.  If the group $S_3$ were an internal direct
product of its proper subgroups $H$ and $K$, then one of the  subgroups,
say $H$, would have to have order 3. In this case $H$ is the subgroup $\{
(1), (123), (132) \}$. The subgroup $K$ must have order 2, but no
matter which subgroup we choose for $K$, the condition that $hk = kh$
will never be satisfied for $h \in H$ and $k \in K$.
\mbox{\hspace{1in}}
\end{example}

 
\begin{thm}
Let $G$ be the internal direct product of  subgroups $H$ and $K$. Then
$G$ is isomorphic to $H \times K$. 
\end{thm}
 

\begin{proof}
Since $G$ is an internal direct product, we can write any element $g
\in G$ as $g =hk$ for some $h \in H$ and some $k \in K$. Define a map
$\phi : G \rightarrow H \times K$ by $\phi(g) = (h,k)$.

 
The first problem that we must face is to show that $\phi$ is a
well-defined map; that is, we must show that $h$ and $k$ are uniquely
determined by $g$. Suppose that $g = hk=h'k'$. Then $h^{-1} h'= k
(k')^{-1}$ is in both $H$ and $K$, so it must be the identity.
Therefore, $h = h'$ and $k = k'$, which proves that $\phi$ is, indeed,
well-defined. 

 
To show that $\phi$ preserves the group operation, let $g_1 = h_1 k_1$
and $g_2 = h_2 k_2$ and observe that 
\begin{align*}
\phi( g_1 g_2 ) & = \phi( h_1 k_1 h_2 k_2 )\\
& = \phi(h_1  h_2 k_1 k_2) \\
& = (h_1  h_2, k_1 k_2) \\
& = (h_1, k_1)( h_2, k_2) \\
& = \phi( g_1 ) \phi(  g_2 ).
\end{align*}
We will leave the proof that $\phi$ is one-to-one and onto
as an exercise.
\end{proof}

 
\begin{example}\label{example:isomorph:Z6_product}
The group ${\mathbb Z}_6$ is an internal direct product isomorphic to $\{
0, 2, 4\} \times \{ 0, 3 \}$. 
\end{example}

 
We can extend the definition of an internal direct product of $G$ to a
collection of subgroups $H_1, H_2, \ldots, H_n$ of $G$, by requiring
that 
\begin{itemize}
 
\item
$G = H_1 H_2 \cdots H_n = \{ h_1 h_2 \cdots h_n : h_i \in H_i \}$;
 
\item
$H_i \cap \langle \cup_{j \neq i} H_j \rangle = \{ e \}$;
 
\item
$h_i h_j = h_j h_i$ for all $h_i \in H_i$ and $h_j \in H_j$.
 
\end{itemize}
We will leave the proof of the following proposition as an exercise. 
 
\begin{thm}
Let $G$ be the internal direct product of subgroups $H_i$, where $i =
1, 2, \ldots, n$. Then $G$ is isomorphic to $\prod_i H_i$. 
\end{thm}

 


 
\markright{EXERCISES}
\section*{Additional Exercises}
\exrule

 
 
{\small
\begin{enumerate}[(1)]
 
%**********************Computations
 
\item
Let $\omega = \cis(2 \pi /n)$ be a primitive $n$th root of
unity.  Prove that the matrices 
\[
A=
\begin{pmatrix}
\omega & 0 \\
0 & \omega^{-1}
\end{pmatrix}
\quad \text{and} \quad
B =
\begin{pmatrix}
0 & 1 \\
1 & 0
\end{pmatrix}
\]
generate a multiplicative group isomorphic to $D_n$.
 

\item
Show that the set of all matrices of the form
\[
B =
\begin{pmatrix}
\pm 1 & n \\
0 & 1
\end{pmatrix},
\]
where $n \in {\mathbb Z}_n$, is a group isomorphic to $D_n$. 


\item
Let $G = {\mathbb R} \setminus \{ -1 \}$ and define a binary operation on
$G$ by 
\[
a \ast b = a + b + ab.
\]
Prove that $G$ is a group under this operation. Show that $(G, *)$ is
isomorphic to the multiplicative group of nonzero real numbers.
  
% TWJ, 2010/04/21
% Made correction to exercise at the suggestion of C. Thon


\item
List all of the elements of ${\mathbb Z}_4 \times {\mathbb Z}_2$.
 

\item
Find the order of each of the following elements.

\begin{enumerate}
 
 \item
$(3, 4)$ in ${\mathbb Z}_4 \times {\mathbb Z}_6$

 \item
$(6, 15, 4)$ in ${\mathbb Z}_{30} \times {\mathbb Z}_{45} \times {\mathbb
Z}_{24}$

 \item
$(5, 10, 15)$ in ${\mathbb Z}_{25} \times {\mathbb Z}_{25} \times {\mathbb
Z}_{25}$

 \item
$(8, 8, 8)$ in ${\mathbb Z}_{10} \times {\mathbb Z}_{24} \times {\mathbb
Z}_{80}$
 
\end{enumerate}
 

\item
Prove that $D_4$ cannot be the internal direct product of two of its
proper subgroups. 
 

\item
Prove that the subgroup of ${\mathbb Q}^\ast$ consisting of elements of
the form $2^m 3^n$ for $m,n \in {\mathbb Z}$ is an internal direct
product isomorphic to ${\mathbb Z} \times {\mathbb Z}$.
 

\item
Prove that $S_3 \times {\mathbb Z}_2$ is isomorphic to $D_6$. Can you
make a conjecture about $D_{2n}$? Prove your conjecture. [{\em Hint:\/}
Draw the picture.] 
 

\item
Prove or disprove: Every abelian group of order divisible by 3
contains a subgroup of order 3.  


\item
Prove or disprove: Every non-abelian group of order divisible by 6
contains a subgroup of order 6. 
 

\item
Let $G$ be a group of order 20. If $G$ has subgroups $H$ and $K$ of
orders 4 and 5 respectively such that $hk = kh$ for all $h \in H$ and
$k \in K$, prove that $G$ is the internal direct product of $H$ and $K$. 
 

\item
Prove the following: Let $G$, $H$, and $K$ be
groups such that $G \times K \cong H \times K$. Then is is also true that $G \cong H$. 
(\emph{Hint}: Show that $G \times e_K$ is a subgroup of $G \times K$, and that $G \times e_K \cong G$; and similarly for $H$.)
 

\item
Prove or disprove: There is a noncyclic abelian group of order 51. 
 

\item
Prove or disprove: There is a noncyclic abelian group of order 52. 
 
%*****************************Theory
 

\item
Let $\phi : G_1 \rightarrow G_2$ be a group isomorphism. Show that
$\phi( x) = e$ if and only if $x=e$. 
 

\item
Let $G \cong H$. Show that if $G$ is cyclic, then so is $H$.
 

\item
Prove that any group $G$ of order $p$, $p$  prime, must be isomorphic
to ${\mathbb Z}_p$. 
 

\item
Show that $S_n$ is isomorphic to a subgroup of $A_{n+2}$.  

\item
Prove that $D_n$ is isomorphic to a subgroup of $S_n$.
 

\item
Let $\phi : G_1 \rightarrow G_2$ and  $\psi : G_2 \rightarrow G_3$  be
isomorphisms. Show that  $\phi^{-1}$ and $\psi \circ \phi$ are both
isomorphisms. Using these results, show that the isomorphism of groups
determines an equivalence relation on the class of all groups.
 

\item
Prove $U(5) \cong {\mathbb Z}_4$. Can you generalize this result to show
that $U(p) \cong {\mathbb Z}_{p-1}$? 
 

\item
Write out the permutations associated with each element of $S_3$ in
the proof of Cayley's Theorem. 
 
%*****************Automorphisms
 

\item
An \bfii{ automorphism\/}\index{Automorphism!of a
group}\index{Group!automorphism of} of a group $G$ is an isomorphism
with itself. Prove that complex conjugation is an automorphism of the
additive group of complex numbers; that is, show that the map $\phi(
a + bi ) = a - bi$ is an isomorphism from ${\mathbb C}$ to ${\mathbb C}$. 
 

\item
Prove that $a + ib \mapsto a - ib$ is an automorphism of ${\mathbb C}^*$. 
 

\item
Prove that $A \mapsto B^{-1}AB$ is an automorphism of $SL_2({\mathbb R})$
for all $B$ in $GL_2({\mathbb R})$. 
 
\item
We will denote the set of all automorphisms of $G$ by
$Aut(G)$\label{noteauto}.  Prove that  $Aut(G)$ is a subgroup of
$S_G$, the group of permutations of $G$. 
 
\item
Find $Aut( {\mathbb Z}_6)$.

\item
Find $Aut( {\mathbb Z})$.
 
\item
Find two nonisomorphic groups $G$ and $H$ such that $Aut(G) \cong Aut(
H)$. 
 
\item
Let $G$ be a group and $g \in G$. Define a map $i_g : G \rightarrow
G$\label{noteinner} 
by $i_g(x) = g x g^{-1}$.  Prove that $i_g$ defines an automorphism of
$G$.  Such an automorphism is called an \bfii{ inner
automorphism}\index{Automorphism!inner}. The set of all inner
automorphisms is denoted by $Inn(G)$\label{noteinneraut}. 
 
\item
Prove that $Inn(G)$ is a subgroup of $Aut(G)$.
 
\item
What are the inner automorphisms of the quaternion group $Q_8$? Is
$Inn(G) = Aut(G)$ in this case? 
 
\item
Let $G$ be a group and $g \in G$.  Define maps $\sigma_g :G
\rightarrow G$ and $\tau_g :G \rightarrow G$\label{noterightreg}
 by $\sigma_g(x) = gx$
and $\tau_g(x) = xg^{-1}$. Show that $i_g := \tau_g \circ \sigma_g$ is
an automorphism of $G$. 
 

\item
Let $G$ be the internal direct product of subgroups $H$ and $K$.  Show
that the map $\phi : G \rightarrow H \times K$ defined by  $\phi(g) =
(h,k)$ for $g =hk$,  where $h \in H$ and  $k \in K$, is one-to-one and
onto. 
 

\item
Let $G$ and $H$ be isomorphic groups. If $G$ has a subgroup of order
$n$, prove that $H$ must also have a subgroup of  order $n$.
 

\item
If $G \cong \overline{G}$ and $H \cong \overline{H}$, show that $G
\times H \cong \overline{G} \times \overline{H}$.
 

\item
Prove that $G \times H$ is isomorphic to $H \times G$.
 

\item
Let $n_1, \ldots, n_k$ be positive integers. Show that
\[
\prod_{i=1}^k {\mathbb Z}_{n_i} \cong {\mathbb Z}_{n_1 \cdots n_k}
\]
if and only if $\gcd( n_i, n_j) =1$ for $i \neq j$.
 

\item
Prove that $A \times B$ is abelian if and only if $A$ and $B$ are
abelian. 
 

\item
If $G$ is the internal direct product of $H_1, H_2, \ldots, H_n$,
prove that $G$ is isomorphic to $\prod_i H_i$. 
 

\item
Let $H_1$ and $H_2$ be subgroups of $G_1$ and $G_2$, respectively. Prove that $H_1 \times H_2$ is a subgroup of $G_1 \times G_2$. 
 

\item
Let $m, n \in {\mathbb Z}$. Prove that $\langle m,n \rangle \cong \langle d \rangle$ if and only if $d = \gcd(m,n)$.
 

\item
Let $m, n \in {\mathbb Z}$. Prove that $\langle m \rangle \cap \langle n \rangle \cong \langle l \rangle$ if and only if $d = \lcm(m,n)$. 
 
\end{enumerate}
}