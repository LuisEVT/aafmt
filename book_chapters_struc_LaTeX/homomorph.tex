% Homotopy groups


%%%%(c)
%%%%(c)  This file is a portion of the source for the textbook
%%%%(c)
%%%%(c)    Abstract Algebra: Theory and Applications
%%%%(c)    Copyright 1997 by Thomas W. Judson
%%%%(c)
%%%%(c)  See the file COPYING.txt for copying conditions
%%%%(c)
%%%%(c)
\chap{Homomorphisms of Groups: the Very Basics}{homomorph}

%% TWJ, 2010/03/31
%% The chapter HOMOMORPHISMS AND FACTOR GROUPS is now
%% two chapters: (10) NORMAL SUBGROUPS AND FACTOR GROUPS
%% (11) HOMOMORPHISMS
 
\section{Definition and basic properties}
 
 
In the previous chapter we talked about isomorphisms, which are bijections between two groups that also preserve
the group operation.\footnote{This chapter is bare-bones, and will be expanded at a later date.} We've seen that isomorphic groups are essentially the ``same'' group (thinking groupwise). 

But why limit ourselves to bijections? We may also consider functions between two groups that preserve group operation, but are not necessarily bijections. This idea leads to the following definition.

\begin{defn} A
\bfii{ homomorphism\/}\index{Group!homomorphism of}\index{Homomorphism!of
groups} between groups $(G, \cdot)$ and $(H, \circ)$ is a map $\phi :
G \rightarrow H$ such that  
\[
\phi( g_1 \cdot g_2 ) = \phi( g_1 ) \circ \phi( g_2 )
\]
for $g_1, g_2 \in G$. The range of $\phi$ in $H$ is called the \bfii{
homomorphic image\/}\index{Homomorphic image}~of~$\phi$.
 \end{defn}
 
Homomorphisms can be used to bring out relationships between groups that are not exactly identical, but nonetheless are structurally similar in some way.  For example, the symmetric group $S_n$ and the group ${\mathbb
Z}_2$ are related by the fact that $S_n$ can be divided into even and
odd permutations that exhibit a group structure like that ${\mathbb
Z}_2$, as shown in the following multiplication table. 
\begin{center}
\begin{tabular}{c|cc}
            & even & odd \\
\hline
even & even & odd \\
odd  & odd  & even
\end{tabular}
\end{center}
We use homomorphisms to study relationships such as the one we have
just described.
 
 
\begin{example}\label{example:homomorph:homo_Zn}
Let $G$ be a group and $g \in G$. Define a map $\phi : {\mathbb Z}
\rightarrow G$ by $\phi( n ) = g^n$. Then $\phi$ is a group
homomorphism, since 
\[
\phi( m + n ) = g^{ m + n} = g^m g^n = \phi( m ) \phi( n ).
\]
This homomorphism maps ${\mathbb Z}$ onto the cyclic subgroup of $G$
generated by $g$. 
\mbox{\hspace*{1in}}
\end{example}
 
 
\begin{example}\label{example:homomorph:homo_GL2}
Let $G = GL_2( {\mathbb R })$ (that is, the group of invertible $2 \times 2$ matrices under matrix multiplication). If
\[
A=
\begin{pmatrix}
a & b \\
c & d
\end{pmatrix}
\]
is in $G$, then the determinant is  nonzero; that is, $\det(A) = ad -bc
\neq 0$.  Also, for any two elements $A$ and $B$ in $G$, $\det(AB) =
\det(A) \det(B)$. Using the determinant, we can define a homomorphism
$\phi : GL_2( {\mathbb R }) \rightarrow {\mathbb R}^\ast$ by
$A~\mapsto~\det(A)$.  
\mbox{\vspace{1in}}
\end{example}
 
 
\begin{example}\label{example:homomorph:homo_T}
Recall that the circle group ${ \mathbb T}$ consists of all complex
numbers $z$ such that $|z|=1$. We can define a homomorphism $\phi$
from the additive group of real numbers ${\mathbb R}$ to ${\mathbb T}$ by
$\phi : \theta \mapsto \cos \theta + i \sin \theta$. Indeed, 
\begin{align*}
\phi( \alpha + \beta )
& =
\cos( \alpha + \beta ) + i \sin( \alpha + \beta ) \\
& =
(\cos \alpha \cos \beta - \sin \alpha \sin \beta)  + i( \sin \alpha 
\cos \beta + \cos \alpha \sin \beta ) \\
& =
(\cos \alpha + i \sin \alpha ) + (\cos \beta + i \sin \beta
) \\
& = \phi( \alpha ) \phi( \beta ).
\end{align*}
Geometrically, we are simply wrapping the real line around the circle 
in a group-theoretic fashion. 
\end{example}

It turns out that a particular type of subgroup is of key importance in the study of homomorphisms:

\begin{defn} Given a group $G$ a subgroup $H \subset G$ is called a 
\bfii{ normal subgroup}\index{Subgroup!normal} if for every $g \in G$ and for every $h \in H$, we have that $ghg^{-1} \in H$.  (Alternatively, we can write this condition as: $gHg^{-1} = H$.)
\end{defn}

\begin{exercise}
Show that a subgroup $H \subset G$ is normal iff every left coset of $H$ is also a right coset of $H$.
\end{exercise}

\begin{exercise}
Let $H \subset G$ be a normal subgroup, and suppose $x_1 \in g_1H$ and $x_2 \in g_2H$. Prove that $x_1x_2 \in g_1g_2H$.
\end{exercise} 

The following proposition lists some basic properties of group
homomorphisms.
 
 
\begin{thm}\label{HomorphismSubgroupProp}
Let $\phi : G_1 \rightarrow G_2$ be a homomorphism of groups. Then 
\begin{enumerate}
 
\rm \item \it
If $e$ is the identity of $G_1$, then $\phi( e)$ is the identity of
$G_2$;  
 
\rm \item \it
For any element $g \in G_1$, $\phi( g^{-1}) = [\phi( g )]^{- 1}$;
 
\rm \item \it
If $H_1$ is a subgroup of $G_1$, then $\phi( H_1 )$ is a subgroup of
$G_2$;
 
\rm \item \it
If $H_2$ is a  subgroup of $G_2$, then $\phi^{-1}(H_2) = \{ g \in G :
\phi(g) \in H_2 \}$ is a subgroup of $G_1$. Furthermore, if $H_2$ is
normal in $G_2$, then $\phi^{-1}(H_2)$ is normal in $G_1$. 
 
\end{enumerate}
\end{thm}
 
 
\begin{proof}
(1)
Suppose that $e$ and $e'$ are the identities of $G_1$ and $G_2$,
respectively; then
\[
e' \phi(e) = \phi(e) = \phi(e e) = \phi(e) \phi(e).
\]
By cancellation, $\phi(e) = e'$.
 
 
(2)
This statement follows from the fact that
\[
\phi( g^{-1}) \phi(g) = \phi(g^{-1} g) = \phi(e) = e.
\]
 
 
(3)
The set $\phi(H_1)$ is nonempty since the identity of $H_2$ is in
$\phi(H_1)$.
Suppose that $H_1$ is a subgroup of $G_1$ and let $x$ and $y$ be in
$\phi(H_1)$. There exist elements $a, b \in H_1$ such that $\phi(a) =
x$ and $\phi(b)=y$. Since 
\[
xy = \phi(a) \phi(b) = \phi(a b ) \in \phi(H_1),
\]
and
\[
x^{-1} = \phi(a)^{-1} = \phi(a^{-1}) \in \phi(H_1),
\]
it follows that $\phi(H_1)$ is a subgroup of $G_2$ (since it is closed under the group operation and inverse).
  
(4)
Let $H_2$ be a subgroup of $G_2$ and define $H_1$ to be
$\phi^{-1}(H_2)$; that is, $H_1$ is the set of all $g \in G_1$ such
that $\phi(g) \in H_2$.  The identity is in $H_1$ since $\phi(e) = e$.
If $a$ and $b$ are in $H_1$, then $\phi(ab^{-1}) = \phi(a)[ \phi(b)
]^{-1}$ is in $H_2$ since $H_2$ is a subgroup of $G_2$.  Therefore,
$ab^{-1} \in H_1$ and $H_1$ is a subgroup of $G_1$. If $H_2$ is normal
in $G_2$, we must show that $g^{-1} h g \in H_1$ for $h \in H_1$ and
$g \in G_1$. But 
\[
\phi( g^{-1} h g) = [ \phi(g) ]^{-1} \phi( h ) \phi( g ) \in
H_2,
\]
since $H_2$ is a normal subgroup of $G_2$.  Therefore, $g^{-1}hg \in
H_1$.\index{Normal subgroup}
\end{proof}
 
 
\medskip
 
 
Let $\phi : G \rightarrow H$ be a group homomorphism and suppose that
$e$ is the identity of $H$. By Proposition~\ref{HomorphismSubgroupProp}, $\phi^{-1} ( \{ e \}
)$ is a subgroup of $G$. This subgroup is called the \bfii{
kernel\/}\index{Kernel!of a group
homomorphism}\index{Homomorphism!kernel of a group} of $\phi$ and will
be denoted by $\ker \phi$\label{kernelofphi}.  In fact, this subgroup
is a normal subgroup of $G$ since the trivial subgroup is normal in
$H$.  We state this result in the following theorem, which says that
with every homomorphism of groups we can naturally associate a normal
subgroup.   
 
 
\begin{theorem}
Let $\phi : G \rightarrow H$ be a group homomorphism. Then the kernel
of $\phi$ is a normal subgroup of $G$.\index{Kernel!as a normal subgroup} 
\end{theorem}
 
 
\begin{example}\label{example:homomorph:homo_G2_to_R}
Let us examine the homomorphism $\phi : GL_2( {\mathbb R }) \rightarrow
{\mathbb R}^\ast$ defined by $A \mapsto \det( A )$. Since 1 is the
identity of ${\mathbb R}^\ast$, the kernel of this homomorphism is all
$2 \times 2$ matrices having determinant one. That is, $\ker \phi =
SL_2( {\mathbb R })$.
\mbox{\hspace{1in}}
\end{example}
 
 
\begin{example}\label{example:homomorph:kernel}
The kernel of the group homomorphism $\phi : {\mathbb R} \rightarrow
{\mathbb C}^\ast$ defined by $\phi( \theta ) = \cos \theta + i \sin
\theta$ is $\{ 2 \pi n : n \in {\mathbb Z} \}$. Notice that $\ker \phi
\cong {\mathbb Z}$. 
\end{example}
 
 
\begin{example}\label{example:homomorph:homo_Z7}
Suppose that we wish to determine all possible homomorphisms $\phi$
from ${\mathbb Z}_7$ to  ${\mathbb Z}_{12}$. Since the kernel of $\phi$ must
be a subgroup of  ${\mathbb Z}_7$, there are only two possible
kernels, $\{ 0 \}$ and all of ${\mathbb Z}_7$.  The image of a subgroup
of ${\mathbb Z}_7$ must be a subgroup of ${\mathbb Z}_{12}$. Hence, there is
no injective homomorphism; otherwise, ${\mathbb Z}_{12}$ would have a
subgroup of order 7, which is impossible. Consequently, the only
possible homomorphism from ${\mathbb Z}_7$ to  ${\mathbb Z}_{12}$ is the one
mapping all elements to zero. 
\end{example}
 
 
\begin{example}\label{example:homomorph:homo_g^n}
Let $G$ be a group. Suppose that  $g \in G$ and $\phi$ is the
homomorphism from ${\mathbb Z}$ to $G$ given by $\phi( n ) = g^n$. If the
order of $g$ is infinite, then the kernel of this homomorphism is $\{
0 \}$ since $\phi$ maps ${\mathbb Z}$ onto the cyclic subgroup of $G$
generated by $g$. However, if the order of $g$ is finite, say $n$,
then the kernel of $\phi$ is $n {\mathbb Z}$.
\end{example}
 

 
 
 
%\section{The Isomorphism Theorems}
% 
% 
%Though at first it is not evident that factor groups correspond
%exactly to homomorphic images, we can use factor groups to study
%homomorphisms. We already know that with every group homomorphism
%$\phi: G \rightarrow H$ we can associate a normal subgroup of $G$,
%$\ker \phi$; the converse is also true. Every normal subgroup of a
%group $G$ gives rise to homomorphism of groups. 
% 
%Let $H$ be a normal subgroup of $G$. Define the \bfii{
%natural\/}\index{Homomorphism!natural} or \bfii{ canonical
%homomorphism}\index{Homomorphism!canonical}  
%\[
%\phi : G \rightarrow G/H
%\]
%by
%\[
%\phi(g) = gH.
%\]
%This is indeed a homomorphism, since
%\[
%\phi( g_1 g_2 ) = g_1 g_2 H =  g_1 H g_2 H = \phi( g_1) \phi( g_2 ). 
%\]
%The kernel of this homomorphism is $H$.	 The following theorems 
%describe the relationships among group homomorphisms, normal 
%subgroups, and factor groups.\index{Group!factor groups}\index{Factor groups} 
% 
% 
% 
%\begin{theorem}[First Isomorphism Theorem]\label{FirstIsoTheorem}\index{First Isomorphism
%Theorem!for groups}
%If $\psi : G \rightarrow H$ is a group homomorphism with $K =\ker
%\psi$, then $K$ is normal in $G$. Let $\phi: G \rightarrow G/K$ be
%the canonical homomorphism.  Then there exists a unique isomorphism
%$\eta: G/K \rightarrow \psi(G)$ such that $\psi =  \eta \phi$.
%\end{theorem}
% 
% 
%\begin{proof}
%We already know that $K$ is normal in $G$. Define $\eta: G/K
%\rightarrow \psi(G)$ by $\eta(gK) = \psi(g)$.  We must first show that
%this is a well-defined map. Suppose that $g_1 K =g_2 K$. For some $k \in
%K$, $g_1 k=g_2$; consequently, 
%\[
%\eta(g_1 K) = \psi(g_1) = \psi(g_1) \psi(k) = \psi(g_1k) = \psi(g_2)
%= \eta(g_2 K). 
%\]
%Since $\eta(g_1 K) = \eta(g_2 K)$, $\eta$ does not depend on the 
%choice of coset representative. Clearly $\eta$ is onto $\psi( G)$. 
%To show that $\eta$ is one-to-one, suppose that $\eta(g_1 K) = 
%\eta(g_2 K)$. Then $\psi(g_1) = \psi(g_2)$. This implies that 
%$\psi( g_1^{-1} g_2 ) = e$, or $g_1^{-1} g_2$ is in the kernel of $\psi$; 
%hence, $g_1^{-1} g_2K = K$; that is, $g_1K =g_2K$.  Finally, we must 
%show that $\eta$ is a homomorphism, but 
%\begin{align*}
%\eta( g_1K g_2K ) & = \eta(g_1 g_2K) \\
%& = \psi(g_1 g_2) \\
%& = \psi(g_1) \psi(g_2) \\
%& = \eta( g_1K) \eta( g_2K ).
%\end{align*}
%\end{proof}
% 
% 
%\medskip
% 
% 
%Mathematicians often use diagrams called \bfii{ commutative
%diagrams\/}\index{Commutative diagrams} to describe such theorems. The
%following diagram ``commutes'' since $\psi = \eta \phi$. 
%
%
%
%\begin{center}
%\tikzpreface{homomorphs_first_isomorphism}
%\begin{tikzpicture}[scale=0.8]
%
%\node at (1.5,2) [above] {$\psi$};
%\node at (0.25,0.65) {$\phi$};
%\node at (2.75,0.65) {$\eta$};
%\draw [->] (0,2)  node [left] {$G$} -- (3,2) node [right] {$H$};
%\node at (1.5,0) [below] {$G/K$};
%\draw [->] (0,1.7) -- (1.3,0);
%\draw [->] (1.7,0) -- (3,1.7);
%
%
%\end{tikzpicture}
%
%\end{center}
%
% 
% 
%\begin{example}\label{example:homomorph:homo_cyclic}
%Let $G$ be a cyclic group with generator $g$. Define a map $\phi :
%{\mathbb Z} \rightarrow G$ by $n \mapsto g^n$.  This map is a surjective
%homomorphism\index{Homomorphism!surjective } since  
%\[
%\phi( m + n) = g^{m+n} = g^m g^n = \phi(m) \phi(n).
%\]
%Clearly $\phi$ is onto. If $|g| = m$, then  $g^m = e$. Hence, $\ker
%\phi = m {\mathbb Z}$ and ${\mathbb Z} / \ker \phi =  {\mathbb Z} / m {\mathbb Z}
%\cong G$. On the other hand, if the order of $g$ is infinite, then
%$\ker \phi = 0$ and $\phi$ is an isomorphism of $G$ and ${\mathbb Z}$.
%Hence, two cyclic groups are isomorphic exactly when they have the
%same order. Up to isomorphism, the only cyclic groups are ${\mathbb Z}$
%and ${\mathbb Z}_n$. 
%\end{example}
% 
% 
%\begin{theorem}[Second Isomorphism Theorem]\label{homomorph:theorem:2nd_isomorph}\index{Second Isomorphism
%Theorem! for groups}
%Let  $H$ be a subgroup of a group $G$ (not necessarily normal in $G$)
%and $N$ a normal subgroup of $G$.  Then $HN$ is a subgroup of $G$,
%$H \cap N$ is a normal subgroup of $H$, and 
%\[
%H / H \cap N \cong HN /N.
%\]
%\end{theorem}
% 
% 
%\begin{proof}
%We will first show that $HN = \{ hn : h \in H, n \in N \}$ is a
%subgroup of $G$.  Suppose that  $h_1 n_1, h_2 n_2 \in HN$. Since 
%$N$ is normal, $(h_2)^{-1} n_1 h_2 \in N$. So 
%\[
%(h_1 n_1)(h_2 n_2) = h_1 h_2 ( (h_2)^{-1} n_1 h_2 )n_2
%\]
%is in $HN$. The inverse of $hn \in HN$ is in $HN$ since
%\[
%( hn )^{-1} = n^{-1 } h^{-1} = h^{-1} (h n^{-1} h^{-1} ).
%\]
% 
% 
%Next, we prove that $H \cap N$ is normal in $H$. Let $h \in H$ and $n
%\in H \cap N$. Then $h^{-1} n h \in H$ since each element is in $H$.
%Also, $h^{-1} n h \in N$ since $N$ is normal in $G$; therefore,
%$h^{-1} n h \in H \cap N$. 
% 
% 
%Now define a map $\phi$ from $H$ to $ HN / N$ by $h \mapsto h N$. The
%map $\phi$ is onto, since any coset $h n N = h N$ is the image of $h$
%in $H$. We also know that $\phi$ is a homomorphism because 
%\[
%\phi( h  h')  = h h' N =  h N h' N =  \phi( h ) \phi( h').
%\]
%By the First Isomorphism Theorem, the image of $\phi$ is isomorphic to
%$H / \ker \phi$; that is,
%\[
%HN/N = \phi(H) \cong H / \ker \phi.
%\]
%Since
%\[
%\ker \phi = \{ h \in H : h \in N \} = H \cap N,
%\]
%$HN/N = \phi(H) \cong H / H \cap N$.
%\end{proof}
% 
% 
%\begin{theorem} {\bf (Correspondence Theorem)}\label{CorrespondTheorem}\index{Correspondence
%Theorem!for groups}
%Let $N$ be a normal subgroup of a group $G$. Then $H \mapsto H/N$
%is a one-to-one correspondence between the set of subgroups $H$
%containing $N$  and the set of subgroups of $G/N$. Furthermore, the
%normal subgroups of $H$ correspond to normal subgroups of~$G/N$. 
%\end{theorem}
% 
% 
%\begin{proof}
%Let $H$ be a subgroup of $G$ containing $N$. Since $N$ is normal in
%$H$, $H/N$ makes sense.  Let $aN$ and $bN$ be elements of $H/N$. Then
%$(aN)( b^{-1} N )= ab^{-1}N \in H/N$; hence, $H/N$ is a subgroup of
%$G/N$. 
%
%
%Let $S$ be a subgroup of $G/N$. This subgroup is a set of cosets of
%$N$.  If  $H= \{ g \in G : gN \in S \}$, then for $h_1, h_2 \in H$, we
%have that $(h_1 N)( h_2 N )= h h' N \in S$ and $h_1^{-1} N \in S$.
%Therefore, $H$ must be a subgroup of $G$. Clearly, $H$ contains $N$.
%Therefore, $S = H / N$. Consequently, the map  $H \mapsto H/H$ is
%onto. 
%
% 
%Suppose that $H_1$ and $H_2$ are subgroups of $G$ containing $N$ such
%that $H_1/N = H_2/N$. If $h_1 \in H_1$, then $h_1 N \in H_1/N$. Hence,
%$h_1 N = h_2 N \subset H_2$ for some $h_2$ in $H_2$. However, since
%$N$ is contained in $H_2$, we know that $h_1 \in H_2$ or $H_1 \subset
%H_2$. Similarly, $H_2 \subset H_1$.  Since $H_1 = H_2$, the map  $H
%\mapsto H/H$ is one-to-one. 
%
% 
%Suppose that $H$ is normal in $G$ and $N$ is a subgroup of $H$.  Then
%it is easy to verify that the map $G/N \rightarrow G/H$ defined by $gN
%\mapsto gH$ is  a homomorphism.  The kernel of this homomorphism is
%$H/N$, which proves that $H/N$ is normal in $G/N$. 
% 
% 
%Conversely, suppose that $H/N$ is normal in $G/N$. The homomorphism
%given by 
%\[
%G \rightarrow G/N \rightarrow \frac{G/N}{H/N}
%\]
%has kernel $H$. Hence, $H$ must be normal in $G$.
%\end{proof}
% 
%\medskip
% 
% 
%Notice that in the course of the proof of Theorem~\ref{CorrespondTheorem}, we have also
%proved the following theorem. 
% 
% 
%\begin{theorem}[Third Isomorphism Theorem]\label{ThirdIsoTheorem}\index{Third Isomorphism
%Theorem!for groups}
%Let $G$ be a group and $N$ and $H$ be normal subgroups of $G$ with $N
%\subset H$.  Then 
%\[
%G/H \cong \frac{G/N}{H/N}.
%\]
%\end{theorem}
% 
% 
%\begin{example}\label{example:homomorph:3rd_isomorph}
%By the Third Isomorphism Theorem,
%\[
%{\mathbb Z} / m {\mathbb Z} \cong ({\mathbb Z}/ mn {\mathbb Z})/ (m {\mathbb Z}/ mn
%{\mathbb Z}). 
%\]
%Since $| {\mathbb Z} / mn {\mathbb Z} | = mn$ and  $|{\mathbb Z} / m{\mathbb Z}| =
%m$, we have $| m {\mathbb Z} / mn {\mathbb Z}| = n$. 
%\end{example}
 
 
\markright{EXERCISES}
\section*{Additional Exercises}
\exrule
 
 
 
{\small
 
 
\begin{enumerate}
 
 
 
\item
Prove that $\det( AB) = \det(A) \det(B)$ for $A, B \in GL_2( {\mathbb R}
)$. This shows that the determinant is a homomorphism from $GL_2(
{\mathbb R} )$ to ${\mathbb R}^*$. 
 
 
 
\item
Which of the following maps are homomorphisms? If the map is a
homomorphism, what is the kernel? 
\begin{enumerate}
 
 \item
$\phi : {\mathbb R}^\ast \rightarrow GL_2 ( {\mathbb R})$ defined by
\[
\phi( a ) =
\begin{pmatrix}
1 & 0 \\
0 & a
\end{pmatrix}
\]
 
 \item
$\phi : {\mathbb R} \rightarrow GL_2 ( {\mathbb R})$ defined by
\[
\phi( a ) =
\begin{pmatrix}
1 & 0 \\
a & 1
\end{pmatrix}
\]
 
 \item
$\phi : GL_2 ({\mathbb R})   \rightarrow {\mathbb R}$ defined by
\[
\phi
\left(
\begin{pmatrix}
a & b \\
c & d
\end{pmatrix}
\right)
= a + d
\]
 
 \item
$\phi : GL_2 ( {\mathbb R})   \rightarrow {\mathbb R}^\ast$ defined by 
\[
\phi
\left(
\begin{pmatrix}
a & b \\
c & d
\end{pmatrix}
\right)
= ad -bc
\]
 
 \item
$\phi : {\mathbb M}_2( {\mathbb R})   \rightarrow {\mathbb R}$ defined by
\[
\phi
\left(
\begin{pmatrix}
a & b \\
c & d
\end{pmatrix}
\right)
= b,
\]
where ${\mathbb M}_2( {\mathbb R})$ is the additive group of $2 \times 2$ matrices with entries in ${\mathbb R}$.
 
\end{enumerate}
 
  
 
\item
Let $A$ be an $m \times n$ matrix.  Show that matrix multiplication,
$x \mapsto Ax$, defines a homomorphism $\phi : {\mathbb R}^n \rightarrow
{\mathbb R}^m$. 
 
 
\item
Let $\phi : {\mathbb Z} \rightarrow {\mathbb Z}$ be given by $\phi(n) = 7n$.
Prove that $\phi$ is a group homomorphism. Find the kernel and the
image of $\phi$.
 
 
\item
Describe all of the homomorphisms from ${\mathbb Z}_{24}$ to ${\mathbb Z}_{18}$. 
 
 
\item
Describe all of the homomorphisms from ${\mathbb Z}$ to ${\mathbb Z}_{12}$. 
 
 
%\item
%In the group ${\mathbb Z}_{24}$, let $H = \langle 4 \rangle$ and $N =
%\langle 6 \rangle$. 
%\begin{enumerate}
% 
% \item
%List the elements in $HN$ (we usually write $H + N$ for these additive
%groups) and $H \cap N$. 
% 
% \item
%List the cosets in $HN/N$, showing the elements in each coset.
% 
% \item
%List the cosets in $H/(H \cap N)$, showing the elements in each coset. 
% 
% \item
%Give the correspondence between $HN/N$ and $H/(H \cap N)$ described in
%the proof of the Second Isomorphism Theorem. 
% 
%\end{enumerate}
% 
% 
%%***************************THEORY******************
% 
% 
%\item
%If $G$ is an abelian group and $n \in {\mathbb N}$, show that $\phi : G
%\rightarrow G$  defined by $g \mapsto g^n$ is a group homomorphism. 
% 
% 
%
% 
%\item
%If $\phi : G \rightarrow H$ is a group homomorphism and $G$ is
%abelian, prove that $\phi(G)$ is also abelian. 
% 
% 
%\item
%If $\phi : G \rightarrow H$ is a group homomorphism and $G$ is cyclic,
%prove that $\phi(G)$ is also cyclic. 
% 
% 
%\item
%Show that a homomorphism defined on a cyclic group is completely
%determined by its action on the generator of the group.
%
%
% 
% 
%\item
%Let $G$ be a group of order $p^2$, where $p$ is a prime number. If $H$
%is a subgroup of $G$ of order $p$, show that $H$ is normal in $G$.
%Prove that $G$ must be abelian. 
% 
% 
%\item
%If a group $G$ has exactly one subgroup $H$ of order $k$, prove that
%$H$ is normal in $G$. 
% 
% 
%\item
%Prove or disprove: ${\mathbb Q} / {\mathbb Z} \cong {\mathbb Q}$.
% 
% 
%%\item
%%Define the \bfii{ centralizer\/}\index{Element!centralizer
%%of}\index{Centralizer!of an element} of an element $g$ in a group $G$
%%to be the set  
%%\[
%%C(g) = \{ x \in G : xg = gx \}.
%%\]
%%Show that $C(g)$ is a subgroup of $G$.  If $g$ generates a normal
%%subgroup of $G$, prove that $C(g)$ is normal in $G$.
%% 
%% 
%%\item
%%Recall that the \bfii{ center\/}\index{Group!center of} of a group $G$ is
%%the set 
%%\[
%%Z(G) = \{ x \in G : xg = gx \mbox{ for all $g \in G$ } \}.
%%\]
%%\begin{enumerate}
%% 
%% \item
%%Calculate the center of $S_3$.
%% 
%% \item
%%Calculate the center of $GL_2 ( {\mathbb R} )$.
%% 
%% \item
%%Show that the center of any group $G$ is a normal subgroup of $G$. 
%% 
%% \item
%%If $G / Z(G)$ is cyclic, show that $G$ is abelian.
%% 
%%\end{enumerate}
%% 
% 
%\item
%Let $G$ be a finite group and $N$ a normal subgroup of $G$. If $H$ is
%a subgroup of $G/N$, prove that $\phi^{-1}(H)$ is a subgroup in $G$ of
%order $|H| \cdot |N|$, where $\phi : G \rightarrow G/N$ is the
%canonical homomorphism. 
% 
% 
%%\item
%%Let $G$ be a group and let $G' = \langle aba^{- 1} b^{-1} \rangle$;
%%that is, $G'$ is the subgroup of all finite products of elements in
%%$G$ of the form $aba^{-1}b^{-1}$.  The subgroup $G'$ is called the
%%\bfii{ commutator
%%subgroup\/}\index{Subgroup!commutator}\label{commutatorsubgroup} of $G$.  
%%\begin{enumerate}
%% 
%% \item
%%Show that $G'$ is a normal subgroup of $G$.
%%
%% \item
%%Let $N$ be  a normal subgroup of $G$.  Prove that $G/N$ is abelian if
%%and only if $N$ contains the commutator subgroup of $G$.
%% 
%%\end{enumerate}
%
% 
%\item
%Let $G_1$ and $G_2$ be groups, and let $H_1$ and $H_2$ be normal subgroups
%of $G_1$ and $G_2$ respectively. Let $\phi : G_1 \rightarrow G_2$ be a
%homomorphism. Show that $\phi$ induces a natural homomorphism
%$\overline{\phi} : (G_1/H_1) \rightarrow (G_2/H_2)$ if $\phi(H_1) \subset
%H_2$. 
% 
% 
%\item
%If $H$ and $K$ are normal subgroups of $G$ and $H \cap K = \{ e \}$,
%prove that $G$ is isomorphic to a subgroup of $G/H \times G/K$.
% 
% 
% 
%\item
%Let $\phi : G_1 \rightarrow G_2$ be a surjective group homomorphism.
%Let $H_1$ be a normal subgroup of $G_1$ and suppose that $\phi(H_1) =
%H_2$.  Prove or disprove that $G_1/H_1 \cong G_2/H_2$.
% 
% 
%\item
%Let $\phi : G \rightarrow H$ be a group homomorphism.  Show that
%$\phi$ is one-to-one if and only if $\phi^{-1}(e) = \{ e \}$.
%
%
%\end{enumerate}
%}
% 
%
%
%\subsection*{Additional Exercises: Automorphisms}
% 
% 
%{\small
%\begin{enumerate}
% 
% 
%\item
%Let $Aut(G)$ be the set of all automorphisms of $G$; that is,
%isomorphisms from $G$ to itself. Prove this set forms a group and is a
%subgroup of the group of permutations of $G$; that is, $Aut(G) \leq S_G$. 
% 
% 
%\item
%An \bfii{ inner automorphism\/}\index{Automorphism!inner} of $G$,
%\[
%i_g : G \rightarrow G,
%\]
%is defined by the map
%\[
%i_g(x) = g x g^{-1},
%\]
%for $g \in G$. Show that $i_g \in Aut(G)$.
% 
% 
%\item
%The set of all inner automorphisms is denoted by $Inn(G)$. Show that
%$Inn(G)$ is a subgroup of $Aut(G)$. 
% 
% 
%\item
%Find an automorphism of a group $G$ that is not an inner automorphism.
% 
% 
%\item
%Let $G$ be a group and $i_g$ be an inner automorphism of $G$, and
%define a map 
%\[
%G \rightarrow Aut(G)
%\]
%by
%\[
%g \mapsto i_g.
%\]
%Prove that this map is a homomorphism with image $Inn(G)$ and kernel
%$Z(G)$. Use this result to conclude that 
%\[
%G/Z(G) \cong Inn(G).
%\]
% 
% 
%\item
%Compute $Aut(S_3)$ and $Inn(S_3)$.  Do the same thing for $D_4$.
% 
% 
%\item
%Find all of the homomorphisms $\phi : {\mathbb Z} \rightarrow {\mathbb Z}$.
%What is $Aut({\mathbb Z})$? 
% 
% 
%\item
%Find all of the automorphisms of ${\mathbb Z}_8$.  Prove that $Aut({\mathbb
%Z}_8) \cong U(8)$. 
% 
% 
%\item
%For $k \in {\mathbb Z}_n$, define a map $\phi_k : {\mathbb Z}_n \rightarrow
%{\mathbb Z}_n$ by $a \mapsto ka$.  Prove that $\phi_k$ is a homomorphism. 
% 
% 
%\item
%Prove that $\phi_k$ is an isomorphism if and only if $k$ is a generator
%of ${\mathbb Z}_n$. 
% 
% 
%\item
%Show that every automorphism of ${\mathbb Z}_n$ is of the form $\phi_k$,
%where $k$ is a generator of ${\mathbb Z}_n$. 
% 
% 
%\item
%Prove that $\psi : U(n) \rightarrow Aut({\mathbb Z}_n)$ is an
%isomorphism, where $\psi : k \mapsto \phi_k$. 
% 
% 
\end{enumerate}
%}
% 
% 
% 
% 
%