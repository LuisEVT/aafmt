\section{Solutions for  ``Equivalence Relations''}
\noindent\textbf{\textit{ (Chapter \ref{EquivalenceRelationsChap})}}\bigskip

\noindent\textbf{Exercise \ref{exercise:EquivalenceRelationsChap:7}:}\\
c. Relations: $\{(a,1)\},\{(b,1)\},\{(a,1),(b,1)\},\emptyset$\\
d. $A$ x $A$=$\{(a,a),(a,b),(b,a),(b,b)\}$\\
Relations: $\{(a,a)\},\{(a,b)\}, \{(b,a)\}, \{(b,b)\}$\\
$\{(a,a),(a,b)\}, \{(a,a),(b,a)\}, \{(a,a),(b,b)\}$\\
$\{(a,b),(b,a)\}, \{(a,b),(b,b)\}, \{(b,a),(b,b)\}$\\
$\{(a,a),(a,b),(b,a)\}, \{(a,a),(a,b),(b,b)\}, \{(a,a),(b,a),(b,b)\}, \{(a,b),(b,a),(b,b)\}$\\
$\{(a,a),(a,b),(b,a),(b,b)\},\emptyset$\\
e. $A$ x $A$=$\{(a,a),(a,b),(a,c),(b,a),(b,b),(b,c),(c,a),(c,b),(c,c)\}$\\
There are $2^9=512$ binary relations on the set $A$\\
\\
\textbf{Exercise \ref{exercise:EquivalenceRelationsChap:RelationDef}:}\\
b. $R$ = $\{(a+bi,c+di)\in C$ x $C | a=c$ and $b=d$ where $a,b,c,d\in R\}$\\
d. $A=\{1,2,3\}$\\
\\
\textbf{Exercise \ref{exercise:EquivalenceRelationsChap:17}:}\\
b. By definition of subset: It's transitive and reflexive; it isn't symmetric.\\
\\
\textbf{Exercise \ref{exercise:EquivalenceRelationsChap:21}:}\\
a. If each element has an arrow starts from and come back to itself $\implies$  reflexive.\\
b. If all relation lines have arrow signs at both ends $\implies$  symmetric.\\
\\
\textbf{Exercise \ref{exercise:EquivalenceRelationsChap:EquivRelShowEx}:}\\
b. It' s an equivalence relation because it's reflexive, symmetric,and transitive.\\
c. Equivalence relation.\\
Prove that it's transitive:\\
Let $(a_1,b_1)\sim (a_2,b_2) \implies a_1b_2=a_2b_1$\\
    $(a_2,b_2)\sim (a_3,b_3) \implies a_2b_3=a_3b_2$\\
Then we have:\\
$a_1b_3$ = $a_1b_3\displaystyle\frac{b_2}{b_2}$ ($b_2\in N$ so $b_2\neq 0$)\\
= $a_1b_2\displaystyle\frac{b_3}{b_2}=a_2b_1\displaystyle\frac{b_3}{b_2}=a_3b_2\displaystyle\frac{b_1}{b_2}$\\
$\implies a_1b_3=a_3b_1$\\
$\implies (a_1,b_1)\sim (a_3,b_3$\\
\\
\textbf{Exercise \ref{exercise:EquivalenceRelationsChap:EquivClassEasyEx}:}\\
b. $[1]=\{1,3,5\}$\\
$[2]=\{2,4\}$\\
$[3]=\{1,3,5\}$\\
$[4]=\{2,4\}$\\
$[5]=\{1,3,5\}$\\
\\
\textbf{Exercise \ref{exercise:EquivalenceRelationsChap:EquivRelPropsPfEx}:}\\
a. Equivalence relation $\implies$  reflexive $\implies a \sim a \implies a\in[a]$\\
b. $a\in [a]$ (from part a) $\implies [a]\neq \emptyset$\\
c. Let $a_1\in \bigcup_{a\in S}[a] \implies a_1\in [a]$. We also have $[a]\subset S \implies a_1\in S$. So $\bigcup_{a\in S}[a]\subset S$.\\
Let $a_2 \in S$. From part (a), we have $a_2 \in [a_2] \implies a_2\in\bigcup_{a\in S}[a] \implies S\subset \bigcup_{a\in S}[a]$.\\
Therefore $\bigcup_{a\in S}[a]=S$.\\
d. Let $a_1,a_2 \in S$ such that $a_1 \sim a_2$. Choose $a_3\in[a_1] \implies a_3 \sim a_1$ (Def. 33)\\
$\implies a_3 \sim a_2$ (because $a_1 \sim a_2$, transitive property)\\
$\implies a_3 \in [a_2]$ (Def. 33)\\
$\implies [a_1]\subset[a_2]$ (1)\\
Choose $a_4\in[a_2]$ then, similarly, we have $a_4\in[a_1]$\\
$\implies [a_2]\subset[a_1]$ (2)\\
From (1) and (2), we have $[a_1]=[a_2]$.\\
e. Let $a_1,a_2 \in S$ such that $a_1 \not\sim a_2$\\
Suppose that $[a_1]\cap[a_2]\neq\emptyset \to$ there is an element $x\in[a_1]$ and $x\in[a_2]$.\\
$x\in[a_1] \implies x\sim [a_1] \implies [a_1]\sim x$\\
$x\in[a_2] \implies x\sim [a_2]$\\
$\implies [a_1]\sim[a_2]$ (transitive) $\implies$  it contradicts to the supposition that $a_1 \not\sim a_2$.\\
Therefore, $[a_1]\cap[a_2]=\emptyset$.\\
\\
\textbf{Exercise \ref{exercise:EquivalenceRelationsChap:DisjointEquivEx}:}\\
a. $a\in[a_1]\cap[a_2]$\\
b. $a\in[a_1]$ and $a\in[a_2]$\\
c. $a\sim a_1$ and $a\sim a_2$\\
d. part (d) tells us that $[a]=[a_1]$ and $[a]=[a_2]$\\
e. Therefore $[a_1]=[a]=[a_2]$\\
\\
\textbf{Exercise \ref{exercise:EquivalenceRelationsChap:ProveModEquiv}:}\\
$a\equiv b\pmod{n}$ iff $a-b=kn$ ($k\in Z$).\\
We have $a-a=0n \implies a\equiv a\pmod{n} \to$ reflexive.\\
(i) For symmetric:\\
$a\equiv b\pmod{n} \implies a-b=kn \implies b-a=-kn$\\
$-k\in Z$ because $-1,k\in Z$\\
$\implies b\equiv a\pmod{n}$\\
(ii) For transitive:\\
Let $a\equiv b\pmod{n}$ and $b\equiv c\pmod{n}$\\
$\implies a-b=kn$ and $b-c=ln$\\
$\implies a-c=(k+l)n$\\
$\implies a\equiv c\pmod{n}$\\
Therefore, equivalent mod n is an equivalence relation.\\
\\
\textbf{Exercise \ref{exercise:EquivalenceRelationsChap:47}:}\\
a. If r is the remainder when $a-b$ is divided by 3, then $\bar{a}-\bar{b}=\bar{r}$\\
b. If r is the remainder when $a.b$ is divided by 3, then $\bar{a}.\bar{b}=\bar{r}$\\
\\
\textbf{Exercise \ref{exercise:EquivalenceRelationsChap:54}:}\\
a. (i) $a_1\equiv a_2\pmod{n}$ and $b_1\equiv b_2\pmod{n}$\\
(ii) $a_1=a_2 + k_1n$ and $b_1= b_2 + k_2n$, where $k_1,k_2 \in Z$.\\
(iii) $(a_1+b_1)- (a_2+b_2) = (k_1+k_2)n$\\
(iv) $a_1+b_1 \equiv (a_2+b_2)\pmod{n}$\\
(v) that $\overline{a_1+b_1}=\overline{a_2+b_2}$\\
(vi) $\overline{a_1}+\overline{b_1}=\overline{a_2}+\overline{b_2}$\\
b. Suppose $\overline{a_1}=\overline{a_2}$ and $\overline{b_1}=\overline{b_2}$.\\
From the definition of equivalence class, it follows that $a_1\equiv a_2\pmod{n}$ and $b_1\equiv b_2\pmod{n}$.\\
By definition 42, it follows that $a_1-a_2=k_1n$ and $b_1-b_2=k_2n$ where $k_1,k_2 \in Z$.\\
By integer arithmetic, if follows that $(a_1-b_1)-(a_2-b_2)=(k_1-k_2)n$.\\
Since $k_1,k_2 \in Z$, it follows from def. 42 that $(a_1-b_1)\equiv (a_2-b_2)\pmod{n}$.\\
It follows from proposition 37 part (d) that $\overline{a_1-b_1}=\overline{a_2-b_2}$.\\
By definition 50 part (c), then $\overline{a_1}-\overline{b_1}=\overline{a_2}-\overline{b_2}$.\\
So it is well-defined.\\
\\
\textbf{Exercise \ref{exercise:EquivalenceRelationsChap:57}:}\\
a. Let define $f:Z_4\implies Z_{12}$ given by $f([a]_4)=[a]_{12}$.\\
If $[a]_{12}=[b]_{12} \implies a\equiv b\pmod{12} \implies a-b=3.4.k \implies a\equiv b\pmod{4}$\\
$\implies [a]_4=[b]_4 \implies f([a]_4)=f([b]_4)$.\\
\\
\textbf{Exercise \ref{exercise:EquivalenceRelationsChap:66}:}\\
a. $A=\{-3,-2,-1,0,1,2,3\}$\\
Reflexive: Let $x\in A \implies f(x)=x^2=f(x) \implies x\sim x$.\\
Symmetric: Let $x_1,x_2\in A$. If $x_1\sim x_2 \implies f(x_1)=x_1^2=f(x_2)=x_2^2$\\
$\implies f(x_2)=f(x_1) \implies x_2 \sim x_1$.\\
Transitive: transitive rule is not violated.\\
% This exercise has been removed
%\\
%\textbf{Exercise \ref{exercise:EquivalenceRelationsChap:67}:}\\
%a. $x^2+y^2=r^2$\\
%b. Suppose that $C_r\cap C_s \neq \emptyset$ where $s\neq r$ ($r,s>0$).\\
%$\implies$  there is $(x_1,y_1)\in (C_r\cap C_s)$\\
%$\implies x_1^2+y_1^2=r^2$ and $\implies x_1^2+y_1^2=s^2$\\
%$\implies r^2=s^2 \implies r=s \to$ it's a contradiction to the supposition that $r\neq s$\\
%$\implies C_r\cap C_s=\emptyset$.\\
%c. Let $(x,y)\in R^2$\\
%$\implies$  distance from point $(x,y)$ to the origin is $d=\sqrt{x^2+y^2}$\\
%$\implies x^2+y^2=d^2 \implies (x,y)\in C_d$.\\
%So each element of $R^2$ is in at least one circle.\\
%d. From (c), we have $\bigcup_{r=0}^{\infty}C_r=R^2$\\
%From (b), we have $C_r\cap C_s=\emptyset$ where $r\neq s$, which also means that $\bigcap_{r=0}^{\infty}C_r=\emptyset$\\
%$\implies \{C_r|r\in [0,\infty)\}$ forms a partition of $R^2$ by definition of partition.\\
%e. $(x_1,y_1)\sim (x_2,y_2)$ iff $x_1^2+y_1^2=x_2^2+y_2^2$\\
