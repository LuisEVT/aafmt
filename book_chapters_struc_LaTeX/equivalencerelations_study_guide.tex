\section{Study guide  for ``Equivalence Relations and Equivalence Classes''  chapter} \label{sec:EquivalenceRelationsChap:study} 
\subsection*{Section \ref{sec.relation}, Binary relations}
\subsubsection*{Concepts}
\begin{enumerate}
\item 
Relations and binary relations (Definition~\ref{relation})
\item
Relations that are not functions
\item
Binary relations and digraphs
\item
Bipartite graphs
\end{enumerate}

\subsubsection*{Notation}
\begin{enumerate}
\item
 $x\mid y$ means $x$ divides $y$
\item 
$:=$ means ``defined as''
\end{enumerate}

\subsubsection*{Competencies}
\begin{enumerate}
\item
Given a set of data list the relations as subsets.  (\ref{exercise:EquivalenceRelationsChap:bball_subsets}) 
\item
Given set(s) be a ble to list all relations or binary relations. (\ref{exercise:EquivalenceRelationsChap:7}, \ref{exercise:EquivalenceRelationsChap:notbinary})
\item	
Using a set of data create a digraph. (\ref{exercise:EquivalenceRelationsChap:DrawBinRelExer1}, \ref{exercise:EquivalenceRelationsChap:DrawBinRelExer3})
\item
Be able to draw bipartite graphs. (\ref{exercise:EquivalenceRelationsChap:DrawBinRelExer2})
\item
Define a binary relation given a set of criteria. (\ref{exercise:EquivalenceRelationsChap:RelationDef})
\end{enumerate}


\subsection*{Section \ref{Partitions1}, Partitions and properties of binary relations}
\subsubsection*{Concepts:}
\begin{enumerate}
\item 
Partition $\mathcal{P}$ of a set $A$ (Definition~\ref{Partition})
\item 
Any partition may be used to define a binary relation (Definition~\ref{partitionrelation})
\item 
The reflexive, symmetirc, and transitive property of relations. (Proposition~\ref{proposition:EquivalenceRelationsChap:partbinrel})
\item 
Equivalence relations are reflexive, symmetric, and transitive. (Definition~\ref{defn:equivalencerelation})
\end{enumerate}

\subsubsection*{Competencies}
\begin{enumerate}
\item
Given a set partition, define and graph the associated binary relation. (\ref{exercise:EquivalenceRelationsChap:RelGraphR2})
\item
Be able to prove whether or not a given binary relation is reflexive, symmetric, and transitive. (\ref{exercise:EquivalenceRelationsChap:relfrompart2}, \ref{exercise:EquivalenceRelationsChap:17}, \ref{exercise:EquivalenceRelationsChap:trickyTransitive})
\item
From a given set create binary relations which have (or do not have) the reflexive, symetric, and/or transitive properties. (\ref{exercise:EquivalenceRelationsChap:BinRelSomePropsEx})
\end{enumerate}


\subsection*{Section \ref{exampleEquivRel}, Examples of equivalence relations}
\subsubsection*{Concepts:}
\begin{enumerate}
\item 
Any function $f$ can be used to define an equivalence relation on the domain of $f$ (Proposition~\ref{proposition:EquivalenceRelationsChap:EquivRelFromFunc}).
\item
Any equivalence relation based on a partition can also be derived from a function (Propositions~\ref{exercise:EquivalenceRelationsChap:partitionfunction} and \ref{proposition:EquivalenceRelationsChap:PartToFunc}).
\item
Any equivalence relation defined in terms of a function can also be derived from a partition (Propositions~\ref{exercise:EquivalenceRelationsChap:66} and \ref{proposition:EquivalenceRelationsChap:FuncIsRange}).

\end{enumerate}

\subsubsection*{Competencies}
\begin{enumerate}
\item
Given a binary relation, show whether or not it is an equivalence relation. (\ref{exercise:EquivalenceRelationsChap:EquivRelShowEx})
\item
Prove the correspondence between partition equivalence relations and function equivalence relations. (\ref{exercise:EquivalenceRelationsChap:partitionfunction})
\item
Using a given partition or function, be able to produce the other one with the same equivalence relation.  (\ref{exercise:EquivalenceRelationsChap:realfunction}, \ref{exercise:EquivalenceRelationsChap:66})
\end{enumerate}


\subsection*{Section \ref{EquivalenceRelationsDefnSect}, Obtaining partitions from equivalence relations}
\subsubsection*{Concepts:}
\begin{enumerate}
\item 
An equivalence class consists of all elements in a set that are equivalent to a particular element (Definition~\ref{DefEquivRel}).
\item
Key properties of equivalence classes.  (Proposition~\ref{proposition:EquivalenceRelationsChap:EquivRelProps}, Proposition~\ref{proposition:EquivalenceRelationsChap:DisjointEquiv})
\item
The equivalence classes for an equivalence relation  form a partition.   (Proposition~\ref{proposition:EquivalenceRelationsChap:EquivRel->Part})
\end{enumerate}

\subsubsection*{Competencies}
\begin{enumerate}
\item
Given a set and an equivalence relation, find the equivalence classes for the equivalence relation.  (\ref{exercise:EquivalenceRelationsChap:EquivClassEasyEx})
\item
Be able to prove properties of equivalence classes. (\ref{exercise:EquivalenceRelationsChap:EquivRelPropsPfEx}, \ref{exercise:EquivalenceRelationsChap:DisjointEquivEx})
\item
Given a set and binary relation, be able to show it is an equivalence relation,  be able to describe and depict the equivalence classes, and give a function that produces the equivalence relation. (\ref{exercise:EquivalenceRelationsChap:ECtoP}, \ref{exercise:EquivalenceRelationsChap:ECtoP2})
\item
Given either a partition or equivalence relation, be able to produce the corresponding equivalence relation or partition respectively. (\ref{exercise:EquivalenceRelationsChap:equivalencePartition})
\end{enumerate}


\subsection*{Section \ref{EquivalenceRelationsModArithSect}, Modular arithmetic redux}
\subsubsection*{Concepts:}
\begin{enumerate}
\item 
Modular equivalence is an equivalence relation.
\item
The equivalence classes  mod $n$ are the remainders mod $n$.  (Proposition~\ref{proposition:EquivalenceRelationsChap:equivModN})
\item
Arithmetic operations for equivalence classes mod $n$. (Definition~\ref{RulesOfModArith}, Definition~\ref{integ_mod_n})
\item
Well-defined operations (Section~\ref{EquivalenceRelationsWellDefSect})
\item
$A/\mathord{\sim}$ (`` $A$ mod twiddle'') is defined as is the set of all equivalence classes of $\sim$ in set $A$.  (Definition~\ref{moduloRelation})
\end{enumerate}

\subsubsection*{Notation}
\begin{enumerate}
\item 
$|S|$ means the number of elements in $S$
\end{enumerate}

\subsubsection*{Competencies}
\begin{enumerate}
\item
Show that an equivalence mod $n$ is an equivalence relation. (\ref{exercise:EquivalenceRelationsChap:ProveModEquiv})
\item
Understand and use arithmetic rules for equivalence classes mod $n$. (\ref{exercise:EquivalenceRelationsChap:47}, \ref{exercise:EquivalenceRelationsChap:Mod3TablesEx})
\item
Create Cayley tables using arithmetic operations defined on equivalence classes mod $n$. (\ref{exercise:EquivalenceRelationsChap:ModArithEx}, \ref{exercise:EquivalenceRelationsChap:ModArithEx2})
\item
Show whether or not an operation is well-defined. (\ref{exercise:EquivalenceRelationsChap:54}, \ref{exercise:EquivalenceRelationsChap:ExpModNotWellDef} -  \ref{exercise:EquivalenceRelationsChap:58})
\end{enumerate}



