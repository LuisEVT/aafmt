\section{Hints for ``Isomorphisms'' exercises}\label{sec:isomorph:hints} 


\noindent Exercise \ref{exercise:isomorph:iso_4}:  Show a counterexample where the sum of two complex numbers is not the same as the sum of their corresponding ordered pairs.

\noindent Exercise \ref{exercise:isomorph:InvCompIso}(a):  According to Definition~\ref{isomorph_defn}, this involves proving two things about $\phi^{-1}$.  What are they?

\noindent Exercise \ref{exercise:isomorph:InvCompIso}(b): You need to prove the same two things as in part (a).  Use results from the Functions chapter.

\noindent Exercise \ref{exercise:isomorph:GpEquivRel}:   Recall that this involves proving the three properties: reflexive, symmetric, and transitive. You may find that Exercise~\ref{exercise:isomorph:InvCompIso}  is useful.

\noindent Exercise \ref{exercise:isomorph:cyclic_noncyclic}:   the proof follows Proposition~\ref{proposition:isomorph:not_isomorph_cyclic} very closely.

\noindent Exercise \ref{exercise:isomorph:PhiBij}(a): Use Exercise~\ref{exercise:isomorph:InvCompIso}.~~(b):  Use Proposition~\ref{proposition:isomorph:abelian_non-abelian}. ~~(c):  Use Proposition~\ref{proposition:isomorph:cyclic_noncyclic}. 

\noindent Exercise \ref{exercise:isomorph:cyclic_inf_isomorph3}(a): Use Proposition~\ref{proposition:isomorph:cyclic_noncyclic}.

\noindent Exercise \ref{exercise:isomorph:zp_isomorph}: This  is a direct result of Proposition~\ref{proposition:cosets:cosets_theorem_7} in the Cosets chapter..

\noindent Exercise \ref{exercise:isomorph:proof_thm_7}: For each group property to be proved, use the corresponding group property for $G$ and $H$ independently.

\noindent Exercise \ref{exercise:isomorph:dirProdAbel}: To show that $G$ is abelian, for arbitrary group elements $g_1, g_2 \in G$ consider the elements $(g_1, id_H)$ and $(g_2,id_H)$ in $G \times H$, where $id_H$ is the identity of the group $H$.  Show that if $(g_1, id_H)$ and $(g_2,id_H)$ commute, then $g_1$ and $g_2$ must also commute. 


\noindent Exercise \ref{exercise:isomorph:prodCycProp}: Since $G \times H$ is cyclic, it must have a generator $(g,h)$.  Show that $g$ is a generator for $G$ and $h$ is a generator for $H$.


\noindent Exercise \ref{exercise:isomorph:direct_commute}:
 Define a function $\phi:G \times H \rightarrow H \times G$ by:  $\phi(g,h) = \underline{~~~~}$ (you fill in the blank).  Show that this function is in fact an isomorphism. 

\noindent Exercise \ref{exercise:isomorph:exel}(c): Consider 2 cases: (i) 9 divides one of the factors $p_i^{e_i}$ in Proposition~\ref{proposition:isomorph:FactorabelianGroup}; (ii) 9 does not divide any of the factors.

\noindent Exercise \ref{exercise:isomorph:isomex}:  Show that $G \times {\var id}_K$ is a subgroup of $G \times K$, and that $G \times {\var id}_K \cong G$; and similarly for $H$.
\medskip

\textbf{Additional exercises}

\noindent Exercise \ref{exercise:isomorph:eoc}: If you take every other vertex in a hexagon, you get an equilateral triangle. Also note that 180-degree rotation is an element of order 2.

