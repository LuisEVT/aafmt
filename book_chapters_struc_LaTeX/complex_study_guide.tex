\section{Study guide  for ``Complex Numbers''  chapter}\label{sec:complex:study} 

Note: all study guides were written by Katrina Smith.

\subsection*{Section \ref{origin_complex}, The origin of complex numbers}
\subsubsection*{Concepts}
\begin{enumerate}
\item 
$n$th roots of a real number
\item
Roots of a real function
\item
Proof by contradiction
\item
Irrational number: cannot be written as a quotient of integers
\item
Definition of $i$ (square root of $-1$)
\item
Definition of complex numbers:  $\mathbb{C}=\{a+bi, \quad a,b \in \mathbb{R} \}$
\end{enumerate}

\subsubsection*{Notation}
\begin{enumerate}
\item
Symbols for number systems: $\mathbb{R}$=real numbers, $\mathbb{Z}$=integers, $\mathbb{N}$=natural numbers (positive integers), $\mathbb{Q}$=rational numbers, $\mathbb{C}$=complex numbers
\item $\exists$ means ``there exists'',  and $\in$ means ``element of''. So $\exists x \in \mathbb{C}, x^3 = -1$ means ``there exists a complex number $x$ such that $x^3=-1$.
\end{enumerate}

\subsubsection*{Competencies}
\begin{enumerate}
\item
Given a real number, prove whether it has any real $n$th roots.  (\ref{exercise:complex:root1}, \ref{exercise:complex:root2}) 
\item
Use proof by contradiction to prove a value or function has no real roots. (\ref{exercise:complex:root1}, \ref{exercise:complex:root2})
\item	
Given an $n$th root which is not an integer, prove that it is irrational. (\ref{exercise:complex:6}, \ref{exercise:complex:4})
\end{enumerate}


\subsection*{Section \ref{complex_arith}, Arithmetic with complex numbers}
\subsubsection*{Concepts:}
\begin{enumerate}
\item 
Complex arithmetic
\item 
Identity \& inverse (additive \& multiplicative)
\item 
Associative law
\item 
Commutative law
\item 
Absolute value or modulus of complex number
\item 
Complex conjugate
\end{enumerate}

\subsubsection*{Key Formulas}
\begin{enumerate}
\item 
Complex addition: $(a + bi) + (c + di) = (a + c) + (b + d)i$
\item 
Complex multiplication (FLOI): $(a + bi)(c + di) = (ac - bd) + (ad + bc)$
\item 
Complex division: $\dfrac{(c + di)}{(a + bi)} = \dfrac{(c + di)(a - bi)}{(a^{2} +  b^{2})}, \text{when}\, (a + bi) \ne 0$
\item 
Modulus of complex number: $\mid z \mid = \sqrt{a^{2} + b^{2}}$
\item 
Complex conjugate of a complex number: $\bar{z} = a - bi$
\end{enumerate}

\subsubsection*{Competencies}
\begin{enumerate}
\item
Simplify expressions involving complex numbers in $a+bi$ form, including inverse and complex conjugation. (\ref{exercise:complex:9}, \ref{exercise:complex:18})
\item
Simplify algebraic expressions with variables in $a+bi$ form, including inverse and complex conjugation. (\ref{exercise:complex:9}d, e, k, \ref{exercise:complex:18}i, j)
\item
Be able to state the associative, inverse, identity, commutative, and distributive properties for different number systems. (\ref{exercise:complex:15})
\item
Prove identities for a complex number $z$ involving algebraic expressions, modulus, complex conjugate \emph{without} converting back to Cartesian form. (\ref{exercise:complex:cxprops}a-i)
\end{enumerate}


\subsection*{Section \ref{complex_graphical}, Alternative representations of complex numbers}
\subsubsection*{Concepts:}
\begin{enumerate}
\item 
Forms of complex number: rectangular and polar form
\item
Converting from rectangular form to polar form and vice versa
\item
Complex multiplication and division using polar form
\item
De Moivre's Theorem (raising complex numbers to integer powers)
\end{enumerate}

\subsubsection*{Key Formulas}
\begin{enumerate}
\item 
Converting from polar form to rectangular: $a = r\cos\theta; b = r\sin\theta$
\item 
Converting from rectangular form to polar: $r = \mid z \mid = \sqrt{a^{2}+b^{2}}; \\\theta = \tan^{-1}\left(\dfrac{b}{a}\right)$ (** be careful about $\tan^{-1}$ -- make sure it's in the right quadrant **)    
\item 
Multiplication of complex numbers: $r \cis \theta \cdot s \cis \phi = rscis(\theta + \phi)$ 
\item 
Division of complex numbers: $\dfrac{r \cis \theta}{s \cis \phi} = \left(\dfrac{r}{s}\right)cis(\theta - \phi)$
\item 
De Moivre's Theorem: $(r \cis \theta)^{n} = r^{n}\cis(n \theta)$

\indent{Notes}
\begin{enumerate}[(a)]
\item
$r \cis \theta \text{ := } r(\cos\theta +  i\sin\theta$); ``:='' means ``is defines as''
\end{enumerate}
\end{enumerate}

\subsubsection*{Competencies}
\begin{enumerate}
\item
Be able to convert back and forth between rectangular form and polar form. (\ref{exercise:complex:21}, \ref{exercise:complex:22})
\item
Perform complex multiplications and divisions using polar form (if the problem is stated in terms of rectangular form, convert to polar form first). (\ref{exercise:complex:25}, \ref{exercise:complex:26})
\item
Raise complex numbers to positive and negative integer powers using de Moivre's theorem. (\ref{exercise:complex:31})
\item
Prove trigonometric formulas for $\cos(n\theta)$ and $\sin(n\theta)$ using de Moivre's theorem. (\ref{exercise:complex:cos form})
\end{enumerate}


\subsection*{Section \ref{complex_roots}, Complex numbers and roots of algebraic equations}
\subsubsection*{Concepts:}
\begin{enumerate}
\item 
$n^{th}$ roots of unity
\item
$n^{th}$ roots of arbitrary complex numbers
\item
The Fundamental Theorem of Algebra
\item
Complex roots of polynomials with real coefficients come in conjugate pairs.
\end{enumerate}

\subsubsection*{Key Formulas}
\begin{enumerate}
\item 
Roots of unity: $z = \cis\left(\dfrac{2k\pi}{n}\right),\text{where} \, k = 0, 1, ... , n - 1 $
\end{enumerate}

\subsubsection*{Competencies}
\begin{enumerate}
\item
Know how to find $n^{th}$ roots of unity for any $n \in \mathbb{N}$. (\ref{exercise:complex:49})
\item
Relate complex conjugation and multiplication by $n^{th}$ roots of unity to rigid motions of a regular $n$-gon. (\ref{exercise:complex:hexagon_rotate} - \ref{exercise:complex:non_comm})
\item
Find all $n^{th}$ roots of a given complex number by (1) Finding a single root using de Moivre's theorem: (2) Multiplying that single root by all $n^{th}$ roots of unity. (Note: there are always $n$ $n^{th}$ roots for any complex number.) (\ref{exercise:complex:58a}, \ref{exercise:complex:60}).
\item
Be able to prove complex conjugation properties of roots of polynomial equations with real coefficients. (\ref{exercise:complex:cubic_conj})
\item
Use complex conjugate properties of roots to reconstruct polynomials. (\ref{exercise:complex:62})
\end{enumerate}







