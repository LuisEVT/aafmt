\chap{In the Beginning}{BeforeWeBegin}

\begin{verse}
Let's start at the very beginning\\
A very good place to start\\
When you read you begin with A B C\\
When you sing you begin with Do Re Me
\end{verse}
\medskip

(Oscar Hammerstein, \emph{The Sound of Music})


\section{Prologue}
If Maria had been the Trapp children's math tutor, she might have continued:  ``When you count, you begin with 1 2 3''. Ordinarily we think of the ``counting numbers''  (which mathematicians call the \emph{natural numbers} or \emph{positive integers}) as the ``very beginning'' of math.\footnote{A famous mathematician once said, ``God made the integers; all else is the work of man.'' (Leopold Kronecker, German mathematician, 1886)}
\medskip

It's true that when we learn math in school, we begin with the counting numbers. But do we really start at the ``very beginning''?   How do we know that $1 + 1 = 2$? How do we know that the methods we learned to add,  multiply, divide, and subtract will always work? We've been taught how to factor integers into prime factors. But how do we know this always works?

 Mathematicians  are the ultimate skeptics: they won't take ``Everyone knows'' or ``It's obvious''  as valid reasons. They keep asking ``why'', breaking things down into the most basic assumptions possible. The very basic assumptions they end up with are called \bfii{axioms}\index{Axiom}. They then take these axioms and play with them like building blocks. The arguments that they build with these axioms are called \bfii{proofs}, and the conclusions of these proofs are called \bfii{propositions}\index{Proposition!mathematical} or \bfii{theorems}\index{Theorem}.  

The mathematician's path is not an easy one. It is exceedingly  difficult to push things back to their foundations. For example, arithmetic was used for thousands of years before a set of simple axioms was finally developed (you may look up ``Peano axioms'' on the web).\footnote{The same is true for calculus. Newton and Leibniz first developed calculus around 1670, but it was not made rigorous until 150 years later.}
Since this is an elementary book, we are not going to try to meet rigorous mathematical standards. Instead, we'll lean heavily on examples, including the integers, rationals, and real numbers. Once you are really proficient with different examples, then it will be easier to follow more advanced ideas.\footnote{Historically, mathematics has usually progressed this way: examples first, 
and axioms later after the examples are well-understood.}

This text is loaded with proofs, which are as unavoidable in abstract mathematics as they are intimidating to many students. We try to ``tone things down'' as much as possible. For example, we will take as ``fact'' many of the things that you learned in high school and  college algebra--even though you've never seen proofs of these ``facts''.  In the next section  we remind you of some of these ``facts''.  When writing proofs or doing exercise feel free to use any of these facts.  If you have to give a reason, you can just say  ``basic algebra''.\index{Algebra!high school and college}

We close this prologue with the assurance that abstract algebra is a beautiful subject that brings amazing insights into the nature of numbers, and the nature of Nature itself. Furthermore, engineers and technologists are finding more and more practical applications, as we shall see in the coming chapters.



\section {Integers, rational numbers, real numbers}



We assume that you have already been introduced to the following number systems: integers, rational numbers, and real numbers.\footnote{This section and the following were written by David Weathers (edited and expanded by C.T.)}  These number systems possess the well-known arithmetic operations of addition, subtraction, multiplication, and division. The following statements hold for all of these number systems. 

\begin{warn}
There are number systems for which the following properties do NOT hold (as we shall see later). So they may be safely assumed ONLY for integers, rational numbers, and real numbers.
\end{warn}

\subsection{Operations and relations }\label{OpsAndRels}

We assume the following properties of these arithmetic operations:

\begin{enumerate}[(a)]
\item
\emph{Commutative}: When two numbers are added together, the two numbers can be exchanged without changing the value of the result. The same thing is true for two numbers being mulitiplied together.
\item
\emph{Associative}: When three or more numbers are added together, changing the grouping of the numbers being added does not change the value of the result.   The same goes for three or more numbers multiplied together. (Note that in arithmetic expressions, the ``grouping'' of numbers is indicated by parentheses.)
\item
\emph{Distributive}:   Multiplying a number by a sum gives the same result as taking the sum of the products.
\item
\emph{Order}:   Given two numbers, exactly one of these three are true: either the first number is greater than the second, or the second number is greater than the first, or the two numbers are equal.
\item
\emph{Identity}:   Addition by 0 or multiplication by 1 result in no change of original number.
\item
The sum of two  positive numbers  is positive. The sum of two negative numbers is negative.
\item
The product of two  nonzero numbers with the same sign is positive. The product of two numbers with different signs is negative.
\item
If  the product of two numbers is zero, then one or the other number must be zero.
\end {enumerate}

\begin{exercise}\label{exercise:BeforeWeBegin:1}
\begin{enumerate}[(a)]
\item
For the properties a,b,c,e  above, give (i) a specific example for addition, using numbers, and (ii) a general statement for multiplication, using variables.  
For example, for property (a) (the commutative property) a specific example would be  $3+5 = 5+3$, and a general statement would be $x \cdot y=y \cdot x$. 
\item 
For properties d,f,g,h above, give a specific example which illustrates the property using numbers.
\end{enumerate}
 \end{exercise}
\begin{exercise}\label{exercise:BeforeWeBegin:2}
\begin{enumerate}[(a)]
\item
Give an example (using numbers)  that shows that subtraction is \emph{not} commutative 
.\item
Give an example (using numbers)  that shows that division is \emph{not} associative. 
\end{enumerate}
\end{exercise}
\begin{exercise}\label{exercise:BeforeWeBegin:3}
Suppose $a > b$,  $b \ge 0$ and $ab = 0$.  What can you conclude about $a$ and $b$? Use one (or more)  of the properties we have mentioned to justify your answer.
\end{exercise}
\begin{exercise}
Which of the above properties must be used to prove each of the following statements?
\begin{enumerate}[(a)]
\item
$(x+y)+(z+w) = (z+w)+(x+y)$
\item
$(x \cdot y) \cdot z = ( z \cdot  x) \cdot y)$
\item
$(a\cdot x + a \cdot y) + a \cdot z = a \cdot ( (x+y) + z)$
\item
$((a \cdot b) \cdot c + b \cdot c) + c \cdot a = c \cdot ((a+b) + a \cdot b)$
\end{enumerate}
\end{exercise} 

Note that the associative property allows us to write expressions without putting in so many parentheses.  So instead of writing $(a+b)+c$, we may simply write $a+b+c$. By the same reasoning, we can remove parentheses from any expression that involves only addition, or any expression that involves only multiplication: so for instance, $(a \cdot (b \cdot c) \cdot d) \cdot e = a \cdot b \cdot c \cdot d \cdots e$. Using the associative and distributive property, it is possible to write any arithmetic expression without parentheses. So for example, $(a \cdot b) \cdot (c + d)$ can be written as $a \cdot b \cdot c + a \cdot b \cdot d$.  (Remember that according to operator precedence rules, multiplication is always performed before addition: thus $3 \cdot 4 + 2$ is evaluated by first taking $3 \cdot 4$ and then adding 2.)

\begin{exercise}
Rewrite the following expressions without any parentheses, using \emph{only} the  associative and distributive properties.  ({\underline Don't use commutative in this exercise!})
\begin{enumerate}[(a)]
\item
$(((x + y) + (y+z))\cdot w) - 2y \cdot w$
\item
$0.5 \cdot ( (x+y) + (y + z) + (z + x))$
\item
$((((((a+b)+c) \cdot d)+ e) \cdot f) + g) + h$
\end{enumerate}
\end{exercise}

\begin{exercise}
For parts (a--c) of the preceding exercise, now apply the commutative property to the results to simplify the expressions as much as possible.
\end{exercise}

\begin{exercise}
Evaluate the following expressions by hand (no calculators!).
\begin{enumerate}[(a)]
\item
$3 \cdot 4 + 5 + 6 \cdot 2$
\item
$3 + 4 \cdot 5 \cdot 6 + 2$
\item
$1 + 2 \cdot 3 + 3 \cdot 4 \cdot 5 + 5 \cdot 6 \cdot 7 \cdot 2$
\end{enumerate}
\end{exercise}

\subsection {Manipulating equations and inequalities}

Following are some common rules for manipulating equations and inequalities. Notice there are two types of inequalities:  \bfii{strict inequalities} (that use the $>$ or $<$ symbols) and \bfii{nonstrict inequalities} (that use the $\ge$ or $\le$ symbols).\index{Inequality!strict}\index{Inequality!nonstrict} 

\begin {enumerate}[(A)]
\item
\emph{Substitution}: If two quantities are equal then one can be substituted for the other in any true equation or inequality and the result will still be true. 
\item
\emph{Balanced operations}: Given an equation, one can perform the same operation to both sides of the equation and maintain equality.  The same is true for inequalities for the operation of addition, and for multiplication or division by a \emph{positive} number.
\item
Multiplying or dividing an inequality by a negative value will reverse the inequality symbol.
\item
The ratio of two integers can always be reduced to lowest terms, so that the numerator and denominator have no common factors.
\end {enumerate}

\begin{exercise}\label{exercise:BeforeWeBegin:4}
Give specific examples for statements (A--D)   given above. You may use either numbers or variables (or both) in your examples.. For (A) and (B), give one example for each of the following cases: (i) equality, (ii) strict inequality, (iii) nonstrict inequality.
\end{exercise}


%\section {Number Theory Rules}
%
%Given that a prime number $p$ evenly divides into a number $q$, it is true that the prime number $p$ must divide one of the factors of $q$.
%
%Given that the product of two numbers is equal to 0, then it is true that one of the numbers must be 0.

\subsection {Exponentiation (VERY important)}

Exponentiation is one of the key tools of abstract algebra. It is \emph{essential} that you know your exponent rules inside and out!  

\begin{enumerate}[(I)]
\item
Any nonzero number raised to the power of 0 is equal to 1.
\footnote{ Technically $0^0$ is undefined, although often it is taken to be 1. Try it on your calculator!}
\item
A number raised to the sum of two exponents  is the product of the same number raised to each individual exponent.
\item
A number raised to the power which is then raised to another power is equal to the same number raised to the product of the two powers.
\item
A number raised to a negative exponent is equal to the reciprocal of the number  raised to a positive .
\item
Taking the  product of two numbers  and raising to a given power is the same as taking the powers of the two numbers separately, then multiplying the results.
\end{enumerate}

\begin{exercise}\label{exercise:BeforeWeBegin:5}
For each of the above items (I--V),  give a general equation (using variables) that expresses the rule.  For example one possible answer to (II) is:  $x^{y+z} = x^y \cdot x^z$ .
\end{exercise}
\begin{exercise}\label{exercise:BeforeWeBegin:6}
Write an equation that shows another way to express a number raised to a power that is the difference of two numbers.
\end{exercise}

\section{Test yourself}
Test yourself with the following exercises. If you feel totally lost, I strongly recommend that you improve your basic algebra skills before continuing with this course. This may seem harsh, but I only mean to spare you agony. All too often students go through the motions of learning this material, and in the end they learn nothing because their basic skills are deficient. If you want to play baseball, you'd better learn how to throw, catch, and  the ball first.

\begin{exercise}\label{exercise:BeforeWeBegin:7}
Simplify the following expressions. Factor whenever possible
\begin{multicols}{2}
\begin{enumerate}[(a)]
\item
$ 2^4 4^2$
\item
$ \dfrac{3^9}{9^3}$
\item
$\left( \dfrac{5}{9} \right)^7 \left( \dfrac{9}{5} \right)^6$
\item
$\dfrac{a^5}{a^7} \, \cdot \, \dfrac{a^3}{a}$
\item
$x(y-1) - y(x-1)$
\end{enumerate}
\end{multicols}
\end{exercise}


\begin{exercise}\label{exercise:BeforeWeBegin:8}
Same instructions as the previous exercise. These examples are  harder. (\emph{Hint}: It's usually best to make the base of an exponent as simple as possible. Notice for instance that $4^7 = (2^2)^7 = 2^{14}$.)
\begin{multicols}{2}
\begin{enumerate}[(a)]
\item
$6^{1/2\cdot}2^{1/6}\cdot3^{3/2}\cdot2^{1/3}$
\item
$(9^3)(4^7)\left(\frac{1}{2}\right)^8\left(\frac{1}{12}\right)^6$
\item
$4^5 \cdot 2^3 \cdot \left(\frac{1}{2}\right)^5 \cdot \left( \frac{1}{4} \right) ^3$
\item
$2^3 \cdot 3^4 \cdot 4^5 \cdot 2^{-5} \cdot 3^{-4} \cdot 4^{-3}$
\item
$\dfrac{x(x-3)+3(3-x)}{(x-3)^2}$
\end{enumerate}
\end{multicols}
\end{exercise}


\begin{exercise}\label{exercise:BeforeWeBegin:9}
Same instructions as the previous exercise. These examples are even harder. (\emph{Hint}: Each answer is a single term, there are no sums or differences of terms.)
\begin{multicols}{2}
\begin{enumerate}[(a)]
\item
$\dfrac{a^5 +a^3 - 2a^4}{(a-1)^2}$
\item
$a^x b^{3x}(ab)^{-2x}(a^2 b)^{x/2}$
\item
$(x+y^{-1})^{-2}(xy+1)^2$
\item
$\dfrac{(3^x+9^x)(1-3^x)}{1-9^x})$
\item
$\dfrac{3x^2 - x}{x-1} + \dfrac{2x}{1-x}$

\end{enumerate}
\end{multicols}
\end{exercise}

\begin{exercise}\label{exercise:BeforeWeBegin:10}
Find ALL  real solutions to the following equations. 
\begin{multicols}{2}
\begin{enumerate}[(a)]
\item
$x^2 = 5x$
\item
$(x - \sqrt{7})(x+\sqrt{7}) = 2$
\item
$2^{4+x} = 4(2^{2x})$
\item
$3^{-x} = 3(3^{2x})$
\item
$16^5 = x^4$
\item
$\dfrac{1}{1 + 1/x} -1= -1/10$
\end{enumerate}
\end{multicols}
\end{exercise}